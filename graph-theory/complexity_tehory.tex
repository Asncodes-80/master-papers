\documentclass[a4paper]{article}
\usepackage{float}
\usepackage{geometry}
\usepackage{listings}
\usepackage{hyperref}
\usepackage{plantuml}
\usepackage{graphicx}
\usepackage{ragged2e}
\usepackage{color}
\usepackage{xepersian}
\usepackage{subfiles}
\newgeometry{left=1.4cm, right=1.4cm, bottom=2.0cm, top=2.0cm}
\settextfont[Scale=1]{XB Roya}

\title{نظریه پیچیدگی}
\author{علیرضا سلطانی نشان}

\begin{document}
\maketitle

\section{مجوز}

به فایل license همراه این برگه توجه کنید. این برگه تحت مجوز GPLv3 منتشر شده است
که اجازه نشر و استفاده (کد و خروجی/pdf) را رایگان می‌دهد.

\section{پیش گفتار}

در حال خواندن مقاله‌ای در مورد طراحی دارو‌ها به وسیله ترکیب و اتصال پروتئین‌ها
بودم که متوجه شدم الگوریتم‌ها و قضیه‌هایی که در آن مطرح شده را نمی‌توانم متوجه
شوم، تصمیم بر آن گرفتم که یک جزوه ساده در مورد آن‌ها بنویسم. منابع مطالبی که
اینجا نگاشته شده است را در بخش مراجع می‌توانید مشاهده کنید. در صورت مغایرت با
مطالب درست می‌توانید آن را تصحیح کنید.

\end{document}
