\documentclass[a4paper]{article}
\usepackage{float}
\usepackage{geometry}
\usepackage{listings}
\usepackage{hyperref}
\usepackage{plantuml}
\usepackage{graphicx}
\usepackage{ragged2e}
\usepackage{color}
\usepackage{xepersian}
\usepackage{subfiles}
\newgeometry{left=1.4cm, right=1.4cm, bottom=2.0cm, top=2.0cm}
\settextfont[Scale=1]{XB Roya}

\title{مطلوب است بدست آوردن شرایط زیر در گراف \lr{Q(n)}}
\author{علیرضا سلطانی نشان}

\begin{document}
\maketitle

\begin{LTR}
    \begin{itemize}
        \item Independent set
        \item Matching
        \item Vertex color, edge color
    \end{itemize}
\end{LTR}

\begin{table}[H]
    \centering
    \begin{tabular}{c|c|c|c} 
        شاخص & \lr{Min/Max} & \lr{Of What?} & شرط \\ \hline
        $\alpha(G(n))$ & \lr{Max} & \lr{Vertces} & رئوس مجاور نباشد \\
        $\alpha'(G(n))$ & \lr{Min} & \lr{Vertces} & پوشش از سمت هر لبه \\
        $\beta(G(n))$ & \lr{Max} & \lr{Edges} & هر راس با حداقل یک لبه پوشانده شده باشد \\
        $\beta'(G(n))$ & \lr{Min} & \lr{Vertces} & هیچ دو راسی نباید مجاور هم باشند \\
    \end{tabular}
    \caption{پاسخ}
    \label{fig:ahpCostStep2}
\end{table}


\end{document}