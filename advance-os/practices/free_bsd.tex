% !TEX program = xelatex
\documentclass[20pt, a4paper]{article}
\usepackage{hyperref}
% Justify paragraphs
\usepackage{graphicx}
\usepackage{ragged2e}
\usepackage{color}
\usepackage{xepersian}
\usepackage{subfiles}
\settextfont[Scale=1]{XB Roya}
\renewcommand{\baselinestretch}{1.5}

\begin{document}
\centerline{سیستم عامل \lr{Free BSD}}
\centerline{درس سیستم عامل پیشرفته}
\centerline{علیرضا سلطانی نشان}
\centerline{\today}
\tableofcontents

\section{نگاه اولیه}

سیستم عامل FreeBSD یک سیستم عامل \lr{Unix-like} است که توسط گروه نرم افزاری
برکلی توسعه داده شده است.

نخستین نسخه این سیستم عامل در سال ۱۹۹۳ منتشر شد. در سال ۲۰۰۵، FreeBSD محبوب ترین
سیستم عامل BSD متن باز بود که حدودا بیشتر از سه چهارم سیستم های BSD نصب شده و
دارای مجوز مجاز را تشکیل می‌داد.

سیستم عامل FreeBSD شباهت‌هایی نسبت به سیستم عامل لینوکس دارد. اما مهمترین تفاوت
این دو در مجوز‌های آنهاست:

در سیستم عامل FreeBSD شما با یک سیستم کامل طرف هستید. این بدان معناست که این
پروژه تمام مسائل مربوط به کرنل، درایور‌ها، برنامه‌های کاربردی و حتی مستندات را
پیاده سازی کرده است، در مقابل، سیستم عامل لینوکس تنها یک کرنل و درایور‌های
مربوطه که بر پایه نرم‌افزار‌های شخص ثالث است را در سیستم نرم‌افزاری خود ارائه
می‌کند.

کد منبع این سیستم عامل تحت مجوز BSD که برخلاف مجوز GPL به صورت کپی لفت که توسط
لینوکس استفاده می‌شود، در دسترس عموم قرار دارد.


پروژه FreeBSD شامل یک تیم بزرگ امنیت است که بر روی تمام نرم‌افزار‌هایی که توسط
این گروه تحقیقاتی توسعه داده می‌شود نظارت و کنترل دارد.

نرم‌افزار‌های شخص ثالث گسترده‌ای می‌توانند از روی باینریشان نصب و راه‌اندازی شود
که این عمل توسط یک مدیریت نرم‌افزار یا اصلاحات \lr{Package manager} میسر  
می‌شود. همچنین کاربران می‌توانند به جای استفاده از یک مدیر بسته مناسب در سیستم
عامل، برنامه مورد نظر را در سیستم بارگیری و اقدام به نصب دستی آن کنند.

بسیاری از کد‌های FreeBSD به بخشی جدایی ناپذیر از سیستم عامل‌های دیگر مانند
داروین که از بنیادهای اصلی سیستم‌عامل‌های macOS iOS iPadOS watchOS و tvOS است،
همچنین سیستم عامل‌های دیگر مانند \lr{TrueNAS} که یک سیستم عامل متن باز براساس
\lr{NAS/SAN} است و همچنین سیستم نرم‌افزاری کنسول‌های بازی پلی استیشن ۳ و پلی
استیشن ۴، تبدیل شده است. از دیگر سیستم‌های BSD می‌توان به OpenBSD، NetBSD و
DragonFlyBSD اشاره کرد که حاوی مقدار زیادی از کد‌های FreeBSD می‌باشد.

\section{تاریخچه}

\subsection{پیش زمینه}

در سال ۱۹۷۴، پروفوسور باب فابری از دانشگاه کالیفرنیا-برکلی مجوز سورس یونیکس را
از شرکت ارائه دهنده سرویس AT\&T دریافت کرد که توسط گروه تحقیق و توسعه سیستم‌های
کامپویتری DARPA پیشتیبانی می‌شد تا بتوانند با تحقیق و توسعه، سیستم یونیکس AT\&T
را ویرایش و بهبود دهند. آنها توسعه زیادی را روی این سیستم اعمال کردند و این
سیستم عامل را بعد از اعمال ویرایشات جدید \lr{Berkeley Unix} یا \lr{Berkeley
Software Distribution} یا اختصارا \lr{(BSD)} نامیدند.  در این نسخه ویژگی‌های
زیادی از جمله \lr{TCP/IP}, حافظه مجازی و \lr{Berkeley Fast File System} را
پیاده‌سازی کردند. پروژه \lr{BSD} در سال ۱۹۷۶ توسط بیل جوی تاسیس شد. اما از
آنجایی که \lr{BSD} حاوی کدی از یونیکس شرکت AT\&T بود، همه دریافت کنندگان این
سیستم عامل باید در ابتدا از AT\&T مجوز ویرایش دریافت می‌کردند تا بتوانند از
\lr{BSD} استفاده کنند.

\section{ویژگی‌های کلیدی}

\subsection{موارد استفاده}

این سیستم عامل شامل مجموعه شگفت انگیزی از نرم‌افزار‌های سمت سرور است که به این
سیستم عامل اجازه می‌دهد تا نقش یک سرویس ایمیل، وب سرویس، فایروال، سرویس انتقال
فایل، سرویس دی‌ان‌اس و حتی یک مسیریاب را ایفا کند.

سیستم عامل FreeBSD می‌تواند روی یک دستگاه دسکتاپ یا روی یک لپتاپ نصب و
راه‌اندازی شود.  توجه داشته باشید که مدیر پنجره‌ها یا \lr{X Window System} به
صورت پیش فرض روی آن نصب نیست، اما کاربر می‌تواند به دلخواه مدیر پنجره‌ای که مورد
نظر دارد را روی این سیستم عامل نصب کند. همچنین جالب است بدانید که می‌توانید
سیستم \lr{Wayland} را روی آن نصب کنید که براساس فروم‌ها کاربران اعلام کردند که
این نصب به صورت غیر رسمی است و به صورت مستقیم توسط آن پشتیبانی نمی‌شود.  به طور
کل تعداد زیادی از محیط‌های دسکتاپ از این سیستم عامل پشتیبانی می‌کنند. از این
محیط‌ها می‌توان به موارد زیر اشاره کرد:

\begin{enumerate}
    \item Lumina
    \item GNOME
    \item KDE
    \item XFCE
\end{enumerate}

\subsection{پشتیبانی از معماری سخت‌افزاری}

این سیستم عامل از معماری‌های مختلفی طی به روز رسانی‌های متعدد، پشتیبانی می‌کند.
این معماری‌ها عبارت‌اند از:

\begin{enumerate}
    \item x86-64 در FreeBSD 13
    \item aarch64
    \item x86-32
    \item PowerPC
    \item RISC-V
    \item SPARC در FreeBSD 12
    \item 32 bit ARM
    \item armv6
    \item armv7
\end{enumerate}

\subsection{شبکه}

FreeBSD کاملا براساس استک \lr{TCP/IP} است، به گونه‌ای که از پروتکل‌های متعددی به
صورت سازگار پشتیبانی می‌کند.  همچنین از نسخه ۶ آدرس‌دهی شبکه، \lr{SCTP}،
\lr{IPSec} و به خصوص از شبکه بی‌سیم \lr{Wi-Fi} پشتیبانی می‌کند.

\subsection{حافظه}

این سیستم عامل، دارای چندین ویژگی منحصر به فرد مربوط به ذخیره سازی است. به روز
رسانی‌های نرم‌افزاری می‌توانند از سازگاری سیستم فایل \lr{UFS} (که به طور گسترده
در BSD استفاده می‌شود.) در صورت خرابی سیستم محافظت کند. همچنین تهیه انسپ‌شات‌های
فوری از سیستم فایل‌ها اجازه می‌دهد تا از یک فایل سیستم \lr{UFS} در یک لحظه در
زمان پشتیبان گیری کند.  همچنین این سیستم‌عامل به لطف \lr{GEOM} از قابلیت
\lr{RAID} از سطح ۰، ۱، ۲ و ۳ پشتیبانی می‌کند.  از دیگر ویژگی‌های مدیریت حافظه
این سیستم عامل به توان به عوامل زیر اشاره کرد.

\begin{enumerate}
    \item قابلیت رمزنگاری و قفل گذاری روی دیسک و دیسکت‌ها
    \item قابلیت گرفتن ژورنال از عملیات داخل حافظه
    \item ادغام و الحاق و شرینک کردن دیسک‌ها
    \item حافظه کشینگ
    \item دسترسی به فضای ذخیره سازی مبتنی بر شبکه یا استفاده از حافظه به اشتراک
    گذاشته شده شبکه ای
\end{enumerate}

\subsection{امنیت}

ویژگی‌های متعددی را ارائه می‌دهد. از این ویژگی‌ها می‌توان به پایه ترین آنها یعنی
\lr{access-control lists (ACLs)} اشاره کرد. که به مدیر این سیستم عامل اجازه
می‌دهد تا دسترسی کاربران را روی منابع سیستمی و غیر سیستم کنترل و در صورت امکان
آنها را محدود کند. در برخی مراجع این لیست کنترل را دسترسی‌ها و مجوز‌های سیستم
عامل می‌نامند.

\end{document}
