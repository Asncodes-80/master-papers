\documentclass[20pt, a4paper]{article}
\usepackage{hyperref}
\usepackage{graphicx}
\usepackage{ragged2e}
\usepackage{color}
\usepackage{xepersian}
\usepackage{subfiles}
\settextfont[Scale=1]{XB Roya}
\renewcommand{\baselinestretch}{1.5}

\begin{document}
\centerline{سیستم عامل \lr{Nooks}}
\centerline{علیرضا سلطانی نشان}
\centerline{\today}

\section{هشدار}

در هنگام نگارش این برگه از صحت اطلاعات جست و جو شده مطمئن نبوده‌ام. این اطمینان
از وجود اطلاعات دقیق‌تر در مورد موضوع تحقیق حاصل خواهد شد. در صورت یافتن اطلاعات
بهتر، لطفا لینک را بخش کامیت این برگه در قسمت نظرات ارسال کنید.

\section{سیستم عامل Nook}

برند \lr{Barnes \& Noble Nook} که یک برند آمریکایی می‌باشد برای اولین بار
کتابخوان الکترونیک را تولید کردند که براساس سیستم عامل اندروید بوده است. اولین
دستگاه Nook در اکتبر سال ۲۰۰۹ در آمریکا منتشر شد.

این دستگاه از یک صفحه نمایش ۶ اینچی بهره می‌برد که قابلیت نمایش رنگ و صفحه قابل
لمس را ارائه می‌دهد. اولین نسخه این دستگاه از اتصال به Wi-Fi بهره‌ می‌برد که بعد
از گذشت چند نسخه اولیه توانستند نسخه‌های بعدی این دستگاه را به اتصال بی‌سیم و 3G
مجهز کنند.

در حقیقت شرکت \lr{Barnes \& Noble Nook} یک شرکت خرده فروشی کتاب است که بعد از
گذشت مدتی ایده‌ای مانند آمازون کایندل به ذهن آن‌ها خطور کرد، کتابخوان الکترونیکی
خودشان را تولید کنند تا بتوانند از بریده شدن درختان برای تولید برگه کتاب جلوگیری
کنند.

آخرین نسخه این کتابخوان \lr{Nook GlowLight 4 Plus} می‌باشد که در وبسایت رسمی
\lr{www.barnesandnoble.com} به فروش می‌رسد. امروزه رقابت بین شرکت‌های نرم‌افزاری
و تولید کننده دستگاه‌ها بسیار زیاد شده است که بقای آنها به تولید بیشتر و ارائه
آپدیت‌های گوناگون برای حفظ بازخورد کاربران، وابسته است.

این برند یک رقیب جدی در برابر شرکت Amazon در قسمت فروش کتاب و دستگاه‌های
دیجیتالی کتابخوان است که شما می‌توانید کتاب‌های این دو مجموعه را هم به صورت
رایگان هم به صورت خرید اشتراک، آنها را مطالعه کنید. همچنین این شرکت‌ها نسخه قابل
استفاده از کتاب‌های فروشگاه خود را به صورت قالب \lr{.epub} هم ارائه می‌دهند.

\end{document}