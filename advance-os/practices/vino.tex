\documentclass[20pt, a4paper]{article}
\usepackage{hyperref}
\usepackage{graphicx}
\usepackage{ragged2e}
\usepackage{color}
\usepackage{xepersian}
\usepackage{subfiles}
\settextfont[Scale=1]{XB Roya}
\renewcommand{\baselinestretch}{1.5}

\begin{document}
\centerline{سیستم عامل \lr{VINO}}
\centerline{علیرضا سلطانی نشان}
\centerline{\today}
\tableofcontents

\section{مقدمه}

سیستم عامل \lr{VINO} یکی از پروژه‌های دانشگاه هاروارد برای تولید یک کرنل قابل
توسعه بر اساس سیستم \lr{NetBSD} می‌باشد. وضعیت این پروژه در دانشگاه به صورت غیر
فعال است.

سیستم عامل \lr{VINO} یک سیستم عامل از نوع \lr{Unix-like} است. در این سیستم عامل
تمام ابزار‌های مد کاربر وجود دارد و نرم‌افزار‌های استاندارد Unix در این سیستم
عامل قابلیت کامپایل و اجرا دارند.

این سیستم عامل همچنین قابلیت‌هایی را برای توسعه آن نرم‌افزار‌های سنتی موجود در
یونیکس را دارا می‌باشد.

در حقیقت این سیستم عامل از NetBSD منتقل شده است. تقریبا تمام برنامه‌های سمت کرنل
و برنامه‌های سمت کاربر این سیستم عامل در هنگام انتقال از صفر نوشته شده اند. 

این سیستم عامل از سیستم‌های چند پردازنده پشتیبانی نمی‌کند. 

\section{اهداف طراحی و پیاده‌سازی این سیستم عامل}

این بخش نوشته مستندات دانشجویان دانشگاه هاروارد در مورد این سیستم عامل می‌باشد:

\subsection{قابل استفاده و ماژولار}

ما امید داشتیم که یک سیستمی از اجزا بسازیم که کاربران همانگونه که می‌خواهند
سیستمشان را برای اجرای برنامه‌ها بسازند. ما علاوه‌بر این به اپلیکیشن‌های
نرم‌افزاری اجازه دادیم که از اجزای سیستم حتی اجزای کرنل استفاده مجدد کنند. این
کار باعث می‌شود که حجم کمتری از کد‌ها توسط برنامه نویسان نوشته شود که در نهایت
موجب اجرای سریع برنامه‌های کاربردی خواهد شد.

\subsection{خط مش کرنل برنامه محور}

ما بر این باور بودیم که کرنل باید ارائه دهنده مکانیزم باشد تا سیاست و خط مش.
سیاست‌ها توسط ادمین سیستم، برنامه‌نویسان و یا حتی کاربران نهایی تعیین و انتخاب
می‌شوند. این سیستم عامل به گونه‌ای طراحی شده است که تمام سیاست‌ها تا حد امکان در
فهرست تنظیمات کاربران قابل شخصی‌سازی باشد. مانند سیاست‌های داخلی خواندن و نوشتن
در سیستم فایل.

\subsection{\lr{Universal resource interface}}

استفاده از واسط‌های مختلف برای دستیابی به نتایج یکسان نباید به صورت ضروری باشد،
فقط به این دلیل که مواردی که با آنها سرو کار داریم به روش‌های نامرتبط متفاوت
هستند. واسط‌های kernel و مد کاربر به گونه‌ای طراحی شده‌اند که تا آنجا که ممکن
است حجم داده‌های تکراری کم باشد. این کار باعث کاهش زمان بارگیری در یک اپلیکیشن
توسط توسعه‌دهندگان می‌شود و این راهکار برای کاهش اندازه‌ها باعث می‌شود که
نیازمندی‌های برنامه نیز کمتر شود.

\subsection{\lr{Jump to no conclusions}}

روش استاندارد طراحی یک سیستم عامل و اجزای ‌آن باید آنقدر تکامل نسبت به ۴۰ سال
گذشته داشته باشند یا در بسیاری از موارد آنقدر معمولی می‌شوند که دیگر هیچ کس حتی
هیچ جایگزینی را به خاطر نمی‌آورد. بعضی اوقات یک پاسخ بهینه برای رسیدگی به
درخواست‌ها پیدا می‌شود. اما باید توجه داشت آن چیزی که که در گذشته به صورت "بهینه
ترین" یا "بهترین پاسخ" برای حل مشکلات مطرح می‌شد نباید انتظار داشت که امروز هم
شاهد همان عملکرد گذشته برای تکنولوژی جدید باشیم. در سیستم عامل VINO ما سعی کردیم
که هیچ یک از این راه‌حل‌های مرسوم را بدیهی جلوه نکنیم. به همین دلیل ما تلاش
می‌کنیم که طراحی جدیدی را توسعه دهیم تا نیازمندی‌های جاری را بهتر مدیریت و کنترل
کند.

\subsection{مجوز}

VINO قابل توزیع براساس مجوز BSD-like می‌باشد. متاسفانه این مجوز برای هیچ یک از
نسخه‌های موجود این سیستم عامل به موقع آماده نبود. اگر شما می‌خواهید روی یکی از
نسخه‌های منتشر شده این سیستم عامل فعالیتی انجام بدهید و یا از آن استفاده کنید
اما شاهد کپی رایت مربوط به دانشگاه هاروارد بودید که شامل هیچ مجوزی نبود، با ما
ارتباط برقرار کنید و ما سعی خواهیم کرد که موارد مربوطه را برای شما آماده کنیم.

\end{document}