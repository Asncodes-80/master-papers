\documentclass[20pt, a4paper]{article}
\usepackage{biblatex}
\usepackage{hyperref}
\usepackage{graphicx}
\usepackage{ragged2e}
\usepackage{color}
\usepackage{xepersian}
\usepackage{subfiles}
\settextfont[Scale=1]{XB Roya}
\renewcommand{\baselinestretch}{1.5}

\begin{document}
\centerline{SPIN}
\centerline{درس سیستم عامل پیشرفته}
\centerline{علیرضا سلطانی نشان}
\centerline{\today}
\tableofcontents

\section{مقدمه}

سیستم عامل SPIN یک پروژه تحقیقاتی که با زبان برنامه نویسی Modula-3 توسعه و
پیاده‌سازی شده است، می‌باشد. این سیستم عامل یک پروژه متن باز است که برای سه هدف
اصلی طراحی و پیاده‌سازی شده است:

\begin{enumerate}
    \item انعطاف پذیری \footnote{Flexibility}
    \item ایمن \footnote{Safety}
    \item اجرای مناسب \footnote{Performance}
\end{enumerate}

این پروژه در دانشگاه واشنگتون \footnote{University of Washington} توسعه داده شد.

در این سیستم عامل، کرنل می‌تواند توسط \lr{Dynamic loading} ماژول‌ها که با
اینترفیس‌ها پیاده‌سازی شده اند دامنه اجرایی را مشخص می‌کند. این دامنه‌ها توسط
زبان برنامه نویسی مشخص شده‌اند. تمام اکستنشن‌های کرنل زیر مجموعه ایمن زبان
Modula-3 با ساختار‌های MetaLanguage و نوع تایپ سیف سیستم نوشته شده‌اند. همچنین
این سیستم اکسنشن‌ ران‌تایم کامپایلر را گزارش می‌کند.

یک مجموعه‌ای از اکستنشن‌های کرنل یک Api را ارائه می‌دهند که می‌تواند سیستم
دیجیتالی یونیکس را برای صدا زدن آن اینترفیس‌ها، شبیه سازی کند.

\section{منظور از \lr{Dynamic loading}}

موضوع \lr{Dynamic loading} یک مکانیزمی است که توسط آن یک برنامه کامپیوتری
می‌تواند در هنگام اجرا \footnote{Run time} یک کتابخانه را چه به صورت یک منبع
برنامه یا به صورت باینری، داخل حافظه رم خوانده، تمام آدرس‌های مربوط به توابع و
متغیر‌هایی که داخل کتابخانه هستند را دریافت کرده، این توابع را اجرا کرده یا به
متغیر‌ها دسترسی داشته و در نهایت پس از انجام کار‌هایش کتابخانه را از داخل حافظه
تخلیه می‌کند و اصطلاحا حافظه را از آن کتابخانه آزاد می‌کند.

\lr{Dynamic loading} به یک برنامه کامپیوتری اجازه اجرا در حالت نبود کتابخانه‌ها
و وابستگی‌های مربوط به آن برنامه، می‌دهد که می‌تواند کتابخانه‌های موجود را پیدا
کند، و به صورت بلقوه بتواند توابع و قابلیت‌های اضافی که در برنامه گنجانده نشده
است را بدست آورد و کاری کند که نرم‌افزار بدون مشکل اجرا شود.

\section{زبان برنامه نویسی Modula}

Modula-3 یک زبان برنامه نویسی است که به عنوان جانشین آپگرید شده نسخه
Modula-2 شناخته می‌شود. این زبان در حالی که در محافل تحقیقاتی تاثیر گذار بوده
است مانند حضور بین زبان‌های Python Java C\# و حتی Nim اما هیچ وقت در حوزه صنعتی
برای اهداف مختلف سازگار و مورد استفاده عموم قرار نگرفت.

ویژگی اصلی این زبان سادگی و ایمنی در حالی که تمام قدرت زبان برنامه‌نویسی-سیستمی
را حفظ کرده است. این زبان با هدف ادامه سنت زبان پاسکال که به تایپ سیفتی معروف
بود در حالی معرفی شد که سازه‌های جدیدی برای برنامه نویسی در دنیای واقعی را مطرح
کرد. در عمل این زمان از موارد زیر به صورت کامل پشتیبانی می‌کند:

\begin{enumerate}
    \item برنامه نویسی \lr{Generic} مشابه با Template. در حقیقت این پشتیبانی به
    برنامه نویس این قابلیت را ارائه می‌دهد که یک تایپ جنریک برای قسمت مورد نظر
    خود در نظر بگیرد. برای مثلا اگر به دنبال تعریف یک کلاس هستید می‌توانید
    داده‌های مربوط به کانستراکتور را از نوع جنریک تعریف کنید که کلاس شما بتواند
    در مرحله نمونه برداری به چند تایپ برای مورد خاص نقش ایفا کند.
    \item برنامه نویسی چند نخی یا \lr{Mutlithread Programming}
    \item قابلیت جلوگیری و پیشبینی خطا یا \lr{Exception error handler}
    \item قابلیت بسیار مهم \lr{Garbage collection}
    \item برنامه نویسی از نوع شئ گرا یا \lr{Object Oriented Programming}
\end{enumerate}

هدف نهایی طراحی این زبان پیاده‌سازی ویژگی‌های مهم برنامه نویسی مدرن دستوری یا
\lr{Imperative Programming}
در فرمی بسیار پایه و ابتدایی بوده است.  بدین ترتیب به قول معروف در این زبان از
پیاده‌سازی ویژگی‌های پیچیده و خطرناک از قبیل چند وراثتی و \lr{Operator
overloading} چشم پوشی شده است.

\section{برنامه نویسی دستوری یا \lr{Imperative Programming}}

در حقیقت در این نوع از برنامه نویسی، تمرکز بر تعریف آن که چگونه یک برنامه به
صورت مرحله به مرحله عمل می‌کند می‌باشد بجای آن که بیشتر روی توصیف انتظار روی
نتیجه آن در سطوح بالاتر وجود داشته باشد.

همانطور که در این شیوه از برنامه نویسی مشخص است، تمام افعال به صورت دستوری خواهد
بود. برای مثال زمانی که میخواهیم داده‌ای را از یک جدولی از پایگاه داده دریافت
کنیم به صورت کلی می‌گوییم که داده مربوط به دانشجوی با شناسه ۱۲ را دریافت کن. در
حقیقت در این شیوه ما نتیجه برنامه را بیان کردیم. اما در شیوه دستوری به اینگونه
عمل نمی‌کنیم. در حقیقت تمام مراحل را مانند الگوریتم مرحله به مرحله توضیح
می‌دهیم. ابتدا به پایگاه داده متصل شو، یک کوئری بنویس که حاوی شرط‌های مورد نظر
باشد، داده‌ها را دریافت کن. بعد از دریافت داده‌ها آنها را نرمال‌سازی کن. سپس
داده‌ها را به سمت جریان (ب) ارسال کن. اگر ارسال داده موفقیت آمیز بود ارتباط با
پایگاه داده را ببند. وضعیت را موفقیت آمیز اعلام کن.

\section{موضوع \lr{Operator overloading}}

یکی از موارد مشخص برنامه نویسی شئ گرا و استفاده از قانون چندریختی یا
\lr{Polymorphism} می‌باشد. این بخش به طور کامل در سندی داخل این مخزن توضیح داده‌
خواهد شد.

\end{document}