\documentclass[20pt, a4paper]{article}
\usepackage{hyperref}
\usepackage{graphicx}
\usepackage{ragged2e}
\usepackage{color}
\usepackage{xepersian}
\usepackage{subfiles}
\settextfont[Scale=1]{XB Roya}
\renewcommand{\baselinestretch}{1.5}

\begin{document}
\centerline{SPIN}
\centerline{درس سیستم عامل پیشرفته}
\centerline{علیرضا سلطانی نشان}
\centerline{\today}
\tableofcontents

\section{مقدمه}

سیستم عامل SPIN یک پروژه تحقیقاتی که با زبان برنامه نویسی Modula-3 توسعه و
پیاده‌سازی شده است، می‌باشد. این سیستم عامل یک پروژه متن باز است که برای سه هدف
اصلی طراحی و پیاده‌سازی شده است:

\begin{enumerate}
    \item انعطاف پذیری \footnote{Flexibility}
    \item ایمن \footnote{Safety}
    \item اجرای مناسب \footnote{Performance}
\end{enumerate}

این پروژه در دانشگاه واشنگتون \footnote{University of Washington} توسعه داده شد.

در این سیستم عامل، کرنل می‌تواند توسط \lr{Dynamic loading} ماژول‌ها که با
اینترفیس‌ها پیاده‌سازی شده اند دامنه اجرایی را مشخص می‌کند. این دامنه‌ها توسط
زبان برنامه نویسی مشخص شده‌اند. تمام اکستنشن‌های کرنل زیر مجموعه ایمن زبان
Modula-3 با ساختار‌های MetaLanguage و نوع تایپ سیف سیستم نوشته شده‌اند. همچنین
این سیستم اکسنشن‌ ران‌تایم کامپایلر را گزارش می‌کند.

یک مجموعه‌ای از اکستنشن‌های کرنل یک Api را ارائه می‌دهند که می‌تواند سیستم
دیجیتالی یونیکس را برای صدا زدن آن اینترفیس‌ها، شبیه سازی کند.

\section{منظور از \lr{Dynamic loading}}

موضوع \lr{Dynamic loading} یک مکانیزمی است که توسط آن یک برنامه کامپیوتری
می‌تواند در هنگام اجرا \footnote{Run time} یک کتابخانه را چه به صورت یک منبع
برنامه یا به صورت باینری، داخل حافظه رم خوانده، تمام آدرس‌های مربوط به توابع و
متغیر‌هایی که داخل کتابخانه هستند را دریافت کرده، این توابع را اجرا کرده یا به
متغیر‌ها دسترسی داشته و در نهایت پس از انجام کار‌هایش کتابخانه را از داخل حافظه
تخلیه می‌کند و اصطلاحا حافظه را از آن کتابخانه آزاد می‌کند.

\lr{Dynamic loading} به یک برنامه کامپیوتری اجازه اجرا در حالت نبود کتابخانه‌ها
و وابستگی‌های مربوط به آن برنامه، می‌دهد که می‌تواند کتابخانه‌های موجود را پیدا
کند، و به صورت بلقوه بتواند توابع و قابلیت‌های اضافی که در برنامه گنجانده نشده
است را بدست آورد و کاری کند که نرم‌افزار بدون مشکل اجرا شود.

\section{زبان برنامه نویسی Modula}

Modula-3 یک زبان برنامه نویسی است که به عنوان جانشین آپگرید شده نسخه دوم
Modula-2 شناخته می‌شود. این زبان در حالی که در محافل تحقیقاتی تاثیر گذار بوده
است مانند حضور بین زبان‌های Python Java C\# و حتی Nim اما هیچ وقت در حوزه صنعتی
برای اهداف مختلف سازگار و مورد استفاده عموم نبود.

ویژگی اصلی این زبان سادگی و ایمنی در حالی که تمام قدرت زبان برنامه‌نویسی-سیستمی
را حفظ کرده است. این زبان با هدف ادامه سنت زبان پاسکال که به تایپ سیفتی معروف
بود در حالی معرفی شد که سازه‌های جدیدی برای برنامه نویسی در دنیای واقعی را مطرح
کرد.

\end{document}