% !TEX program = xelatex
\documentclass[20pt, a4paper]{article}
\usepackage{hyperref}
% Justify paragraphs
\usepackage{graphicx}
\usepackage{ragged2e}
\usepackage{color}
\usepackage{xepersian}
\usepackage{subfiles}
\settextfont[Scale=1]{XB Roya}
\renewcommand{\baselinestretch}{1.5}

\begin{document}
\centerline{سیستم عامل \lr{Tiny OS}}
\centerline{علیرضا سلطانی نشان}
\centerline{\today}
\tableofcontents

\section{نگاه اولیه}

TinyOS یک سیستم عامل امبدد و کامپوننت محور است که برای دستگاه‌هایی کم مصرف و بی
سیم مورد استفاده قرار می‌گیرد. مانند سنسور‌های شبکه‌ای بی سیم، دستگاه‌های شخصی،
خانه‌های هوشمند و متر‌های هوشمند. این سیستم عامل با استفاده از زبان \lr{nesC}
برنامه نویسی شده است.

این سیستم عامل با همکاری دانشگاه کالیفرنیا-برکلی، محققان شرکت اینتل و شرکت
تکنولوژی Crossbow به عنوان یک پروژه نرم افزاری متن باز تحت مجوز BSD توسعه داده
شده است.

این سیستم عامل در فضا استفاده شده است که می‌توان پیاده‌سازی آن را در
\lr{ESTCube-1} مشاهده کرد.

\section{پیاده سازی}

برنامه‌های کاربردی این سیستم عامل به زبان nesC که یکی از زیر بخش‌های بهینه شده
زبان C است، توسعه داده شده‌اند که مورد مناسبی برای اعمال محدودیت‌های حافظه
سنسور‌ها و تجهیزات IoT می‌باشد. برای طراحی برنامه‌های فرانت-اندی از ابزار‌های
زبان جاوا و شل اسکریپت استفاده می‌کنند تا بتوانند در این سیستم عامل تعامل کاربر
را جذب کنند. TinyOS واسط‌های نرم‌افزاری برای عملیاتی مانند جا به جایی بسته‌های
شبکه‌ای، مسیریابی، دریافت اطلاعات از محیط به وسیله سنسور‌ها و اجرای فرمان‌ها روی
یک عمل کننده سخت‌افزاری به صورت کامل و مجزا ارائه داده است.

TinyOS کاملا به صورت \lr{non-blocking} است. این سیستم تنها یک \lr{run-time
stack} دارد. بدین ترتیب تمام عملیات ورودی/خروجی‌ها که بیشتر از ۱۰۰ میکروثانیه
زمان می‌برد، به صورت ناهمزمان بوده و دارای یک کال‌بک می‌باشد.

\section{نیم نگاهی به تاریخچه}

این سیستم عامل همانند FreeBSD از یک پروژه دانشگاه برکلی برای اهداف \lr{DARPA
NEST} شروع شد. توسعه جالب این سیستم عامل به گونه‌ای بود که هزاران دانشجو و
توسعه‌دهنده از سرتاسر دنیا به رشد آن می‌پرداختند.

برای اطلاعات بیشتر لیستی از انتشارات این سیستم عامل در زیر آمده است:

\begin{enumerate}
    \item اوت ۲۰۱۲: TinyOS نسخه ۲/۱/۲
    \item آپریل ۲۰۱۰: TinyOS نسخه ۲/۱/۱
    \item اوت ۲۰۰۸: TinyOS نسخه ۲/۱/۰
    \item ژولای ۲۰۰۷: TinyOS نسخه ۲/۰/۲ منشتر شد. که با تغییرات جزئی روی
    اینترفیس‌های برنامه همراه بود
    \item آپریل ۲۰۰۷: TinyOS نسخه ۲/۰/۱  منتشر شد که تغییرات آن از دانشگاه
    کمبریج حاصل شد
\end{enumerate}

\end{document}
