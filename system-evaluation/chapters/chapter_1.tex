\section{تعاریف اولیه}

\subsection{ابزار‌ها و تکنیک‌های ارزیابی سیستم}

\begin{enumerate}
    \item مدل‌سازی ریاضیاتی (\lr{Modeling})
    \item اندازه‌گیری، در دنیای واقعی (\lr{Measurement})
    \item شبیه‌سازی (\lr{Simulation})
\end{enumerate}

\subsection{متریک‌ها یا معیار‌ها}

 معیار‌هایی برای بررسی و ارزیابی عملکرد وجود دارد که بایستی در هنگام آنالیز و
 بررسی سیستم نسبت به آنها تنظیم شود. دقیقاً معیار‌ها و شرایطی هستند که می‌خواهیم
 سیستم را بر مبنای آن اندازه‌گیری کنیم.

\begin{enumerate}
    \item \lr{Response time}
    \item \lr{Throughput}
    \item \lr{Enegry consumption}
    \item \lr{Memory consumption}
    \item \lr{Clock usage}
    \item \lr{Capacity consumption}
\end{enumerate}

\subsection{مخاطبان ارزیابی}

مخاطبانی که مورد ارزیابی کارایی سیستم‌ها قرار می‌گیرند معمولاً در دسته‌بندی زیر
قرار می‌گیرند:

\begin{itemize}
    \item برنامه‌نویسان (\lr{Developers and Programmers})
    \item طراحان سیستم (\lr{Designers})
    \item مدیران سیستم (\lr{System Administrators})
    \item کاربران (\lr{Users})
\end{itemize}

\subsection{طراحی سیستم یا \lr{System design}}

تمامی افراد اکوسیستم یک نرم‌افزار، سخت‌افزار یا شبکه در مسئله ارزیابی کارایی
سیستم درگیر خواهند بود. هدف اصلی در ارزیابی سیستم ارائه بالاترین کارایی با
کمترین هزینه می‌باشد. مهندسی علمی برای ایجاد توازن بین کارایی و هزینه‌ها
می‌باشد. هزینه‌ها شامل زمان، نیروی انسانی و پول می‌باشند. هیچ وقت ادعای نظری
برتری کارایی یک چیز را ثابت نمی‌کند.

\subsection{فضای \lr{System Selection}}

فضای \lr{Selection} بحث در انواع سیستم‌ها را دارد. چیزی که کمک می‌کند تا بهترین
انتخاب را داشته باشیم با بررسی ارزیابی کارایی آن سیستم.

یکی از مهم‌ترین کاربرد‌های ارزیابی کارایی سیستم، آنالیز (تجزیه و تحلیل) می‌باشد.
یک سیستمی که تمام فرآیند توسعه را طی کرده باشد و نیاز به بررسی رخداد‌ها دارد که
مشخص شود چرا در این هنگام باعث کاهش کارایی آن شده است. اگر کد و آزمون‌های سیستمی
به خوبی در هنگام توسعه و آزمایش انجام شده باشد هزینه‌های پشتیبانی در آینده کاهش
می‌یابد. ارزیابی کارایی کاربرد بسیار زیادی در جنبه‌های انتخاب سیستم در لیستی از
انتخاب‌ها، طراحی سیستم‌ها و اپلیکیشن‌ها و آنالیز سیستم‌های کنونی دارد.

\subsection{تنظیم سیستم یا \lr{System tuning}}

وقتی که یک سیستم را با مفاهیم ارزیابی کارایی تنظیم می‌کنیم که بتوانیم بهترین
کارایی را برای سیستم خود طراحی واجرا کنیم.

\subsection{پیدا کردن مشکلات و گلوگاه‌ها \lr{Bottleneck identification}}

پیدا کردن مشکلات و گلوگاه‌هایی در سیستم که منجر به کاهش کارایی سیستم مورد نظر
شده است. این فاز کلاً به صورت شناسایی مشکلات سیستم کنونی معرفی می‌شود.

\subsection{حجم کاری یا \lr{Workload}}

تمام ویژگی‌هایی که بار کاری در سیستم ایجاد می‌کند. تا زمانی که به سیستم ورودی
داده نشود نمی‌توانیم بررسی کنیم که خروجی به چه شکلی خواهد بود و آیا می‌توانیم از
خروجی اندازه‌گیری کنیم؟ پس به ورودی داده‌ها \lr{Workload} می‌گوییم. این که قرار
است چه نوع داده‌ای به سیستم وارد شود، تعداد مخاطب سیستم چقدر است و تمامی
جنبه‌هایی که بر روی اندازه‌گیری کارایی سیستم تاثیرگذار است را در این بخش بررسی
می‌کنیم. باید بدانیم که سیستم مورد ارزیابی برای چه نوع ورودی‌هایی تنظیم شده و
تمام متریک‌های دیگر روی عملکرد آن تاثیرگذار است را شناسایی کنیم.

به عبارتی دیگر، درخواست‌هایی که توسط کاربران یا برنامه‌های دیگر جهت استفاده از
سیستم مورد نظر وارد می‌شود را \lr{Workload} می‌گوییم.

\subsection{\lr{Capacity planning}}

تعیین تعداد و اندازه کامپوننت‌ها، ماژول‌‌ها و بخش‌های استفاده شده در سیستم را
گوییم. ظرفیت‌های سیستم را بررسی می‌کنیم تا بدانیم برای آن چقدر منابع باید اختصاص
دهیم. در حقیقت از دسته فعالیت‌هایی برای آینده سیستم می‌باشد. برای مثال یک
نرم‌افزار قرار است پنج سال در دسترس کاربران باشد، باید در آن به صورت پویا مشخص
کنیم که کاربران چقدر افزایش خواهند داشت و داده‌های آن‌ها با منابعی که مستقر
کردیم سازگاری ظرفیتی دارد یا بایستی برنامه‌ریزی مجدد روی منابع جدید نسبت به
افزایش درخواست‌ها، داشته باشیم. بیشتر با مقیاس پذیری همراه است.

\subsection{پیشبینی یا \lr{Forecasting}}

پیشبینی کارایی سیستم در لود‌های کاری آینده. همانند برنامه‌ریزی برای ظرفیت‌ها
می‌باشد. بیشتر روی لود‌هایی که در آینده روی عملکرد سیستم تاثیرگذار است همراه
می‌باشد.

\subsection{الزامات انجام اندازه‌گیری‌های کارایی}

برای اندازه‌گیری کارایی یک سیستم بایستی دو بخش را داشته باشیم:

\begin{enumerate}
    \item ایجاد داده و بار کاری یا \lr{Load generator}
    \item پایش وضعیت سیستم با بار کاری‌ای که ایجاد شده یا \lr{Monitor}
\end{enumerate}

برای مثال موارد ارزیابی لیست شده زیر را در نظر بگیرید و بگویید که کدام ابزار
می‌تواند برای سنجش کارایی مناسب باشد:

\begin{itemize}
    \item میزان مصرف یک \lr{LAN} که معمولاً با استفاده از نرم‌افزار شبیه‌سازی
    توان مصرفی \lr{iPerf} قابل سنجش می‌باشد.
    \item مدت زمانی که طول می‌کشد یک پاسخ از \lr{Web server} در سمت سرویس‌گیرنده
    دریافت شود.
    \begin{itemize}
        \item \lr{Python Locust}: جهت شبیه‌سازی ارسال درخواست‌های تعداد زیاد در
        برنامه‌های \lr{Backend} و \lr{Api}ها
        \item \lr{Apache benchmark}: تنها برای اندازه‌گیری میزان تحمل بار
        درخواست در سرویس‌های راه‌اندازی شده \lr{Apache} مورد استفاده قرار
        می‌گیرد.
        \item \lr{JMeter}
    \end{itemize}
    \item بررسی کیفیت صدا در شبکه‌های \lr{VoIP}
\end{itemize}

\subsection{استفاده از تکنیک‌های آماری برای مقایسه جایگزین‌ها}

وقتی در یک \lr{System selection} چندین سیستم را داریم بایستی مقایسه‌ای از نظر
عملکردی بر روی آن‌ها انجام شود که بدانیم کدام سیستم به نیاز ما نزدیک‌تر می‌باشد.
برای اینکار در نظر گرفتن چند نکته الزامی می‌باشد:

\begin{enumerate}
    \item اصولاً فرایند ارزیابی با یک ورودی یا \lr{Workload} انجام نمی‌شود بلکه
    در حالت‌های مختلف با ورودی‌های مختلف بایستی سیستم را مورد ارزیابی قرار داد.
    با اینکار ممکن است به رخداد‌های غیرقطعی برسیم که می‌تواند روی عملکرد سیستم
    تاثیرگذار باشد.
    \item هر کدام از متریک‌ها می‌تواند از فرمول ریاضیاتی خاص خودش استفاده کند و
    نتیجه را ارائه دهد. می‌تواند یکی از متریک‌ها با میانگین‌گیری باشد و دیگری با
    بدست آوردن واریانس.
\end{enumerate}

یکی از قسمت‌های مهم ارزیابی استفاده از حالت‌های مختلف ورودی‌ها جهت مقایسه کارایی
سیستم می‌باشد. فرآیند ارزیابی بایستی بار‌ها امتحان شود تا بتوان به نتیجه‌گیری
مناسبی رسید. از یک یا چند روش آماری جهت جمع‌بندی کار‌هایمان در مقالات و
آزمایش‌ها بایستی استفاده کنیم تا دلیل برتری کارایی سیستم (الف) از سیستم (ب) مشخص
شود. جدول شماره \ref{fig:packetLostLinks} تعداد بسته‌های گم شده را در لینک‌های
(\lr{A}) و (\lr{B}) را نشان می‌دهد.

\begin{LTR}
    \begin{table}[H]
        \centering
        \label{fig:packetLostLinks}
        \caption{مقایسه گم شدن بسته‌ها در لینک‌های \lr{A} و \lr{B}}
        \begin{tabular}{ccc}
            \textbf{File size} & \textbf{Link A} & \textbf{Link B} \\ \hline
            1000 & 5 & 10 \\
            1200 & 7 & 3 \\
            1300 & 3 & 0 \\
            5 & 0 & 1 \\
        \end{tabular}
    \end{table}
\end{LTR}

سوال مهمی که این میان ممکن است مطرح شود آن است، آزمایش‌های اندازه‌گیری و
شبیه‌سازی چه تعداد بار بایستی انجام شود که قادر به پوشش فضای مورد نظر باشد؟

غالباً فاکتور‌های بسیار زیادی روی کارایی تاثیر می‌گذارند، بایستی اثراتی که به
صورت انفرادی مهم هستند را جدا کنیم. تعدادی از پارامتر‌ها به صورت ثابت یا
\lr{Static} هستند. برای مثال در سناریو‌های خود نوع سیستم عامل و مرورگری که کاربر
استفاده می‌کند را به صورت ثابت به ترتیب ویندوز و کروم در نظر می‌گیریم. دلیل اصلی
این کار آن است که تمام پارامتر‌هایی که سیستم می‌پذیرد را نمی‌توان کامل ارزیابی
کنیم چرا که از نظر هزینه‌ای مدیریت دشوار و پیچیده‌ای دارد. راهکار اصلی آن است که
بر اساس تجربه یکسری از پارامتر‌ها را حذف می‌کنیم.

قابل توجه است که همه پارامتر‌ها در ابتدا برای ارزیابی کارایی سیستم بایستی نوشته
شوند و به صورت هوشمندانه تعدادی از این پارامتر‌ها را ثابت می‌کنیم.

\begin{itemize}
    \item پارامتر‌هایی که به صورت ثابت نگهداری می‌شوند را \lr{Static parameter}
    می‌گوییم.
    \item پارامتر‌هایی دستخوش تغییرات قرار می‌گیرند را \lr{Dynamic parameter} یا
    \lr{Factor parameter} می‌گوییم.
\end{itemize}

\subsubsection*{نکات}

\begin{itemize}
    \item عملیات ارزیابی کارایی بسیار دشوار و زمان‌بر می‌باشد.
    \item سطح تعدادی از پارامتر‌ها را می‌توان زیاد کرد.
    \item فرایند ارزیابی کارایی به صورت \lr{iterative} و تکرارشونده هستند.
    \item همیشه باید کل سیستم را بررسی و آنالیز کنیم که پارامتر‌های ثابت و
    \lr{factor} مشخص شود.
    \item عملیات ارزیابی آنقدر تکرار می‌شود که بدانیم سیستم روی چه پارامتر‌هایی
    باعث می‌شود که ارزیابی را با دقت بررسی کنیم.
    \item انجام آزمایش‌ها براساس نتایج و سناریو‌ها را \lr{Experiment} می‌گوییم.
\end{itemize}

\subsection{شبیه‌ساز و نکات تنظیمات}

یکی از نکات مهم در ارزیابی کارایی سیستم انجام شبیه‌سازی درست و مناسب می‌باشد:

\begin{itemize}
    \item انتخاب زبان درست، انتخاب \lr{Seed} برای اعداد تصادفی، مدت زمان اجرای
    شبیه‌سازی و آنالیز‌ها
    \item قبل از موارد بالا بایستی اعتبار شبیه‌ساز تایید شود و شبیه‌ساز کاملاً
    با معیار‌های ارزیابی ما بایستی \lr{Tune} شود.
\end{itemize}

برای مثال، جهت مقایسه کارایی دو الگوریتم کش جایگزینی خواهیم داشت:

\begin{itemize}
    \item به چه اندازه‌ای شبیه‌سازی باید در حال اجرا باشد؟
    \item چه کاری باید انجام دهیم که با زمان اجرای کوتاه‌تر به دقت اجرای بلند
    مدت برسیم؟
    \item استفاده از مدل‌های صف‌بندی برای ارزیابی کارایی
    \item غالباً می‌توان سیستم‌های کامپیوتری را با نرخ سرویس‌ها و بارکاری‌ای که
    به سیستم می‌رسد به دو روش مدل کرد:
    \begin{itemize}
        \item استفاده از چند سرویس‌دهنده یا \lr{Multiple servers}
        \item استفاده از چند صف‌بندی یا \lr{Multiple queues}
        \item برای مثال می‌گوییم که کدام روش می‌تواند باعث ارزیابی بهتر کارایی
        وب سرویس می‌شود، استفاده از دو تک‌پردازنده یا استفاده از چهار وب سرویس
        تک پردازنده؟
    \end{itemize}
\end{itemize}

\subsection{هنر ارزیابی کارایی}

ارزیابی را نمی‌توان به صورت مکانیکی تولید کرد. نیازمندی اصلی ارزیابی، دانش دقیق
نسبت به سیستم و دقت بالا در انتخاب متدولوژی، \lr{Workload}ها و ابزار‌ها را دارا
می‌باشد. با تکنیک‌ها و ابزار‌هایی که در اختیار داریم باید بررسی کنیم که با
استفاده از آن‌ها به چه شکلی می‌توانیم به ارزیابی کارایی سیستم برسیم. پروژه‌ها
اکثراً شامل مستندات \lr{Test coverage} می‌باشد که در آن تکلیف دو طرف یعنی هم
کارفرما هم پیمانکار را مشخص می‌کند.

\subsection{ارزیابی اصولی}

نکات مهمی در مورد ارزیابی اصولی به صورت \lr{Best practice}ها مطرح می‌شود:

\begin{itemize}
    \item یک هدف واحد را در نظر بگیریم.
    \begin{itemize}
        \item در فرایند ارزیابی کارایی تمام اهداف در ابتدا طراحی و مشخص شده‌اند
        و اصلاُ اهداف به صورت کلی و جامع نیستند.
    \end{itemize}
    \item هدف مغرضانه یا \lr{Biased goals} نداشته باشیم.
    \begin{itemize}
        \item هیچ وقت از سیستمی طرفداری و جانبداری بی‌دلیل نکنیم و براساس
        داده‌سازی‌هایی که کرده‌ایم نشان ندهیم که سیستم تولید شده توسط ما بهتر از
        سیستم مقابل کارایی دارد. باید منصفانه تمام ارزیابی‌ها انجام شود.
    \end{itemize}
    \item تمام \lr{Workload}هایی که در سیستم مورد نظر استفاده می‌شود بایستی جامع
    و کامل باشد نه اینکه به صورت خاص تعیین شوند که برای سیستم طراحی شده توسط ما
    کارایی خوبی داشته باشد و برای سیستم مورد مقایسه کارایی بد. تنها با داده‌های
    خیلی زیاد و یا خیلی کم ارزیابی را انجام ندهیم بلکه تمام \lr{Workload}ها
    بایستی از سطح مناسبی از تنوع (\lr{Diversity}) برخوردار باشند.
    \item تکنیک‌های اشتباه ارزیابی: از مدل، شبیه‌سازی و ابزار اندازه‌گیری مناسبی
    استفاده کنیم.
    \item از یک سطح مناسب و متعادلی از جزئیات استفاده کنیم.
    \begin{itemize}
        \item نمی‌توانیم تعداد زیادی از موارد را مانند مثال سیستم عامل‌ها و
        مرورگر‌ها را برای سیستم خود مورد ارزیابی قرار دهیم.
        \item نمی‌توانیم تعداد کمی از مورد‌ها را انتخاب کنیم.
    \end{itemize}
    \item نداشتن حساسیت روی آنالیز‌ها
    \begin{itemize}
        \item آن ارزیابی‌ که بر روی سیستم انجام می‌شود تنها یک مشاهده می‌باشد نه
        یک دامنه یا \lr{fact} طبیعی که بتوانیم با ادعا‌هایی سیستم را زیر سوال
        ببریم و اثبات کارمان را انجام دهیم.
    \end{itemize}
    \item نمایش درستی از آن‌چه که بدست آورده‌ایم را فراهم کنیم.
    \begin{itemize}
        \item خواندن صرف برای همه کار پیچیده‌ای است، سعی کنیم با نمایش نمودار‌ها
        و گراف‌ها نتایج را ملموس‌تر سازیم.
        \item در استفاده از شکل‌ها نهایت احتیاط را داشته باشیم. اگر نتایج به
        صورت نامناسب مصورسازی شده باشد باعث می‌شود ارزیابی سیستم زیر سوال رود.
    \end{itemize}
    \item استفاده مناسب از محدودیت‌ها و مفروضات
    \begin{itemize}
        \item احتمال دارد در قدم اول فرض انبوه ترافیک شبکه را در \lr{TCP} داشته
        باشیم با اینکه بیشتر ترافیک از سمت \lr{UDP} ناشی می‌شود.
        \item لزومی ندارد که مفروضات در سیستم مورد نظر همیشه درست باشند.
    \end{itemize}
\end{itemize}