جلسه سوم:

زوایای مختلف اهداف در مطالب قبلی

در هدف اول برای اینکه بخوایم یک فرایند pe رو انجام بدیم باید تکنیکمان را به صورت درست انتخاب کنیم.

تکینک‌های اندازه‌گیری که در اولین section بوده.

متریک‌ها چیزایی هتسن که میخوایم اونو اندازه گیری کنیم و اونو بهتر کنیم.

workload -> input

دو تا از سیستم‌ها رو بخوایم مقایسه بکنیم چه متریک‌هایی و ورک لود‌هایی رو باید استفاده کنیم.

دو تا trx رو با هم مقایسه کنیم. 

بررسی دو تا از الگوریت‌ها retransmission در شبکه


h'v fo,hdl hc hknhciB'dvd hsjthni ;kdl fhdn nrj nhaji fhadl ;i nv sdsjlBihd
hknhci 'dvd fi n, l,gtiBhd ;i kdhc nhvdl

تولید دیتا یا لود جنریتور و مانیتور

یک ماژولی باید باشه که بتونه دیتا جنریت کنه لود تولید کنه و یک ماژول داشته باشیم
که بتونه وضعیت سیستم رو پایش و بررسی کنه که بتونیم در نهایت پارامتر‌ها رو
اندازه‌گیری کنیم.

+ iperh for Utilization on a LAN
+ Response time from a Web server
+ Audio quality in a VoIP network

objective 3 of 6

اصولاً فرایند تست با یک ورودی امتحان نمی‌شود بلکه در حالت‌های مختلف باید به
سیستم ورودی داشته بشه. 

هر کدوم از متریک‌ها میتوان فرمول ریاضی متفاوتی رو ارائه کرد.

یکی از قسمت‌های مهم ارزیابی استفاده از حالت‌های مختلف جهت مقایسه کارایی سیستم
هستش.

از یک یا چند روش آماری جهت جمع‌بندی کار‌هامون توی مقالات استفاده میذکنیم.

فرایند ارزیابی باید بار‌ها امتحان شود تا بتوان به نتیجه‌گیری مناسب رسید.

objective 4 of 6

آزمایش‌های اندازه‌گیری و شبیه‌سازی چه تعداد باید باشه که بتونه اون فضای مورد نظر
رو بپوشونه.

یکسری از پارامتر‌ها به صورت ثابت هستن

مثلا فقط فکر میکنیم که کاربر از windows استفاده می‌کنه.

یا مثلاً فقط با Chrome وارد سیستم میخواد بشه.

همه پارامتر‌هایی که سیستم می‌دذیره رو نمیشه کامل تست کنیم از نظر هزینه‌ای مدیریت
دشواری دارد.

راه‌کار چیه:

وقتی می‌بینی سیستم دچار مشکل شده بر اساس تجربه یکسری از پارمتر‌ها رو حذفش کنیم.

همه پارامتر‌ها در ابتدا برای PE باید نوششته بشه و باید بهصورت هوشمندانه و فعال
یکسری از این پارمتر‌ها فیکس کنیم.

برخی پارامتر‌ها رو دچار تغییر می‌کنیم رو می‌گیم factor.

اونایی که داینامیک هستن رو میگیم factor

لول یکسری از پارامتر‌ها رو میشه زیاد کرد 

خود فرایند‌ ارزیابی iterative و تکرار شونده هستش.

همیشه باید کل سیستم رو انلایز کنیم که بعضیا استاتیک بشه بعضیا داینامیک بشه.

انقدر تکرار میکنیم که بفهمیم سیستم روی چه چیزایی باعث میشه که ارزایبی رو بررسی کنیم.

زمانبر و خیلی دشواره فرایند pe

experiment انجام آزمایش‌ها براساس ریزالت‌هایی که بدست میاد

objective 5 of 6

خود سیمولاتور رو باید خوب  تنظیم کنیم. 

اون شبیه ساز باید خوب تیو بشه.

ببین این اسلایدو که سوالاتش رو بنویسی.

بخش objective 6 of 6 رو بخون.

the art of performance evaluation

با تکینک‌هایی که در اختیار داریم اون ابزار‌هایی که داریم باید ببینیم که با اینا
چه شکلی میتونیم به ارزیابی کارایی برسیم.

پروژه‌ها شامل مستندات test coverage هستن تا تکلیف هر دو طرف رو مشخص بکنه که این
سیستم مورد نظر چطوری ار میکنه.


Commn mistakes

سه حالت داره که خرابی‌هایی پیش بیاد:

اهداف تعریف نشده نداریم. یعنی به صورت کلی نیستش.

یکسری اهدافی رو نسبت بهش بایاس بشیم.

مثلا بریم توی یه جلسه‌ای که سه مدل سرور میخواد انتخاب کنه که مدیر عامل بای
دیفالت نظرش روی سرویس A باشه.

performance analysis is like a jury

ورکلود جامع باشه. نه اینکه به صورت خاص باشه که عملکرد خوبی داشته باشه یا مثلا
عملکرد بدی داشته باشه.

evaluation technique کلا مسیر رو میتونه نا هموار کنه

اون لولی که توی کار وجود داره.

مثلا توی یه پروژه ای باشیم که مرورگر رو ۵ نوع در نظر بگیریم.

آنالیز و senstive بودن بررسی‌ها

اون چیزی که از سیستم میگییرم یه مشاهدست نه یک فکت. که یه وقت سیستم رو باهاش زیر سوال ببری.

سیستم شما نسبت به ورودی‌هامون فرایند تستش چطوریه

بحث نمایش نتایج هستش.

ریپورته

پایان‌نامه

یه گزارشه

نمایش نتایج ارزیابی کارایی.

عدم استفاده از شکل خوب

اگه نتونیم خوب گراف بیاریم بیورن ممکنه کل کارایی سیستم زیر سوال بره.

it is no the numer of graphs, but the number of graphs that help make decisions

may assume most traffic TCP, whereas some links may have significant UDP traffic

May lead to applying results where assumptions do not hold.

A systematic approach

بیان اهداف و تعریف مرز‌ها

انتخاب متریک‌های p 

لیست کردن پارامتر‌ها و ورکلود‌های سیستمی

انتخاب فاکتور‌ها و مقادیر

انتخاب تکنیک ارزیابی

انتخاب ورکلود

طراحی سناریو تست

آنالیز و شبیه‌سازی و ران کردن داده‌ها

تقسیر نتایج و نمایش اون‌ها

در صورت نیاز میشه این مراحله رو تکرار کرد.


متریک‌ها از سه جنس سرعت دقت و و در دستسرس بودن هستش.

اکسپریمنت رو طراحی میکنیم

فاز یک

many factors, few levels

see which factors matter

فاز دوم

few factors, more levels

see where the range of impact for the factor is

حتما مثال rpc رو بخون و کامل بنویس.

experiments میشه د رمورد سمینار

