\documentclass[a4paper]{article}
\usepackage{forest}
\usepackage{float}
\usepackage{geometry}
\usepackage{listings}
\usepackage{hyperref}
\usepackage{graphicx}
\usepackage{ragged2e}
\usepackage{color}
\usepackage{xepersian}
\usepackage{subfiles}
\newgeometry{left=1.4cm, right=1.4cm, bottom=2.0cm, top=2.0cm}
\settextfont[Scale=1]{XB Roya}
\renewcommand{\baselinestretch}{1.5}
\definecolor{dkgreen}{rgb}{0,0.6,0}
\definecolor{gray}{rgb}{0.5,0.5,0.5}
\definecolor{mauve}{rgb}{0.58,0,0.82}
\definecolor{commentColor}{rgb}{0.6,0.6,0.60}

\title{ارزیابی کارایی سیستم \\ آقای دکتر مهدی امینیان}
\author{علیرضا سلطانی نشان}

\begin{document}
\maketitle
\tableofcontents

\section*{مجوز}

به فایل license همراه این برگه توجه کنید. این برگه تحت مجوز GPLv3 منتشر شده است
که اجازه نشر و استفاده (کد و خروجی/pdf) را رایگان می‌دهد.


مطالب قبلی به صورت کامل نوشته شود.

Capacity planning:

ظرفیت برنامه‌ریزی 

تخمین تعداد و اندازه کامپوننت‌ها و ماژول‌ها

یک کاری که باید انجام بدیم تو این حوزه اینه که بتونیم برای اون سیستمی که میخوایم
مستقرش کنیم یک فعالیتی انجام بدیم که مربوط به pe هستش 

چقدر برنامه ریزی دارین

چقدر ظرفیت دارین چقدر ارشد چقدر دکتری چقدر کارشناسی

که بتوینم توش مشخص کنیم که چقدر باید ظرفیت بهش اختصاص بدیم.

فعالیت‌هایی برای آینده هستش. اون دو مورد آخر

برای این نرم‌افزار که قراره ۵ سال دستتش باشه باید مشخص کنیم که توی این ۵ سال
چقدر ظرفیت رو درگیر خواهد کرد.

Forecasting

پیشبینی کارایی سیستم در لود‌های آینده

مضاف بر اینکه بحث فورکستینگ یک بحث مهمه که تو شرکت‌های future trend ها رو به
شرکت‌های دیگه می‌فرسشن.

تمام اقدامات به صورت یکهویی انجام نمی‌شود. که سریع پروداکت میدن سریع مشتری
میگیرن مثل شرکت‌های تاپ ۱۰ اینطوری نیست که ترند‌های سال‌های آینده رو بنویسن.

نمودار‌های گارتنر دیده شود. TODO:

EETimes: News of Hardware

Edge Impulse Powering AI Solutions with Industrial Computers

\end{document}