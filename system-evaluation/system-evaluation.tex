\documentclass[a4paper]{article}
\usepackage{forest}
\usepackage{float}
\usepackage{geometry}
\usepackage{listings}
\usepackage{hyperref}
\usepackage{graphicx}
\usepackage{ragged2e}
\usepackage{color}
\usepackage{xepersian}
\usepackage{subfiles}
\newgeometry{left=1.4cm, right=1.4cm, bottom=2.0cm, top=2.0cm}
\settextfont[Scale=1]{XB Roya}
\renewcommand{\baselinestretch}{1.5}
\definecolor{dkgreen}{rgb}{0,0.6,0}
\definecolor{gray}{rgb}{0.5,0.5,0.5}
\definecolor{mauve}{rgb}{0.58,0,0.82}
\definecolor{commentColor}{rgb}{0.6,0.6,0.60}

\title{ارزیابی کارایی سیستم \\ آقای دکتر مهدی امینیان}
\author{علیرضا سلطانی نشان}

\begin{document}
\maketitle
\tableofcontents

\section*{مجوز}

به فایل license همراه این برگه توجه کنید. این برگه تحت مجوز GPLv3 منتشر شده است
که اجازه نشر و استفاده (کد و خروجی/pdf) را رایگان می‌دهد.

\section{ابزار‌های ارزیابی سیستم}

\begin{enumerate}
    \item مدل‌سازی ریاضیاتی (\lr{Modeling})
    \item اندازه‌گیری، در دنیای واقعی (\lr{Measurement})
    \item شبیه‌سازی (\lr{Simulation})
\end{enumerate}

\section{مخاطبان ارزیابی}

مخاطبانی که مورد ارزیابی کارایی سیستم‌ها قرار می‌گیرند معمولاً در دسته‌بندی زیر
قرار می‌گیرند:

\begin{itemize}
    \item برنامه‌نویسان (\lr{Developers and Programmers})
    \item طراحان سیستم (\lr{Designers})
    \item مدیران سیستم (\lr{System Administrators})
    \item کاربران (\lr{Users})
\end{itemize}

\section{تعاریف اولیه}

\subsection{طراحی سیستم یا \lr{System design}}

تمامی افراد اکوسیستم یک نرم‌افزار، سخت‌افزار یا شبکه در مسئله ارزیابی کارایی
سیستم درگیر خواهند بود. هدف اصلی در ارزیابی سیستم ارائه بالاترین کارایی با
کمترین هزینه می‌باشد. مهندسی علمی برای ایجاد توازن بین کارایی و هزینه‌ها
می‌باشد. هزینه‌ها شامل زمان، نیروی انسانی و پول می‌باشند. هیچ وقت ادعای نظری
برتری کارایی یک چیز را ثابت نمی‌کند.

\subsection{فضای \lr{System Selection}}

فضای \lr{Selection} بحث در انواع سیستم‌ها را دارد. چیزی که کمک می‌کند تا بهترین
انتخاب را داشته باشیم با بررسی ارزیابی کارایی آن سیستم.

یکی از مهم‌ترین کاربرد‌های ارزیابی کارایی سیستم، آنالیز (تجزیه و تحلیل) می‌باشد.
یک سیستمی که تمام فرآیند توسعه را طی کرده باشد و نیاز به بررسی رخداد‌ها دارد که
مشخص شود چرا در این هنگام باعث کاهش کارایی آن شده است. اگر کد و آزمون‌های سیستمی
به خوبی در هنگام توسعه و آزمایش انجام شده باشد هزینه‌های پشتیبانی در آینده کاهش
می‌یابد. ارزیابی کارایی کاربرد بسیار زیادی در جنبه‌های انتخاب سیستم در لیستی از
انتخاب‌ها، طراحی سیستم‌ها و اپلیکیشن‌ها و آنالیز سیستم‌های کنونی دارد.

\subsection{تنظیم سیستم یا \lr{System tuning}}

وقتی که یک سیستم را با مفاهیم ارزیابی کارایی تنظیم می‌کنیم که بتوانیم بهترین
کارایی را برای سیستم خود طراحی واجرا کنیم.

\subsection{پیدا کردن مشکلات و گلوگاه‌ها \lr{Bottleneck identification}}

پیدا کردن مشکلات و گلوگاه‌هایی در سیستم که منجر به کاهش کارایی سیستم مورد نظر
شده است. این فاز کلاً به صورت شناسایی مشکلات سیستم کنونی معرفی می‌شود.

\subsection{حجم کاری یا \lr{Workload}}

تمام ویژگی‌هایی که بار کاری در سیستم ایجاد می‌کند. تا زمانی که به سیستم ورودی
داده نشود نمی‌توانیم بررسی کنیم که خروجی به چه شکلی خواهد بود و آیا می‌توانیم از
خروجی اندازه‌گیری کنیم؟ پس به ورودی داده‌ها \lr{Workload} می‌گوییم. این که قرار
است چه نوع داده‌ای به سیستم وارد شود، تعداد مخاطب سیستم چقدر است و تمامی
جنبه‌هایی که بر روی اندازه‌گیری کارایی سیستم تاثیرگذار است را در این بخش بررسی
می‌کنیم. باید بدانیم که سیستم مورد ارزیابی برای چه نوع ورودی‌هایی تنظیم شده و
تمام متریک‌های دیگر روی عملکرد آن تاثیرگذار است را شناسایی کنیم.

\subsection{\lr{Capacity planning}}

تعیین تعداد و اندازه کامپوننت‌ها، ماژول‌‌ها و بخش‌های استفاده شده در سیستم را
گوییم. ظرفیت‌های سیستم را بررسی می‌کنیم تا بدانیم برای آن چقدر منابع باید اختصاص
دهیم. در حقیقت از دسته فعالیت‌هایی برای آینده سیستم می‌باشد. برای مثال یک
نرم‌افزار قرار است پنج سال در دسترس کاربران باشد، باید در آن به صورت پویا مشخص
کنیم که کاربران چقدر افزایش خواهند داشت و داده‌های آن‌ها با منابعی که مستقر
کردیم سازگاری ظرفیتی دارد یا بایستی برنامه‌ریزی مجدد روی منابع جدید نسبت به
افزایش درخواست‌ها، داشته باشیم. بیشتر با مقیاس پذیری همراه است.

\subsection{پیشبینی یا \lr{Forecasting}}

پیشبینی کارایی سیستم در لود‌های کاری آینده. همانند برنامه‌ریزی برای ظرفیت‌ها
می‌باشد. بیشتر روی لود‌هایی که در آینده روی عملکرد سیستم تاثیرگذار است همراه
می‌باشد.
\end{document}