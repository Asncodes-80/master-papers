% !TEX program = xelatex
\documentclass[20pt, a4paper]{article}
\usepackage{hyperref}
% Justify paragraphs
\usepackage{graphicx}
\usepackage{ragged2e}
\usepackage{color}
\usepackage{xepersian}
\usepackage{subfiles}
\settextfont[Scale=1]{XB Roya}
\renewcommand{\baselinestretch}{1.5}

\begin{document}
\centerline{الگوریتم‌های موازی}
\centerline {خانم دکتر دامرودی}
\centerline{علیرضا سلطانی نشان}
\centerline{\today}

\tableofcontents

\section{مفاهیم اولیه}

نگارش این بخش تکمیل نشده است.

\section{وابستگی‌ها}

هیچ برنامه‌ای نمی‌تواند سریع‌تر از برنامه‌ای که مدت زمان بیشتری، زمان رسیدن به
نتیجه را سپری می‌‌کند، اجرا شود. زیرا محاسباتی که نسبت به محاسبات قبلی به
یکدیگر، در زنجیره‌ای از فرایند‌ها وابستگی دارند، باید به ترتیب اجرا شوند چرا که
به صورت سریالی می‌باشند.  همه الگوریتم‌ها شامل وابستگی‌هایی هستند که می‌توان
آنها را تشخیص و به صورت مناسب آنها را حذف کرد تا بتوان محاسبات مستقل به صورت
موازی را حل کرد.

\subsection{انواع وابستگی‌های داده}

وابستگی‌های داده به سه دسته زیر تقسیم می‌شوند:

\begin{enumerate}
    \item \lr{Data Flow Dep \footnote[1]{Dependency}} ماهیت برنامه
    \item \lr{Anti Data Dep} برنانویس
    \item \lr{Output Data Dep} برنامه نویس
\end{enumerate}

\subsection{منبع} 

اگر دو دستورالعمل بخواهند به صورت همزمان کار کنند، به گونه‌ای که روی یک منبع
عملیات خواندن و نوشتن را انجام دهند، دو حالت به وجود می‌آید:

\begin{enumerate}
    \item یا در ماهیت خود برنامه وجود دارند
    \item یا برنامه نویس ایجاد می‌کند که قابلیت حذف دارد
\end{enumerate}

نکته: یکی از وظایف اصلی کامپایلر حذف وابستگی‌هایی است که برنامه نویس ایجاد
می‌کند و قبل از مرحله اجرا انجام می‌شود و باعث کارایی بالا برنامه می‌شود.


% Should fix by a simple diagram
وابستگی‌های زیر را در نظر بگیرید:

S[i]: O ... = .... O

S[j]: O ... = .... O

\subsection{وابستگی \lr{WBR} یا \lr{Write Before Read}}

این وابستگی در ابتدا روی متغیر مورد نظر داده‌ای را می‌نویسد و سپس آن را در
فرایندی دیگر می‌خواند. به این نوع از وابستگی، وابستگی \lr{Date Flow} گفته
می‌شود.

برای مثال:

\begin{enumerate}
    \item s[i]: A = B + C , s[j] K = A + C
    \item s1: A = B + C , s2: E = A - D
\end{enumerate}

\sbusection{وابستگی \lr{RBW} یا \lr{Read Before Write}}

در این از وابستگی در ابتدا عملیات نوشتن روی متغیر صورت می‌گیرد، سپس از آن متغیر
برای خواندن مقدار استفاده می‌کند. به این نوع از وابستگی \lr{Ani Data}

برای مثال:

\begin{enumerate}
    \item s[i]: B = C + A , s[j]: A = K + J
    \item s[i]: B = A * D , s[j]: A = F - 5
\end{enumerate}

\subsection{وابستگی \lr{WBW} یا \lr{Write Before Write}}

این نوع از وابستگی به شکل خطی می‌باشد. یعنی در مرحله اول فرایند ابتدا بر روی آن
مقداری می‌نویسد، سپس مجددا این عمل را تکرار می‌کند. به این نوع وابستگی
\lr{Output Data} گفته می‌شود.

برای مثال:

\begin{enumerate}
    \item s[i]: A = B + C , s[j]: A = K + J
    \item s[1]: A = B + C , s[2]: A = D + E
\end{enumerate}

نکته: برای نمایش وابستگی‌ها از گراف استفاده می‌کنند.

نکته: اگر تمام مسیر‌های گراف به یک جهت باشد (هم جهت باشد) و حلقه وجود نداشته
\footnote{Performancee} باشد، گراف مورد نظر به شکل آبشاری خواهد بود. که در این
حالت می‌توان با اجرای مناسبی پردازشی را انجام داد.

% شکل مربوط به نوع هر وابستگی یعنی یال‌ها کشیده شود

نکته: نتیجه وابستگی‌ها به شکل جدول وابستگی‌ها یا \lr{Dependencies table} نشان
داده می‌شود.

% شش حالت استیت‌ها رو بکش و با دپس مناسب به هم وصل کن

\footnote{\lr{Scaler 3 of (5, 2) Vector => (15, 6)} یادآوری:}

% مسئله و گراف وابستگی اول نوشته شود.

در وابسته اول، مهم‌ترین نکته در آن است که بایستی ارتباطات حداقل دو اندیس را در
نظر داشته باشیم تا بتوانیم وابستگی مورد نظر را پیدا کنیم. برای مثال در این
وابستگی در صورتی که تنها روی یک اندیس به دنبال وابستگی بگردیم نمی‌توانیم حلقه را
پیدا کنیم. در صروتی که در ذات این برنامه حلقه یا وابستگی نوع \lr{Output date
deps} وجود دارد.

% مسئله و وابستگی دوم نوشته شود.

% تا ابتدای صفحه ۲۶ خوانده شد

\end{document}