% !TEX program = xelatex
\documentclass[20pt, a4paper]{article}
\usepackage{hyperref}
% Justify paragraphs
\usepackage{graphicx}
\usepackage{ragged2e}
\usepackage{color}
\usepackage{xepersian}
\usepackage{subfiles}
\settextfont[Scale=1]{XB Roya}
\renewcommand{\baselinestretch}{1.5}

\begin{document}
\centerline{الگوریتم‌های موازی}
\centerline {خانم دکتر دامرودی}
\centerline{علیرضا سلطانی نشان}
\centerline{\today}

\tableofcontents

\section{مفاهیم اولیه}

نگارش این بخش تکمیل نشده است.

\section{وابستگی‌ها}

هیچ برنامه‌ای نمی‌تواند سریع‌تر از برنامه‌ای که مدت زمان بیشتری، زمان رسیدن به
نتیجه را سپری می‌‌کند، اجرا شود. زیرا محاسباتی که نسبت به محاسبات قبلی به
یکدیگر، در زنجیره‌ای از فرایند‌ها وابستگی دارند، باید به ترتیب اجرا شوند چرا که
به صورت سریالی می‌باشند.  همه الگوریتم‌ها شامل وابستگی‌هایی هستند که می‌توان
آنها را تشخیص و به صورت مناسب آنها را حذف کرد تا بتوان محاسبات مستقل به صورت
موازی را حل کرد.

\subsection{انواع وابستگی‌های داده}

وابستگی‌های داده به سه دسته زیر تقسیم می‌شوند:

\begin{enumerate}
    \item \lr{Data Flow Dep \footnote[1]{Dependency}} ماهیت برنامه
    \item \lr{Anti Data Dep} برنانویس
    \item \lr{Output Data Dep} برنامه نویس
\end{enumerate}

\subsection{منبع} 

اگر دو دستورالعمل بخواهند به صورت همزمان کار کنند، به گونه‌ای که روی یک منبع
عملیات خواندن و نوشتن را انجام دهند، دو حالت به وجود می‌آید:

\begin{enumerate}
    \item یا در ماهیت خود برنامه وجود دارند
    \item یا برنامه نویس ایجاد می‌کند که قابلیت حذف دارد
\end{enumerate}

نکته: یکی از وظایف اصلی کامپایلر حذف وابستگی‌هایی است که برنامه نویس ایجاد
می‌کند و قبل از مرحله اجرا انجام می‌شود و باعث کارایی بالا برنامه می‌شود.


وابستگی‌های زیر را در نظر بگیرید:

S[i]: O ... = .... O

S[j]: O ... = .... O

\end{document}