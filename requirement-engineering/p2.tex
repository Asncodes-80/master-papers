\documentclass[a4paper]{article}
\usepackage{forest}
\usepackage{float}
\usepackage{pgf-pie}
\usepackage{pgfplots} 
\usepackage{geometry}
\usepackage{listings}
\usepackage{hyperref}
\usepackage{plantuml}
\usepackage{graphicx}
\usepackage{ragged2e}
\usepackage{color}
\usepackage{xepersian}
\usepackage{subfiles}
\newgeometry{left=1.4cm, right=1.4cm, bottom=2.0cm, top=2.0cm}
\settextfont[Scale=0.75]{XB Roya}
\renewcommand{\baselinestretch}{1.5}
\definecolor{dkgreen}{rgb}{0,0.6,0}
\definecolor{gray}{rgb}{0.5,0.5,0.5}
\definecolor{mauve}{rgb}{0.58,0,0.82}
\definecolor{commentColor}{rgb}{0.6,0.6,0.60}

\title{مهندسی نیازمندی‌ها خانم دکتر سپیده آدابی}
\author{علیرضا سلطانی نشان}

\begin{document}
\maketitle
\section*{جملات \lr{Strong conflict} و \lr{Weak conflict} در مورد سیستم
انتخاب واحد دانشجو}

\subsection*{تضاد ضعیف}

\begin{enumerate}
    \item دانشجوی پسر در صورت معافیت تحصیلی تا زمان مقرر می‌تواند به تحصیل و
    انتخاب واحد خود بپردازد.
    \item شهریه درس می‌تواند تا پایان ترم جاری پرداخت شود.
    \item انتخاب واحد در هر ترم در مقطع کارشناسی ارشد نهایتاً ۱۴ واحد می‌باشد.
    \item شرایط پرداخت قسط (شهریه ثابت) تا نیمه دوم ترم جاری می‌باشد.
\end{enumerate}

\subsection*{تضاد قوی}

\begin{enumerate}
    \item در صورت عدم پرداخت شهریه ثابت، دانشجو مجاز به انتخاب واحد در ترم جاری
    نیست.
    \item دانشجو نمی‌تواند دو درسی که در یک ساعت و یک روز تشکیل می‌شود را به
    صورت همزمان در یک ترم داشته باشد.
    \item دانشجوی حذف ترم مجاز به انتخاب واحد نمی‌باشد.
    \item دانشجوی مرخصی مجاز به انتخاب واحد نمی‌باشد.
    \item امکان مشاهده کارنامه و برنامه امتحانی قبل از پرداخت شهریه درس وجود
    ندارد.
\end{enumerate}

\subsection*{المان‌های دانشی}

\begin{enumerate}
    \item \lr{Actor}: دانشجو
    \item \lr{Concept}: مدیریت دروس، اخذ واحد
    \item \lr{Objective}: تحصیل در ترم جاری، انتخاب حداقل واحد، پرداخت شهریه
    \item \lr{Task}: مدیریت پرداخت‌ها، مدیریت دروس انتخاب شده، هندل انتخاب واحد
    \item \lr{Requirement}: دانشجو بتواند به میزان حداقل و حداکثر تعداد واحد،
    درس اخذ کند.
    \item \lr{Domain}: \lr{isBasePaymentSuccessful} $\leftarrow$
    \lr{studentCanSelectUnit}
\end{enumerate}


\end{document}