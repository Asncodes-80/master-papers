\documentclass[a4paper]{article}
\usepackage{forest}
\usepackage{float}
\usepackage{pgf-pie}
\usepackage{pgfplots} 
\usepackage{geometry}
\usepackage{listings}
\usepackage{hyperref}
\usepackage{plantuml}
\usepackage{graphicx}
\usepackage{ragged2e}
\usepackage{color}
\usepackage{xepersian}
\usepackage{subfiles}
\newgeometry{left=1.4cm, right=1.4cm, bottom=2.0cm, top=2.0cm}
\settextfont[Scale=1]{XB Roya}
\renewcommand{\baselinestretch}{1.5}
\definecolor{dkgreen}{rgb}{0,0.6,0}
\definecolor{gray}{rgb}{0.5,0.5,0.5}
\definecolor{mauve}{rgb}{0.58,0,0.82}
\definecolor{commentColor}{rgb}{0.6,0.6,0.60}

\title{مهندسی نیازمندی‌ها خانم دکتر سپیده آدابی}
\author{علیرضا سلطانی نشان}

\begin{document}
\maketitle
\subsubsection*{تفاوت پایایی و روایی در پرسشنامه‌ها}

یک پرسشنامه خوب باید دو ویژگی پایایی و روایی را به همراه داشته باشد. 

\begin{itemize}
    \item پایایی قابلیت اطمینان پرسشنامه به همراه دقت در اندازه‌گیری می‌باشد.
    یعنی اگر همان پرسشنامه در همان شرایط بخواهد به صورت مجدد صورت گیرد، امتیاز
    یا مقدار حاصل از پرسشنامه هیچ تغییری نخواهد کرد.
    \item روایی به معنای آن است که میزان مطابقت نتایج بدست آمده از پرسشنامه با
    دنیای واقعی به چه اندازه‌ای می‌باشد.
\end{itemize}

\end{document}