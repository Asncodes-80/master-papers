\documentclass[a5paper]{article}
\usepackage{forest}
\usepackage{float}
\usepackage{pgf-pie}
\usepackage{pgfplots} 
\usepackage{geometry}
\usepackage{listings}
\usepackage{hyperref}
\usepackage{plantuml}
\usepackage{graphicx}
\usepackage{ragged2e}
\usepackage{color}
\usepackage{xepersian}
\usepackage{subfiles}
\newgeometry{left=1.4cm, right=1.4cm, bottom=2.0cm, top=2.0cm}
\settextfont[Scale=0.75]{XB Roya}
\renewcommand{\baselinestretch}{1.5}
\definecolor{dkgreen}{rgb}{0,0.6,0}
\definecolor{gray}{rgb}{0.5,0.5,0.5}
\definecolor{mauve}{rgb}{0.58,0,0.82}
\definecolor{commentColor}{rgb}{0.6,0.6,0.60}

\title{مهندسی نیازمندی‌ها خانم دکتر سپیده آدابی}
\author{علیرضا سلطانی نشان}

\begin{document}
\maketitle
\subsubsection*{تفاوت پایایی و روایی در پرسشنامه‌ها}

یک پرسشنامه خوب باید دو ویژگی پایایی و روایی را به همراه داشته باشد. 

\begin{itemize}
    \item پایایی قابلیت اطمینان پرسشنامه به همراه دقت در اندازه‌گیری می‌باشد.
    یعنی اگر همان پرسشنامه در همان شرایط بخواهد مجدداً صورت گیرد، امتیاز یا
    مقدار حاصل از پرسشنامه هیچ تغییری نخواهد کرد.
    \item روایی به معنای آن است که میزان مطابقت نتایج بدست آمده از پرسشنامه با
    دنیای واقعی به چه اندازه‌ای می‌باشد.
\end{itemize}

\subsection*{مثال}

فرض کنید می‌خواهید بیماری که به تازگی در یک دانشگاه شیوع پیدا کرده است را بررسی
کنید. علائمی را به عنوان پرسش‌نامه بین دانشجویان دانشگاه منتشر می‌کنید. این
علائم باید پایداری داشته باشند و در هر دوره که این پرسش‌نامه برگزار می‌شود
دقیقاً با یک معیار و اندازه انجام شود نه با متغیر‌های متفاوت. همچنین علائم اعلام
شده از سوی دانشجویان بایستی صحت داشته باشد، نتیجه پرسش‌نامه به گونه‌ای نشود که
با محیط پیرامون نتیجه کاملاً متفاوتی داشته باشد و جمع‌آوری داده را با شک و شبه
همراه سازد پس بایستی پرسش‌نامه و نتیجه آن روایی داشته باشد.

\end{document}