جلسه چهارم:


کیفیت جمله‌ها میتواند صرفا کیفیت درستی نداشته .

مجموعه ای از قابلیت‌ها میشه اسکوپ که در دامین‌ها تعریف می‌شوند.

برای ساخت سبد باید فرایند‌های مهندسی نیازمندی را برویم:

چهار قدم مهندسی نیازمندی:

تمام مراحل تکرار پذیر هستند.

ساعت گرد. داده‌ها جمع‌آوری میشود ولی همه داده‌ها انتخاب نمی‌شود. یکی از آنها
انتخاب می‌شود.

دامنه و استخراج نیازمندی‌ها

ارزیابی و توافق

ممکنه چیزایی جمع‌اوری کرده باشیم که کاملا نامربوط به اسکوپ باشد. هر آنچه به ما
می‌گویند ممکنه ریسک‌هایی باشد کگه می‌ةواند قابلیت بخش به نرم‌افزار باشد.

مثلا ریسک این وجود دارد که ممکنه اینترنت قطع بشود.

نیازمندی‌هایی که به ما وارد میکند الزاما همراستا نیست

میخواهد کارنامه ببیند دانشجو، استاد میگه ثبت نمره میخوام کارنامه رو ببینم. 
اگه یکی رو برآورده کنیم ممکنه کانفلیکت به وجد آید.

اولویت نیازمندی ها را آیا تعین کرده‌ایم؟ لیست دروس رو نشون ندیم ولی دانشجو
بتواند درس انتخاب کند. (امکان ندارد.)

همه این تصمیمات رو باید ارزیابی کنیم.

دسته آخر: یک سبدی که خیلی آشفته بود تبدیل به سبدی میشود که همه روی آن توافق
دارند و به طراح داده‌میشود.

نیازمندی‌هایی که روی آنها توافق شده است.

سبد بعد از آن به طراح تحویل داده می‌شود که زبان مشترک بین طراح و نیازمندی میشه
اشکال بصری که مطالب صریح و سریع انتقال شود.

سند یک قالب می‌خواد که استاندارد است. با چی بنویسیم با نماد گرافیکال نمایش بده.

بخش آخر:

ادغام، اعتبارسنجی:
سبد دستخوش تغییرات است. تا برسد به ۸۰ درصد ثابت و ۲۰ درصد تغییر کننده. این تغییر
۲۰ درصدی می‌تواند ساید افکت ایجاد کند. پس برای رفع ساید افکت باید اعتبارسنجی
حتما انجام شود.

می‌تواند چندین دور بزد.

اسلاید ۵۲:

آیا هر سیستمی نیازمند مهندسی نیازمندی است؟ خیر. 

سیستم‌های لگیسی حتما سند نیازمندی می‌خواهند. 

کلا اونایی که ورکفلو‌های اصلی رو می‌چرخونن سند نیازمندی می‌خواد

پروژه‌های استارتآپی که قراره خدمات به مردم بده که جنس خدمت یکیه و نحوه انجام آن
متفاوت است. این سیستم‌ها هم سند نیازمندی براشون اهمیت داره.

سند نیازمندی قابلیت reuse را به پروژه‌های مشابه می‌دهد.

یک مبنعی برای پروژه‌های مشابه میشه نه یه الگو.

اسلاید ۵۲

سند نیازمندی: اول خواسته در آورده می‌شه بعد قرارداد پروژه در میاد.

سازمان‌ها براسساس Request for proposal کار میکنند. که مهندس نیازمندی و متخصصین
اوجنا می‌نویسن که معمولا واحد‌های it مسئول آنها هستند.

پروتوتایپ در خصوص بخری نیازمندی‌ها که مبهم است که واضح نیست یک پروتوتایپ درست
می‌کنیم که بفهمیم اون نیاز چیست. می‌تواند روی کاغذ باشد باشد یا یه ui اشد. که
منظور اولیه رو برساند. می‌تواند تو سطح فاکنشن باشد هم می‌تواند نان فانکشن باشد.

چرا این سند دو طرفه‌است. توی محیط یه چیزی به ما گفتن. یعنی یه نیازی به ما منقل
شده که ما پروتوتایپ درست کردیم. بعد اون پروتوتایپ رو بهش نشون میدیم که ببینیم
درست فهمیدیم یا نه. ممکن نیازمندی اضافه بشه یا حذف بشه تا سبد اسکوپ ما کامل شود.

تخمین پروژه: باید سند نیازمندی‌ها دیده شود که بفهیمیم چقدر خواسته داریم تا زمان
مشخص کنیم، تا هزینه و اندازه تخمین بزنیم. ورژن بندی اینجا انجام می‌شود. یکی از
نیازمندی‌ها غیرعملیاتی مربوط به توسعه بوده. روی سبد تاثیر گذار است. که بایستی
براساس زمان معقول باشد.

acceptance test باید با نیازمندی‌های مشتری مطابقت داشته باشد. باید سناریو تست
وجود داشته باشد. سناریو‌های تست از سند نیازمندی می‌آید.

معماری:

چه قید‌هایی، چه ترتیبی تا بتواند نان فانکشن‌ها رو چک کنه که نان فانکشن‌ها در سند
نیازمندی نوشته شده است. مانند availability, useability, و دیگر خواسته‌ها.

معماری یکسری الزامات را ایجاد می‌کند. اون الزامات از جنس فانکشن‌هاستند. که روی
سند نیازمندی‌ها تاثیر می‌ذاره.

ریپورت صفحه ۵۳ نمودار رو نگاه اینجا بنویس.

expectation منظور همان Assumptionهاست.