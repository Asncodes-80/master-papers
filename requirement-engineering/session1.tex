\section{مهندسی نیازمندی}

\subsection{تعریف}

طبق تعریف کتاب پرسمن، نیازمندی‌ها تنها ثابت در حال تغییر می‌باشند. مهندسی
نیازمندی مهم‌ترین فاز انجام هر کاری در مهندسی نرم‌افزار می‌باشد. زیرا مشتری
دائماً در حال تغییر درخواست‌های خودش است به همین خاطر نیازمندی‌های برآورد شده
ملزوم به بروز شدن هستند. هر تغییری که صورت می‌گیرد به دلیل ماهیت پیچیده
نرم‌افزار بایستی پایدار \footnote{\lr{Stable}} باشد. پایداری به منظور بررسی
تغییرات از جوانب مختلف مانند امنیت و آزمون عملکرد صحیح می‌باشد. نیازمندی‌ها
کاملا پر دردسر هستند زیرا خیلی از دلایل شکست پروژه‌ها عدم بررسی نیازمند‌ها بوده
است. درست است که با آزمون و خطا تجربه به دست می‌آید ولی این تجربه‌ها در
پروژه‌های مقیاس بزرگ می‌تواند خطر آفرین باشد چرا که خود تجربه‌ها نیز نیازمند
بررسی و آزمون هستند که بتوانیم از آنها در پروژه‌های بعدی یا فعلی خود استفاده
کنیم.

دو کلمه اصلی در مهندسی نیازمندی‌ها وجود دارد:

\begin{enumerate}
  \item کلمه چه چیزی \footnote{\lr{What}}: دقیقاً‌ آن چیزی است که سیستم بایستی
  قادر به انجام آن باشد. مثلاً کاربر باید بتواند در نرم‌افزار لاگین کند.
  \item کلمه چطور \footnote{\lr{How}}: همانطور که از نامش پیداست چطور انجام شدن
  کار را تعریف می‌کند. برای مثال بالا می‌توان گفت سیستم لاگین باید کاملاً امن
  باشد. در این سیستم لاگین کاربران مختلف اعم از استاد، دانشجو و رئیس دانشگاه
  باید بتوانند زیر پنج ثانیه احراز هویت انجام دهند.
\end{enumerate}

زمانی که می‌گوییم نرم‌افزار ثبت نام درس، دقیقا بالاترین سطح تجرید
\footnote{\lr{Abstract}} را در نیازمندی بیان کرده‌ایم.

\subsection*{نکات}

\begin{itemize}
  \item مفاهیم کیفی به اندازه مفاهیم اجرایی مهم هستند. درست است نرم‌افزار باید
  اجرا شود اما این اجرا شدن باید صحیح باشد. امنیت نرم‌افزار خود خواسته می‌تواند
  تخریب شود، یعنی نرم‌افزاری نوشته می‌شود که می‌تواند ورودی‌های اشتباه و نادرست
  را بپذیرد، پس در این صورت امنیت و کارایی درست را زیر سوال می‌برد.
  \item سوال چه چیزی به صورت عملیاتی است و سوال چگونه به صورت غیر عملیاتی
  \item همیشه باید بین مسائلی که در مهندسی نرم‌افزار پیش می‌آید یک سبک سنگینی
  \footnote{\lr{Trade off}} صورت گیرد. معمولاً \lr{Benchmarks} ها به ما این
  امکان را می‌دهند. یعنی نرم‌افزار می‌تواند به چند شکل مختلف توسعه پیدا کند اما
  با گرفتن \lr{Benchmark} ها می‌توانیم بررسی کنیم که کدام یک از آنها در قسمت
  عملیاتی و عملکرد صحیح بهتر بوده‌اند. به عبارت دیگر، روش‌ها را نمی‌توان بدون
  بررسی و با میل شخصی انتخاب کرد، بلکه باید روش‌ها بررسی و سبک سنگین شوند.
  \item فرایند‌ها در مهندسی نیازمندی را process گویند
  \item توضیح و بازنویسی نیازمندی‌ها، کار پایه مهندس نیازمندی است.
  \item تمام مراحل در فرایند به یکدیگر وابسته می‌باشند، فرایند اساساً در مورد
  جزئیات صحبت نمی‌کند بلکه به ماهیت کلی و تجرید می‌پردازد. برای مثال فرایند
  جمع‌آوری داده و تحلیل و دیگر مراحل کاملاً به صورت مرحله‌ای و بازگشت پذیر
  می‌باشد. خروجی فرایند بعد از طی کردن تمام مراحل، نیازمندی را مشخص می‌کند.
  \item هیچ وقت فرایند با نیازمندی‌ها هم ارز نیست، بلکه نیازمندی خروجی فرایند
  می‌باشد. در حقیقت به خروجی فرایند، سند نیازمندی یا \lr{Requirement Document
  (RD)} می‌گویند.
  \item در فرایند تکینک‌ها و استاندارد‌ها دیده می‌شود.
\end{itemize}

\section{نکته تجرید}

هر موقع در مورد تجرید صحبت شد، در واقعیت امر میزان سطح پرداختن به جزئیات را
توضیح می‌دهد.

\section{متدولوژی}

متدولوزی \footnote{\lr{Methodology}} یک جهان‌بینی کلی، در تولید نرم‌افزار است
(دید از بالا برای انجام کار‌ها و وظایف). تمام متدولوزی‌ها را برای تولید استفاده
می‌کنند و تمام راهنمایی‌ها توضیحات دارند. در حقیقت تمام متدولوژی‌ها از خواستگاه
تولید نرم‌افزار ایجاده شده‌اند و حتی می‌شوند. نکته مهم آن است که فرایند‌ها درون
متدولوژی‌ها هستند. متدولوژی یک نقشه است که آن را معمار نرم‌افزار با دیدگاه
کاملاً جامع انتخاب می‌کند.

\section{دلیل متدولوژی‌های مختلف}

ماهیت و ذات پروژه‌ها متفاوت و پیچده‌ است، پس در این جهت متدولوزی‌های مختلفی برای
مهار آنها ارائه شده است که توع تولید را متفاوت می‌کند. متدولوژی بایستی کاملا
منعطف باشد. مراحل و فرایند‌ها در متدولوژی‌ها متغیر می‌باشد.

\section{ماهیت مدل}

انسان همیشه با خواندن مشکل دارد. خواندن دائماً با مشکلات محاوره‌ای همراه است.
محاوره با ابهام همراه است. در پروژه مهندسی نرم‌افزار، وقتی افراد بخواهند با
یکدیگر در مورد پروژه صحبت کنند، زبان میان آنها مدل‌های بصری و گرافیکی می‌باشد.
افراد بعد از جمع‌آوری اطلاعات و تحلیل آنها، بایستی با آنها به مفهوم بصری برسند
تا به کارشناسان دیگر آن را انتقال دهند. به بیانی دیگر، مدل زبان مشترک برای انجام
فرایند‌ها، بیان گرافیکی با حفظ سطح تجرید است.

انسان روی جمله‌های ترکیبی مشکل دارد: 

\begin{equation}
  \label{eq:مقایسه عملگر‌ها}
  (A \& B) v (c) \rightarrow x
\end{equation}

یا

\begin{equation}
  \label{eq:مقایسه عملگر‌ها}
  A \& (B v C) \rightarrow x
\end{equation}

راهکار: استفاده از \lr{Decision table} که بتوان منطقی به نتیجه رسید.

زبان مدلسازی: ریاضی و گرافیک (بصری)

\subsection*{عملیات به دو دسته تقسیم می‌شوند}

\begin{enumerate}
  \item عملیات ریاضی: $y = x$
  \item عملیات بصری: نمودار‌ها و مختصات
\end{enumerate}

\section*{نکات}

\begin{itemize}
  \item تجرید میزان پرداختن به جزئیات است
  \item سطح تجرید نسبت به هر کلاس و مدل‌های مختلف* متفاوت است
  \item خروجی هر فاز فرایند در متدولوژی مدل می‌شود. در حقیقت در متدولوژی مشخص
  می‌شود که مدل بخش مورد نظر به چه شکلی باشد.
  \item از آنجایی که زبان بین انسان و ماشین زبان برنامه نویسی (کامپایلر و گرامر)
  می‌باشد، زبان بین افراد برای نمایش بصیری نتیجه فرایند‌ها مدل می‌باشد.
  \item عملیات ریاضی صرفاً محاسباتی نیستند، بلکه می‌توانند در قسمت آنالیز هم
  بررسی و انجام شوند
\end{itemize}

\section{الگو}

الگو، راهنمایی برای حل مسائل مشابه می‌باشد. مشابه بودن مسائل به دلیل پر تکرار
بودن آنها در پروژه‌های مختلف است.

\section{استاندارد}

مجموعه‌ای از قواعد \footnote{\lr{Rules}} یا دستورات است. اجرای دستور ما را به
خواسته می‌رساند. مانند تمام \lr{Rule} هایی که روی فایروال شبکه اعمال می‌شوند. یا
اینکه یکسری قواعد محیطی را بیان می‌کند.

\section{مهندسی نیازمندی}

مهندسی نیازمندی یعنی مدلی که همه روی آن توافق دارند. یکسری حساب و کتاب،
استاندارد .مدل‌ها و غیره که خوش تعریف هستند بدون هیچ‌گونه ابهام، مطرح می‌شوند.

% زبان \lr{formal}:

% csp
% ctl
% ltl
% z
