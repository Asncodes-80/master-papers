\documentclass[a4paper]{article}
\usepackage{geometry}
\usepackage{hyperref}
\usepackage{xepersian}
\newgeometry{left=1.4cm, right=1.4cm, bottom=2.0cm, top=2.0cm}
\settextfont[Scale=1.0]{XB Roya}

\title{مهندسی نیازمندی‌ها خانم دکتر سپیده آدابی}
\author{علیرضا سلطانی نشان}

\begin{document}
\maketitle

\section*{سوال}
یک سیستم مدیریت یادگیری \footnote{\lr{Learning Management System}} مانند ادوبی
کانکت، اسکای‌روم یا دان را انتخاب کنید. دو جمله بنویسید که تضاد ضعیف یا قوی
داشته باشد و سپس برای آن راه‌حل مطرح کنید.

\subsection*{راهنمایی}

می‌توانید از نقش‌های زیر در جمله‌سازی (\lr{Statement}) خود استفاده کنید:

\begin{itemize}
    \item مدیر شبکه
    \item دانشجو
    \item استاد
    \item پشتیبانی
    \item مدیر آموزش
\end{itemize}

\subsection*{جمله اول}

\begin{itemize}
    \item نقش‌ها \begin{itemize}
        \item دانشجو
        \item استاد
    \end{itemize}
    \item تعاریف: \begin{itemize}
        \item \lr{Lecturer}: به مدرسی گفته می‌شود که امکان ارائه در \lr{LMS} را
        دارد.
        \item \lr{Student}: به دانشجویی گفته می‌شود که تنها امکان شرکت در جلسه
        را دارد و نمی‌تواند به صورت پیش‌فرض ارائه دهد.
    \end{itemize}
    \item جمله‌ای با تضاد قوی
    \item موقع ثبت‌نام دانشجویان در این سیستم، همه‌ی آن‌ها در نقش \lr{Student}
    در سیستم ثبت شده‌اند.
    \item تمام اساتید در هنگام ثبت‌نام در این سیستم، به عنوان \lr{Lecturer} ثبت
    شده‌اند.
    \item تضاد: هیچ دانشجویی امکان ارائه را ندارد.
    \item راه‌حل استفاده از رویکرد ضعیف کردن جمله است. در حالت عادی دانشجو در
    حال آموزش دیدن بود و استاد در حال تدریس از روی سیستم خود، یعنی دو طرف از
    قابلیت‌های خود راضی بودند اما مسئله از آنجایی شروع می‌شود که دانشجویی بخواهد
    مطلبی را از صفحه سیستم خود ارائه دهد. به همین خاطر تا قبل از شرایط مرزی
    مشکلی وجود نداشت پس دانشجو با دریافت مجوز مناسب (به روزرسانی مجوز دانشجو)
    می‌تواند مطالب خود را ارائه دهد.
\end{itemize}

\newpage

\subsection*{جمله دوم}

\begin{itemize}
    \item نقش‌ها \begin{itemize}
        \item دانشجو
        \item مدیر شبکه
    \end{itemize}
    \item جمله‌ای با تضاد قوی
    \item دانشجو می‌خواهد محیط ارائه خود را با کیفیت \lr{1920x1080} به نمایش
    بگذارد.
    \item مدیر شبکه پهنای باند را برای هر ارائه دهنده روی اندازه خروجی \lr{768x830}
    تنظیم کرده است تا ترافیک کمتری مصرف شود که این باعث تاخیر کمتر بین صدا و
    تصویر ارائه‌دهنده خواهد شد.
    \item دو رویکر می‌تواند با توجه به نیاز استفاده شود: \begin{itemize}
        \item اول: خاص‌سازی صورت گیرد، تنها کسانی که اندازه ارائه آنها با مقدار
        مشخص شده برابر است امکان ارائه داشته باشند.
        \item دوم: به دلیل سطح پیچیدگی مورد اول، «پرهیز از شرایط مرزی» استفاده
        معقولی خواهد بود. برای ارائه‌دهنده در سیستم مشخص می‌شود که چگونه
        می‌تواند خودش را با شرایط خواسته شده سازگار کند تا قادر به ارائه باشد.
    \end{itemize}
\end{itemize}

\subsection*{جمله سوم}

\begin{itemize}
    \item نقش‌ها \begin{itemize}
        \item دانشجو
        \item استاد
    \end{itemize}
    \item تضاد ضعیف: چرا که دانشجو کماکان می‌تواند از قابلیت \lr{Student} بودن
    خود در کلاس استفاده کند و طرفین کاملاً تا قبل از شرایط مرزی (که سوال پرسیدن از طریق
    مایکروفن می‌باشد) راضی هستند.
    \item دانشجو می‌خواد از قابلیت صحبت کردن خود در کلاس استفاده کند.
    \item استاد زمانی اجازه می‌دهد که دانشجو \lr{Raise hand} کرده باشد.
    \item رفع تضاد با رویکرد ضعیف کردن جمله، مشخصاً با دادن مجوز جدید دانشجو
    قادر به صحبت کردن می‌باشد ولی نکته حائز اهمیت در این میان آن است که بعد از
    پایان صحبت دانشجو استاد می‌تواند قابلیت صحبت کردن با مایکروفن رو از دانشجو
    بگیرد، در این میان رویکرد سوم که \lr{Restore} کردن می‌باشد می‌تواند مورد
    استفاده قرار بگیرد.
\end{itemize}

\subsection*{جمله چهارم}

\begin{itemize}
    \item نقش‌ها \begin{itemize}
        \item دانشجو
        \item مدیر شبکه
        \item استاد
        \item پشتیبان
        \item مدیر آموزش
    \end{itemize}
    \item جمله‌ای با تضاد قوی
    \item استاد می‌خواهد به طور کامل بتواند چت‌های دانشجویان را با یکدیگر بررسی
    کند.
    \item مدیر آموزش درخواست بررسی چت دانشجویان را بی‌دلیل تایید نمی‌کند.
    \item استفاده از رویکرد ضعیف کردن تضاد: استاد در حالت عادی می‌تواند قابلیت
    صحبت کردن و ارائه دادن دانشجو را کنترل کامل کند. اما زمانی می‌تواند چت‌های
    دانشجویان را بررسی کند که در ابتدا دلیل خود را به مدیر آموزش مطرح کند و در
    صورت تایید، مدیر آموزش به پشتیبان اعلام می‌کند که تنها در مدت ۱۰ دقیقه
    بتواند چت دانشجویان مشخص شده را بررسی کند..
\end{itemize}
\end{document}