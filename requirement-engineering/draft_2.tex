ایف‌ها رو از داخل کتاب تو فصل نهم بخون:

هر ریسکی به درخت ختم نمیشه

یک ریسک میتونه مانعی باشه برای رسیدن به گول

الزامان باید به یک المان در درخت گول متصل شود.

باید نسبت به اون ۶ کتگوری گول‌ها بررسی بشه که ریسک‌ها بهش متصل بشه.
مثل چسب داره اینکارو دو طرفه برامون انجام میده.

اول باید اون ۶ تا رو حساب کنیم که اینا مستقل نیستن

وقتی در مورد درخت ریسک صحبت میکنیم باید بچه‌های ریسک رو بنویسیم.

همون درخت ریسکه فقط نماداش و نود‌هاش تغییر میکنه. و از متوازی الاضلاع برعکس
استفاده می‌کنه.

مقدار criticality از ماتریس DDP بدست می‌آید.

افقی باید نقاط رو پر رگ کنیم.
انقدر ریسک را می‌شکنیم که بتوانیم حاشیه رو نویسیم

اگر and باشه min عدد رو می‌نویسیم

اگه or باشد max عدد رو می‌نویسیم

اون عدد چیه؟ همون احتمال وقوع ریسک و کریتیکالیتی که رخ دادن ریسکه.

برای نوشتن اعداد احتمال و کریتیکالیتی از annotation استفاده می‌کنیم.

راه‌حل‌های از جنس قابلیت هستند.

اون شکل وسطی که عکس گرفتی یعنی دومیه متوازی الاضلاعه میاد راه‌حل میگه نسبت به
ریسکی که رخ داده.

ساخت درخت ریسک؛

\subsection{تاتولوژیTautology}

هر چیزی از تاتولوژی نتیجه بگیره اند یا اورش میشه تو پر

از قواعد دمورگان و ساختمان گسسته استفاده می‌کند.

داخل خود گول خیلی کمک میکنه. اگه داخل گول باشه

a: reverse .... iff B: wheels turning

تاتولوژی شکستش کامله

اگه بین شروط تاتولوژی نبود یعنی تاتولوژی برای شکست این ریسک کاری ازش بر نمیاد.

هر ریسک ناشناخته‌ای رو نمیتونه واسمون بکشنه.

همه جا کشش نداره.

\subsection{شناخت شزایط لازم}

اکثر گول ‌های درخت از نوع اچیو و منیتین هستند.

if a then b

ریسک یعنی a هست و not b. یعنی بی و نتونستیم تجربه کنیم.

\subsection{الگو}

% فصل ۱۰ و ۱۱ در مورد یو ام ال
% دیاگرام 