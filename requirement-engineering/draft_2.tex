جدول فصل سه ص ۶۹ رو بخون

ضریب اهمیت به ان اف آر‌ها داده شده است. (وزن‌هاشون)

\subsection{اولویت‌ها}

چگونه ورژن‌بندی بکنیم؟

اولویت بعد از انتخاب رخ می‌دهد.
که سبک نیازمندی ثابت است و انتخاب در ورژن‌ها می‌تواند متغیر باشد.

مهم برای امتحان

روش AHP Analytic Hierarchy Process

انواع دارد

اما فکر کنم تو درس یکی رو داره استفاده میکنه که بدردش می‌خوره.

نسبت سبد را مشخص میکند

یه شلوار خریدم ۱ تومن گرونه؟ خب باید بگیم نسبت به چی.

مثایسه دو به دو میکند و سپس همه نسبت اون دوتایی‌ها به بقیه بررسی میشن.

نسبت ماژیک با کیف و غیره

نسبت مقایسه ماژیک کیف و ماژیک لباس و ماژیک و غیره چیه؟

AHP تکنیک

در نرم‌افزار دو معیار داره که میسنجه:

کاست و ارزش

ارزش: یعنی این نیازمندی رو محققش کنیم چند درصد به پروژه رسیدیم و اهدافمون رو طی
کردیم؟

کاست: دوست داریم هزینه پایین باشه و ارزش بالا باشد.

محور کاست 
محور ارزش‌ها

تلاقی دو خط هر چقدر ارزش بیشتر و هزینه کمتر داشته باشه اولویتش برامون بالاتره.

اون رزولوشن‌های هر محور را AHP میگن.

شیب خط از کجا میاد؟

استفاده از سیستم‌های فازی:

پارامتر‌ها مدیریت پروژه و خروجی میشه شیب خطه.

اگه پول کافی نداشته باشیم باید یه کار کوچک‌تر بگیریم ارزش اول رو برسیم بهش.

نیروی کارآموز دارم بهش اول باید یاد بدم یه کار رو بریم جلو پس الکی هزینه ایجاد
نکنم که چند تا کار رو باز کنم که باعث بشه کار‌ها نیمه نصفه تمام شود.

مثل یه چاقو میمونه که نیازمندی باید اونو مشخص کنه.

قضیه سیستم‌های فازی در \lr{Expert system} وجود دارد.

مهندس نیازمندی باید نقاط رو تعیین کنه

AHP
دو بار استفاده میکنیم یه بار برای کاست یه بار هم برای ولیو

تلاقی اون دوبار کاست و ولیو میشه نقطه‌ای که د راونجا اولویت مشخص می‌شود.

تکنیکی برای اولویت بندی نیازمندی‌ها می‌باشد.

شیب و چاقو را مدیر پروژه تعیین می‌کند.

برشی که زده میشه ورژنی که باید اون ارزش با هزینه تهیه و پیاده‌سازی بشه.

ص ۷۴

از عدد گزاری فرد استفاده می‌کند.

اگه یه مقدار بیشتر باشه اعداد فرد رو یه مقدار بیشتر می‌کنه.

هر چیزی نسبت به ارزش‌ها می‌تواند معکوس باشد. یعنی نسبت به قطر وارون میشه.

ماتریس این صفحه بهش میگن آر

ص ۷۶ جدولش ماتریس آر پریمه

جدول آر همون جریان ماژیک بالاست

آر پریم میشه خط بعدیش و دو تایی مقایسه می‌کند.

یه ردیف ۵ تایی رو جمع میکنه و تقیسم بر تعداد میکنه.

از نظر ولیو مشخص کردیم.

حالا جدول دوم به نسبت کاست ساخته می‌شه.

نقطه‌ای که روی نمودار ایجاد میشود پروداکت یا خروجی روش AHP بر روی هزینه‌ها و
ارزش‌هاست.

این تکنیک یک تکنیک انسان محور است.

نمودار‌ها:

نیازمندی سیستم میگیم هدف چند عامله
و سافور میشه سینگل عامله

یسری گول‌ها استراتژیک داریم ریز میشه به گوول‌های کوچک هست.

متوازی الاضلاع میاد گول‌ها رو مشخص میکنه

بولد شدن برای نشان لیف‌ها و آسسامپشن‌ها و نیازمندی سیستم‌ها که دیگه شکست و مشتق
نخواهیم داشت.

نقاط تو پر دارند complete and و complete or را مشخص می‌کنند.

توصیفی‌ها گول نیستن

توصیفی‌ها که میشه domain property میشه ذوزنقه

میگه سرعت مخالف صفر یعنی قطار حرکت میکنه در‌ها قبل حرکت بسته باشد.

dp میشه توصیف اون متوازی الاضلاعه

گول یه جملست و میدانیم که میتواند مالتی ایجنت و سینگل ایجنت باشد

عامل یعنی آن چیزی یا المانی که گول را محقق می‌کند
اگر گول سیستم ریکا بود مالتی هستن
اگر اسمپشن و سافور بودن یعنی یه عامل رو دارن محقق می‌کنند.

عامل: افراد دیوایس‌ها و سافت ور تو بی

برگ‌ها تک عامله هستند 

عامل میشه ۶ ضلعی agent

نکته: استاندارد نوشتاری برای گول

چون این استاندارد‌ها یه روش نوشتاری میدن.

استیت یعنی صفات خاص با مقدایر خاص

مسئول تغییر استیت میشه عوامل

مثال‌ها نوشته شود در مورد قطعار

قطار به خط قرمز رسید بوق بزند.

میشه مینیتنه تا زمانی که قرمز رو نبینه.

قطار به سرعت ۱۲۰ رسید به مدت ۱۰ ثانیه بوق بزند.

تاچ کردن سرعت ۱۲۰ یعنی رسیدن یا اچیو

اطلاعات قرض گیرندگان کتاب برای هیچ کس آشکار نشود.

Avoid آشکار نوشد نشود

گول: بار کاری پرسنل می‌نیموم شود. 
میشه گول نرم
که یه عملی توش دیده میشه
یا مدام انجام میشه
یا یک لحظه انجام می‌شود.

صفحه ای که برای کاربر طراحی می‌کنیم قابل استفاده برای نا بینایان باشد.

هر چی بیشتر بهتر میشه max(usability)

هزینه تولید نرم‌افزار باید کم شود.

min(Cost)

آیا همه نان فانکشن‌ها هستن نه
چون برای قرض گرفتنه نباید دیگران مطلع بشن در حقیقت ایمنی هم هست ولی به عنوان
\lr{NFR} دیده نمی‌شود. در حقیقت رفتاری بود.

رفتاری و نرم رو برای نوشتن گزاره‌ها گفته شده است.

گول‌های نرم دارن اولویت بین آپشن یک یا دو انتخاب شود.

برای انتخاب بین دو شاخه از system ref استفاده میشود که بین دو انتخاب اگه یه
انتخاب داشته باشیم میشه sytem to be.

تو نمودار ممکنه system as is داشته باشیم یعنی چیزی باشه که الانم داره استفاده
میشه .یعنی چیزی میشه که الان داریمش.

% کانفلیکت رو نگفتی 
% اون چارت def اینا رو هم بگو