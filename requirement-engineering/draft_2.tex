نکته: استاندارد نوشتاری برای گول

چون این استاندارد‌ها یه روش نوشتاری میدن.

استیت یعنی صفات خاص با مقدایر خاص

مسئول تغییر استیت میشه عوامل

مثال‌ها نوشته شود در مورد قطعار

قطار به خط قرمز رسید بوق بزند.

میشه مینیتنه تا زمانی که قرمز رو نبینه.

قطار به سرعت ۱۲۰ رسید به مدت ۱۰ ثانیه بوق بزند.

تاچ کردن سرعت ۱۲۰ یعنی رسیدن یا اچیو

اطلاعات قرض گیرندگان کتاب برای هیچ کس آشکار نشود.

Avoid آشکار نوشد نشود

گول: بار کاری پرسنل می‌نیموم شود.
میشه گول نرم
که یه عملی توش دیده میشه
یا مدام انجام میشه
یا یک لحظه انجام می‌شود.

صفحه ای که برای کاربر طراحی می‌کنیم قابل استفاده برای نا بینایان باشد.

هر چی بیشتر بهتر میشه max(usability)

هزینه تولید نرم‌افزار باید کم شود.

min(Cost)

آیا همه نان فانکشن‌ها هستن نه
چون برای قرض گرفتنه نباید دیگران مطلع بشن در حقیقت ایمنی هم هست ولی به عنوان
\lr{NFR} دیده نمی‌شود. در حقیقت رفتاری بود.

رفتاری و نرم رو برای نوشتن گزاره‌ها گفته شده است.

گول‌های نرم دارن اولویت بین آپشن یک یا دو انتخاب شود.

برای انتخاب بین دو شاخه از system ref استفاده میشود که بین دو انتخاب اگه یه
انتخاب داشته باشیم میشه sytem to be.

تو نمودار ممکنه system as is داشته باشیم یعنی چیزی باشه که الانم داره استفاده
میشه .یعنی چیزی میشه که الان داریمش.

% کانفلیکت رو نگفتی 
% اون چارت def اینا رو هم بگو

------------------------------

ریسک نات هدف هستش

گول مدل رو میکشه

فصل نهم:

ریسک مدل رو می‌نویسه

خود گل‌ها رو چگونه نشون بدیم
چند عامله و تک عامله (اسمپشن و سافور) گل‌ها متوازی الاضلاع

هر کدام از المان‌ها درخت فرقی نداره که دیاگراممان چطوریه و چی هستش

انوتیشن معمولاً نوت هستن باید ببینم المان چیه مهم نیست دو چیز نوشته میشه نام و
تعریف بقیه اختیاری می‌باشد، چرا الزامیه چون یه ایجنتی رو یه اسمی گذاشتیم روش از
من در آوردی، پس باید یه تعریفی واسه متقضای تحصیل نوشه بشه. مثلا نوشته حداقل
فاصله رو نگه دارن مینتین و منظور به هدف داره، پس باید تعریف بشه این یعنی چی؟

باید بگیم این تعریف مال کدوم المان من هستش.

اسمشون رو اون بالای مستطییل خط چین دار می‌نویسم:

name:
Def:

بقیش اختیاریه

تایپ گول هستش: یا رفتاری یا نرمه

به داخل مستطیل نگاه کن.

کتگوری نوع گول از نظر عملیاتی و غیر عملیاتی هستش.

چراییش رو چند صفحه بعد می‌نویسیم.

این بعدا توی ریسک بدرد می‌خوره.

یسری از گول‌های ما ریسکاشون مهمه که در بیارن اختیاری نیست.

مثلا اگه سکیوریتی باشه بایستی ریسکش در بیاد. ولی بعضیا الزام آور نیست و بسته به
انتخاب مشتریه.

۹ تا از مجموعه گول‌های عملیاتی و غیر عملیاتیمون باید در بیاد.

annotation حاشیه گذاشتن

feature ويژگی که تو حاشیه گول نوشته میٰشه.

منبعی که از اون گول مشخص کردیم میشه سورس
به چه دردی میخوره، مثلا سوال داشته باشیم، ابهام داشته باشیم یا در ورژن‌های بعدی
اونو تغییر بدیم میتونیم پیدا کنیم کسی که بهمون کمک کنه.

issue یعنی مسئله زمانی که داریم سند می‌کنیم ممکنه یه سوال برامون پیش بیاد که
برای مون ابهام داشته باشه.

قطار‌هایی که در داخل سیستم تعریف شده‌اند را هر چند قت یکباری باید نگهداری ببری.

کدومش؟ اونایی که تو انبار گذاشتیم یا اونایی که دارن الان حرکت میذکنن؟

باید یه دسته از ایشو درست کنیم که مثل یک چک نویس مرتب و منظرم داریم می‌نویسم که
چیه.

هر جایی که سورس گفته می‌تونه بهمون کمک کنه تا ایشو رو حل کنیم.

معمولاٍ میذاریم که چند تا ایشو جمع بشه تا بریم ابهاماتمون رو رفع کنیم.

اولویت‌ها از همون AHP استفاده میکنم که به صورت مستند روی دیاگرام نشون بدیم.

Formal Spec ما میتونستیم فرمال استفاده کنیم ولی باید سیستم عملیات بحرانی بلد
باشیم که زبان منطثی دارد.

زبان‌هایی مثل زد و csp می‌تواند استفاده شود.

Fit Critrion: معیار برازنده: فقط برای سافت گوله
میگفتیم بار کاری باید کمینه بشه معیاری که منو راضی میکنه چقدره؟ نمیتونیم بگیم
فقط کاهش؟ بگیم چقدر؟ این سنجشی برای اون هستش. کمی میکنه مقدار و مفهوم کیفی رو
که
میخوایم و مفهوم کیفی سافت گوله.

name is required
def is required

another is not required

نماد دامین پراپرتی با home نمایش داده میشه.

کانفلیکت رو با نمادی خاص نشان می‌دهیم: نمادی مثل فلش یا جرفه می‌ماند اونایی که
کانفیکلیت داره رو به هم وصل می‌کنه و علامت تضاد رو نشون میده.

تصمیم قبلی همون تصمیم فعلی ما میتونه باشه که میشه ref

فصل نهم:

کتاب مرجع در مورد ریسک از کلمه obstacle استفاده کرده.

متوازی الاضلاع برعکس هستش برای نمایش ریسک که نات گول هستش.

ریسک میگه من در کارنت هستم و به اون استیت بعدی نرسیدم.

ریسک در مورد بخش بعد از then صحبت می‌کند.

اگر تمرز قطار رو کشیدم باید استاپ بشه. از نوع اچیو هستش. ناتش میشه که به اون
استیته که باید صفر بشه نرسه.

ایف‌ها رو از داخل کتاب تو فصل نهم بخون:

۹ تا گول داریم که باید ریسکاشون رو دربیاری چه مشتری بگه چه مشتری نگه:

safety -> Hazar
security -> Threat
statistician -> Dissertation
information -> Misinformation
accuracy -> Inaccuracy
usability -> usability

در حقیقت ۶ تاس اون ۹ تا رو ادیت کن به ۶.

هر ریسکی به درخت ختم نمیشه

یک ریسک میتونه مانعی باشه برای رسیدن به گول

الزامان باید به یک المان در درخت گول متصل شود.

باید نسبت به اون ۶ کتگوری گول‌ها بررسی بشه که ریسک‌ها بهش متصل بشه.
مثل چسب داره اینکارو دو طرفه برامون انجام میده.

اول باید اون ۶ تا رو حساب کنیم که اینا مستقل نیستن

وقتی در مورد درخت ریسک صحبت میکنیم باید بچه‌های ریسک رو بنویسیم.

همون درخت ریسکه فقط نماداش و نود‌هاش تغییر میکنه. و از متوازی الاضلاع برعکس
استفاده می‌کنه.

مقدار criticality از ماتریس DDP بدست می‌آید.

افقی باید نقاط رو پر رگ کنیم.
انقدر ریسک را می‌شکنیم که بتوانیم حاشیه رو نویسیم

اگر and باشه min عدد رو می‌نویسیم

اگه or باشد max عدد رو می‌نویسیم

اون عدد چیه؟ همون احتمال وقوع ریسک و کریتیکالیتی که رخ دادن ریسکه.

برای نوشتن اعداد احتمال و کریتیکالیتی از annotation استفاده می‌کنیم.

راه‌حل‌های از جنس قابلیت هستند.

اون شکل وسطی که عکس گرفتی یعنی دومیه متوازی الاضلاعه میاد راه‌حل میگه نسبت به
ریسکی که رخ داده.

ساخت درخت ریسک؛

\subsection{تاتولوژیTautology}

هر چیزی از تاتولوژی نتیجه بگیره اند یا اورش میشه تو پر

از قواعد دمورگان و ساختمان گسسته استفاده می‌کند.

داخل خود گول خیلی کمک میکنه. اگه داخل گول باشه

a: reverse .... iff B: wheels turning

تاتولوژی شکستش کامله

اگه بین شروط تاتولوژی نبود یعنی تاتولوژی برای شکست این ریسک کاری ازش بر نمیاد.

هر ریسک ناشناخته‌ای رو نمیتونه واسمون بکشنه.

همه جا کشش نداره.

\subsection{شناخت شزایط لازم}

اکثر گول ‌های درخت از نوع اچیو و منیتین هستند.

if a then b

ریسک یعنی a هست و not b. یعنی بی و نتونستیم تجربه کنیم.

\subsection{الگو}

% فصل ۱۰ و ۱۱ در مورد یو ام ال
% دیاگرام 