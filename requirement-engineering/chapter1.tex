\section{فصل اول}

\subsection{اصطلاحات}

\subsubsection{\lr{World problem} یا \lr{Environment}}

دنیای مسئله جایی است که مشکلی در آن رخ داده است و کسی وجود دارد که این مشکل را
در ابتدا بررسی و بعد از آن حل می‌کند. در حقیقت دنیا، محیط عملیاتی ما در مهندسی
نیازمندی است. این دنیا می‌تواند سینما باشد یا دانشگاه. جنس این مسائل می‌تواند
مشکل باشد که بایستی برطرف شود یا قابلیتی که می‌خواهیم در آینده اتفاق بیوفتد.

\subsubsection{\lr{Machine}}

ماشین راه‌حلی برای حل مسئله‌ای می‌باشد که پیش آمده است. ماشین می‌تواند به صورت
آماده خریداری شود یا توسط تیم توسعه از صفر توسعه داده شود. ما باید در سند
نیازمندی این نوع از نیازمندی را مشخص کنیم. ماشین در حقیقت نرم‌افزاری است که قرار
است داشته باشیم \footnote{\lr{Software to be}}. مدیر نیازمندی با توجه به هزینه
می‌تواند برای مهندس نیازمندی تعیین کند که آیا داشتن نرم‌افزار آماده هزینه کمتری
برایش دارد یا توسعه آن نرم‌افزار از صفر توسط تیم توسعه خود.

\subsubsection{\lr{Context}}

کلمه \lr{Context} به معنای زمینه می‌باشد. تمام رفتار‌ها و شکل‌های انجام کار را
نشان می‌دهد. مشخص می‌کند که چه نیازمندی‌های علمی را باید بدانیم تا بتوانیم در
نرم‌افزار آن را پیاده‌سازی کنیم. زمینه‌های مرتبطی برای توسعه‌ که باید به علوم
آنها واقف شویم. برای مثال هنگام توسعه یک نرم‌افزار تشخیص پیوند مولکولی و طراحی
پروتئین نیازمند آن هستیم که در مورد شاخه‌های علمی بایولوژی، بایوتک و ژنتیک علومی
را کسب کنیم. این علوم می‌تواند توسط تحقیقات و پژوهش‌های فردی بدست آید یا اینکه
در راستای تحصیل در یک رشته می‌توانیم در رشته دیگر به تحصیلات آکادمیک بپردازیم و
به نوعی مدرک کارشناسی آن حوزه را بدست آوریم که بتوانیم به صورت کامل روی موضوع
عملیاتی خود واقف و مسلط شویم.

\subsubsection{\lr{Statement} یا جمله}

\lr{Statement} یک جملست که ترکیبی از پدیده‌ها می‌باشد. برای مثال گفته می‌شود،
وقتی ترمز خودرو فشرده شد، در‌ها قفل شود و کاربر بتواند وضعیت دنده خود را تغییر
دهد. بعضی از این پدیده‌ها در دنیای مسئله یا محیط اتفاق می‌افتد. فعل‌های محیطی را
به هم متصل می‌کند و به فعل‌های نرم‌افزاری دخالتی ندارد.

\subsubsection{\lr{Phenomena} یا پدیده‌ها}

تمام اتفاقاتی که در مسئله (یا جمله) رخ می‌دهد را پدیده یا \lr{Phenomena} گویند.
برخی پدیده‌ها دقیقاً داخل نرم‌افزار رخ می‌دهد، مانند خطای TLS یا خطای پیدا نشدن
صفحه. برخی پدیده‌ها بین ارتباطات رخ می‌دهد مانند نرمال‌سازی دیتابیس. پدیده خرید
کردن یک پدیده محیطی است. وقتی برای کاربر اعلانی ارسال می‌شود در واقع این اعلانات
پدیده‌ بین محیط و نرم‌افزار است.

\subsubsection{\lr{System as is}}

\subsubsection{\lr{System to be}}

\subsubsection{مفروضات یا \lr{Assumption}}

% TODO: از نوع Descriptive است

\subsubsection{ویژگی دامنه یا \lr{Domain property}}

% بررسی واژگان

% چرا باید مهندسی‌نیازمندی‌ها وجود داشته باشد

% بررسی آمار و ارقام

% واژگان:

% المان: انجام دهنده کار

system as is به محیط مربوط است.

به عکس گرفته شده مراجعه شود. مهندسی نیازمندی بیشتر به قرمز و حاشور حساسیت انجام
میده.

Prescriptive

Descriptive

Phenomena

نرم‌افزار به اجزای زیر نیاز دارد:

People: عوامل
Device: دستگاه‌های واسط برای دریافت داده و انتقال به بخش software
exists software: تمام توابع اجرایی مثل پروتکل‌ها و غیره

قرار است گزارش‌هایی را تهیه کنند که به آنها سیستم تحویل داده شود.

Env/Problem world

تولز‌ها واسط بین انسان و انجام کار هستند.

دیوایس‌ها مثل سنسور‌ها

Prescriptive:

تجوزی است که نیاز سیستم مشخص می‌کند که چیکار باید کرد.

- system requirement: وظایف اعلان‌های موجود در سیستم را مشخص می‌کند.  در‌های
قطار موقع حرکت قفل شود. استیت منت میشه و فعل‌ها بسته شدن و حرکت کردن.

سنسور‌های قطار و اون نرم‌افزار باهم میشه سافتور تو بی، و در کنار هم میشه سیستم
تو بی.

یک requirement سیستم مجموعه‌ از assumption ها و sw requirement هاست.

system to be مجموعه از المان‌های‌ محیطی و sw to be

روی دکمه تایید زدن assumption هستش که کاربر باید دخالت کنه.

- SW requirement: 
- Assumption: مفروضات. تک کار‌های کوچکی که به محیط میدیم میشه مفروضات

System:

- system as is: چیزی که هست. المان‌ها و ارتباطاتی که الان وجود دارد.
مانند افراد و دستگاه‌ها

- system to be: چیزی که باید رخ بدهد.

Descriptive:

- Domain property: یک عبارت توصیفی است. یک حقیقت از فیزیک است. قابل مذاکره نیست.
نمیشه بگیم بعدا. اصلا کم و زیادش نمیشه کرد.

برای مثال نمی‌توان دو تا درس در دو زمان یکسان برداشت. یعنی از نظر فیزیکی
نمی‌توان همزمان در دو کلاس حاضر شد. و این پیام را نیازمندی نرم‌افزار برنامه نویس
مشخص کرده است که این پیام را نشان بدهد.

SW. Requirement:

- Functional requirement
- non functional requirement

نکته:

یک assuption یک پدیدست که statement را مشخص می‌کند.


----------------------

جلسه سوم:

المان‌ها 
محیطی: اون سه تا مردم و دیوایس‌ها و نرم‌افزار‌های موجود
سافت ور تو بی

همه statement هستند.

Prescriptive:
- System requirement
- Sofwreare requirement
- Assumption

Descriptive:
- Domain properties

Organization:
+ Domain1 to n: دامنه‌های در دل سازمان‌ها هستند، دامنه پژوهشی و دامنه مالیآدم‌ها
و ارتباطاتی که در دامنه‌ها وجود دارد.
+ Scopes: مجموعه‌هایی هستند که که نرم‌افزار میتواند در آن‌ها ورود داشته باشد.
مثلا فعالیت‌های مربوط به ثبت نام را شامل می‌شود. که به آن system scope ثبت‌نام
گفته می‌شود.

مهندس نیازمندی باید مراقب باشد که اسکوپ‌ها را کنترل و مدیریت کند که نرم‌افزار از
دست خارج نشود و باعث پیچیده‌تر شدنش نشود.

از یک دامنه به دامنه دیگر این ویژگی‌های تغییر می‌کند. در حالی که ساختار این
دامنه‌ها باید حفظ شود.

اسکوپ مجموعه‌ای از system requirement‌ها می‌باشد. که تصمیم میگیرم که کدوما باید
نرم‌افزاری باشد و کدوما باید محیطی باشد.

مثال کتابخانه فیزیکی و کتابخانه دیجیتال قید شود. که دامنه‌ها همیشه ثابت نیستند و
می‌تواند تغییر کند.

بعد y میشه هدف: مثلا پیاده‌سازی این قابلیت هدفش رضایت مشتری است.

همیشه از اهداف شروع می‌کنیم و به نیازمندی سیستمی می‌رسیم و نیازمندی سیستمی را به
نرم‌افزاری و محیطی مشکنیم.

% TODO: Drew three
درخت هدف، نیازمندی سیستم، نرم‌افزار‌ها و مفروضات در برگ‌های درخت نوشته شود.

what: چیزی که قرار پیده بشه در سافتور 
who: محیط و سافت‌ور

- سازمانی‌ها بیشترشون هدفگرا.
- کار‌های استارتآپی یوزر‌هاشون زیاد و بیشتر agent محور هستند.

سطح system requirement بالاست چرا که اصلاً مشتری هیچیزی از آن نمی‌داند چرا که
تنها هدف و خواسته و درخواست خودش را مطرح می‌کند.

یک معنای دقیق از چیزایی که می‌نویسم.
Definition: اصطلاحاتی که در سیستم وجود دارد.


مانیتور کردن یعنی بررسی و تحلیل ورودی‌های داده
کنترل کردن یعنی اعمال

system requirement:
بررسی میکند که حرکت دارد یا نه یعنی ماینتور میکنه و چک میکنه در باز هست یا بسته
پس system requirement ارتباط بین مانیتور و کنترل است

ارتباط مانیتور با input میشه assumption

یه assumption دیگه هم داره بعد از اینکه تحلیل داده‌های ورودی انجام شد نوبت
دریافت output و اعمال نتیجه یعنی کنترل است.

به اسلاید مراجعه شود. اسلاید ۲۸ تا ۳۰

اسکوپ‌ها از system requirementها نشکیل شده است.


Model checker

software requirement:
+ functional requirement: چه کاری در سیستم باید انجام شود
+ non-functional requirement: چطوری باید انجام شود. برای این دسته باید مجموعه‌ای
از اقدامات که بار اجرایی دارند رو استفاده کنیم.

دسته‌بندی functions:
+ Information: اعلانات و اطلاع رسانی هر چیزی که سند و ریسیو داشته باشد
+ Satisfaction: تعین state یک کار است که در جریان معنا دارد.
+ Stim-response: محرک پاسخ: وقتی دکمه زده شد آلارم صدا کند.

اسلاید ۳۵ non-functional رو بررسی کنید.

سه تای اول برای محصوله
و آخری برای مدیر پروژست

constraint: قید با محدودیت فرق دارد. limit

قید: مقید شدن به چیزی، مثلا نرم‌افزاری که توسعه شده باید قابلیت نصب در موبایل را
داشته باشد.

limitation: چیزی که نمیتوان آن را برطرف کرد. بلکه باید نرم‌افزار با آن کنار
بیاد. بار منفی دارد.

QoS: Quality of Service: پارامتری را نشان میدهد که میخوایم آن را از نظر کیفی
تامین کنیم. برای مثال اهداف امنیتی.
SLA: Service Level Agreement: توافق سطح سرویس از نظر کیفی

در قرار داد sla مقدار قابل قبول qosهایی که دانبالش هستیم را بیان می‌کنیم.

Compliance: قواعد و هنجار‌ها، الزاما ثابت نیست. (که پذیرفته شده‌اند)

نرم‌افزار باید تابع این هنجار‌ها باشد. قواعدی که در نرم‌افزار قید می‌شود. برای
مثلا فاصله بین دو ماشین که با شهرداری صحبت میکنیم که تو شهر یه مقداره، توی جاده
یه مقدار دیگه.

قید معماری: بعضی قید‌ها مربوط به نصب هستند. و برخی دیگر توزیع شده.

نصب: مثلا باید نرم‌افزار روی پلتفرم موبایل یا عینک گوگل نصب شود. مشخصات نصب
بازی. که حتی می‌تواند پایین‌تر از سطح سکو باشد مثلا نصب در یک سیستم‌عامل مخصوص
باید باشد. حتی میتونه قید و بند‌های سخت‌افزاری باشد. مثلا نصب تنها روی آرم باشد.

قید توزیع: ورودی و خروج از دو در مختلف در دانشگاه. چون داده ورودی و خروجی در
دوجای مختلف بلکه برای رسیدن به توافق در این توزیع باید این داده‌ها را بررسی کرد.
منظور در حقیقت توزیع ورودی و خروجی داده است. بلکه سرور یک جای مجتمع می‌تواند هم
باشد.

قید و بند توسعه:
کاری نداره که محصول چیست. نگرانی اصلی پیش مدیر پروژه باشد. هم متدولوژی درست هم
تصمیم درست باید اتخاذ کند. بلکه هزینه هم از این قضیه نگرانی دارد. و همچنین طراح
معمار نرم‌افزار هم دچار نگرانی می‌شود. 