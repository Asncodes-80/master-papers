% \newpage

% توضیحات فصل اول بسیار اولیه است. در حالتی که در مورد دیاگرام‌های uml صحبت میکند.

% Setting the scene

% \section{آماده‌سازی صحنه}

% بررسی واژگان

% چرا باید مهندسی‌نیازمندی‌ها وجود داشته باشد

% بررسی آمار و ارقام

% واژگان:

% Problem world

% دنیای مسئله جایی است که یک مشکلی اتفاق افتاده است کسی می‌تواند این مشل را حل کند
% که می‌تواند آن را بررسی و حل کند. اون دنیا می‌تونه می‌تونه سینما باشه، مدرسه
% باشد، دانشگاه باشد، شرکت باشد. یا مشکل است یا قابلیتی است که می‌خواهیم اتفاق
% بیوفتد.

% Machine

% راه‌حل برای حل اون مسئله است یعنی software to be. ماشین می‌تواند خریداری شود یا
% توسط تیم توسعه نوشته شود. ما باید در سند نیازمندی‌ها این نیازمندی را مشخص کنیم.

% Context

% رفتار‌ها و شکل‌های انجام کار را نشان می‌دهد. مشخص می‌کند که چه نیازمندی‌های علمی
% را باید بدانیم تا بتوانیم در نمر‌افزار آن را پیاده‌سازی کنیم. زمینه‌های مرتبطی
% برای توسعه‌ آن‌ها باید به علوم آن واقف شویم.

% المان: انجام دهنده کار

% Statement: یک جملست که ترکیبی از پدیده‌هاست. مثلا پا رو روی ترمز بذارم، در‌ها
% باز شود. ترکیب جملات است. بعضی از این پدیده‌ها در problem world تفاق می‌وفتد.
% فعل‌های محیطی، اشتراکی را به هم متصل می‌کند و به فعل‌های نرم‌افزاری کاری ندارد.

% بعضی از پدیده‌ها دقیقا داخل نرم‌افزار است. مثل فعل ارور ۴۰۴. ارور tls.

% برخی پدیده‌ها بین ارتباطات رخ می‌دهد. مثل نرمال‌سازی دیتابیس.

% اعلانات بین راه ارتباطی محیط و نرم‌افزار است.

% پدیده خرید کردن دقیقا یک پدیده محیطی است.

% system as is به محیط مربوط است.

% به عکس گرفته شده مراجعه شود. مهندسی نیازمندی بیشتر به قرمز و حاشور حساسیت انجام
% میده.

% Prescriptive

% Descriptive

% Phenomena

% نرم‌افزار به اجزای زیر نیاز دارد:

% People: عوامل
% Device: دستگاه‌های واسط برای دریافت داده و انتقال به بخش software
% exists software: تمام توابع اجرایی مثل پروتکل‌ها و غیره

% قرار است گزارش‌هایی را تهیه کنند که به آنها سیستم تحویل داده شود.

% Env/Problem world

% تولز‌ها واسط بین انسان و انجام کار هستند.

% دیوایس‌ها مثل سنسور‌ها

% Prescriptive:

% تجوزی است که نیاز سیستم مشخص می‌کند که چیکار باید کرد.

% - system requirement: وظایف اعلان‌های موجود در سیستم را مشخص می‌کند.  در‌های
% قطار موقع حرکت قفل شود. استیت منت میشه و فعل‌ها بسته شدن و حرکت کردن.

% سنسور‌های قطار و اون نرم‌افزار باهم میشه سافتور تو بی، و در کنار هم میشه سیستم
% تو بی.

% یک requirement سیستم مجموعه‌ از assumption ها و sw requirement هاست.

% system to be مجموعه از المان‌های‌ محیطی و sw to be

% روی دکمه تایید زدن assumption هستش که کاربر باید دخالت کنه.

% - SW requirement: 
% - Assumption: مفروضات. تک کار‌های کوچکی که به محیط میدیم میشه مفروضات

% System:

% - system as is: چیزی که هست. المان‌ها و ارتباطاتی که الان وجود دارد.
% مانند افراد و دستگاه‌ها

% - system to be: چیزی که باید رخ بدهد.

% Descriptive:

% - Domain property: یک عبارت توصیفی است. یک حقیقت از فیزیک است. قابل مذاکره نیست.
% نمیشه بگیم بعدا. اصلا کم و زیادش نمیشه کرد.

% برای مثال نمی‌توان دو تا درس در دو زمان یکسان برداشت. یعنی از نظر فیزیکی
% نمی‌توان همزمان در دو کلاس حاضر شد. و این پیام را نیازمندی نرم‌افزار برنامه نویس
% مشخص کرده است که این پیام را نشان بدهد.

% SW. Requirement:

% - Functional requirement
% - non functional requirement

% نکته:

% یک assuption یک پدیدست که statement را مشخص می‌کند.