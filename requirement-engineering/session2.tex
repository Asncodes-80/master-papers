\subsection{دلیل استفاده از زبان UML}

در مهندسی نیازمندی زبان مشترک بین تیم توسعه و طراحی با مشتری (کسی که درخواست
دارد) زبان \lr{UML} است. زبان درخواست کننده محاوره‌ای است و می‌تواند از آن هر
برداشتی داشت.

\section{بررسی شروع کار مهندسی نیازمندی}

\subsection{بررسی \lr{UML to goal}}

قبل از انجام هر کاری بایستی اقدامات مهمی در شروع مهندسی صورت گیرد. تهیه
نمودار‌هایی که با یکدیگر ارتباط مهمی دارند و لازمه ورود به بخش طراحی معماری
نرم‌افزار است.

\subsubsection{نمودار هدف}

اولین نموداری که در مهندسی باید کشیده شود نمودار هدف \footnote{\lr{Goal
diagram}} است. اهداف در نهایت به نیازمندی‌هایی می‌رسد که قرار است در سیستم محقق
شود. بیان نیازمندی یعنی بیان اهداف.

\subsubsection{نمودار ریسک}

رسیک‌ها اتفاقات محیطی هستند که باید اقداماتی نسبت به آن‌ها در سیستم پیاده شود.
مانند برقرار امنیت یا مشکلات کند بودن سرویس‌دهی مربوط به لود بالانسینگ. آن
مواردی که به عنوان ریسک در اهداف پیدا می‌شود هم نیازمند کشیدن نمودار ریسک است.

\subsubsection{نمودار \lr{Agent}}

برخی از اقدامات توسط نرم‌افزار انجام می‌شود و برخی دیگر توسط کاربر (عامل). برخی
از اهداف ممکن است به یکسری قابلیت‌های محیطی مربوط شوند. یعنی نرم‌افزار هیچ قوه
تحلیلی برای مشتری ندارد بلکه مشتری است که با دخالت خود می‌تواند به هدف مورد نظر
برسد. عامل کسی است که تعیین میکند قرار است چه عملیاتی رخ دهد.

\subsection{مهندسی نرم‌افزار و مهندسی نیازمندی}

در مهندسی نرم‌افزار مجموعه‌ای از ترتیب‌های \footnote{\lr{Discipline}} مخصوص به
آن وجود دارد مانند:

\begin{enumerate}
    \item مدیر پروژه \lr{Project manager}
    \item مالک پروژه \lr{Product owner}
    \item بخش‌های زیرساختی مانند زیرساخت شبکه و پشتیبانی و سرویس
    \item بخش پیاده‌سازی \lr{Implementation}
    \item بخش بررسی استاندارد‌ها و متدولوژی‌ها
    \item بخش مستندات \lr{Documentation}
    \item بخش آزمون \lr{Test}
\end{enumerate}

مهندسی نیازمندی یکی از زیر بخش‌های مهم مهندسی نرم‌افزار است.

\subsection{مهندسی نیازمندی و مدیریت نیازمندی}

مهندسی کلمه‌ای است که داشتن یک فرایند مرحله به مرحله را الزام‌آور می‌کند. یعنی
برای مهندسی یک پروژه نرم‌افزاری باید تمام جنبه‌های نرم‌افزاری به همراه ابزار‌ها
را بشناسیم که با صحیح و خطا و آزمایش موجب تولید یک محصول نهایی نشویم.

برای مثال فرایند مهندسی نیازمندی چهار مرحله‌ای زیر:

\begin{enumerate}
    \item جمع‌آوری نیازمندی‌ها
    \item تمیز کردن داده‌ها و معنادار کردن آنها
    \item بیان زبان برای مطرح کردن داده‌ها
    \item صحت‌سنجی و اعتبارسنجی کار‌ها
\end{enumerate}

مدیریت یعنی توزیع منابع. این منابع می‌تواند زمان، نیروی انسانی و ارزش‌های مالی
مانند پول و غیره باشد. مدیریت نیازمندی شامل مجموعه‌ای از ترتیب‌ها و توضیحات است
که بیشتر به مدیریت پروژه مربوط می‌شود. مدیر پروژه سهم بین هر بخش از توسعه را
تقسیم می‌کند. وظیفه مدیر نیازمندی، تقسیم وظایف به زیر عوامل است، اینکه بتواند
منابع اصلی را بین افراد و زیر بخش‌های خود (مفهوم چتری) تقسیم کند.

فعالیت اصلی زیر بخش مدیریت نیازمندی، مهندسی نیازمندی‌ها می‌باشد.