مهندسی نیازمندی

بین مشتریان و تیم توسعه و طراحی، زبان کسی که قرار طراحی انجام بده زبان uml است.
زبان درخواست دهنده محواره‌ای هستش. یک زبان مشترک بین طراح و درخواست دهنده محصول

شروع کار مهندسی نرم‌افزار از بررسی روش زیر است:

تا زمانی که اهداف سیستم رو نداشته باشیم نمی‌توانیم به ریسک و agent رسیم.

بررسی Goal to Uml

اولین دیاگرامی که کشیده میشه دیاگرام هدف است.
اهداف در نهایت به نیازمندی‌هایی می‌رسد که قرار است در سیستم محقق شود.
بیان نیازمندی یعنی اهدافمون

ریسک‌های اتفاقات محیطی هستند که باید اقدامتی نسبت به آنها بهع سیستم وارد شود.
مانند امنیت، یا مشکلات بالانسینگ. آن چیزی که پیدا میکنیم برای ریسک هم دیاگرام
می‌کشیم.

اهداف به یکسری قابلیت‌ها می‌رسند مثلا محیطی هستند یعنی نرم‌افزار برای مشتری اصلا
تصمیم گیری نمی‌کند. یکسری اقدامات توسط نرم‌افزار انجام میش‌شود و یکسری اقدامات
توسط عامل انجام می‌شود. اون عامل کسی که تعیین می‌کنه که قرار مشتری توی برنامه
چیکار بکنه. چقدرش با نرم‌افزاره در واقع توضیح وسعت سیستم رو نشون میدهند.

تمام این سه مدل با مشتری است.  یعنی از این دیاگرام به دیاگرام‌های دیگه می‌رسیم
که برای ورود به بخش طراحی معماری استفاده می‌شود.

use case diagram: کاربران

class  diagram: خیلی از ویژگی‌ها را کمتر دارد تجرید بالاتری دارد

object diagram: آبجکت تنها در دیتیل معنا دارد.

software eng vs requirement engineering:

مهندسی نرم‌افزار یک سری قوانین و دسیپلین مخصوص به خودش را دارد مانند 
pm
po
Application based and infrastructure based like docker and kuber
Generate standarads and methodologies
Impl

یکی از قوانین و دسیپلین مهندسی‌های نرم‌افزار مهندسی می‌باشد.

requirement eng vs requirement management

مدیریت نیازمندی: 

مهندسی کلمه‌ای است که داشتن یک فرایند را الزام آور می‌کند. یعنی تمام اقدامات را
بدانیم تمام نرم‌افزار‌ها و ابزار‌ها رو بشناسیم و بدون صحیح و خطا جلو بریم.

فرایند نیازمندی چهار مرحله می‌باشد.

جمع‌آوری نیازمندی‌ها
تمیز کردن و خوش رو کردن نیازمندی‌ها
مشخص کنیم که اون رو با چه زبانی بیان کنیم
صحت سنجی و اعتبارسنجی کنیم که کارمون درسته یا نه

مدیریت یعنی توزیع منابع. این منبع زمان و نیروی انسان به همراه پول است.

مدیر پروژه سهم بین هر بخش از توسعه را تقسیم می‌کنید.

وظیفه مدیر نیازمندی، مهم‌ترین وظیفه تقسیم وظایف به زیر عوامل است. بتواند منابع
اصلی را بین افراد و زیر بخش‌های خودش تقسیم کند. مفهوم چتری. یکی از فعالیت‌هاش
مهندسی نیازمندی است.

مدیریت نیازمندی شامل یکسری قوانین و توضیحات است که بیشتر به مدیریت پروژه ارتباط
دارد.

collaboration diagram

Sequence diagram: 

state diagram

component diagram: که بیشتر در سطح پایین‌تر در هنگام طراحی و توسعه برنامه
می‌رسیم.

سرفصل‌ها:

مبانی مهندسی نیازمند‌ها
درک دامنه مسئله و استخراج نیازمندی
ارزیابی نیازمندی
توصیف و مستندسازی و نیازمندی‌ها
تضمین کیفی نیازمندی
تکامل تیازمندی‌ها و ردیابی
مقصودگرایی در مهندسی نیازمندی‌ها
مدل‌سازی اهداف سیستم با استفاده از مدل‌های مقصود
تحلیل خطر با استفاده از مدل‌های مقصود
تحلیل خطر با استفاده از مدل‌های مقصود
مدل‌سازی نیازمندی‌ها با استفاده از نمودار‌های سناریوگرا
مدل‌سازی نیازمندی‌ها با استافده از نمودار‌های uml
مدل‌سازی عملیات سیستم
مدل‌سازی رفتار
وارسی و اعتبارسنجی نیازمندی‌ها
مدیریت نیازمند‌ها