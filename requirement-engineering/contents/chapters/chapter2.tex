فصل دوم، معادل فاز یک استخراج داده.

بستنی فروشی میشه سازمان

هر سطح پایینی میشه دامنه (ظرف‌ها)
و هر هر دامنه شامل اسکوپ

مجموعه مشکلات در اسکوپ‌ها. system as is

دو دسته برای جمع‌آوری داده داریم:

تکنیک‌های فراورده گرا یا artifact:

هر آنچیزی که در پروژه تولید یا استفاده می‌شود. از آرتفیکت‌هایی استفاده میکنمی که
نیاز بکشیم بیرون. قواعد آموزشی میشه ازش نیاز کشید بیرون. پروتوتایپ‌ها هم از آن
نیاز کشیده میشود بیرون.

آرتیفکت منبع نیازمندی است.

دسته دوم ذینفع گراk

آدم‌ها منبع نیازمندی‌ها هستند. مثل جلسات.

مستندات موجود را مطاله می‌کنیم. سند‌های موجود در سازمان را ارتیفکت می‌گویند.

مشکلات آرتیفکت:

background

حجم مستندات زیاد است.
جزيیات نامرتبط مثلا بخش بایگانی اسنادی را نگهداری میکند که ممکن است نامرتبنط
باشد.
اسناد ممکنه outdate باشد. یعنی یه جورایی با ذینفعان هم درگیر میشن.

هرس کردن مستندات:

بررسی بخش‌هایی که معتبر است و حذف بخش‌هایی که نامعتبر است. مثل خواندن درس در
محدوده فصل‌های مشخص شده تا تمام فصل‌ها.

روش دوم:

دیتا کالکشن: آرتیفکت اینجا دیتا هستش. این دیتا می‌تواند متادیتا باشد مانند فرم
ثبت نام. جمله ندارد بلکه براساس داده‌ها به جمله می‌رسند. مثلا در جمع آوری داده
در مورد مستندات نرم‌افزار مثلا ۵ تا نرم‌افزار نوشته مثلا تو طراحی ui جست و جوشون
براساس درختی بوده. یعنی سوابق طراحیشون رو دیده. براساس داده‌هایی که جمع کردیم به
استیت منت نان-فانکشن می‌نویسیم. نوشتن جمله و تفسیرش توسط مهندس نیازمندی براساس
داده‌های جمع‌اوری شده است.

یکی از مشکلاتش اینه که ممکنه درست نباشه این تفسیر مهندس نیازمندی. یعنی لزوماً
حرف آخر را ممکن است با تفسیر نا درست بیان کند.

یه زمانی دیتا جمع‌کردن خودش زمان بره و کار مشکلیه. 

یک نکته مثبت: 
در روش قبلی که سند قبلی میخوندیم سند‌های علمیاتی بودن ولی در این روش (دیتا
کالکشن) غیرعملیاتی است.

اینا تماماً روش‌‌های پایه است. 

requirement elicitation:
- Text mining

پرسشنامه آرتیفکت:

هم می‌سازیمش هم ازش استفاده و بهره برداری می‌شود.

انتها باز

انتها بسته مانند انتخاب‌های چک مارکی یا رادیو باکس. 

بعضی از سوالات کیفی است یعضی کمی. 
کیفی: (خوب، خیلی خوب)
کمی: درصدی

دلیل: تنوع زیاد کاربران و عدم تمرکز لوکیشن‌ها فرهنگ فناوری در کاربران متفاوته.
پس برای پوشش دایورسیتی از پرسش نامه استفاده میکنیم.

سریع، ارزان و می‌تواند از راه دور نیازمندی‌ بسیاری از افراد را جمع‌آوری کنیم. 

پرسشنامه باید روایی و کارایی داشته باشد.

روایی یعنی چی در پرسشنامه چیست؟

پرسشنامه اینترویو: از آدمی که میداند بپرس. سوالاتی که جواب نداریم را در اینترویو
می‌پرسیم. مثلا بعد از ثبت نام نوتیف بدیم؟ مثلا میگن نه فقط در تغییر کلاس نوتیف
بده.

تا آخر کوئسشنری. صفحه ۱۰.

موضوعات گروه‌های سه نفری و یک مقاله چدید از requirement engineering Springer. یا
حتی کنفرانسش.