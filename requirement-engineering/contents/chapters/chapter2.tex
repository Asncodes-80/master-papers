\section{فصل دوم، درک دامنه و جمع‌آوری نیازمندی‌ها}

این فصل معادل فاز (فرایند) اول مهندسی نیازمندی یعنی استخراج داده‌ها می‌باشد.
تمام مشکلاتی که در \lr{Scope} می‌باشد در حقیقت \lr{System as is} را مشخص می‌کند.

\subsection{دسته‌بندی جمع‌آوری داده}

جمع‌آوری داده‌ها را می‌توانیم به دو دسته زیر تقسیم کنیم، (درک دامنه و جمع‌آوری
داده‌ها ترکیبی از تکنیک‌های متفاوت می‌باشد):

\begin{enumerate}
    \item تکنیک‌های فرآورده‌گرا یا \lr{Artifact driven}: هر آن چیزی که در پروژه
    تولید یا استفاده می‌شود.
    \item \begin{enumerate}
        \item می‌توانیم از قواعد آموزشی نیازمندی‌هایی را خارج کنیم.
        \item \lr{Prototype}ها
        \item مستندات موجود در سازمان‌ها
    \end{enumerate}
    \item تکنیک‌های ذینفع‌گرا یا \lr{Stakeholder driven}: هر آن چیزی که در
    ارتباط با آدم‌ها در سازمان باشد.
    \item \begin{enumerate}
        \item جلسات
    \end{enumerate}
\end{enumerate}

\subsection{تکنیک‌های جمع‌آوری اطلاعات فرآورده‌گرا}

\subsubsection{\lr{Background study}}

\begin{itemize}
    \item سازمان: نمودار‌های سازمانی، بیزینس پلن‌ها، گزارش‌های مالی، صورتجلسه
    \footnote{\lr{Meeting minutes}}
    \item دامنه‌ها: کتاب‌ها، نظرسنجی‌ها، مقالات، مقررات و استاندارد‌ها،
    گزارش‌های سیستم‌های مشابه در دامنه مشابه
    \item سیستم کنونی یا \lr{System as is}: جریانات کاری مستند شده، فرایند‌ها،
    قوانین بیزینسی، مستندات مبادله شده، گزارش‌های مربوط به شکایات، مستندات مربوط
    به تغییر خواسته‌های مشتری و غیره.
\end{itemize}

یکی از نیازمندی‌های مهم برای ذینفعان می‌باشد تا آن‌ها را نسبت به جلسه بعدی‌شان
آماده کند.

مهم‌ترین مشکلات:

\begin{enumerate}
    \item حجم مستندات به شدت زیاد است
    \item جزئیات نامرتبط برای مثال بخش بایگانی اسنادی را نگهداری می‌کند که ممکن
    است کاملاً با یکدیگر نامرتبط باشد.
    \item اسناد ممکن است منسوخ شده یا \lr{Outdated} باشند.
\end{enumerate}

راه حل مشکلات این بخش:

استفاده از تکنیک هرس کردن مستندات می‌باشد. بررسی بخش‌هایی که معتبر است و حذف
بخش‌هایی که منسوخ شده و غیرمعتبر می‌باشد. این تکنیک مانند خواندن فصل‌های مشخص
شده از یک درس می‌باشد تا اینکه کل فصل‌های مطرح شده را بخواند.

\subsubsection{\lr{Data collection, questionnaires}}

جمع‌آوری داده‌هایی که مستندسازی نشده‌اند. مانند حقایق و ارقام. حقایق و ارقام به
صورت صریح در مستندات موجود نیستند. این داده‌ها می‌تواند به صورت \lr{Meta data}
با شد مانند فرم ثبت‌نام، جمله ندارد بلکه براساس داده‌ها می‌توان به یک جمله رسید.
بر اساس داده‌هایی که جمع‌آوری کرده‌ایم می‌توانیم جملات \lr{Functional} بنویسم.
نوشتن جمله و تفسیر توسط مهندس نیازمندی‌ها بر اساس داده‌های جمع‌آوری شده انجام
می‌پذیرد.

این داده‌ها مانند موارد زیر می‌باشد:

\begin{itemize}
    \item داده‌های مربوط به دیجیتال مارکتینگ، آمار استفاده، ارقام اجرایی و
    عملکردی، هزینه‌ها
    \item استفاده از تکنیک‌های نمونه‌گیری آماری
\end{itemize}

\subsubsection*{مشکلات}

\begin{itemize}
    \item ممکن است تفسیر مهندس نیازمندی لزوماً درست نباشد.
    \item داده کاوی مطمئن و درست ممکن است بسیار زمانبر باشد.
\end{itemize}

در روش قبل که اسنادی که می‌خواندیم اسناد عملیاتی بودند اما در این روش اسنادی که
مطالعه می‌شود کاملاً غیرعملیاتی هستند (مانند معیار‌ها و کیفیت ارائه سرویس).

\subsubsection*{روش‌های احتمالی}

\begin{itemize}
    \item \lr{requirement elicitation}
    \item \lr{Text mining}
\end{itemize}

\subsubsection*{پرسشنامه}

لیستی از سوالاتی که توسط ذینفعان مشخص شده را آماده می‌کنیم که هر کدام یک جواب
مناسب را می‌تواند در برگیرد.

نمونه‌ها میتواند:

\begin{itemize}
    \item انتخاب یک گزینه از چند گزینه. مانند استفاده از \lr{Radio button}
    \item سوالاتی که وزن‌دار هستند:
    \item \begin{itemize}
        \item کیفی: عالی، خوب، بد
        \item کمی: اعلام مقدار به صورت درصدی
    \end{itemize}
\end{itemize}

\subsubsection*{ویژگی‌های یک پرسشنامه خوب}

\begin{enumerate}
    \item تنوع زیاد کاربران و عدم تمرکز موقعیت مکانی و فرهنگ مختلف که در تمام
    کاربران متغیر می‌باشد. پس برای پوشش تنوع و گوناگونی
    \footnote{\lr{Diversity}} کاربران از پرسشنامه استفاده می‌کنیم.
    \item سریع، ارزان و قابل دسترس از راه دور نیازمندی بسیاری از کاربران را
    جمع‌آوری می‌کنیم.
    \item پرسشنامه‌ای خوب است که روایی و کارایی داشته باشد.
\end{enumerate}

\subsubsection*{تفاوت پایایی و روایی در پرسشنامه‌ها}

یک پرسشنامه خوب باید دو ویژگی پایایی و روایی را به همراه داشته باشد. 

\begin{itemize}
    \item پایایی قابلیت اطمینان پرسشنامه به همراه دقت در اندازه‌گیری می‌باشد.
    یعنی اگر همان پرسشنامه در همان شرایط بخواهد به صورت مجدد صورت گیرد، امتیاز
    یا مقدار حاصل از پرسشنامه هیچ تغییری نخواهد کرد.
    \item روایی به معنای آن است که میزان مطابقت نتایج بدست آمده از پرسشنامه با
    دنیای واقعی به چه اندازه‌ای می‌باشد.
\end{itemize}

\subsubsection{\lr{Repertory grids, Card sorts for concept acquisition}}

جمله مجموعه‌ای از اسم‌ها را با فعل به یکدیگر متصل می‌کند تا یک جمله کامل را
تشکیل دهد. برای مثال جمله «دانشجو باید بتواند درس انتخاب کند.» اسم‌ها به ترتیب،
«دانشجو» و «درس» هستند و فعل این جمله که این دو اسم را به یکدیگر متصل می‌کند
«انتخاب کردن» می‌باشد.

اسم‌ها تبدیل به کارت می‌شوند و تمام کارت‌ها معادل به کلاس هستند. تمام کلاس‌ها در
فضای مسئله بررسی می‌شوند و فضای راه‌حل در حقیقت خروجی ارتباط آنها (جمله) است.
یکی از مثال‌های فضای راه‌حل اتصال به دیتابیس می‌باشد.

\subsubsection*{فضای مسئله}

دقیقاً وضعیت موجود را نمایش می‌دهد. تمام چیز‌هایی که می‌بینیم در حقیقت فضای
مسئله می‌باشد.

\subsubsection*{فضای راه‌حل}

فضای راه‌حل نتیجه ارتباط جملات و کلاس‌ها هستند که طراح مشخص می‌کند.

\subsubsection*{مثال}

برای مثال می‌توان به دانشجو و شماره دانشجویی اشاره کرد. نام و نام خانوادگی،
تاریخ تولد، سال ورودی دانشگاه، رشته ورودی، گرایش رشته و غیره تمام مسائلی هستند
که موجودیت دانشجو را تعریف می‌کنند پس فضای مسئله می‌باشند.

طراح سیستم دانشگاهی با توجه به این فضای‌ مسئله ورودی‌ها را بررسی می‌کند و یک
خروجی برای مشخص کردن یکتا بودن دانشجو تولید می‌کند و آن هم شماره دانشجویی
می‌باشد که یکی از مهم‌ترین فرآورده‌های فضای راه‌حل است.

\subsubsection*{نکته}

کاملاً بستگی به نیاز سیستم دارد که مشخص کنیم یک اسم کلاس باشد یا نه. زیرا یک اسم
هم می‌تواند کلاس باشد یا می‌تواند به عنوان ویژگی کلاس دیگری یا \lr{Attribute}
باشد. برای مثال کتابخانه می‌توان اشاره کرد که اگر بخواهیم «کتاب‌ها» و
«نویسندگان» را کلاس جداگانه در نظر بگیریم می‌توانیم کوئری‌هایی در این بابت داشته
باشیم که یک کتاب را چه نویسندگانی تالیف کرده‌اند و یا یک نویسنده چه کتاب‌هایی
دارد. یا می‌توانیم نیاز سیستم را در این ببینیم که یکی از \lr{Attribute}های کتاب
نویسنده باشد به جای آن که یک کلاس جداگانه داشته باشد.

در حالت کلی می‌توان گفت که قانون سفت و سختی برای تشکیل کلاس از روی کارت‌ها وجود
ندارد و کاملاً نیاز سیستم مشخص می‌کند که کلاس باشند یا \lr{Attribute}.

\subsubsection*{گریدبندی کردن کارت‌‌ها}



% \subsubsection{\lr{Scenarios, Storyboards for problem world exploration}}

% \subsubsection{\lr{Prototypes, Mock-ups for early feedback}}

% \subsubsection{\lr{Knowledge reuse: Domain-independent, Domain specific}}

% \subsection{تکنیک‌های جمع‌آوری اطلاعات ذینفع‌گرا}

% \subsubsection{\lr{Interviews}}

% تمام سوالاتی که برای آن‌ها جوابی نداریم را در بخش \lr{Interview} می‌پرسیم. مثلا
% بعد از ثبت‌نام کاربر برای او اعلاناتی ارسال شود؟ یا مثلاً می‌گویم که در سناریو
% تغییر کلاس کاربر اعلانات ارسال شود.

% \subsubsection{\lr{Observation and ethnographic studies}}

% \subsubsection{\lr{Group sessions}}