\newpage

\section{فصل سوم}

در فصل یک و دو در مورد قدم اول  مهندسی نیازمندی یعنی جمع‌آوری اطلاعات صحبت شد.
در این فصل در مورد بررسی و ارزیابی داده‌های جمع‌آوری شده مرحله قبل صحبت خواهیم
کرد. نتیجه‌ای که این مرحله دارد آن است که همه اعضای تیم به یک اجماع و توافق بر
سر تمام مواردی که انتخاب شده است برسند.

\subsection{چهار کار اصلی ارزیابی داده‌های جمع‌آوری شده}

\begin{enumerate}
    \item \lr{Inconsistency management}: مدیریت ناسازگاری, نویسنده کتاب در شرایط
    خاصی که دو چیز با هم سازگاری ندارند را می‌گوید ناسازگار است و گاهی در برخی
    قسمت‌های کتاب از کلمه تضاد یا \lr{Conflict} استفاده کرده است. تضاد زمانی رخ
    می‌دهد که جملات با هم تضاد داشته باشد. \begin{enumerate}
        \item اگر بین جملات تناقص باشد به آن می‌گویند تضاد یا \lr{Conflict} که
        در نیازمندی‌های نرم‌افزاری، نیازمندی سیستم و \lr{Assumption}ها رخ
        می‌دهد.
        \item اگر تناقص بین المان‌ها باشد می‌گویند المان‌ها ناسازگاری دارند.
    \end{enumerate}
    \item \lr{Risk analysis}: بررسی ریسک‌ها \begin{enumerate}
        \item در حقیقت تمام اتفاقاتی را می‌گوید که ممکن است در برابر آنها
        کارهایی انجام بدهیم یا قابلیتی را طراحی کنیم که معمولاً محیطی،
        نرم‌افزاری و دامنه‌ای هستند.
        \item برای مثال: فراموشی گذرواژه یک بررسی ریسک بوده است، که پیامک شدن
        گذرواژه یا \lr{OTP} به صورت عامل محیطی یا \lr{Assumption} بوده، طراحی
        لاگین و فرم فراموشی گذرواژه از نوع نیازمندی نرم‌افزاری که در برابر ریسک
        تمهیداتی در نظر گرفته شده است.
        \item نکته مهم در ریسک‌ها آن است که هیچ وقت در زمان جمع‌آوری داده‌ها
        ریسک را بررسی نمی‌کنیم چون ممکن است ناخودآگاه برخی موارد را ریسک در نظر
        بگیریم و در جمع‌آوری آنها حساس شویم.
        \item می‌تواند در خصوص مجموعه اقداماتی باشد که در سیستم تکرار پذیر‌اند
        مانند تحلیل‌گر ریسک در سیستم‌های مخشص مانند سیستم‌های مالی
        \item ریسک \lr{Not} یک جمله می‌باشد.
        \item ریسک برای یک جمله می‌باشد، اما تضاد‌ها برای دو یا چند جمله می‌باشد
        (به تمرین \lr{p2.pdf} مراجعه شود).
    \end{enumerate}
    \item انتخاب بین گزینه‌ها: بعد از ریسک‌ها گزینه‌هایی که به نظرم مناسب بوده
    است که فیلتر کردیم را بایستی بین آنها یکی را انتخاب کنیم که در سیستم نهایی
    خود استفاده کنیم.
    \item اولویت‌بندی کردن کار‌ها: همه کار‌ها در یک سطح اهمیت نخواهند بود. پس
    نیازمند اولویت‌بندی کار‌های مشخص شده در مرحله قبل هستیم. یکی از بارزترین
    مثال‌ها نسخه‌بندی کردن کار‌ها می‌باشد.
\end{enumerate}

\subsection{ناسازگاری‌ها}

ناسازگاری بین المان‌های دانشی اتفاق می‌افتد که مرتبه تکرار بسیار زیادی در مهندسی
نیازمندی دارد. معمولا دو بُعد ناسازگاری وجود دارد:

\begin{itemize}
    \item \lr{Inter-viewpoint}: مربوط به \lr{NFR}ها نیست و معمولاً ذینفعان تمرکز
    و نگرانی‌های خودشان را دارند. برا مثال کارشناس دامنه در برابر بخش بازاریابی.
    \item \lr{Intra-viewpoint}: خواسته‌های مختلف کاربران که به صورت عملیاتی
    هستند. حلشان با استفاده از الگوریتم‌ها امکان‌پذیر می‌باشد.
\end{itemize}

ناسازگاری‌ها به ۳ دسته تقسیم می‌شوند تا قبل از طراحی توسط طراح سیستم همه با این
مفاهیم به اجماع برسند:

\subsubsection{تصادم معنایی یا \lr{Terminology clash}}

استفاده از چندین نام برای یک مفهوم مشترک را می‌گوید.

\begin{itemize}
    \item کسی در دانشگاه درس می‌دهد نام‌های مختلفی دارد: استاد، دکتر، مدرس
    \item کسی که کتاب را از کتابخانه قرض می‌گیرد: کاربر، قرض‌گیرنده، متشری یا
    \lr{Patron}
\end{itemize}

این تضاد معنایی به گونه‌ای است که هر معنا یک کلاس خاص خواهد بود که هیچ ربطی
ندارند تا به یکدگیر متصل شوند.

\subsubsection{تصادم در تعیین و طراحی یا \lr{Designation clash}}

استفاده از یک نام برای چند مفهوم مختلف را می‌گوید.
برای مثال: کسانی که در دانشگاه کار می‌کنند را کارمند می‌گویند. این کارمندان شامل،
آبدارچی، رییس دانشگاه، مدرسان و اعضای هیات علمی می‌باشد. دقیقاً در این رابطه
منظور از کارمندان کدام است. قواعد به طور کلی متفاوت هستند و تعاریف مختلف اسامی
مخصوص به خودشان را دارند.

یا مثالی دیگر در رابطه با کارمندان دانشگاه این است که دولت می‌خواد حقوق کارمندان
دانشگاه را افزایش دهد. الان چه قشری از دانشگاه قرار است حقوقشان افزایش پیدا کند؟
اساتید؟ اعضای هیات علمی؟ معاونین و رییس دانشگاه؟ دقیقاً کدام بخش قرار است اثر
بخشی این مسئله صورت گیرد؟

\subsubsection{تصادم ساختاری یا \lr{Structure clash}}

کلاسی به نام درس داریم که یک صفت به عنوان زمان دارد. در یک قسمت می‌گوییم که کلاس
آزمایشگاهی دو ساعت می‌باشد و در یک قسمت می‌گوییم که کلاس آزمایشگاهی بین ساعت ۱۰
تا ۱۲ ظهر می‌باشد. در دو زمان هستند اما نوع و ساختار متفاوتی دارند. از نظر منقط
دارند در مورد زمان صحبت می‌کنند ولی ساختارشان متفاوت است که باعت شکست در سیستم
خواهد شد.

تمام مشکلات ۳ مورد ناسازگاری را می‌تواند در فهرست واژگان یا \lr{Glossary} سند
نیازمندی‌ها \lr{RD} مطرح کرد تا همه بتوانند با تمام قواعد و معنای سیستم به صورت
اصولی آشنا شوند. در حقیقت مطرح کردن این واژگان وظیفه مهندس نیازمندی است و طراح
سیستم بایستی تمام این موارد را مطالعه کند و در کامل کردن مطالب نقش داشته باشد.
می‌تواند نوع کلاس‌های خود را تعیین کند. تایپی مشخص را برای سیستم تعریف کند و
غیره.

\subsubsection*{نکات}

\begin{itemize}
    \item نکته: منظور از \lr{Handle} کردن یعنی راست و ریس کردن ناسازگاری‌هایی که
    بعد از جمع‌آوری اطلاعات رخ داده است.
    \item سازمان‌های با تعریف \lr{Ontology} یا هستی شناسی، تفاوت بین المان‌‌های
    دانشی را مطرح می‌کنند.
    \item هستی شناسی ارتباط بین معنا‌ها با معنا‌های دیگر، که در نهایت موجب ایجاد
    نود و معنای جدید می‌شود که بسیار وابسته به دامنه است.
\end{itemize}

\subsection{تضاد‌ها}

تضاد ها به دو دسته تقسیم می‌شوند:

\subsubsection{تضاد قوی یا \lr{Strong conflict}}

در هیچ شرایطی نمی‌توانیم هر دو جمله را با هم در سبد نیازمندی خود نگهداریم. از
نظر منطقی امکان پذیر نمی‌باشد. برای مثال دو جمله زیر بیان می‌شود:

\begin{itemize}
    \item دانشجو بتواند کارنامه خود را ببیند.
    \item استاد در هنگام ثبت نمره بتواند کارنامه دانشجو را ببیند.
\end{itemize}

در دو جمله بالا اگر هر دو خواسته را بخواهیم برقرار کنیم حتماً به تضاد بر
می‌خوریم. در این شرایط طراح انتظار دارد که مهندس نیازمندی تکلیف کار او را روشن
کند که دقیقاً باید چه سیستمی طراحی کند و چه دسترسی‌هایی را بین هر دو کاربر
برقرار سازد. مثال بیشتر در تمرین دوم در فایل \lr{p2.pdf}.

در مثال کتابخانه سنتی، دو کاربر هیچ وقت نمی‌توانند یک کتاب با \lr{ISBN} و جلد
یکسان را از کتابخانه قرض بگیرند.

یا برای مثالی شفاف‌تر، از نظر تضاد‌ها می‌توانیم به شرایط \lr{NFR}ها اشاره کنیم.
هیچ وقت نمی‌توان بهترین امنیت را با بالاترین سرعت داشت، زیرا از نظر منطق
الگوریتم‌های امنیتی شرایط را پیچیده‌تر می‌کنند و خودآگاه باعث کاهش سرعت ورودی و
خروجی داده‌ها در سیستم خواهند شد.

\subsubsection{تضاد ضعیف یا \lr{Weak conflict}}

تضادی است که تا یک مرزی همه چیز خوب پیش می‌رود و هم سیستم راضی است هم کاربر، اما
بعد از آن شرایطی به وجود می‌آید که سیستم را متاثر می‌کند. مانند \lr{Deadline}ها،
تا زمانی که رخ نداده است هیچ مشکلی پیش نمی‌آید ولی به محض اینکه از زمانش می‌گذرد
در سیستم تضاد ایجاد می‌کند.

برای مثال کتابخانه سنتی می‌توان گفت وقتی مهلت تحویل کتاب توسط خواننده کتاب، ۳
هفته باشد، تا قبل از سه هفته اگر تحویلی انجام شود سیستم هیچ تضادی ندارد اما به
محض اینکه وارد هفته چهارم می‌شود و خواننده، کتاب را به کتابخانه تحویل نداده باشد
در سیستم کتابخانه تضاد ایجاد می‌کند.

\subsubsection*{راه‌حل این دو تضاد}

\begin{itemize}
    \item مهم‌ترین رویکرد مدیریت کردن است.
    \item برای بهتر کردن فرآیند‌ها در خصوص تضادها، استفاده از تکنیک‌های
    الکوریتمیک ضروری می‌باشد.
\end{itemize}

\subsubsection*{نکات}

\begin{itemize}
    \item رفع کردن تمام تضاد‌ها به صورت برد-برد امکان پذیر نیست.
    \item منظور از مدیریت کردن در مورد تضاد‌ها به معنای آن است که جملات را در
    کنار یکدیگر راضی نگه داریم.
\end{itemize}

\newpage

\subsection{مدیریت تضاد‌ها \lr{Managing conflicts}}

\begin{figure}[H]
    \centering
    \includegraphics[width=0.9\textwidth]{images/managing_conflicts.png}
    \caption{چهار قدم چرخشی مدیریت تضاد‌ها}
\end{figure}

مدیریت تضاد‌های ضعیف و قوی در ۴ قدم انجام می‌شود:

\begin{enumerate}
    \item \lr{Identify overlapping statements}: شناسایی عباراتی که با هم مشترک
    هستند و در مورد یک مفهوم مشترک صحبت می‌کنند. شباهت‌های می‌توانند فاعل، فعل و
    مفعول باشند. عملیاتی که در کنار یکدیگر دچار تضاد نمی‌شوند. \begin{enumerate}
        \item پدیده‌های باز و بسته شدن در‌های خطار مفاهیم رایج در سبد
        نیازمندی‌های آن است.
        \item پدیده‌های بدست آوردن کتاب، قرض گرفتن و بازگرداندن کتاب نیز از
        مفاهیم رایج مرتبط به \lr{Book copy} می‌باشد.
    \end{enumerate}
    \item \lr{Detect conflicts among them, document these}: از میان جملات
    جمع‌آوری شده باید بررسی کنیم که ببینیم چه نظراتی با هم همپوشانی دارند. اگر
    از میان همپوشانی‌ها تضادی پیدا شد بایستی تضاد‌ها را داکیومنت کنیم. راه‌های
    تشخیص تضاد: \begin{enumerate}
        \item \lr{Informally} یا غیررسمی: به صورت غیررسمی اعلام می‌کنیم که
        همپوشانی عبارات با هم رضایت بخش هستند و تحت چه شرایطی راضی کننده نیستند؟
        به گونه‌ای که به صورت چشمی منطقی نیستند.
        \item استفاده از روش‌های اکتشافی یا \lr{Heuristics} (استفاده از درخت):
        براساس یک جدول مشخص می‌کند که جملات چگونه می‌توانند با یکدیگر تضاد داشته
        باشند.
        \item استفاده از روش رسمی یا \lr{Formally}: تکنیک‌های اثبات قضیه. در
        حالت رسمی نرم‌افزار‌های بحرانی را نمی‌توان \lr{UML} کرد چرا که نیاز به
        اثبات دارند. نمایش و اعتبارسنجی هم با استفاده از زبان‌های رسمی امکان
        پذیر می‌باشد.
        \item استفاده از الگو‌های تضاد که نسخه‌ای سبک‌تر از تکنیک‌های رسمی
        هستند. نتیجه به صورت گرافیکال می‌باشد.
    \end{enumerate}
    \item \lr{Generate conflict resolutions}: رزولوشن یک مفهوم است که کتاب مرجع
    برای مدیریت تضاد از آن استفاده می‌کند. هر  راه‌حلی که به ذهن مهندس نیازمندی
    رسید باید کامل آن را بیان کند.
    \item \lr{Evaluate resolutions, select preferred}: باید یکی از راه‌حل‌هایی
    که در مرحله پیشین ارزیابی کردیم را بررسی کنیم و بهترین آنها را انتخاب کنیم.
\end{enumerate}

\subsubsection*{نکات}

\begin{itemize}
    \item همانطور که می‌دانیم \lr{Statements} سبد ما می‌باشد و راه‌حل‌هایی که
    بدست می‌آوریم باید از جنس سبد باشد. اولین راه‌حل \lr{Drop} کردن می‌باشد.
    همچنین از دیگر راه‌حل‌های تغییر جمله و سازگار کردن آن است، حتی ما می‌توانیم
    برای حل تضاد جمله به آن جمله‌ای مناسب را اضافه کنیم.
    \item مهندس نیازمندی باید در \lr{Intra-viewpoint}ها بازه را تعیین کند.
    \item برای رفع تضاد ممکن است جمله‌ای را حذف، اضافه یا حتی تغییر دهیم. در این
    حین ممکن است بازم ایجاد تضاد صورت گیرد به همین خاطر ۴ قدم مدیریت تضاد‌ها به
    صورت چرخشی می‌باشد.
    \item ما باید عواملی که با هم تضاد دارند را بشناسیم که بتوانیم آن‌ها را
    مدیریت کنیم.
    \item باید بدانیم که کدام موارد باعث ایجاد تضاد می‌شوند و بعد از تغییر سبد
    می‌توانند دردسرزا شوند.
    \item معمولاً \lr{Conflict}خیز‌ها معمولاً در \lr{Overlap}های زیادی شرکت
    می‌کنند.
    \item جملاتی را باید استفاده کنیم که در \lr{Overlap}های زیادی شرکت داشتن و
    شرکتشان خوب و بدون تضاد بوده.
\end{itemize}

\subsection{تکنیک‌های داکیومنت کردن}

\begin{figure}[H]
    \centering
    \includegraphics[width=0.9\textwidth]{images/systematic_process_table.png}
    \caption{تشخیص تضاد‌ها و همپوشانی‌ها}
\end{figure}

\subsubsection*{نکته}

\begin{itemize}
    \item باقی مانده نشان‌دهنده تضادهای بین دو جمله می‌باشد.
    \item خارج قسمت نشان‌دهنده همپوشانی مناسب و بدون تضاد است.
\end{itemize}

\begin{equation}
    Conflicts(S1) = remainderOf(1002/1000) \rightarrow 2
\end{equation}

\begin{equation}
    nonConflictingOverlaps(S1) = quotientOf(1002/1000) \rightarrow 1
\end{equation}

\begin{equation}
    Conflicts(Total) = remainderOf(2006/1000) \rightarrow 6
\end{equation}

\begin{equation}
    nonConflictOverlaps(Total) = quotientOf(2006/1000) \rightarrow 2
\end{equation}


