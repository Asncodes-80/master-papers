\newpage

\section{فصل سوم}

در فصل یک و دو در مورد قدم اول  مهندسی نیازمندی یعنی جمع‌آوری اطلاعات صحبت شد.
در این فصل در مورد بررسی و ارزیابی داده‌های جمع‌آوری شده مرحله قبل صحبت خواهیم
کرد. نتیجه‌ای که این مرحله دارد آن است که همه اعضای تیم به یک اجماع و توافق بر
سر تمام مواردی که انتخاب شده است برسند.

\subsection{چهار کار اصلی ارزیابی داده‌های جمع‌آوری شده}

\begin{enumerate}
    \item \lr{Inconsistency management}: مدیریت ناسازگاری, نویسنده کتاب در شرایط
    خاصی که دو چیز با هم سازگاری ندارند را می‌گوید ناسازگار است و گاهی در برخی
    قسمت‌های کتاب از کلمه تضاد یا \lr{Conflict} استفاده کرده است. تضاد زمانی رخ
    می‌دهد که جملات با هم تضاد داشته باشد. \begin{enumerate}
        \item اگر بین جملات تناقص باشد به آن می‌گویند تضاد یا \lr{Conflict} که
        در نیازمندی‌های نرم‌افزاری، نیازمندی سیستم و \lr{Assumption}ها رخ
        می‌دهد.
        \item اگر تناقص بین المان‌ها باشد می‌گویند المان‌ها ناسازگاری دارند.
    \end{enumerate}
    \item \lr{Risk analysis}: بررسی ریسک‌ها \begin{enumerate}
        \item در حقیقت تمام اتفاقاتی را می‌گوید که ممکن است در برابر آنها
        کارهایی انجام بدهیم یا قابلیتی را طراحی کنیم که معمولاً محیطی،
        نرم‌افزاری و دامنه‌ای هستند.
        \item برای مثال: فراموشی گذرواژه یک بررسی ریسک بوده است، که پیامک شدن
        گذرواژه یا \lr{OTP} به صورت عامل محیطی یا \lr{Assumption} بوده، طراحی
        لاگین و فرم فراموشی گذرواژه از نوع نیازمندی نرم‌افزاری که در برابر ریسک
        تمهیداتی در نظر گرفته شده است.
        \item نکته مهم در ریسک‌ها آن است که هیچ وقت در زمان جمع‌آوری داده‌ها
        ریسک را بررسی نمی‌کنیم چون ممکن است ناخودآگاه برخی موارد را ریسک در نظر
        بگیریم و در جمع‌آوری آنها حساس شویم.
        \item می‌تواند در خصوص مجموعه اقداماتی باشد که در سیستم تکرار پذیر‌اند
        مانند تحلیل‌گر ریسک در سیستم‌های مخشص مانند سیستم‌های مالی
        \item ریسک \lr{Not} یک جمله می‌باشد.
        \item ریسک برای یک جمله می‌باشد، اما تضاد‌ها برای دو یا چند جمله می‌باشد
        (به تمرین \lr{p2.pdf} مراجعه شود).
    \end{enumerate}
    \item انتخاب بین گزینه‌ها: بعد از ریسک‌ها گزینه‌هایی که به نظرم مناسب بوده
    است که فیلتر کردیم را بایستی بین آنها یکی را انتخاب کنیم که در سیستم نهایی
    خود استفاده کنیم.
    \item اولویت‌بندی کردن کار‌ها: همه کار‌ها در یک سطح اهمیت نخواهند بود. پس
    نیازمند اولویت‌بندی کار‌های مشخص شده در مرحله قبل هستیم. یکی از بارزترین
    مثال‌ها نسخه‌بندی کردن کار‌ها می‌باشد.
\end{enumerate}

\subsection{ناسازگاری‌ها}

ناسازگاری بین المان‌های دانشی اتفاق می‌افتد که مرتبه تکرار بسیار زیادی در مهندسی
نیازمندی دارد. معمولا دو بُعد ناسازگاری وجود دارد:

\begin{itemize}
    \item \lr{Inter-viewpoint}: مربوط به \lr{NFR}ها نیست و معمولاً ذینفعان تمرکز
    و نگرانی‌های خودشان را دارند. برا مثال کارشناس دامنه در برابر بخش بازاریابی.
    \item \lr{Intra-viewpoint}: خواسته‌های مختلف کاربران که به صورت عملیاتی
    هستند. حلشان با استفاده از الگوریتم‌ها امکان‌پذیر می‌باشد.
\end{itemize}

ناسازگاری‌ها به ۳ دسته تقسیم می‌شوند تا قبل از طراحی توسط طراح سیستم همه با این
مفاهیم به اجماع برسند:

\subsubsection{تصادم معنایی یا \lr{Terminology clash}}

استفاده از چندین نام برای یک مفهوم مشترک را می‌گوید.

\begin{itemize}
    \item کسی در دانشگاه درس می‌دهد نام‌های مختلفی دارد: استاد، دکتر، مدرس
    \item کسی که کتاب را از کتابخانه قرض می‌گیرد: کاربر، قرض‌گیرنده، مشتری یا
    \lr{Patron}
\end{itemize}

این تضاد معنایی به گونه‌ای است که هر معنا یک کلاس خاص خواهد بود که هیچ ربطی
ندارند تا به یکدگیر متصل شوند.

\subsubsection{تصادم در تعیین و طراحی یا \lr{Designation clash}}

استفاده از یک نام برای چند مفهوم مختلف را می‌گوید.
برای مثال: کسانی که در دانشگاه کار می‌کنند را کارمند می‌گویند. این کارمندان شامل،
آبدارچی، رییس دانشگاه، مدرسان و اعضای هیات علمی می‌باشد. دقیقاً در این رابطه
منظور از کارمندان کدام است. قواعد به طور کلی متفاوت هستند و تعاریف مختلف اسامی
مخصوص به خودشان را دارند.

یا مثالی دیگر در رابطه با کارمندان دانشگاه این است که دولت می‌خواد حقوق کارمندان
دانشگاه را افزایش دهد. الان چه قشری از دانشگاه قرار است حقوقشان افزایش پیدا کند؟
اساتید؟ اعضای هیات علمی؟ معاونین و رییس دانشگاه؟ دقیقاً کدام بخش قرار است اثر
بخشی این مسئله صورت گیرد؟

\subsubsection{تصادم ساختاری یا \lr{Structure clash}}

کلاسی به نام درس داریم که یک صفت به عنوان زمان دارد. در یک قسمت می‌گوییم که کلاس
آزمایشگاهی دو ساعت می‌باشد و در یک قسمت می‌گوییم که کلاس آزمایشگاهی بین ساعت ۱۰
تا ۱۲ ظهر می‌باشد. در دو زمان هستند اما نوع و ساختار متفاوتی دارند. از نظر منقط
دارند در مورد زمان صحبت می‌کنند ولی ساختارشان متفاوت است که باعت شکست در سیستم
خواهد شد.

تمام مشکلات ۳ مورد ناسازگاری را می‌تواند در فهرست واژگان یا \lr{Glossary} سند
نیازمندی‌ها \lr{RD} مطرح کرد تا همه بتوانند با تمام قواعد و معنای سیستم به صورت
اصولی آشنا شوند. در حقیقت مطرح کردن این واژگان وظیفه مهندس نیازمندی است و طراح
سیستم بایستی تمام این موارد را مطالعه کند و در کامل کردن مطالب نقش داشته باشد.
می‌تواند نوع کلاس‌های خود را تعیین کند. تایپی مشخص را برای سیستم تعریف کند و
غیره.

\subsubsection*{نکات}

\begin{itemize}
    \item نکته: منظور از \lr{Handle} کردن یعنی راست و ریس کردن ناسازگاری‌هایی که
    بعد از جمع‌آوری اطلاعات رخ داده است.
    \item سازمان‌های با تعریف \lr{Ontology} یا هستی شناسی، تفاوت بین المان‌‌های
    دانشی را مطرح می‌کنند.
    \item هستی شناسی ارتباط بین معنا‌ها با معنا‌های دیگر، که در نهایت موجب ایجاد
    نود و معنای جدید می‌شود که بسیار وابسته به دامنه است.
\end{itemize}

\subsection{تضاد‌ها}

تضاد ها به دو دسته تقسیم می‌شوند:

\subsubsection{تضاد قوی یا \lr{Strong conflict}}

در هیچ شرایطی نمی‌توانیم هر دو جمله را با هم در سبد نیازمندی خود نگهداریم. از
نظر منطقی امکان پذیر نمی‌باشد. برای مثال دو جمله زیر بیان می‌شود:

\begin{itemize}
    \item دانشجو بتواند کارنامه خود را ببیند.
    \item استاد در هنگام ثبت نمره بتواند کارنامه دانشجو را ببیند.
\end{itemize}

در دو جمله بالا اگر هر دو خواسته را بخواهیم برقرار کنیم حتماً به تضاد بر
می‌خوریم. در این شرایط طراح انتظار دارد که مهندس نیازمندی تکلیف کار او را روشن
کند که دقیقاً باید چه سیستمی طراحی کند و چه دسترسی‌هایی را بین هر دو کاربر
برقرار سازد. مثال بیشتر در تمرین دوم در فایل \lr{p2.pdf}.

در مثال کتابخانه سنتی، دو کاربر هیچ وقت نمی‌توانند یک کتاب با \lr{ISBN} و جلد
یکسان را از کتابخانه قرض بگیرند.

یا برای مثالی شفاف‌تر، از نظر تضاد‌ها می‌توانیم به شرایط \lr{NFR}ها اشاره کنیم.
هیچ وقت نمی‌توان بهترین امنیت را با بالاترین سرعت داشت، زیرا از نظر منطق
الگوریتم‌های امنیتی شرایط را پیچیده‌تر می‌کنند و خودآگاه باعث کاهش سرعت ورودی و
خروجی داده‌ها در سیستم خواهند شد.

\subsubsection{تضاد ضعیف یا \lr{Weak conflict}}

تضادی است که تا یک مرزی همه چیز خوب پیش می‌رود و هم سیستم راضی است هم کاربر، اما
بعد از آن شرایطی به وجود می‌آید که سیستم را متاثر می‌کند. مانند \lr{Deadline}ها،
تا زمانی که رخ نداده است هیچ مشکلی پیش نمی‌آید ولی به محض اینکه از زمانش می‌گذرد
در سیستم تضاد ایجاد می‌کند.

برای مثال کتابخانه سنتی می‌توان گفت وقتی مهلت تحویل کتاب توسط خواننده کتاب، ۳
هفته باشد، تا قبل از سه هفته اگر تحویلی انجام شود سیستم هیچ تضادی ندارد اما به
محض اینکه وارد هفته چهارم می‌شود و خواننده، کتاب را به کتابخانه تحویل نداده باشد
در سیستم کتابخانه تضاد ایجاد می‌کند.

\subsubsection*{راه‌حل این دو تضاد}

\begin{itemize}
    \item مهم‌ترین رویکرد مدیریت کردن است.
    \item برای بهتر کردن فرآیند‌ها در خصوص تضادها، استفاده از تکنیک‌های
    الکوریتمیک ضروری می‌باشد.
\end{itemize}

\subsubsection*{نکات}

\begin{itemize}
    \item رفع کردن تمام تضاد‌ها به صورت برد-برد امکان پذیر نیست.
    \item منظور از مدیریت کردن در مورد تضاد‌ها به معنای آن است که جملات را در
    کنار یکدیگر راضی نگه داریم.
\end{itemize}

\newpage

\subsection{مدیریت تضاد‌ها \lr{Managing conflicts}}

\begin{figure}[H]
    \centering
    \includegraphics[width=0.9\textwidth]{images/managing_conflicts.png}
    \caption{چهار قدم چرخشی مدیریت تضاد‌ها}
\end{figure}

مدیریت تضاد‌های ضعیف و قوی در ۴ قدم انجام می‌شود:

\begin{enumerate}
    \item \lr{Identify overlapping statements}: شناسایی عباراتی که با هم مشترک
    هستند و در مورد یک مفهوم مشترک صحبت می‌کنند. شباهت‌ها می‌توانند فاعل، فعل و
    مفعول باشند. عملیاتی که در کنار یکدیگر دچار تضاد نمی‌شوند. \begin{enumerate}
        \item پدیده‌های باز و بسته شدن در‌های قطار مفاهیم رایج در سبد
        نیازمندی‌های آن است.
        \item پدیده‌های بدست آوردن کتاب، قرض گرفتن و بازگرداندن کتاب نیز از
        مفاهیم رایج مرتبط به \lr{Book copy} می‌باشد.
    \end{enumerate}
    \item \lr{Detect conflicts among them, document these}: از میان جملات
    جمع‌آوری شده باید بررسی کنیم که ببینیم چه نظراتی با هم همپوشانی دارند. اگر
    از میان همپوشانی‌ها تضادی پیدا شد بایستی تضاد‌ها را داکیومنت کنیم. راه‌های
    تشخیص تضاد: \begin{enumerate}
        \item \lr{Informally} یا غیررسمی: به صورت غیررسمی اعلام می‌کنیم که
        همپوشانی عبارات با هم رضایت بخش هستند و تحت چه شرایطی راضی کننده نیستند؟
        به گونه‌ای که به صورت چشمی منطقی نیستند.
        \item استفاده از روش‌های اکتشافی یا \lr{Heuristics} (استفاده از درخت):
        براساس یک جدول مشخص می‌کند که جملات چگونه می‌توانند با یکدیگر تضاد داشته
        باشند.
        \item استفاده از روش رسمی یا \lr{Formally}: تکنیک‌های اثبات قضیه. در
        حالت رسمی نرم‌افزار‌های بحرانی را نمی‌توان \lr{UML} کرد چرا که نیاز به
        اثبات دارند. نمایش و اعتبارسنجی هم با استفاده از زبان‌های رسمی امکان
        پذیر می‌باشد.
        \item استفاده از الگو‌های تضاد که نسخه‌ای سبک‌تر از تکنیک‌های رسمی
        هستند. نتیجه به صورت گرافیکال می‌باشد.
    \end{enumerate}
    \item \lr{Generate conflict resolutions}: رزولوشن یک مفهوم است که کتاب مرجع
    برای مدیریت تضاد از آن استفاده می‌کند. هر  راه‌حلی که به ذهن مهندس نیازمندی
    رسید باید کامل آن را بیان کند.
    \item \lr{Evaluate resolutions, select preferred}: باید یکی از راه‌حل‌هایی
    که در مرحله پیشین ارزیابی کردیم را بررسی کنیم و بهترین آنها را انتخاب کنیم.
\end{enumerate}

\subsubsection*{نکات}

\begin{itemize}
    \item همانطور که می‌دانیم \lr{Statements} سبد ما می‌باشد و راه‌حل‌هایی که
    بدست می‌آوریم باید از جنس سبد باشد. اولین راه‌حل \lr{Drop} کردن می‌باشد.
    همچنین از دیگر راه‌حل‌های تغییر جمله و سازگار کردن آن است، حتی ما می‌توانیم
    برای حل تضاد جمله به آن جمله‌ای مناسب را اضافه کنیم.
    \item مهندس نیازمندی باید در \lr{Intra-viewpoint}ها بازه را تعیین کند.
    \item برای رفع تضاد ممکن است جمله‌ای را حذف، اضافه یا حتی تغییر دهیم. در این
    حین ممکن است بازم ایجاد تضاد صورت گیرد به همین خاطر ۴ قدم مدیریت تضاد‌ها به
    صورت چرخشی می‌باشد.
    \item ما باید عواملی که با هم تضاد دارند را بشناسیم که بتوانیم آن‌ها را
    مدیریت کنیم.
    \item باید بدانیم که کدام موارد باعث ایجاد تضاد می‌شوند و بعد از تغییر سبد
    می‌توانند دردسرزا شوند.
    \item معمولاً \lr{Conflict}خیز‌ها در \lr{Overlap} های زیادی شرکت می‌کنند.
    \item جملاتی را باید استفاده کنیم که در \lr{Overlap}های زیادی شرکت داشتن و
    شرکتشان خوب و بدون تضاد بوده.
\end{itemize}

\subsection{تکنیک‌های داکیومنت کردن}

\begin{figure}[H]
    \centering
    \includegraphics[width=0.9\textwidth]{images/systematic_process_table.png}
    \caption{تشخیص تضاد‌ها و همپوشانی‌ها}
\end{figure}

\subsubsection*{نکته}

\begin{itemize}
    \item باقی مانده نشان‌دهنده تضادهای بین دو جمله می‌باشد.
    \item خارج قسمت نشان‌دهنده همپوشانی مناسب و بدون تضاد است.
\end{itemize}

\begin{equation}
    Conflicts(S1) = remainderOf(1002/1000) \rightarrow 2
\end{equation}

\begin{equation}
    nonConflictingOverlaps(S1) = quotientOf(1002/1000) \rightarrow 1
\end{equation}

\begin{equation}
    Conflicts(Total) = remainderOf(2006/1000) \rightarrow 6
\end{equation}

\begin{equation}
    nonConflictOverlaps(Total) = quotientOf(2006/1000) \rightarrow 2
\end{equation}

\subsection{تکنیک‌های رفع تضاد}

برای رفع تضاد ۴ روش مطرح شده است که هر کدام از آن‌ها می‌توانند منجر به تولید
نیازمند‌های جدید شوند تا تضاد موجود در جمله را رفع کنند:

\subsubsection{خاص‌سازی منبع یا هدف تضاد}

تضاد در سطح جمله رخ می‌دهد، یعنی یک قانون به کل وارد می‌شود که یکسری جزئیات
دارد. نقض قانون بالایی به جز وارد می‌شود. هر کدام از جزئیات قوانین خودشان را
دارند و چون جز هم قانون خودش را دارد باعث ایجاد تضاد می‌شود. قانون جدید (جمله
جدید) در مورد کل سیستم نبوده و بلکه در مورد یک جز خاص می‌باشد. به جای اعمال
قانون به کل سیستم بایستی به یک نود و قشر مشخص این قانون جدید اعمال شود. معمولاً
در فاعل و مفعول رخ می‌دهد.

برای مثال:

\begin{itemize}
    \item به کاربران (\lr{Users}) اجازه داده شود که بتوانند از وضعیت کتابی که به
    امانت گرفته شده است مطلع شوند.
    \item دانشجویان نباید از وضعیت کتاب امانت گرفته شده مطلع باشند.
\end{itemize}

خاص‌سازی باید روی منبع یا رابطه کل به جز اعمال شود. در مثال بالا مشخص نیست که
کاربران دقیقاً چه قشری هستند و آیا شامل قشر دانشجویان می‌شود؟ پس بایستی قانونی
تعریف کنیم که مشخص شود چه گروهی قادر به مطلع شدن وضعیت باشند و چه گروهی
نمی‌توانند. به همین دلیل مجوز‌هایی برای \lr{Staff users} صادر می‌کنیم و مجوز
دیگری به نام \lr{Students}. در این دو گروه به روشنی می‌توان قابلیت‌هایشان را
شخصی‌سازی نمود. گروه خاص ما فاعل بوده است. چه کسانی بتوانند و چه کسانی نتوانند؟

\subsubsection{ضعیف‌تر کردن جملاتی که تضاد دارند}

در این روش معمولاً جمله سخت‌تر را ضعیف (\lr{Weak}) می‌کنیم. دقیقاً جزئی که قانون
را می‌بندد.

برای مثال:

\begin{itemize}
    \item پدر می‌گوید ساعت ۱۰ شب خانه باش اما مادر شما می‌گوید که هر چقدر بودی
    مشکلی نداره، در این جمله تضاد قوی را مشاهده خواهیم کرد.
    \item قرض گیرنده کتاب، باید کپی کتاب را سر مهلت سه هفته‌ای تحویل دهد مگر
    اینکه یک مجوز (\lr{Permision}) برای استفاده بیشتر کپی کتاب برای دانشجو صاد
    شود.
    \item مثال دقیق‌تر: دانشجو‌ها می‌توانند تا سه هفته کپی کتاب را از کتابخانه
    قرض بگیرند اما در صورتی که عضو انجمن علمی دانشگاه باشند می‌تواند ۵ هفته کتاب
    را داشته باشند.
\end{itemize}

\subsubsection{ری‌استور کردن}

در این روش تا زمانی که به تضاد بر نخورده‌ایم پیش می‌رویم و بعد از برخورد به تضاد
سیستم را به حالت قبل از تضاد خواهیم برد.

برای مثال دانشجو می‌خواهد کتاب را بیشتر از ۳ هفته قرض بگیرد، اما کتابخانه تنها ۳
هفته امکان قرض گرفتن را برای دانشجو فراهم کرده است. برای حل این تضاد کتابخانه از
ری‌استور کردن استفاده می‌کند و می‌گوید برای قرض گرفتن بیشتر از ۳ هفته، سر موعد
مهلت قرض گرفتن کتاب را تمدید کن.

\subsubsection{پرهیز از شرایط مرزی}

آخرین راه‌حل که سخت‌تر از بقیه می‌باشد این روش است که در مورد تضاد‌های ضعیف یا
(\lr{Weak conflict}) صادق خواهد بود. در این روش تلاش بر این است که شرایط مرزی را
برای تضاد‌ها به گونه‌ای کنترل کنیم که هیچ وقت رخ ندهند تا هدف سیستم را از بین
برود.

برای مثال، فرض کنید کتابخانه از یک کتاب مخصوص، تنها سه کپی دارد. اگر هر کدام از
این سه کپی را سه دانشجو مختلف قرض بگیرد، دانشجوی چهارم نمی‌تواند این کتاب را
درخواست کند. یعنی ریشه‌یابی یکسری کتاب که مرجع آن مشخص است که دیگر هیچ کپی از آن
در کتابخانه موجود نیست به این صورت رسماً رسالت کتابخانه زیر سوال رفته است. سوالی
که می‌تواند مطرح شود این است که آیا این نگرانی برای همه کتاب‌ها وجود دارد؟ باید
این مسئله بررسی گردد که برای چه کتاب‌هایی نیاز داریم شرط جدیدی را وضع کنیم. برای
کامل کردن مثال، فرض کنید از آن کتابی که در ابتدای فرضمان سه کپی داشتیم تنها دو
کپی قابل قرض دادن به دانشجو باشد و کپی آخر کتاب تنها زمانی قابل استفاده است که
خواننده کتاب درون کتابخانه باشد و نخواهد آن را به بیرون از کتابخانه ببرد.

برای مثال بالا ممکن است به دنبال الگوریتم‌های دسته‌بندی برویم که بتوانیم رضایت
را برای همه طرفین برقرار کنیم تا همه بتوانند از تمام کپی‌ها به صورت مناسب
استفاده کنند.

در شرایط مثال بالا سیستم کتابخانه بسیار اساسی‌تر خواهد شد و باید در صفات کلاس
مربوطه و نیازمندی‌ها خود اعلام کنیم که چند کپی از کتاب‌ها قابل قرض و چند کپی
قابل استفاده در محل کتابخانه می‌باشد.

\subsection{تمرین اول}

در یک سیستم مانند اسنپ، مسافر می‌خواهد نزدیک‌ترین ماشین به او تخصیص داده شود،
مدیر سیستم می‌خواهد در راستای طرح تشویقی خود رانندگانی با امتیاز بالاتر را به
مشتری تخصیص دهد. آیا تضادی می‌بینید؟ اگر بله از چه نوعی است و راه‌حل آن چیست؟

بله تضاد دارند، دو جمله وجود دارد:

\begin{itemize}
    \item کاربر به دنبال نزدیک‌ترین راننده اسنپ می‌باشد
    \item مدیر می‌خواهد راننده‌ای انتخاب شود که بالاترین امتیاز را داشته باشد.
    \item تضاد در جایی رخ می‌دهد که ممکن است راننده‌ای با امتیاز بالا در شعاع
    دورتری قرار داشته باشد.
    \item در این سناریو مشکلی برای راننده، مدیر و کاربر پیش نمی‌آید. پس تضاد
    ضعیف است. ما می‌توانیم با رویکرد \lr{Restore} کردن این تضاد را به گونه‌ای
    پوشش دهیم که همه راضی باشند.
    \item لزوماً امتیاز راننده می‌تواند ۵ ستاره اولین شعاع نزدیک به کاربر نباشد
    براساس درخواست کاربر مشخص می‌شود که کدام راننده با امتیاز بالا بایستی فیلتر
    شود و از بین آنها کدام راننده می‌خواهد درخواست کاربر را بپذیرد. این بدان
    معناست که درخواست کاربر برای رانندگانی که در همان شعاع هستند که امتیاز آن‌ها
    کم باشد، ارسال نمی‌شود.
\end{itemize}

\subsection{مدیریت ریسک}

معمولاً در فاز‌های اولیه یک پروژه نرم‌افزاری، مهندسان نیازمندی و ذینفعان
انتظارات عجیبی را دارند:

\begin{itemize}
    \item محیط و نرم‌افزار همانگونه که انتظار دارند رفتار کند.
    \item برنامه توسعه نرم‌افزاری پروژه همانگونه که برنامه‌ریزی شده است رو به جلو
    باشد.
\end{itemize}

اما در حقیقت جا به جایی از \lr{System-as-is} به \lr{Sysmte-to-be} ممکن است دچار
ریسک‌های مختلفی شود.

\subsubsection{شدت ریسک یا \lr{Severity}}

درجه از دست دادن رضایت نسبت به یک هدف را شدت ریسک یا \lr{Severity} می‌گویند.

ریسک‌ها نقض (\lr{Not}) یک نیاز می‌باشند. وقتی یک نیاز به درستی انجام نشود یا به
هر دلیلی دیر انجام شود و نقض یک جمله باشد می‌توان گفت که به ریسک تبدیل شده است.
نکته حائز اهمیت آن است که ریسک روی یک جمله می‌باشد و روی دو جمله مانند تضاد‌ها
تاثیر ندارد.

ریسک‌ها به دو دسته تقسیم می‌شوند:

\begin{enumerate}
    \item مرتبط با محصول یا \lr{Product-related}
    \item مرتبط با فرایند یا \lr{Process-related}
\end{enumerate}

\subsubsection{مرتبط با محصول یا \lr{Product-related}}

بیشترین ارتباط را به مهندس نیازمندی دارد. الزاماتی که در طراحی یک سیستم بایستی
در نظر گرفته شود تا بتوانیم سیستم را در برابر آن‌ها تجهیز کنیم؛ لذا افرادی در
حوزه مدیریت ریسک کار می‌کنند که متخصص آن دامنه و سیستم هستند. یعنی کاملاً در
مورد دامنه تجربه و اطلاعات مناسب را دارا هستند. این افراد معمولاً جایگاه‌های
ثابتی در دامنه خود داشتند. مانند سیستم‌های حسابداری بانکی، سیستم‌های \lr{CRM} و
غیره. تمام حالات سیستم را دیده‌اند و در استراتژی‌های مختلف در برابر ریسک‌های
مرتبط را تجربه کرده‌اند. به عبارتی ساده‌تر یعنی به طور کل این افراد شناخت بسیار
کاملی نسبت به آن دامنه دارند.

این ریسک‌ها می‌توانند مانند مثال‌های زیر باشند:

\begin{itemize}
    \item ریسک در برابر ارسال و دریافت اطلاعات داخل برنامه‌ای؛ پیامی که ارسال
    می‌شود و قرار است به یک نفر برسد ریسک موارد زیر را دارد: \begin{itemize}
        \item پیام برای آن شخص مشخص ارسال نشود و به تمام کاربران داخل شبکه بدون
        اجازه ارسال شود. (\lr{Broadcastly send})
        \item پیام با تاخیر در شبکه ارسال شود و به دست دریافت کننده پیام برسد.
        \item شبکه شنود شود و محتوای پیام را بتوان به صورت غیرقانونی در شبکه
        مشاهده نمود.
        \item پیام قابلیت ویرایش پس از ارسال را داشته باشد.
    \end{itemize}
    \item سیستم حاوی احراز هویت می‌باشد و ریسک آن: \begin{itemize}
        \item اگر کاربر گذرواژه خود را فراموش کرده باشد؟ پس بایستی استراتژی
        مناسب در برابر این ریسک را در نظر بگیریم و برای این سیستم احراز هویت
        گزینه فراموشی گذرواژه را طراحی کنیم.
    \end{itemize}
\end{itemize}

\subsubsection{مرتبط با فرایند یا \lr{Process-related}}

تمام اتفاقاتی که در ارتباط مستقیم با محصول نمی‌باشد را شامل می‌شود. برای مثال
ممکن است ارزش پولمان کم‌تر شود یا یکی از اعضا/پرسنل‌مان استعفا دهد.اینگونه
ریسک‌ها مرتبط با مدیر پروژه می‌باشد.

\subsection{چرخه مدیریت ریسک}

\begin{itemize}
    \item فرایند پیدا کردن ریسک در پروژه‌های نرم‌افزاری یک فرایند تکرارپذیر
    می‌باشد که شامل سه مرحله‌ زیر می‌باشد: \begin{enumerate}
    \item \lr{Risk identification}: شناسایی ریسک: دقیقاً ریسکی در سیستم به صورت
    مشخص رخ می‌دهد؟ یا اتفاق افتاده است؟
    \item \lr{Risk assessment}: ارزیابی ریسک: آیا عواقب احتمالی بدی دارد؟
    می‌توان از آن جلوگیری کرد؟ آیا می‌توان تاثیرات رخدادش را مدیریت و کنترل کرد؟
    \item \lr{Risk control}: کنترل ریسک: مدیریت و کنترل ریسک به عنوان نیازمندی
    جدید
\end{enumerate}
    \item در این میان نکته بسیار مهم آن است که در چرخه تکرار بررسی ریسک ممکن است
    هر عملیات و اقداماتی منجر به ایجاد ریسک جدیدی شود.
    \item مدیریت ضعیف ریسک‌ها عامل اصلی شکست در پروژه‌های نرم‌افزاری می‌باشد.
    \begin{enumerate}
        \item اشتباه فکر کردن به جریانات پروژه که انگار قرار نیست هیچ فرایندی
        مشکل داشته باشد.
        \item عدم شناسایی و دستکم گرفتن ریسک‌ها که باعث ناقص و ناکافی در نظر
        گرفتن نیازمندی‌ها در پروژه شود.
    \end{enumerate}
\end{itemize}

\subsection{شناسایی ریسک}

در این قسمت به چهار تکنیک شناسایی ریسک در پروژه‌های نرم‌افزاری می‌پردازیم:

\subsubsection{چک لیست‌های ریسک}

بررسی چک لیست‌های ریسک هم می‌تواند در مورد ریسک‌های محصول باشد و هم در مورد
فرایند‌ها. برای مثال حوزه مالی اولین حوزه‌ای نیست که قبلاً وجود نداشته باشد و
دقیقاً این اولین سیستمی باشد که از قابلیت‌های مالی استفاده می‌کند. در حقیقت این
حوزه از قبل چندین بار مورد استفاده قرار گرفته شده است و توسط متخصصان مختلفی مورد
آزمون و تلاش بسیاری بوده که توانسته به بیشتر چالش‌ها و ریسک‌های آن پاسخ دهند. به
این ترتیب می‌توانیم تمام ریسک‌های آن را از قبل تهیه کنیم و بتوانیم در سیستم خود
آن‌ها را بررسی کنیم که اگر برخی قابلیت‌ها منجر به تولید ریسک شد بتوانیم راهکاری
برای آن طراحی و پیاده‌سازی کنیم. در حقیقت یک لیست راهنما از پیش تعیین شده
می‌باشد.

برای سناریو زیر می‌توان تمام ریسک‌ها را از قبل پیشبینی کرد و لیستی از باید‌ها را
برای آن به شکل زیر بررسی می‌کنیم:

وقتی ارسال کننده پیام بخواهد پیامی را برای دریافت کننده‌ای ارسال کند ریسک‌های
احتمالی موارد زیر خواهد بود:

\begin{itemize}
    \item درگاه پیام تغییر کند: استفاده از رویکردی امن.
    \item پیام در شبکه شنود شود: رمزنگاری و استفاده از شبکه‌های توزیع شده.
    \item تاخیر در ارسال پیام: بررسی زیرساخت‌های شبکه‌ای و حتی الگوریتم‌های
    ارسال و دریافت پیام.
\end{itemize}

نکته: محصولات همگی مشخص هستند و فرایند‌های آن‌ها کاملاً عمومیت دارد.

\subsubsection{بازبینی مولفه‌ها}

مولفه‌ها مخصوص محصول می‌باشد؛ المان‌ها در حقیقت همان مولفه‌ها هستند. آدم‌ها،
نرم‌افزار‌های موجود و نرم‌افزار‌هایی که قرار است توسعه داده شود، تماماً المان
محسوب می‌شوند. برای مثال در سیستم قطار المان سرعت‌سنج را مورد بررسی قرار می‌دهیم
تا ریسک‌هایش را متوجه شویم:

\begin{itemize}
    \item آیا می‌تواند وظیفه‌اش را به درستی انجام ندهد؟ بله ممکن است. وظیفه‌ آن
    بررسی حرکت قطار و سرعت آن است. عوامل مختلفی وجود دارد که می‌تواند سبب درست
    کار نکردن و یا توقف کار کردن آن شود.
    \item اگر سرعت فیزیکی قطار با سرعت اندازه‌گیری شده برابر نباشد یعنی این
    دستگاه مشکلی دارد.
    \item یکی از ریسک‌ها آن است که در حین حرکت یکی از مسافر‌ها اقدام به خراب کرد
    دستگاه کند.
\end{itemize}

یا مثال اپلیکیشن‌های وب و موبایل:

\begin{itemize}
    \item اگر کاربر به اشتباه دستش روی گزینه پاک کردن بخورد جا به جا مورد انتخاب
    شده حذف شود یک ریسک در نظر گرفته می‌شود.
    \item برای این ریسک طراحی دیالوگ را می‌توان در نظر گرفت که از کاربر تایید مجدد
    برای انجام کار خودش گرفته شود.
    \item همچنین بعد از طراحی و پیاده‌سازی دیالوگ می‌توانیم در مورد پیاده‌سازی
    قابلیت لیست موارد پاک شده بپردازیم؛ یعنی سیستم را به گونه‌ای بنویسیم که
    قابلیت حذف آن به صورت \lr{Soft delete} باشد.
\end{itemize}

همه ریسک‌ها را نمی‌توان مدیریت کرد بلکه باید برخی از آنها پذیرفته شود. به همین
خاطر هر سیستمی بنا به مقدار آستانه تحمل خودش ریسک‌ها را می‌پذیرد. ریسک‌ها را
بررسی می‌کند اگر از مقدار آستانه کوچک‌تر بود مدیریتش را به کاربران می‌سپارد.
بارزترین مثال سیستم انتخاب واحد آموزشیار که به روشنی امکانات لود بالانس کردن
درخواست‌های کاربران زیاد را با پایین‌ترین کیفیت می‌تواند مدیریت کند. به خاطر
اینکه به کل سیستم صدمه‌ای وارد نمی‌کند، طراح سیستم از مدیریت این ریسک صرف‌نظر
می‌کند.

\subsubsection{تعریف عواقب یا \lr{Consequence}}

تمام عواقب \lr{Consequence} یک ریسک بایستی برآورد و بررسی شود. اگر نتوانیم یک
ریسک را مدیریت کنیم پس ممکن است رخ دهد. بعد از رخ دادن آن باید عواقب و تاثیرات
(\lr{Side effect}های) آن را در نظر بگیریم که چقدر می‌تواند مضر باشد و به سیستم
صدمه وارد کند. نوشتن تمام عواقب یک ریسک می‌تواند در کاهش نارضایتی‌ها تاثیرگذار
باشد.

\subsubsection*{آیا سیستم‌ها و سازمان‌ها از یک آستانه تحمل یکسان و مشخصی استفاده
می‌کنند؟}

خیر؛ هر سازمانی تحت شرایط و پروتکل‌های خاص خودش کار می‌کند و هر کسی نمی‌تواند
طبق میل و اراده خودش عمل کند. یک سازمان بررسی می‌کند که این مقدار آستانه چقدر
ارزش دارد.

پرسیدن چهار سوال زیر برای بررسی مولفه‌های ریسک الزامی می‌باشد؛ بایستی مولفه‌های
خیلی بزرگ را کوچک کنیم تا بتوانیم به سادگی به پاسخ سوالات زیر برسیم.

نکته: اینکه یک سناریو تبدیل به ریسک شود و به وقوع بپیوندد بایستی عواقب بعد از آن
را کنترل کنیم. مهندس نیازمندی لازم است که آثار ریسک را کمتر کند یا حداقل بتواند
آن‌ها را مدیریت کند. قدم‌های مدیریت ریسک متفاوت می‌باشد.

\begin{enumerate}
    \item \lr{Can it fall?} - آیا امکان رخ دادن همچین رسیکی وجود دارد؟ 
    \item \lr{How?} - بیشتر برای پیامد‌ها و بعد از درد رخ دادن ریسک است. 
    \item \lr{Why?} - چرا همچین ریسکی به وجود آمده است؟ دقیقاً به ریسک اشاره
    دارد.
    \item \lr{What are possible consequences?} - پیامد و عواقبی که ممکن است به
    همراه داشته باشد چه مواردی هستند؟ آیا از آن‌ها می‌توان چشم‌پوشی کرد؟ یا
    بایستی برای رفع آن‌ها هزینه‌ای داشته باشیم و راهکاری مناسب را ارائه دهیم؟
\end{enumerate}

\subsubsection{درخت ریسک}

به جزئیات این تکنیک در فصل نهم بیشتر پرداخته می‌شود.

تمام نود‌های درخت ریسک‌ها هستند. از ریشه شروع می‌کنیم و به ریسک‌های کوچک‌تر
شکسته می‌شود. برای مثال:

\begin{itemize}
    \item اگر خانه آتش بگیرد: \begin{enumerate}
        \item انفجار گاز
        \item اتصالی سیم برق داخل ساختمان
        \item یعنی یک اتفاق بزرگ و بد (آتش گرفتن یک خانه) می‌تواند چند عامل کوچک
        در رخ دادن آن تاثیر داشته باشند.
    \end{enumerate}
\end{itemize}

\subsubsection*{چه زمانی نیامند کشیدن درخت ریسک هستیم؟}

زمانی که ریسک به اندازه‌ای بزرگ و پیچیده باشد که نتوانیم آن را بفهمیم و درک
کنیم، نیازمند آن هستیم که با استفاده از درخت ریسک، یک ریسک را به عوامل مهم و
تاثیرگذارش بشکنیم تا ببینیم می‌توانیم آن را کنترل کنیم یا با سطح آستانه تحمل
سیستم ما حل خواهد شد.

\subsubsection*{نکات}

\begin{itemize}
    \item در این بین بعضی از معیار‌هایی که در درخت مشخص می‌کنیم ممکن است به صورت
    آماری باشند؛ یعنی طی ۵۰ سال مثلاً دوبار خانه آتش گرفته است.
    \item در تمام صنایع ریسک وجود دارد.
    \item ریسک‌ها را یا با مستطیل نمایش می‌دهیم یا با بیضی.
    \item اگر ریسک نیاز به شکستن داشته باشد آن را با نماد مستطیل نمایش می‌دهیم.
    \item اگر به کوچک‌ترین حالت ریسک رسیده باشیم یعنی آن را بتوانیم کامل به
    ساده‌ترین روش درک کنیم و نیاز به شکست نداشته باشد از نماد بیضی استفاده
    می‌کنیم.
    \item نماد‌های دیگری مانند \lr{AND} و \lr{OR} در این درخت استفاده می‌شوند.
    \item برگ‌ها و نود‌های پایانی درخت همیشه با نماد بیضی نمایش داده می‌شوند.
    \item برای آنکه پیدا کردن ریشه اتفاقاتی که رخ می‌دهد، ساده‌تر باشد از درخت
    ریسک استفاده می‌کنیم تا بتوانیم اتفاق بالایی را شناسایی کنیم و سپس بعد از آن
    با الگوریتم \lr{Cutset} درخت را ساده‌تر می‌کنیم.
    \item میزان بزرگی توپ، بزرگی خطر را نشان نمی‌دهد.
    \item پارامتر دیگر در بررسی بزرگی توپ ریسک احتمال وقوع هر عاقبت می‌باشد.
    \item اول عواقب پیدا می‌شود و سپس احتمال هر عاقبت سنجیده می‌شود.
    \item احتمال وقوع ریسک اگر کم باشد نمی‌توانیم بگوییم که میزانش نیز کمتر بوده
    است.
    \item احتمال وقوع ممکن است کم باشد اما اگر رخ بدهد ممکن است سیستم را از کار
    خارج کند.
\end{itemize}

\subsubsection{فلسفه درد}

درد صفر شدنی نیست اما قابل کم شدن است. چون اگر عواقب هم مورد بررسی قرار گیرد
می‌تواند درد داشته باشد؛ اگر درد نداشته باشد باید شک کنیم که آیا ریسک روی سیستم
ما رخ داده است؟ یا روی سیستم دیگری بوده؟ جعبه کمک‌های اولیه بهبود و کاهش اثر
اتفاقاتی است که مدتی است که رخ داده.

\subsubsection{نکات گره‌های \lr{AND} و \lr{OR}}

دقیقاً همانند مدار منطقی عملگر‌های زیر به این شکل کار می‌کند:

\begin{itemize}
    \item نود \lr{AND}: تمام زیر نود‌ها بایستی اتفاق بیوفتند تا نود والد رخ دهد.
    \item نود \lr{OR}: تنها نیاز است یک نود رخ دهد تا اتفاق والد رخ دهد.
\end{itemize}

\begin{figure}[H]
    \centering
    \includegraphics[width=0.9\textwidth]{images/risk_tree_sample.png}
    \caption{درخت ریسک از مسئله قطار}
\end{figure}

\subsubsection{شرط‌های \lr{Cutset}}

\begin{itemize}
    \item اگر از نوع \lr{AND} باشد یک نود درست شود و هر چیزی که به آن وصل بود را
    به آن منتقل می‌کنیم.
    \item اگر از نوع \lr{OR} بود به ازای هر نود مقایسه \lr{OR} صورت می‌گیرد.
\end{itemize}

\begin{figure}[H]
    \centering
    \includegraphics[width=0.9\textwidth]{images/cutset_risk_tree.png}
    \caption{ساده‌سازی درخت ریسک}
\end{figure}

\subsubsection{استفاده از تکنیک‌های جمع‌آوری داده}

با افراد و آدم‌ها در ارتباط می‌باشد. کسانی که ذینفع هستند در قسمت استخراج
اطلاعات می‌توانند نقش پر رنگی را ایفا کنند. اگر از متخصصین این حوزه سوال بپرسیم
می‌توانیم رخداد‌ها و پیامد‌های مختلفی که می‌تواند یک ریسک داشته باشد را یاد
بگیریم. مهم‌ترین راه‌ها: مانند استفاده از روش \lr{Interview} و روش \lr{Group
session}.

نکته: همان‌گونه که راه‌حل رفع تضاد رزولوشن بود، برای حل و کنترل ریسک از اقداماتی
مخصوص استفاده می‌کنیم تا بتوانیم به نسبت متعادلی ریسک را کنترل کنیم و حتی آن را
رفع کنیم (استفاده از تکنیک‌های \lr{Countermeasures} که در بالاتر گفته شد).

در این روش ابتدا راه‌حل را ارزیابی می‌کنیم و سپس بهترین راه‌حل را نسبت به بقیه
انتخاب و در سیستم در حال طراحی (\lr{System-to-be}) استفاده می‌کنیم.

\subsection{ارزیابی ریسک یا \lr{Risk assessment}}

هدف اصلی از ارزیابی ریسک، ارزیابی احتمال خطر + شدت، احتمال عواقب برای کنترل
خطرات بالا با اولویت بالا می‌باشد.

متغیر‌هایی که می‌توانیم برای ارزیابی کیفی مورد استفاده قرار دهیم:

\begin{itemize}
    \item برای احتمالات: (بسیار محتمل، محتمل، ممکن، بعید و...)
    \item برای شدت: (فاجعه‌بار \footnote{\lr{Catastrophic}}، شدید
    \footnote{\lr{Severe}}، بالا، متوسط و....)
\end{itemize}

\subsubsection*{چند عدد بزرگی ریسک را می‌سازند؟}

احتمال وقوع ریسک:

\begin{equation}
    P_{r} \epsilon (0, 1) \rightarrow 1 (State)
\end{equation}

احتمال وقوع عواقب:

\begin{equation}
    P_{c1} \epsilon (0, 1) P_{c2} \cdots \rightarrow n (State)
\end{equation}

شدت و دردی که در سیستم پس از عواقب پدیدار می‌شود:

\begin{equation}
    S_{c1} \{1, 2, 3, 4, 5\} S_{c1} \cdots \rightarrow n (State)
\end{equation}

اعدادی که بزرگی ریسک را می‌سازند (بزرگی توپ ریسک را می‌سازد):

\begin{equation}
    2n + 1
\end{equation}

سوال مهم آن است که چگونه می‌توان این اعداد را با هم ترکیب کرد که میزان خطر ریسک
را پیدا کرد؟ عواملی که مورد بررسی قرار گرفته است از یک جنس نیستند که بتوانیم
آن‌ها را با یکدیگر ترکیب کنیم.

\subsubsection{محاسبه میزان خطر ریسک}

\begin{equation}
    EX\footnote{\lr{Exposure}: نشان‌داده شدن}(r) = \sum_{i = 1}^{n} P_{c_{i}} \times S_{c_{i}}
\end{equation}

متغیر‌ها:

\begin{enumerate}
    \item n تعداد عاقبت
    \item P احتمال رخداد
    \item c \lr{Consequence}
    \item S \lr{Severity}
\end{enumerate}

اگر $n = 3$ باشد آنگاه میزان ریسک با توجه به محاسبات انجام شده بالا به شکل زیر
حاصل می‌شود:

\begin{itemize}
    \item $n = 3$
    \item $max = 15$: که از طریق محاسبات شماره ۱۵ و ۱۷ بدست آمده است.
    \item $min = 0$
\end{itemize}

فرض شود مقدار \lr{EX} برابر با ۷ شده است:

\begin{equation}
    EX = \frac{EX - min}{max - min} = \frac{7 - 0}{15 - 0} = \frac{7}{15}
\end{equation}

عدد $\frac{7}{15}$ میزان خطر ریسک بدست آماده می‌باشد که از طریق آن می‌توان
مقایسه‌ای بین آستانه تحمل درد دامنه خود کنیم که تا چقدر می‌تواند این ریسک برای
سازمان تاثیرگذار باشد.

\begin{equation}
    EX \epsilon [0, 15]
\end{equation}

\subsubsection{راه‌حل‌های ریسک}

راه‌حل‌های رفع ریسک پنج مورد می‌باشد:

\begin{enumerate}
    \item $P_{r} \searrow$: یا احتمال وقوع ریسک را کم می‌کنیم.
    \item $P_{r} = 0$: یا احتماع وقوع ریسک را صفر می‌کنیم. با طراحی که در حال
    انجام هستیم ریسک‌ها را شناسایی کرده‌ایم و می‌توانیم احتمال وقوع آن‌ها را به
    صفر برسانیم. مانند \lr{Iran access} کردن شبکه یک سرور در حالی که از شبکه
    خارج مورد حمله قرار گرفته بود. در این رویکرد اصلاً دیگر نمی‌تواند از شبکه
    خارجی حمله‌ای صورت گیرد.
    \item $P_{c_{i}} \searrow$: احتمال وقوع عواقب را کنترل می‌کنیم و آن را کاهش
    می‌دهیم. تمام سرویس‌های مورد نظر را از سرور آلوده به سرور دیگر در شبکه داخلی
    منتقل می‌کنیم. دکتر مقیم در قطار برای کاهش تاثیر ریسک و مدیریت عواقب می‌باشد.
    \item $P_{c_{i}} = 0$: احتمال وقوع عواقب را صفر می‌کنیم. بعد از آلوده شدن
    سرور، برق کل دیتاسنتر را قطع می‌کنیم که حمله به سرور‌های دیگر رخ ندهد (به
    عنوان مثال غیرحرفه‌ای).
    \item $S_{c_{i}} \searrow$: درد صفر شدنی نیست کم شدنی است.
\end{enumerate}

\subsubsection*{نکات}

\begin{itemize}
    \item در این راه‌حل‌ها، هر لایه که جلوتر می‌رویم لایه قبلی رخ داده است که
    بتوانیم ادامه ریسک‌ها را مدیریت و کنترل کنیم.
    \item برای بررسی ریسک‌ها بین صفر کردن و کم کردن احتمال هر کدام، از عبارت
    «یا» استفاده خواهیم کرد. یا باید ریسک صفر شود یا اگر نتوانستیم ریسک را صفر
    کنیم آن را حداقل کنترل و کم‌تر کنیم.
    \item دید مهندس نیازمندی نسبت به ریسک بسیار مهم است. ریسکی که می‌توان صفر
    کرد دیگر نیازی نیست که استراتژی برای کم کردن احتمالش را در نظر بگیریم.
    \item اگر سه لایه را بررسی کنیم، ممکن است با در نظر گرفتن استراتژی‌هایی مشکل
    را در همان لایه اول یا دوم حل کرده باشیم پس لزومی ندارد که تمام لایه‌ها را
    بررسی و برای آن‌ها استراتژی مناسب را در نظر بگیریم.
    \item ممکن است برای لایه‌ای از ریسک اصلاً راه‌حلی نداشته باشیم.
\end{itemize}

\subsubsection{مثال راه‌حل‌های ریسک}

\subsubsection*{مثال ۱}

\begin{itemize}
    \item \textbf{اتفاق بد محتمل این است که راننده قطار در هنگام حرکت به خواب
    برود. یک راه‌حل برای صفر کردن احتمال وقوع این ریسک ارائه دهید.} پیاده‌سازی
    قابلیت حرکت و مسیریابی خودکار یا \lr{Autopilot} بر روی قطا‌ر‌ها؛ با این
    نرم‌افزار می‌توان کنترل قطار را از انسان به سیستم انتقال داد و در صورتی که
    راننده احساس خستگی کرد می‌تواند بر روی حرکت قطار نظارت داشته باشد. به این
    شکل احتمال تصادف قطار را به دلیل خواب راننده آن صفر کرده‌ایم. استفاده از
    راننده قطار کمکی در سفر احتمال ریسک را کمتر می‌کند چرا که کمک راننده نیز
    می‌تواند احساس خستگی کند.
    \item \textbf{یک راه‌حل برای احتمال وقوع عاقبت این ریسک را بیان کنید که صفر
    کننده آن باشد.} وقتی در مورد عاقبت این ریسک صحبت می‌کنیم در حقیقت نه
    نرم‌افزار \lr{Autopilot} در قطار پیاده‌سازی شده است و نه راننده توانسته
    خستگی خود را کنترل کند و این بدان معناست که راننده در هنگام رانندگی قطار به
    خواب رفته است. برای این منظور می‌توانیم نرم‌افزاری از نوع \lr{AI} طراحی کنیم
    که اگر متوجه خوابیدن (بسته شدن چشم‌های راننده) در هنگام حرکت قطار شد، اعلانی
    به تمام مراکز و دیگر قطار‌ها ارسال کند که از حرکت خودشان منصرف شوند تا باعث
    برخورد دو قطار به یکدیگر نشوند و همچنین می‌توانند خطوط ریل را به گونه‌ای
    طراحی کنند که بعد از اعلان به این مراکز قطار را از طریق ریل‌ها نگه دارند و
    ماموری را برای جریمه راننده قطار ارسال کنیم تا از هر گونه تصادف در آینده
    جلوگیری کند.
    \item \textbf{راننده به خواب رفته است، خوابیدن راننده باعث برخورد دو قطار به
    یکدیگر شده است یعنی در حالت عواقب باز هم ریسک را کنترل نکرده‌ایم و حالا در
    مرحله درد یا \lr{Severity} هستیم.} در هر ۱۰ کیلومتر مسیر سازمان‌ها و
    بیمارستان‌های مجهز طراحی و پیاده‌سازی شوند که به سریع‌ترین حالت ممکن برای
    نجات جان مسافران دو قطار حاضر شوند. درون هر قطار، چند پزشک به همراه تجهیزات
    کامل در نظر گرفته شود که در هنگام تصادف در صورت امکان به آسیب دیدگان کمک
    کنند تا آن‌ها را به بیمارستان نزدیک برسانند. بعد از تصادف در‌ها و پنجره‌های
    قطار باز شوند که مسافرانی که جان آن‌ها سالم است بتوانند به خارج از قطار
    بروند و از هر احتمال انفجار و از بین رفتن جان‌ها جلوگیری شود.
\end{itemize}

\subsubsection*{مثال ۲}

تعریف ریسک: دانشجو نتواند وارد سیستم \lr{LMS} شود.

\begin{itemize}
    \item \textbf{کم کردن احتمال وقوع ریسک}: تعریف چندین سرور جانبی برای ورود
    موفقیت آمیز به کلاس‌ها.
    \item \textbf{دانشجو از طریق سرو‌ر‌های جانبی نتوانست وارد کلاس شود و الان در
    وضعیت ریسک در عاقبت هستیم}: قید و قانونی را می‌توانیم برای شروع هر کلاس وضع
    کنیم که اساتید ۱۵ دقیقه اول کلاس را به دور مباحث قبلی بپردازند یا اساتید دو
    بار حضور و غیاب را انجام دهند. همچنین می‌توان الگوریتمی را توسعه داد که به
    مدت زمان حضور دانشجو در کلاس می‌پردازد و حضور و غیاب توسط آن شکل می‌گیرد.
    \item \textbf{در لایه سوم، نه توانستیم احتمال وقوع را کنترل کنیم و نه در
    عاقبت توانستیم حاضر شویم. نوبت آن است که درد را برای دانشجو کمتر کنیم}: کلاس
    توسط اساتید در سامانه ضبط می‌شوند و دانشجو در صورت عدم حضور موفق در کلاس
    می‌تواند به کلاس ضبط شده مراجعه کند و از درس خود عقب نیوفتد.
\end{itemize}

به غیر از پنج تکنیکی که برای کنترل ریسک در بالاتر بررسی کردیم، استفاده از
تکنیک‌های جمع‌آوری اطلاعات از طریق ذینفعان می‌تواند کمک دیگری در فرایند بررسی
ریسک برای مهندسی نیازمندی باشد. استفاده از افراد خبره (\lr{Expert}) در این زمینه
می‌تواند به ما کمک کند که تمام چالش‌های پیاده‌سازی را بررسی کنیم و برای هر کدام
از آن‌ها از الگو‌های مطرح شده در سیستم‌های پیشین استفاده کنیم. تجربه‌های قبلی به
ما در کامل شدن سیستم و کمتر شدن ریسک‌های سیستمی کمک می‌کنند. برای مثال دو تیک
خوردن پیام در پیا‌مرسان‌های امروزی یک الگو برای مطمئن شدن از ارسال موفقیت آمیز و
خواندن شدن پیام توسط مخاطب و مقصد می‌باشد. فرقی نمی‌کند که این الگو‌ها برای
محصول باشد یا برای فرایند، دلیل آن که تمام سیستم‌های شبیه به یکدیگر هستند آن است
که یکسری از الگو‌های آن‌ها استاندارد و جامع می‌باشند.

نکته جالب دقیقاً از آن جایی است که الگو‌ها نیز می‌توانند در سیستم حاوی ریسک
باشند. برای مثال در گذشته که رمز پویا وجود نداشت افراد با استفاده از رمز دوم در
حساب بانکی خود می‌توانستند خرید خود را انجام دهند. اما تنها استفاده از رمز دوم
برای کاربر ایمن نمی‌باشد، پس استفاده از الگوی رمز‌های داینامیک یا پویا مطرح شد و
تمام بانک‌ها به این الگو پیوستند. در نظر داشته باشید اگر سرور کند شود و پیام رمز
پویا برای کاربر ارسال نشود در حقیقت ریسکی را برای کاربر ایجاد کرده‌ایم که برای
خرید و پرداخت وجه با مشکل رو به رو شده است. درست است که این الگو می‌تواند حاوی
ریسک باشد ولی از آنجایی که برای ما امنیت در اولویت است می‌توانیم از این ریسک چشم
پوشی کنیم و کاربر را وادار به صبر کردن و طراحی دکمه‌ای به نام "ارسال مجدد رمز"
کنیم.

\subsection{انتخاب راه‌حل مناسب برای رفع و کنترل ریسک}

بعد از آن که ریسک‌های سیستم مورد نظر را مهندس نیازمندی برآورد کرد ممکن است برای
کنترل یک ریسک به چندین راه‌حل مختلف برسیم. با چه معیار یا معیار‌هایی می‌توانیم
متوجه شویم که کدام راه‌حل برای کنترل ریسک مناسب است؟ دو روش برای انتخاب راه‌حل
ریسک مطرح شده است که میزان خوب بودن راه‌حل را برای کنترل ریسک مشخص می‌کند.

\subsubsection{اقدامات}

\lr{Countermeasure} به راه‌حل و اقدامات در برابر ریسک‌ها گفته می‌شود.

\subsubsection{ریسک‌ها بایستی مستند شوند}

برای توضیح نیازمندی راه‌حل‌های مطرح شده در کنترل ریسک‌ها، در ارزیابی سیستم
بایستی سندی در این رابطه تهیه شود. برای هر ریسک باید موارد زیر مشخص شود:

\begin{enumerate}
    \item شروط و رخداد‌هایی که باعث اتفاق افتادن ریسک می‌شوند.
    \item تخمین احتمال رخداد ریسک
    \item موارد محتمل و عواقب آن‌ها
    \item تخمین احتمال در وقوع ریسک، عواقب و درد‌های ناشی از آن‌ها
    \item مشخص کردن راه‌حل‌ها و اقدامات متقابل با ریسک‌ها در جهت کاهش ریسک
    \item انتخاب راه‌حل مناسب با استفاده از روش‌های بررسی آن‌ها
\end{enumerate}

\subsubsection{جنبه‌های استفاده از معیار‌ها}

استفاده از هر معیار دیگری میزان درگیری موارد زیر را متناسب با ریسک در بر دارد:

\begin{enumerate}
    \item میزان تاثیر راه‌حل در هزینه‌ها (\lr{Cost-effectiveness})
    \item میزان تاثیر راه‌حل در ریسک‌های دیگر
    \item میزان تاثیر راه‌حل در \lr{NFR}ها (مانند امنیت یا \lr{QoS})
\end{enumerate}

نکته: راه‌حل دادن باعث کوچک‌تر شدن توپ تاثیر می‌شود. همچنین اگر یک راه‌حلی مطرح
شود که در کوچک کردن چند توپ ریسک نقش داشته باشد می‌تواند کارآمد و الگو پذیر باشد
تا بتوانیم با مجموعه‌ای از راه‌حل‌ها به اندازه‌ای کمتر از آستانه مورد نظر خود
برسیم.

\subsubsection{روش \lr{RRL}}

\begin{equation}
    RRL(r, cm) = (Exp(r) - Exp(r/cm))/cost(cm)
\end{equation}

\begin{LTR}
    \begin{itemize}
        \item \lr{EXP(r)}: Exposure of rist r
        \item \lr{EXP(r/cm)}: New exposure of r if countermeasure cm is selected
    \end{itemize}
\end{LTR}

\begin{enumerate}
    \item در این روش راه‌حلی انتخاب می‌شود که بیشترین \lr{RRL}\footnote{\lr{Risk
    Reduction Leverage}} را داشته باشد
    \item .در این فرمول صورت کسر \lr{Effectiveness} بودن را مشخص می‌کند و در
    مخرج هزینه‌ها مشخص شده است.
    \item مقدار \lr{RRL} بیشتر باشد یعنی هزینه‌های آن را کاهش داده‌اند (مخرج
    کوچک‌تر اثر بخشی صورت را بیشتر نشان می‌دهد).
    \item روش \lr{RRL} نقش یک راه‌حل را در تمام ریسک‌ها بررسی نمی‌کند.
    \item این روش تنها در راه‌حل‌های تکی کار می‌کند. ما باید بررسی کنیم که یک
    راه‌حل در چند ریسک تاثیرگذار است.
    \item تمام راه‌حل‌ها را تک به تک بررسی می‌کند.
    \item تک به تک دیدن راه‌حل‌ها نسبت به ریسک‌ها در این روش باعث می‌شود که بعضی
    چیز‌ها در نظر گرفته نشود.
    \item یک فرض غلطی را دارد که می‌گوید یک ریسک را می‌توان با یک راه‌حل کنترل
    کرد. در حالی اگر یک راه‌حلی وجود داشته باشد که در کنترل چندین ریسک نقش داشته
    باشد بسیار ارزشمند‌تر از آن است که یک ریسک را با چندین راه‌حل مدیریت کنیم.
    \item $1(Risk) \rightarrow n(CM)$: روش مناسبی نخواهد بود.
\end{enumerate}

\subsubsection{روش \lr{DDP}}

این روش \footnote{\lr{Defect Detection Prevention}} هم نیز مانند روش \lr{RRL} به
صورت معیاری برای بررسی مناسب بودن راه‌حل برای کنترل ریسک می‌باشد با این تفاوت که
راه‌حل‌ها نسبت به ریسک‌ها به صورت تک به تک عنوان نمی‌شود و این روش به صورت
\lr{Generalization} عمل خواهد کرد. تکنیک و ابزاری است که در سال ۲۰۰۳ توسط ناسا
توسعه داده شده است.

\subsubsection*{تعاریف در این روش}

\begin{LTR}
    \begin{itemize}
        \item \lr{Objective} $\rightarrow$ \lr{requirement}
        \item \lr{Risk} $\rightarrow$ \lr{failure mode}
        \item \lr{Countermeasure} $\rightarrow$ \lr{PACT}
    \end{itemize}
\end{LTR}

برای انجام روش \lr{DDP} بایستی سه مرحله را طی کنیم:

\begin{enumerate}
    \item محاسبه ماتریس تاثیر ریسک به صورت دقیق
    \item محاسبه ماتریس تاثیر راه‌حل بر روی ریسک مورد نظر به صورت دقیق
    \item تعیین تعادل بهینه ریسک نسبت به هدف با تقسیم کاهش ریسک بر روی هزینه
    اجرای راه‌حل ($\frac{risk Reduction}{countermeasure cost}$)
\end{enumerate}

\subsubsection{مرحله اول در \lr{DDP}}

برای مرحله اول که رسم جدول تاثیر-نتیجه (جدول شماره \ref{fig:impactMatrix}) ریسک
می‌باشد باید در نظر داشته باشیم که هدف، موارد زیر می‌باشد:

\begin{itemize}
    \item اولویت‌بندی ریسک‌ها براساس امتیاز \lr{Critical impact} نسبت به تمام
    اهداف
    \item هایلایت کردن و برجسته کردن ریسک‌پذیر‌ترین اهداف
\end{itemize}

\subsubsection*{نکات}

\begin{itemize}
    \item تمام اعداد داخل جدول اعم از وزن‌ برای ریسک‌ها و اهداف و همچنین میزان
    تاثیرات، توسط اسناد قبلی در دامین و افراد خبره بدست آمده است!
    \item تلاقی سطر و ستون تاثیر ریسک روی هدف را نشان می‌دهد ($Impact(r, obj)$)
    \item اگر تاثیر ریسک صفر باشد یعنی هیچ تاثیری در هدف نداشته است.
    \item اگر تاثیر ریسک یک باشد یعنی به هدف تاثیر گذاشته است و ممکن است آن ریسک
    هدف را از بین ببرد.
    \item فرمول بدست آوردن آخرین سطر جدول: $Criticality(r) = Likelihood(r) * \sum_{obj}(Impact(r, obj) * weight(obj))$
    \item فرمول به دست آوردن آخرین ستون از جدول: $Loss(obj) = weight(obj) * \sum_{r}(Impact(r, obj) * Likelihood(r))$
    \item نتیجه \lr{Risk criticality} دردسر کل ریسک را بر اهداف نشان می‌دهد.
\end{itemize}

\begin{LTR}
    \begin{table}[H]
        \centering
        \begin{tabular}{ccccccc}
            Objectives & \makecell{Late returns \\ weight: \lr{0.7}} & \makecell{Stolen copies \\ weight: \lr{0.3}} & \makecell{Lost copies \\ weight: \lr{0.1}} & \makecell{LongLoan by staff \\ weight: \lr{0.5}} & Loss obj \\ \hline
            \makecell{Regular avialability of \\ book copies weight: \lr{0.4}} & \lr{0.30} & \lr{0.60} & \lr{0.60} & \lr{0.20} & \lr{0.22} \\ \hline
            \makecell{Comprehensive library \\ coverage weight: \lr{0.3}} & \lr{0} & \lr{0.20} & \lr{0.20} & \lr{0} & \lr{0.02} \\ \hline
            \makecell{Staff load reduced \\ weight: \lr{0.1}} & \lr{0.30} & \lr{0.50} & \lr{0.40} & \lr{0.10} & \lr{0.04} \\ \hline
            \makecell{Operational costs \\ decreased weight: \lr{0.2}} & \lr{0.10} & \lr{0.30} & \lr{0.30} & \lr{0.10} & \lr{0.05} \\ \hline
            Risk criticality & \lr{0.12} & \lr{0.12} & \lr{0.04} & \lr{0.06} &  \\
        \end{tabular}
        \caption{Impact matrix - example for library system}
        \label{fig:impactMatrix}
    \end{table}
\end{LTR}

\subsubsection{مرحله دوم \lr{DDP}}

برای مرحله دوم نیازمند رسم جدول تاثیر-نتیجه (جدول شماره
\ref{fig:effectivenessMatrix}) هستیم که هدف موارد زیر می‌باشد:

\begin{itemize}
    \item تخمین کاهش ریسک به وسیله راه‌حل جایگزین
    \item برجسته‌سازی بیشترین تاثیر راه‌حل‌های مطرح شده.
\end{itemize}

\subsubsection*{نکات}

\begin{itemize}
    \item تمام اعداد داخل جدول اعم از وزن‌ برای ریسک‌ها و اهداف و همچنین میزان
    تاثیرات راه‌حل، توسط اسناد قبلی در دامین و افراد خبره بدست آمده است!
\end{itemize}

\begin{itemize}
    \item تلاقی سطر و ستون کاهش و حذف ریسک را نشان می‌دهد اگر راه‌حل اعمال شده
    باشد $Reduction(cm, r)$.
    \item اگر مقدار صفر باشد یعنی راه‌حل پیشنهادی هیچ تاثیر نداشته است.
    \item اگر مقدار ۱ باشد یعنی به طور کامل از وقوع ریسک پرهیز کرده است.
    \item فرمول بدست آوردن آخرین سطر جدول به ازای هر ریسک: $combinedReduction(r) = 1 - \prod_{cm}(1 - Reduction(cm, r))$
    \item فرمول بدست آوردن آخرین ستون جدول به ازای هر راه‌حل: $overallEffect(cm) = \sum_{r}(Reduction(cm, r) * Criticality(r))$
    \item نتیجه \lr{Overall effect of countermeasure} مشخص می‌کند که این راه‌حل
    چقدر خوب بوده است.
    \item احتمال وقوع ریسک برایمان اهمیتی ندارد اما به هر دلیلی اینکه آن ریسک
    چقدر برایمان دردسرساز بوده، خیلی اهمیت دارد.
    \item در این مرحله به شما ثابت شد که بر خلاف روش \lr{RRL} یک راه‌حل می‌تواند
    در کاهش چند ریسک تاثیرگذار باشد. پس این راه‌حل ارزشمند است.
\end{itemize}

\begin{LTR}
    \begin{table}[H]
        \centering
        \begin{tabular}{ccccccc}
            Countermeasure & \makecell{Late returns \\ weight: \lr{0.7}} & \makecell{Stolen copies \\ weight: \lr{0.3}} & \makecell{Lost copies \\ weight: \lr{0.1}} & \makecell{LongLoan by staff \\ weight: \lr{0.5}} & \makecell{Overall effect of \\ countermeasure} \\ \hline
            \makecell{Email reminder sent} & \lr{0.70} & \lr{0} & \lr{0.10} & \lr{0.60} & \lr{0.12} \\ \hline
            \makecell{Fine subtracted from \\ registration deposit} & \lr{0.80} & \lr{0} & \lr{0.60} & \lr{0} & \lr{0.12} \\ \hline
            \makecell{Borrower unregistration \\ + insertion on black list} & \lr{0.90} & \lr{0.20} & \lr{0.80} & \lr{0} & \lr{0.16} \\ \hline
            \makecell{Anti-theft device} & \lr{0} & \lr{1} & \lr{0} & \lr{0} & \lr{0.12} \\ \hline
            \makecell{Combined risk \\ reduction} & \lr{0.99} & \lr{1} & \lr{0.93} & \lr{0.60} &  \\
        \end{tabular}
        \caption{Effectiveness matrix - example for library system}
        \label{fig:effectivenessMatrix}
    \end{table}
\end{LTR}

\subsubsection{مرحله سوم \lr{DDP}}

در مرحله پایانی به تعیین تعادل بهینه بین عامل کاهش‌دهنده ریسک و هزینه راه‌حل‌ می‌پردازیم.

\begin{itemize}
    \item هر \lr{Countermeasure} مزایایی دارد اما ممکن است پیاده‌سازی آن شامل
    هزینه‌هایی شود.
    \item هزینه هر \lr{Countermeasure} توسط متخصص خبره آن دامنه تخمین زده
    می‌شود.
\end{itemize}

روش \lr{DDP} را می‌توان بصری‌سازی نمود:

\begin{itemize}
    \item کشیدن چارت تعادل ریسک یا \lr{Risk balance charts}: باقی‌مانده تاثیر هر
    ریسک بر روی تمام اهداف (\lr{Objectives}) اگر راه‌حلی \lr{cm} انتخاب شود.
    \item ترکیب بهینه راه‌حل‌ها برای تعادل ریسک نسبت به قید و محدودیت‌های
    هزینه‌ای که در نهایت باعث موارد زیر می‌شود: \begin{itemize} 
        \item به حداکثر رساندن رضایت از اهداف تحت آستانه مشخصی از هزینه‌ها
        \item به حداقل رساندن هزینه‌ها، بالاتر از آستانه‌ای مشخص از رضایت
    \end{itemize}
\end{itemize}

\subsection{ارزیابی جایگزین‌ها برای تصمیم‌گیری}

\subsubsection*{نکات}

\begin{itemize}
    \item آپشن‌ها ثابت هستند.
    \item آیتم‌های \lr{NFR} و وزن آن‌ها می‌تواند به روز شود.
    \item وزن‌ها ضریب اهمیت به \lr{NFR} هستند.
\end{itemize}

فرایند مهندسی نیازمندی چندین گزینه جایگزین را به نوع مختلفی معرفی می‌کند:

\begin{itemize}
    \item راه‌های جایگزین برای راضی نگه داشتن اهداف یک سیستم
    \item جایگزین‌سازی مسئولیت‌ها در بین اجزای سیستم
    \item جایگزین‌سازی برای رزولوشن یک تضاد
    \item جایگزین‌سازی اقدمات و راه‌حل‌ها برای کاهش و کنترل  یک ریسک
\end{itemize}

نکته مهم: جایگزین (آلترناتیو) انتخابی بایستی به همراه مذاکره باشد:

\begin{enumerate}
    \item توافق بر معیار‌های ارزیابی از قبیل \lr{NFR}ها
    \item مقایسه گزینه‌ها به نسبت معیار‌ها
    \item انتخاب بهترین گزینه
\end{enumerate}

\subsubsection{استدلال‌های کیفی \lr{Qualitative reasoning}}

هدف اصلی استدلال‌های کیفی تخمین کیفی مشارکت هر گزینه در مقام نیازمندی‌های
\lr{NFR} می‌باشد:

\begin{itemize}
    \item \lr{Very positively (++)}
    \item \lr{Positively (+)}
    \item \lr{Negatively (-)}
    \item \lr{Very negatively (--)}
\end{itemize}

برای مثال جهت ارزیابی بهتر زمان‌بندی و اطلاع یک جلسه به اعضای شرکت جدول بررسی
کیفی زیر را خواهیم داشت:

\begin{LTR}
    \begin{table}[H]
        \centering
        \begin{tabular}{cccc}
            Options & Fast response & \makecell{Reliable \\ response} & \makecell{Minimal \\ inconvenience} \\ \hline
            Email reminder sent & $-$ & $+$ & $-$ \\ \hline
            \makecell{Get constraints \\ from e-agenda} & $++$ & $--$ & $++$ \\ \hline
        \end{tabular}
        \caption{Qualitative reasoning for NFR}
        \label{fig:qualitativeReasoningNFR}
    \end{table}
\end{LTR} \subsubsection{‌استدلال‌های کمی \lr{Quntitative reasoning}}

بایستی جدولی (ماتریسی) وزن‌دار به عنوان استانداردی برای این تکنیک بسازیم که:

\begin{itemize}
    \item امتیاز (\lr{score}) هر گزینه (\lr{option}) معیاری جهت ارزیابی می‌باشد.
    \item انتخاب گزینه‌ای (\lr{option}) که بالاترین امتیاز (\lr{score}) را میان
    بقیه معیار‌ها دارد.
    \item برای هر \lr{Option} خواهیم داشت: \lr{opt}
    \item برای هر معیار \lr{Criterion} خواهیم داشت: \lr{crit}
    \item $Score(opt, crit)$: تخمین درصد امتیاز یک گزینه نسبت به یک معیار
    \begin{itemize}
        \item ۰ تا ۱ خواهیم نوشت به گونه‌ای که می‌گوییم: معیار در \lr{x} درصد
        مواقع راضی است.
    \end{itemize}
    \item آخرین خط ماتریس برای هر \lr{Option} مجموع امتیاز هر گزینه را نسبت به
    معیار را بیان می‌کند
    \item \lr{Total}: $totalScore(opt) = \sum_{crit}(Score(opt, crit) * Weight(crit))$
\end{itemize}

\begin{LTR}
    \begin{table}[H]
        \centering
        \begin{tabular}{cccc}
            Evaluation criteria & \makecell{Significance \\ weighting} & \makecell{Get constraints \\ by email} & \makecell{Get constraints \\ from e-agenda} \\ \hline
            Fast response & $0.30$ & $0.50$ & $0.90$ \\ \hline
            Reliable response & $0.60$ & $0.90$ & $0.30$ \\ \hline
            Minimal inconvenience & $0.10$ & $0.50$ & $1.00$ \\ \hline
            Total & $1.00$ & $0.74$ & $0.55$ \\
        \end{tabular}
        \caption{Option Score table (matrix)}
        \label{fig:optionScore}
    \end{table}
\end{LTR}

\subsection{اولویت‌بندی انتخاب‌ها}

همیشه به یاد داشته باشید که اولویت‌بندی بعد از انتخاب راه‌حل‌ها مورد بررسی قرار
می‌گیرد که در آن نیازمندی ثابت است و انتخاب در ورژن‌ها می‌تواند متغیر باشد. به
جمع‌آوری و ارزیابی نیازمندی‌ها بایستی اولویت اختصاص داده شود:

\begin{enumerate}
    \item رزولوشن تضاد‌ها
    \item محدودیت منابع مانند پول، پرسنل و زمان
    \item توسعه افزایشی
    \item برنامه‌ریزی مجدد در حالی که مسئله‌ای  پیشبینی نشده رخ داده است.
\end{enumerate}

در این بین اصولی وجود دارد که می‌تواند در اولویت‌بندی نیازمندی‌ها موثر باشد:

\begin{enumerate}
    \item سطوح اولویت‌بندی را مرتب کنیم و همیشه اعداد سطوح را کوچک نگه‌داریم.
    \item سطوح مرتبط و کیفی (بیشتر از یه چیزی بودن یا به جای دیگری)
    \item نیازمندی‌های قابل مقایسه
    \item نیازمندی‌ها متقابلاً وابسته نیستند (یک مورد می‌تواند گرفته شود و توسط
    سازمان پذیرفته شود و مورد دیگر می‌تواند \lr{Drop} شود).
    \item اولویت‌ها توسط افراد اصلی سازمان پذیرفته شده باشند (به اولویت‌ها
    اعتقاد داشته باشند).
\end{enumerate}

\subsubsection{اولویت‌بندی براساس ارزش-هزینه یا \lr{Value-cost}}

سوال: یک شلوار خریدم یک میلیون تومان، گران است؟

جواب: نسبت به چه چیزی؟

در چنین مواقعی بایستی یک چیز را به یک یا چند چیز دیگر مقایسه کنیم (اشاره به
مفهوم \lr{trade-off}) و سپس نسبت به آنها اقدام به انتخاب و اولویت‌بندی کنیم.
برای اولویت‌بندی براساس ارزش-هزینه از روش \lr{AHP}\footnote{\lr{Analytic
Hierarchy Process}} استفاده می‌شود. در این روش نسبت سبد را مشخص خواهیم نمود. این
روش نموداری را ترسیم می‌کند که در محور \lr{x}ها درصد هزینه‌ای که شامل می‌شود و
در محور \lr{y}ها درصد ارزشی که آن کار دارد را می‌نویسد.

تکنیک \lr{AHP} دو معیار دارد که در نرم‌افزار آن را مورد سنجش خود قرار می‌دهد:

\begin{enumerate}
    \item هزینه‌ها (\lr{Costs}): همیشه دوست داریم هزینه‌ هر کاری پایین باشد و در
    عین حال ارزشمند باشد.
    \item ارزش‌ها (\lr{Values}): به معنای آن است که اگر این نیازمندی را محقق
    کنیم چند درصد به پروژه رسیده‌ایم؟ آیا اهدافمان را طی کرده‌ایم؟
\end{enumerate}

\begin{figure}[H]
    \centering
    \begin{tikzpicture}
        \begin{axis}[
            xlabel={هزینه درصد},
            ylabel={ارزش درصد},
            xmin=0, xmax=50,
            ymin=0, ymax=50,
            % grid=major,
            width=10cm, height=10cm,
            scatter/classes={
                a={mark=o,draw=blue},
                b={mark=square*,draw=red}
            }
        ]
        \addplot[
            scatter, only marks,
            scatter src=explicit symbolic,
        ]
        table[meta=label] {
            x y label
            5 48 POD
            9 28 HPL
            16 17 PCR
            25 4 MLC
            48 8 MA
        };
        % Slope 1: about high priority
        \addplot[
            domain=0:100,
            samples=100,
            color=red,
            thick
        ] {0.5 * x};

        % Slope 2: about low priority
        \addplot[
            domain=0:100,
            samples=101,
            color=red,
            thick
        ] {2 * x};

        % Note about slopes (High, Medium and Low) priorities
        \node at (axis cs:6,40)[anchor=west]{بالا};
        \node at (axis cs:30,35)[anchor=west]{میانی};
        \node at (axis cs:31,15)[anchor=west]{پایین};

        % Points label
        \node at (axis cs:5,48)[anchor=west]{POD};
        \node at (axis cs:9,28)[anchor=west]{HPL};
        \node at (axis cs:16,17)[anchor=west]{PCR};
        \node at (axis cs:25,4)[anchor=west]{MLC};
        \node at (axis cs:43,8)[anchor=west]{MA};
        \end{axis}
    \end{tikzpicture}
    \caption{نمودار ارزش-هزینه بدست آمده از روش \lr{AHP}}
    \label{fig:scatter}
\end{figure}

روش \lr{AHP} مقایسه دو به دو انجام می‌دهد و سپس همه موارد نسبت به نتیجه مقایسه
دوتایی‌ها مورد بررسی و مقایسه مجدد قرار می‌گیرند. به عنوان مثال در ابتدا \lr{x}
را نسبت به \lr{y} و \lr{z} مقایسه می‌کند. بعد از آن \lr{y} انتخاب می‌شود. سپس
\lr{y} نسبت به \lr{xz} و \lr{ab} و دیگر موارد مورد بررسی قرار می‌گیرد.

\subsubsection*{نکات}

\begin{itemize}
    \item به طور کلی \lr{AHP} تکنیکی برای اولویت‌بندی نیازمندی‌ها می‌باشد.
    \item شیب‌ها مانند یک چاقو می‌ماند که نیازمندی و مدیر پروژه باید آن را مشخص
    کند که چه بخشی چه درجه‌ای از اولویت را داراست.
    \item برشی که برای اولویت‌ها زده می‌شود ورژنی را مشخص می‌کند که آن ارزش با
    هزینه مناسب تهیه و پیاده‌سازی می‌شود.
    \item هر چقدر موارد مورد نظر هزینه کمتر و ارزش بیشتری داشته باشند برایمان
    اولویت بالاتری در نسخه کنونی سازمان خواهد داشت.
    \item شیب خطوط (بالا، متوسط و پایین) با استفاده از سیستم‌های فازی حاصل
    می‌شود که خروجی پارامتر‌های مدیریت پروژه این شیب خواهد بود.
    \item اگر به عنوان مثال پول کافی برای انجام یک کار نداریم پس سعی کنیم ابتدا
    یک کار کوچک‌تر و ارزشمند را انجام دهیم و به پایان برسانیم و سپس به دنبال هدف
    بعدی بریم. برای مثال اگر یک نیروی کارآموز داریم اول باید به وی روند انجام
    کار و پروژه را یاد دهیم تا صرفاً بیهوده هزینه ایجاد نکنیم که چند پروژه و
    چندین تسک برای او تعریف کنیم که باعث شود کار‌ها نیمه تمام باقی بماند.
    \item قضیه سیستم‌های فازی در \lr{Expert system} وجود دارد.
    \item نقاط نمودار هزینه-ارزش را مهندس نیازمندی مشخص می‌کند.
    \item این تکنیک یک تکنیک انسان محور می‌باشد.
    \item نقطه‌ای که روی نمودار ایجاد می‌شود محصول یا خروجی روش \lr{AHP} بر روی
    هزینه‌ها و ارزش‌هاست اشاره به شکل شماره \ref{fig:scatter}.
\end{itemize}

\subsubsection{قدم اول روش \lr{AHP} برای معیار ارزش‌ها}

مقیاسی را مشخص کنید که در آن بتوانید نیازمندی‌ها و راه‌حل‌ها را با یکدیگر مقایسه
کنید. در این مرحله از عدد گذاری فرد استفاده می‌کنیم و به هر عددی معیاری مشخص را
اختصاص می‌دهیم تا بتوانیم اولویت خود را بیان کنیم.

\begin{LTR}
    \begin{itemize}
        \item $1$: \lr{Contributes equally}
        \item $3$: \lr{Contributes slightly more}
        \item $5$: \lr{Contributes strongly more}
        \item $7$: \lr{Contributes very strongly more}
        \item $9$: \lr{Contributes extremely more}
    \end{itemize}

    \begin{table}[H]
        \centering
        \begin{tabular}{cccccc}
            Crit: value & \makecell{Produce \\ optimal date} & \makecell{Handle preferred \\ locations} & \makecell{Parameterize conflict \\ resolution strategy} & \makecell{Multi-lingual \\ communication} & \makecell{Metteing \\ assistant} \\ \hline
            \makecell{Produce \\ optimal date} & $1$ & $3$ & $5$ & $9$ & $7$ \\ \hline
            \makecell{Handle preferred \\ locations} & $1/3$ & $1$ & $3$ & $7$ & $7$  \\ \hline
            \makecell{Parameterize conflict \\ resolution strategy} & $1/5$ & $1/3$ & $1$ & $5$ & $3$ \\ \hline
            \makecell{Multi-lingual \\ communication} & $1/9$ & $1/7$ & $1/5$ & $1$ & $1/3$ \\ \hline
            \makecell{Metteing \\ assistant} & $1/7$ & $1/7$ & $1/3$ & $3$ & $1$ \\
        \end{tabular}
        \caption{R-Matrix: AHP Comparison matrix with relative requirements on
        the meeting scheduler}
        \label{fig:ahpValueComparison}
    \end{table}
\end{LTR}

\begin{itemize}
    \item هر نیازمندی نسبت به خودش مقدار ۱ یعنی \lr{Contributes equally} را
    می‌گیرد.
    \item ماتریس جدول شماره \ref{fig:ahpValueComparison} نشان می‌دهد که: $R_{ji} = 1
    / R_{ij}$ به شرطی که $(i >= 1)$ و $(j <= N)$
    \item هر نیازمندی نسبت به ارزش‌ها می‌تواند معکوس باشد یعنی نسبت به قطر وارون
    می‌شود (به نسبت فرمول بالا).
\end{itemize}

\subsubsection{قدم دوم روش \lr{AHP} برای معیار ارزش‌ها}

در این قدم نحوه توزیع معیار بین نیازمندی‌ها را ارزیابی می‌کنیم. هر عنصر از
ماتریس مقایسه با نتیجه تقسیم این عتصر بر مجموع عناصر ستون آن جایگزین می‌شود.

\begin{LTR}
    Criterion distribution = eigenvalues of comparison matrix
\end{LTR}

نوبت به نرمال‌سازی مقادیر ماتریس می‌رسد که بر اساس فرمول شماره
\ref{equation:normalizedAhpValueFirst} عمل می‌کنیم:

\begin{equation}
    R'_{ij} = \frac{R_{ij}}{\sum_{i}R_{ij}} 
    \label{equation:normalizedAhpValueFirst}
\end{equation}

\begin{LTR}
    \begin{table}[H]
        \centering
        \begin{tabular}{ccccccc} Crit: value & \makecell{Produce \\ optim. date} & \makecell{Handle preferred \\ locations} & \makecell{Param. conflict \\ resolution strategy} & \makecell{Multi-lingual \\ communication} & \makecell{Metteing \\ assistant} & \makecell{Relative \\ value} \\ \hline
            \makecell{Produce \\ optimal date} & $0.56$ & $0.65$ & $0.52$ & $0.36$ & $0.38$ & $0.49$ \\ \hline
            \makecell{Handle preferred \\ locations} & $0.19$ & $0.22$ & $0.31$ & $0.28$ & $0.38$ & $0.28$ \\ \hline
            \makecell{Parameterize conflict \\ resolution strategy} & $0.11$ & $0.07$ & $0.10$ & $0.20$ & $0.16$ & $0.13$ \\ \hline
            \makecell{Multi-lingual \\ communication} & $0.06$ & $0.03$ & $0.02$ & $0.04$ & $0.02$ & $0.03$ \\ \hline
            \makecell{Metteing \\ assistant} & $0.08$ & $0.03$ & $0.03$ & $0.12$ & $0.05$ & $0.07$ \\
        \end{tabular}
        \caption{R'-Matrix: AHP has rules for ensuring consistent estimates \&
        ratios}
        \label{fig:ahpValueStep2}
    \end{table}
\end{LTR}

میانگین بین خطوط: مجموع عناصر در خط اول ماتریس نرمال شده تقسیم بر تعداد عناصر در
طول خط فرمول شماره \ref{equation:normalizedAhpValueSecond}.

\begin{equation}
    Contrib(R_i, Crit) = \sum{j}R'_{ij}/N
    \label{equation:normalizedAhpValueSecond}
\end{equation}

حال اگر دقت کرده باشید تمام ماتریس‌های \lr{R} و \lr{R'} بالا به نسبت معیار
\lr{Value} یا همان ارزش‌ها بدست آمدند، الان نوبت آن است که این دو ماتریس را
براساس معیار \lr{Cost} یا هزینه‌ها بدست آوریم.

\subsubsection{قدم اول روش \lr{AHP} برای معیار هزینه‌ها}

\begin{LTR}
    \begin{table}[H]
        \centering
        \begin{tabular}{cccccc}
            Crit: costs & \makecell{Produce \\ optimal date} & \makecell{Handle preferred \\ locations} & \makecell{Parameterize conflict \\ resolution strategy} & \makecell{Multi-lingual \\ communication} & \makecell{Metteing \\ assistant} \\ \hline
            \makecell{Produce \\ optimal date} & $1$ & $1/3$ & $1/5$ & $1/5$ & $1/7$ \\ \hline
            \makecell{Handle preferred \\ locations} & $3$ & $1$ & $1/5$ & $1/5$ & $1/7$  \\ \hline
            \makecell{Parameterize conflict \\ resolution strategy} & $5$ & $5$ & $1$ & $1/3$ & $1/5$ \\ \hline
            \makecell{Multi-lingual \\ communication} & $5$ & $5$ & $3$ & $1$ & $1/3$ \\ \hline
            \makecell{Metteing \\ assistant} & $7$ & $7$ & $5$ & $3$ & $1$ \\
        \end{tabular}
        \caption{R-Matrix: AHP Comparison matrix with relative requirements on
        the meeting scheduler}
        \label{fig:ahpCostComparison}
    \end{table}
\end{LTR}

\subsubsection{قدم دوم روش \lr{AHP} برای معیار هزینه‌ها}

\begin{LTR}
    \begin{table}[H]
        \centering
        \begin{tabular}{ccccccc} Crit: cost & \makecell{Produce \\ optim. date} & \makecell{Handle preferred \\ locations} & \makecell{Param. conflict \\ resolution strategy} & \makecell{Multi-lingual \\ communication} & \makecell{Metteing \\ assistant} & \makecell{Relative \\ value} \\ \hline
            \makecell{Produce \\ optimal date} & $0.05$ & $0.02$ & $0.02$ & $0.04$ & $0.08$ & $0.04$ \\ \hline
            \makecell{Handle preferred \\ locations} & $0.14$ & $0.05$ & $0.02$ & $0.04$ & $0.08$ & $0.07$ \\ \hline
            \makecell{Parameterize conflict \\ resolution strategy} & $0.24$ & $0.27$ & $0.11$ & $0.07$ & $0.11$ & $0.16$ \\ \hline
            \makecell{Multi-lingual \\ communication} & $0.24$ & $0.27$ & $0.32$ & $0.21$ & $0.18$ & $0.25$ \\ \hline
            \makecell{Metteing \\ assistant} & $0.33$ & $0.38$ & $0.53$ & $0.63$ & $0.55$ & $0.48$ \\
        \end{tabular}
        \caption{R'-Matrix: AHP has rules for ensuring consistent estimates \&
        ratios}
        \label{fig:ahpCostStep2}
    \end{table}
\end{LTR}