\newpage

\section{فصل نهم}

یکسری از اهداف ریسک‌هایشان مهم است که مشخص شوند و اصلاً نمی‌توان آن‌ها را در بعد
اختیاری دید. برای مثال اگر ریسک مورد نظر از نوع امنیتی باشد بایستی ریسک و راه‌حل
آن نیز مشخص شود. ولی برخی از ریسک‌ها الزام‌آور نیست و بسته به نیاز و انتخاب
مشتری می‌باشد.

همانطور که در فصل پیشین اشاره شد، در کتاب مرجع بجای استفاده از کلمه ریسک از کلمه
\lr{Obstacle} استفاده شده است.

ریسک‌ها در حقیقت مشخص می‌کنند که در وضعیت جاری هستند و به \lr{State} بعدی
نرفته‌اند. ریسک در مورد بخش بعد از \lr{Then} صحبت می‌کند. اگر ترمز قطار را
کشیدیم باید قطار شروع به توقف کند. از نوع هدف رفتاری و \lr{Achieve} می‌باشد.
\lr{Not} آن می‌شود به آن \lr{State} که باید صفر شود نرسیده است.

\subsection{متوازی الاضلاع برعکس}

برای نمایش ریسک از شکل متوازی الضلاع برعکس استفاده می‌کنیم که نشان‌دهنده
\lr{Not} هدف می‌باشد.

\subsection{اهدافی که باید ریسک آنها بدست آید}

۶ هدف وجود دارد که بایستی ریسکشان را بدست آوریم. چه خواسته مشتری باشد چه نباشد:

\begin{enumerate}
    \item \lr{Hazard}: از دسته اهداف \lr{Safety} می‌باشد.
    \item \lr{Threat}: از دسته اهداف امنیتی می‌باشد مانند: \lr{Disclosure,
    Corruption, DOS (Denial-of-Service), Availability, Respectively}
    \item \lr{Dissatisfaction}: از نوع در خواست‌های عوامل \lr{Satisfaction}
    می‌باشد.
    \item \lr{Misinformation}: اهداف \lr{Information}
    \item \lr{Accuracy}: ریسک ناسازگاری بین وضعیت مقادیر کنترل شونده به وسیله
    عوامل نرم‌افزاری و وضعیت تطابق تعداد موارد کنترل شده به وسیله عواملی محیطی
    است (\lr{Inaccuracy}).
    \item \lr{Unusability}: نسبت به اهداف \lr{Usability} می‌باشد.
\end{enumerate}