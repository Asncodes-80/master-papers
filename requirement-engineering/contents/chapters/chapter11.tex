\newpage

\section{فصل یازدهم}

\section(\lr{Agent Diagram})

از این بخش سوال امتحانی خواهیم داشت.

عامل‌ها براساس وظایفشون مشخص میشه چی رو ببین و چی رو نبینن.

با تحمیل زنجیره آسیب مشخص میشه که تو ایجنت فعلی که داریم استفاده میکنم یا غلط
دادیم وظیفه رو یا تو انجام وظیفش تخوب نیست باید به عامل دیگه‌ای بدیمش.

کنترل یک ایجنت مانیتور یه ایجنت دیگست.

سه دسته دیاگرام عامل داریم:

Agent diagram کامل ترینه که گول داریم اجینت داریم و کلاس داریم.

Context diagram: همون مجموعه ارتباطی آسامپشن‌ها و غیره هستند.

depedency diagram: برای زنجیره آسیبه

اسلاید هشت فایل ۱۱

Ag1

مسئول یه گوله

زنجیره آسیب ارتباطات اینتپونت هر عامل دیگر را مشخص میکند.

همیشه از انتها به ابتدا می‌خوانیم.

تمرین:

در یک مثال یک گول در نظر بگیرید و یک زنجیره آسیب براش بسازید.

نوشته می‌شود. (نسبت به سکشن‌هایی بالایی ببین ببخشید واقعاً)

نهاد کلاس‌های تغییر میکند.

بقیش دیگه تو طراحی می‌مونه

قط کلاس‌ها و اتریبیوت‌های که از قضای مسئله می‌آیند.