\newpage

\section{فصل هشتم}

در این فصل در مورد نمایش بصری اهداف، ریسک‌ها و تمام مطالبی که در فصل‌های پیشین
خوانده‌ایم می‌پردازیم.

نیازمندی سیستم یا \lr{Sysmtem requirement} یک هدف چند عامله و \lr{Software
requirement} یک هدف تک عامله می‌باشد. یکسری اهداف استراتژیک وجود دارد که به
اهداف کوچک‌تری ریز می‌شوند تا قابل فهم مهندس نیازمندی باشند. از اشکال هندسی برای
بیان اهداف و زیر مجموعه آنها، برگ‌ها و غیره استفاده می‌کنیم.


\subsection{نکات تکمیلی جهت رسم نمودار‌های \lr{UML}}

\begin{itemize}
    \item شکل متوازی الاضلاع اهداف را مشخص می‌کند.
    \item پر رنگ شدن یا \lr{Bold} شدن اشکال برای نشان دادن برگ‌ها،
    \lr{Assumption}ها و نیازمندی‌های سیستم است به آن معنا که دیگر شکست و مشتق
    گرفتن در آن قسمت نخواهیم داشت و آن مورد آخرین نود در نمودار خواهد بود.
    \item نقاط تو پر کامل بودن یا نبودن را مشخص می‌کند.
    \item تمام مواردی توصیفی‌ها (\lr{Descriptive}ها) مانند \lr{Domain proper} ها
    با با ذوزنقه نمایش داده می‌شود.
    \begin{itemize}
        \item سرعت قطار مخالف با صفر باشد و در‌های آن قفل باشد. به عنوان دامنه
        هدف محسوب می‌شود.
    \end{itemize}
    \item اشکال توصیفی، هدف (\lr{Goal}) نیستند.
    \item \lr{Goal} یک جمله می‌باشد و می‌تواند به دو صورت زیر باشد:
    \begin{itemize}
        \item \lr{Multi-agent}: چند عامله
        \item \lr{Single-agent}: تک عامله
    \end{itemize}
    \item عامل یعنی آن المانی که \lr{Goal} را محقق می‌کند.
    \begin{itemize}
        \item اگر هدف، \lr{System requirement} بود یعنی چند عامله می‌باشد.
        \item اگر هدف \lr{Assumption} و \lr{Software requirement} بود یعنی تک
        عامل هستند.
    \end{itemize}
    \item عامل شامل افراد، دستگاه‌ها و سنسور‌ها یا تمام کتابخانه‌ها و نرم‌افزاری
    موجود در حال حاضر
    \item برگ‌ها تک عامله هستند.
    \item عوامل را با ۶ ضلعی نمایش می‌دهیم.
\end{itemize}

