% Chapter 3 continue

تاکتیک‌ها مجنر به تولید نیازمندی‌های جدید میشه:

اسلاید ۲۴

خاص سازی منبع یا هدف کانفیلیکت

تضاد در سطح جمله است که یه قانونی به کل وارد میشود که یکری جز دارد نقض قانون
بالایی به جز وارد میشود. هر کدوم قانون خودشون رو دارن چون جز هم قانون خودشو داره
داره باعث ایجاد تضاد میشه.

قانون جدید (جمله جدید) در مورد کل سیستم نبوده بلکه در مورد یک جز خاص بوده.

مثال:

یوزر‌ها بتونن خبر داشته باشن که کی چه کتابی رو قرض گرفته.

به دانشجو‌ها اجازه ندیم که متوجه این قضیه بشه.

به جای اعمال قانون به کل سیستم بایتسی به یک نود و قشر مشخص این اعمال انجام شود.

توی فاعل و مفعول رخ می‌دهد.

اینجا در مفعول پیش آمده است.

خاص سازی باید روی سورس یا 

رابطه کل به جز

باید ببینیم وجود دارد و بعدش باید تصمیم بگیریم که در چه سطحی باید تغییر ایجاد
کنیم.

----

تکنیک ویک کردن:

مثلا میگه ۱۰ خونه باش و مادرت میگه هر چقدر بودی کمشکلی نداره پس تضاد قوی داره.

جمله سخت‌تر را ویک می‌کنم.

اونی که قانون رو می‌بنده

قرض گیرنده باید کپی کتاب‌ّا را سر سه هفته بیارند.

مگر اینکه یک پرمیژن براش صادر بشه

دانشجو‌ها میتونن تا ۳ هفته قرض بگیرن و در صورتی که عضو انجمن هیات علمی باشن
می‌توانن که زمان زیادتری را داشته باشن.

---

تکنیک ری‌استور کردن:

تا جایی که تضاد خوردیم بریم جلو
و بعد از اون سیستم رو ببریم به حالت قبل از تضاد.

کگتاب ۵ هفته نگداری بشه. سر سه هفته بیا کتابخونه و تمیددی بکن کارتو.

---

آخرین راه‌حل که سخت تر از همه است که د رمورد تضاد‌های ویک است میگه در مورد از
شرایط مرزی باید پرهیز کنیم.

همون مثال ظرف سه هفته کتاب:

ریشه یابی
یسری کتاب هست که مرجع هستش
اگه اینا رو بدیم بره
هر بار که آدم جدید میاد این کتاب رو میخواد بادی بهش بگه که ندارم این کتاب رو 
رسما رسالت کتاب خونه زیر سوال میره.
برای همه کتاب‌های این نگرانی آیا وجود دارد؟
میگه نه کتاب‌های خاص فقط
اونا باید در دسترس باشن برای مطالعه
این کتاباها چند نسخه دارن
خوبه که یکی شون رو غیر قابل قرض نگه دارم که یکی حداقل باشه ازش
هر کی اومد من میتونم اون یدونه رو بهش بدم.
باید بتونه اونو تو همون محل بخونه که کارش تموم بشه که بده به درخواست بعدی.

ممکن است که سراغ الگوریتم‌های دسته‌بندی بریم که بتونیم رضایت رو برای همه طرفین
داشته باشیم.

تغییرات در سیستم بسیار اساسی تر میشه که باید مر بکشیمبگیم که چندتا قابل قرضه چند
تا غیرقابل قرض هستسش.

تمرین:

در یک سیستم مانند اسنپ، مسافر می‌خواهد نزدیک‌ترین ماشین به او تخصیص داده شود،
مدیر سیستم می‌خواهد در راستای طرح تشویقی خود رانندگانی با امتیاز بالاتر را به
مشتری تخصیص دهد آیا تضادی می‌بینید؟ اگر بله از چه نوعی است و راه‌حل آن چیست؟

امتیاز بالا داشته باشه

بتوینم کاربر را راضی کنیم 

تضاد از نوع ویکه چون ممکنه اونی که امتیاز بالاتر باشه دورتر باشه.

ویکه چون اونی که خیلی نزدیکه ممکنه امتیاز پایین تره رو بگیره.

از نوع ری استوره چون تازمان یکه به مشکل نخورده پیش میره.

تا زمان یکه پیدا نکرده دایره جست و جو رو برگتر میکنه

الزاماً میتونه ۵ ستاره نباشه تو یه دایره که دور کاربره میاد بررسی میکنه کی بهش
نزدیک‌تره (خواسته کاربر) و کدوم راننده هست که بیشترین ستاره رو داشته باشه.
(خواسته مدیر اسنپ).

تمرین:

دو جمله بگید که برای رفع تضاد استراتژی ویک کار کند.
ویک به مفهوم مشترک کار می‌کند.

% کارمند برای بررسی ساعات شناوری خود بایستی حداقل ۲ ساعت در شرکت حضور داشته باشد.

% هر کارمندی میتواند از قابلیت شناوری خود تا ساعت ۸ شب استفاده کند به شرطی که ۲
% ساعت حداقل حاضر شود.

% برنامه تا زمانی که آزمایش نشود نمی‌تواند آپلود شود. مگر اینکه 

% دانشجو از آزمایشگاه می‌تواند ۵ ساعت استفاده کند مگر اینکه دانشجوی ارشد باشد.

% دانشجو میتواند از سرویس اختصاص کارت گرافیک به

تمرین: یک سیستم LMS مانند اسکای‌روم، ادوبی کانکت، دو جمله بنویسید که فرقی نداره
ویک باشه یا استرانگ، بعد راه‌حل برای حلش انتخاب کنیم.

نقش‌ها:
مدیر شبکه
دانشجو
استاد
پشتیبانی
مدیر آموزش

سوال امتحانی

آپلود شدن فایل‌ها

اسلاید ۲۶ در خصوص ارزیابی راه‌حل‌ها و انتخاب

ریسک هم راه‌حل داره

همون چیزایی که برای قبلی استفاده میکردیم برای ریسک هم استفاده می‌کنیم.

اولین معیار ارتباط با \lr{NFR} می‌باشد.

وقتی هزینه مهم است، زمان پاسخ مهم است، امنیت همه است 
باید تو راه‌حل پیشنهادی همه اینا دیده بشن.

پس نان فاکشن‌ها بسیار حائز اهمیت هستند.

دومی میگه این راه حله خودش نیاد کافنلیکت زا و ریسک زا نباشد.

با جمله‌های تضاد دار بایستی بررسی کرد.

تکنیک‌هاش هم صحبت می‌شود.
