مثال برای راه‌حل‌های ریسک:

مثال ۱:

اتفاق بد محتمل این است که راننده قطار در هنگام حرکت به خواب برود. یک راه‌حل برای
صفر کردن احتمال وقوع این ریسک ارائه بدید.

استفاده از راننده دوم احتمال را کم می‌کند، اما استفاده از نرم‌افزار به جای انسان
عمل کند و حتی اتوپایلت برود.

راه‌حل برای احتمال وقوع عاقبت این ریسک را صفر کند.

عاقبت می‌تواند تصادف کردن باشد، یک پیام از خواب بودن راننده قطار یک بدیم به بقیه
قطار‌ها که بعد ز یه فاصله‌ای بایستند.

یک عاقبت دیگر می‌تواند خروج قطار از ریل باشد: بایستی سرعت کنترل شود.

شدت درد را بررسی کنیم. درد را کم کنیم. اگر تصادف شد اتوماتیک به ۱۱۵ تماس بگیرد.
یا آبولانس بین هر یک کیلومتر داشته باشیم.


مثال ۲:

دانشجو نتواند وارد سیستم \lr{LMS} شود.

لایه اول کم کردن احتمال:

برای کم کردن احتمال استفاده از سرور جانبی می‌تواند مورد استفاده باشد.
% با توجه به هزینه برای صفر کردن استفاده از تکنیک‌های لودبالانسیک مناسب در صورتی
% که بتواند دانشجو را به صورت خود کار هدایت کند. یا کلاس حضوری شود.

لایه دوم بیان عاقبت:

کاهش عاقبت: آن است که استاد ۱۰ دقیقه اول کلاس را به دوره درس بپردازد.  چون
دانشجو باعث می‌شود که دیرتر در ابتدای کلاس وارد شود می‌توانیم قانون بالا و
همچنین این قانون را وضع کنیم. دوبار حضور و غیاب انجام شود. در حقیقت قانون در
سیستم داریم میذاریم. چون یک قید داریم وضع کنیم. یا اینکه میتوانیم الگوریتم
بنویسی که دیوریشن دانشجو را بررسی کند که اگر بیشتر از ۵۰ دقیقه کلاس بود حضورش
اوکی میشه.

لایه سوم بیان درد:

چون نتونستم درس بخونم به دلیل عدم ورودم، کلاس ضبط می‌شود.  برای کم کردن درد بعد
از کلاس می‌تواند کلاس ضبط شده را مشاهده کند.

نکته:

دید ما به ریسک بسیار مهم است، ریسکی که می‌توان صفر کرد دیگر نیاز نیست که
استراتژی برای کم کردن احتمالش بیان کنیم.

ممکنه در حالت سوم دیگر وارد نشویم چون در لایه دوم حلش کرده‌ایم.

ممکن است راه‌حلی برای یک لایه نداشته باشیم.

% تایتلر‌های صفحه ۴۸ تا ۵۰ نوشته شود.
کاهش میزان عاقبت را \lr{Mitigate risk consequence}.

به غیر از این ۵ تاکتیک:

استفاده از تکنیک جمع‌آوری اطلاعات استیک هولدربیس

استفاده از الگو‌ها و کار‌های تجربه شده قبلی.

دو تیک خوردن پیام یک الگو برای مطئن شدن از فرستاده شدن پیام از مبدا به مقصد
می‌باشد.

فرقی نمیکنه که برای محصول باشه یا فرایند.

دلیل آنکه همه سیستم‌ها شبیه هم هستند به خاطر آن است که الگوی استاندارد دارند.

خود اون الگو‌ها می‌توانند حاوی ریسک باشد. حالا که طرف میخواد رمز پویا بزنه با
اینکه رمز پویا امنیت را بالا می‌برد اما اگر سرور کند شود یک الگوی حاوی ریسک
تولید شده که قابل تحمل است چرا که امنیت در اولویت بالاتر است.

حالا نوبت انتخاب کردن راه‌حل می‌باشد:

قسمتی بسیار مهم است که سوال امتحانی پذیر است.

مفهوم \lr{DDP} راه‌کاری است که میزان خوب بودن یک راه‌حل را برای ریسک مشخص میکند.

\lr{Defect Detection Prevention}

قبل از ddp یک معیاری سنجیده میشد به نام lr{RRL}.
هر دو میزان خوب بودن ریسک را بیان می‌کنند.

راه‌حل \lr{Countermeasure}

روی دو معیار این خوب بودن را مشخص میکنن:

معیار هزینه راه‌حل

معیار تاثیر راه‌حل

راه‌حل دادن باعث کوچک‌تر شدن توپ تاثیر یعنی این توپه بزرگ‌تر بشه یا کوچک‌تر.
تاثیر پذیری بیشتر بشه توپ کوجک‌تر می‌شود.

RRL(r, cm) = (Exp(r) - Exp(r/cm))/cost(cm)

صورت کسر effectiveness بودن را مشخص می‌کند و مخرج هزینه را مشخص می‌کند. 

در rrl هی بزرگ‌تر باشد و مخرجش کوچک‌تر باشد.

در DDP هم همینطوره ولی فرمولش فرق می‌کنه.

یک راه‌حل بگیم که تو چند توپ ریسک نقش داشته باشه و کوچکش کرده باشد.

استفاده از مجموعه از راه‌حل‌ها برای رسیند به اندازه که به زیر آستانه برسد.

رویکرد rrl این است که کانترمیژن تکی را جواب می‌دهد. باید بررسی کنیم که کانترمیژن
روی چند چیز تاثیر می‌گذارد. 

نقش یک راه‌حل در تمام ریسک‌ها را بررسی نمیکند مدل rrl.

تمام اشکالات را یک به یک بررسی میکند.

تک به تک دیدن کانترمیزن نسبت به ریسک‌ها در rrl انجام میشود که نهایتاً باعث
می‌شود بعضی از چیز‌ها در نظر گرفته نشود.

رابطه cm با ریسک را یک به یک می‌بند.

یک فرض خیلی غلطی میکند که یک ریسک با یک راه‌حل می‌تواند حل شود.

در حالی که یک راه‌حل در کاهش چندین ریسک نقش داشته باشد و از این نظر ارزشمند باشد
و همیشه یک ریسک با چندین راه‌حل می‌تواند مدیریت شود.

رابطه باید
k -> n 
دیده شود تا 1 -> n

روش ddp میتواند به صورت generalization عمل کند.


ماتریس تاثیر ریسک

ابجکتیو‌ها تاثیر میذاره ریسک

ماتریس تاثیر راه‌حل

راه حل روی ریسک تاثیر میذاره

که ریسک‌ها باقی میمونه و راه‌حلها میاد توی ردیف‌ها

تخمین بالانس خب از کاهش ریسک تقسیم بر راه‌حل‌های هزینه

اسلاید قبل از ۶۰ ببین:

تلاقی سطر رو سوتون میشه تاثیر ریسک روی هدف

اگر صفر باشد یعنی هیچ تاثیری ندارد
اگر یک باشد یعنی آبجکتیو تاثیر گذاشته
یعنی ریسکه هدف رو از بین میبره

پس برای بدست آوردن ستون آخر باید افقی حرکت کنیم:

توی حرکت افقی هدف ثابته و ریسک‌ها و تاثیر‌هاشون متغییر هستن.

تاثیر داشته یعنی چقدر هدف از دست رفته

اسلاید ۵۹

weight(object) \sigma r (impact(r, obj) * likelihood(r))

حرکت ستونی ریسک ثابته و تاثیرش روی آبجکتیو بررسی میشه.

اسلاید ۵۸

Criticality(r) = likelihood(r) * sigma object(impact (r, object) x weight(obj))

باید یسری تاکتیک استفاده شود. 

تلاقی سطر و ستون تاثیر راه‌حل را روی آبجکتیو را بیان می‌کند.

صفر باشد که هیچی
۱ باشد یعنی کلا پرهیز کرده از ریسکش

Overall effect of countermeasure

احتمال اهمیتی نداره ولی این ریسک به هر دلیلی چقدر دردسرساز بوده برای مون اهمیت
داره.

سطر آخر دردسر کلش رو مشخص میکنه یا \lr{Risk criticality}

overallEffect(cm) = \sigma r(Reduction(cm, r) x Criticality(r))

ستون \lr{Overall effect of countermeasure} مشخص میکند این راه‌حل چقدر خوب است؟

میتونی بگی که این سبد راه‌حل روی ریسک ۱ تا ان چقدر تاثیر می‌گذارد.

یک ریسکی که به خاطر راه‌حل ‌های مختلف ریدیوس میشه. 

۱ - ریداکش میشه باقی مانده

چرا ضرب میشه؟ چون همپوشانی دارند.

(فرمول ریداکشن هستش)

۱ - باقی مانده = ریداکشن

برای کامبایند میشه بالایی

۱ - ریداکشن میشه باقی مانده. 

ضربه میگه تکه‌های باقی مانده در کاهش یافته

یک سبد پیشنهاد میده و به همراه لیبل‌هاش مطرح میکنه.

۲۰ تا راه‌حل میذاره برای هر کار

جمله نتیجه به صورت \lr{NP Hard} است.

تا اسلاید ۶۷

تا آخر فصل سه میگه که بره فصل هشت.