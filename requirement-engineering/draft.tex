% \newpage 

% اسلاید ۳۸

% در‌های درخت در هنگام حرکت باز نباشد

% در‌های قطار در هنگام حرکت باز باشد.

% عامل باز بودن رو ما نمی‌دونیم 
% پس درخت ریسک می‌خوایم

% اسلاید ۴۰ دیده شود:

% اسلاید ۴۱ تفسیر شود.

% کل لیست میشه \lr{2n+1} به عکس گرفته شده توجه شود که چگونه می‌توان به این قاعده
% رسید.

% چگونه می‌توان اینا رو ترکیب کرد.

% روش‌های ترکیب متفاوت می‌باشد.

% n: میشه عاقبت

% راه‌حل‌های ریسک ۵ تا می‌باشند. یا احتمال وقوع را کم میکنند یا صفر می‌کنند.

% یادت باشه با عکس بازه بین ۱ و ۰ هستش که در ۱۰۰ می‌تواند ضرب شود.

% روی عواقب باید تاکتیک نظیم کنیم.

% Audit
