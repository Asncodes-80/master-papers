% Chapter 3

\section{فصل سوم}

مرحله ارزیابی

بعد از این مرحله باید همه به اجماع رسن از سر تمام چیزایی که انتخاب شده است.

۴ کار اصلی:

مدیریت ناسازگاری

تضاد

نویسنده در یک شرایط خاصی که دو چیز با هم نمی‌سازند را از این کلمه استفاده کرده و
در جاهای از کلمه کانفلیکت استفاده کرده.

اگر بین جمله باشد میشه تضاد: بین swreq, sysreq, ass

اگر بین المان‌ها باشد میشه consistency.


آنالیز ریسک‌ها

اتفاقیی که ممکن است در برابرشون کار‌هایی انجام بدیم یا قابلیت بدیم

محیطی assault
نرم‌افزاری swreq
دامنه‌ای systemreq

فراموشی گذرواژه 
sms شدن assumption میشه
لاگین و فرم فراموشی میشه swreq
ولی در برابر ریسک داریم تمهیداتی رو در نظر می‌گیریم.


ولی هیچ وقت در زمان جمع‌آوری داده ما ریسک‌ها رو بررسی نمی‌کنیم.

می‌تواند در خصوص مجموعه اقداماتی که در سیستم تکرار می‌شوند ریسک‌ها را بررسی
کنیم.

تحلیل گر ریسک در سیستم‌های مشخص مانند سیستم‌ها مالی

نقدم سوم انتخاب بین گزینه‌ها

قدم چهارم

اولویت کردن کار‌ها

همه کار‌ها در یک سطح اهمیت نیستند. پس نیازمندی اولویت بندی هستند. یکی از
مهم‌ترین مثال‌هاش نسخه‌بندی کار‌هاست.

ریسک نات یک جمله است.


\section{ناسازگاری بین المان‌ها و کانفلیکت بین جملات که دومی خیلی مهم است}

پنج دسته بندی داریم
۳ تا ناسازگاری و ۲ تا تضاد دارد.

اسلاید ۸ ۳ تا ناسازاگاری را نشان می‌دهد

ناسازگاری‌های نسبت به المان‌ها:

\subsection{Terminology clash تصادم معنایی}

یک مفهوم است ولی با مفتواتی صدا زده می‌شود. 

مثلا کسی که در دانشگاه درس میدهد یک بار به عنوان مدرس است یک بار به عنوان استاد
است.

هیچ وقت این دو رو نمیشه به هم در کلاس‌ها متصل کرد.

قبل از اینکه جملات تحویل طراح بشه باید این مافهیم به اجماع برسه.

\subsection{Designation clash: تصادم معنایی}

چند مفهوم یک نام. دانشگاه: یه عده‌ای که هستند در دانشگاه کار میکنند رو میگن هیات
علمی به یه عده‌ای که دارن کار میکنن میگن کارمند. قواعد اصلاً با هم متفاوت است.
تعریف‌ها متفاوت ولی داریم میگیم یکی هستن که اشتباهه.

اسمامی متفاوتی باید باشد.

\subsection{Structure clash: تصادم ساختاری}

فرض شود که یک درس آزمایشگاه تعریف کرده‌ایم یه attr دارد به عنوان زمان. در جدول
آزمایشگاهی ها میگیم دو ساعته. یه جا میگیم بین آزمایشگاهی بین ۱۰ تا ۱۲ است. هر دو
زمان هستند ولی تایپشون و ساختارشون متفاوته. از نظر منطق زمان هستند ولی ساختارشون
متفاوته که باعث شکست در سیستم می‌شود.

همه اینا مشکلات را با Glossary حل می‌کنند.

مهندس نیازمندی اینا را مطرح میکنه و طراح اینا رو یاد میگیره. البته طراح هم
می‌تواند کاملش کند. می‌تواند نوع کلاس را به صورت دلخواه به عنوان نوعی مشخصَ، نود
باشد، کلاس باشد، attr باشد.

این سه تا مشکل رو glossary هندلش می‌کنه.

هندل کردن یعنی راست و ریس کردن.

از ontology در سازمان‌ها متوجه میشن که چه تفاوت‌هایی این المان‌ها با هم دارند.

ontology:

هستان شناسی ارتبطا بین معنا‌ها و معنا‌ها و تشکیل یه نود و معنای جدید. واسته به
دامین است.

ریسک مال یه جملست
کانفلیکت برای چند جمله است.

\section{Conflics دو دسته}

بین جملات رخ می‌دهد. 

\subsection{Strong conflict}

در هیچ شرایطی نمیذتوانیم هر دو جمله را باهم در سبد نیازمندی خود نگهداریم.

دانشجو تنها بتواند کارنامه ببیند، استاد هم گفته تو ثتب نمره بتواند کارنامه
دانشجو را ببیند. اگر خواسته دانشجو رو اجرا کنیم، استاد نمی‌بینه
اگه استاد رو بگیریم هر دو دارن می‌بینن
طراح میگه تکلیف من رو روشن کن که بفهومم باید چی طراحی کنم
چه پرمیژنی رو بیارم بالا.

\subsection{Weak conflict}

تا یه جایی همه چیز خوبه
تا اینکه یه شرایطی مرزی پیش میاد که همه چی خراب میشه.

مثل ددلاین‌ها که تایه جایی که رخ نداده مشکلی نداره ولی به محض اینکه از زمانش
بگذره همه رو نارحات می‌کنه.

همه کانفلیکت ها در صورتی که همه شرایط برد-برد باشه رفعشون امکان پذیر نیست.

مدیریت کردن یعنی یجوری جملات رو با هم راضی نگه داریم.


راه حل این دو تا:

مدیریت کردن همم‌ترین رویکرد است.

تکنیک‌ها الگوریتمیک هستند که فرایند‌ها را بهتر می‌کنند.

inter-viewpoints: ربطی به نان فانکشن‌ها ربطی ندارند. قضیه کتابخونه

intra-viewpoint: خواسته‌های افراد مختلف عملیات هستند. معمولا الگوریتمیک هستند و
حلشون فانکشنال هستن.


مدیریت فرایند:
برای استرانگ‌ها وویکها 

۴ قدم اصلی:

شناسایایی عمباراتی که با هم مشترک هستند (اشاره به مفهومی مشترک)

راجع به مفهومی مشترک صحبت بشه

قدم دوم (اسلاید ۱۲)

جملات جمع‌آوری شده
باید ببینمی مشترکات چه نظراتی با هم همپوشانی دارند.
بعد این کانفیلیکت‌هایی که پیدا کردی رو باید داکیومنت کنی.

قدم سوم:

رزولوشن مفهومی است که کتاب برای مدیریت تضاد استفاده می‌کند.

هر راه‌حلی که به ذهن رسید باید کامل گفته شود.

قدم آخر:

باید یکی از اون راه‌حلی‌هایی که ارزیابی کردیم رو بررسی کنیم و بهترین اونا رو
انتخاب کنیم.

راه‌حل باید از جنس سبد باشد سبد همون جمله statement هستش. اولی راه‌حل دراپ
کردنه. جمله رو تغییر بده یا سازگارش کنه. یا اینکه یه جمله را بهش اضافه کنیم.


مهندس نیازمندی باید تو ntra-viewpintها بازه‌ها رو تعیین بکنه.

رزولوشن از جنش statement.

اضافه کردن، حدف کردن و تغییر دادن. پس ممکنه تضاد جدید ایجاد کنند به خاطر همین
سلاید صفحه ۱۲ چرخشی است.

ما باید بدانیم که چه عواملی با هم کانفلیکت دارن و باید مدیریت بشن.

باید بدانین کدام موارد کانفلیکت خیر هستند و بعد از تغییر سبد می‌توانند دردسرزا
بشن.

کانلفلیکت خیزها کسایی هستند که در اورلپ‌های زیادی شرکت داشتن.

باید جملاتی رو ببینیم که توی overlapهای زیادی شرکت داشتن ولی شرکتشون خوب بوده.

در این قضیه اسلاید ۱۹ دیده بشه.

\subsection{Identify overlapping statements}

شباهت می‌تواند دنش فاعل مفعول و فعل باشد.

عملیاتی که در کنار یکدگیر دچار کانفلیکت نمی‌شود.

\subsection{Detect conflicts amoung them, doc these}

تاکتیک‌ها

غیررمسی: به صورت چمشی منطقی جور در نمیاد. استفاده می‌شود. 

استفاده از درخت: طبق یک جدول مشخص میکنه که جملات چطوری می‌توانند مومجب کانلفیکت
بشن.

سومی: رسمی: نرم‌افزار‌های بحرانی را نمی‌توان uml کرد چرا که نیاز به اثبات دارد.

نمایش و اعبتار سنجیش هم با زبان‌های فرماله

سبک شده زبان‌های فرمال میشه پترن‌ها یه چیزی بین دوتا نوتیشن‌های uml داره و
مفهایم رمسی هم داراست. پس به صورت گرافیکاله.

تمرین:

گروه‌های پنج نفره:

pedalogy

از چه تکنیکی استفاده کردیم که با هم کار کنیم:
اطلاعات چطوری چک بشه
چطوری به هم آموزش بدیم
در نهایت تو گزارش بنویسینی که تو یادگیری عمیق آموزشیمون انتخاب کنیم.

روی موضوع Conflict pattern‌ها

تکنیک داکیومنت کردن:

یک جدول بایستی ساخته بشه (صفحه ۱۹)
شرط کانفلیکت داشتن: بایستی اورلپ داشته باشد تا مشخص بشه که داره یا نه

در قطر اصلی همه صفر می‌گیرند مانند جریان بازتابی.

در تقسیم باقی مانده داره کانفلیکت را نشان می‌دهد. خارج قسمت یشه اورلپ‌ها


اسلاید ۲۰ خوانده شود.

تکنیک‌هایی برای رفع کانفلیکت اتفاق میوفته: (جلسه آینده)

دوتا اتست منت بنویسیم که کانلفیکت قوی دارن
دو تا استیت منتی که کانفلیکت ضعیف دارن.