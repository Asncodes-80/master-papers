% Chapter 2

% \subsubsection{\lr{Repertory grids, Card sorts for concept acquisition} یا
% مشبک‌سازی}

% چرا مرتب سازی:
% مشبک کردن جدل بندی کردن

% جمله یک سری اسم را اگر با فعل به هم وصل کنیم میشه جمله کامل
% در سیستم ثبت نام دانشجو باید بتواند درس انتخاب کند.
% یعنی فعل قابل تغییر است یعنی میتنه اسم‌ها ور به هم وصل کند. تا یک جمله بسازی.
% کارت‌ها همه تبدیل به کلاس میشه. 
% تمام کارت‌ها معادل به کلاس است. تمام کلاس‌هایی تو فضای مسئله دیده میشن نه
% فضا‌های راهحل.  راهحل مثل اتصال به دیتابییس.

% لزومی نداره تمام مسائل داخل اسکوپ باشه. 
% دانشجو باید بتواند با استاد خود در تعامل قرار بگیرد.
% جمله‌ها می‌تواند اوت اف د اسکوپ باشد.
% اما روشی سریع است
% ذهن را باز میکند جون با کلمه و فعل در ارتباط است.

گرید بندی کردن کارت‌ها

% بعضی اسم‌ها کلاس مستقل نیستند. بلکه به صورت مشصخه یا attribute یک کلاس هستند.
% قانون سف و سختی برای تشکیل کلاس از روی کارت‌ها وجود نداره.

کلاس با محیط مسئله و طراح همراه است

کتگوری کردن از طراحه
اطلاعات اجمالی از ویژگی‌ها محیطی است.

فضای مسئله: محیطی است یعنی همینی که هست. وضعیت موجود را نمایش می‌دهد.
فضای راه‌حل رو طراح اگه نیاز باشه مطرح میکنه.

شماره دانشجویی چون بایستی از اطلاعاتی محیطی طراحی.

در حقیقت خروجی این پوت‌های ماست

افزودن value که مهندس نیازمندی آن را در نظر می‌گیرد.

attribute رنج دارد. تایپش چیه. اندازش چیه.

سادست ارزانه
ولی خیلی از این جمله‌ها می‌تواند دقت پایین داشته باشد، نامتربط باشد و یا ممکن
است به اسکوپ ما مرتبط نباشه.

کسایی می‌تونن هندل کنن که اسکوپ رو خیلی خوب درک کرده باشن.


\subsubsection{\lr{Scenarios, Storyboards for problem world exploration}}

سناریو، داستان است که میذتواند شرایط کنونی و شرایط آینده را بگوید. 
یعنی system-as-is
system-to-be

سناریو جاهایی استفاده می‌شود که پیچیدگی سیستم زیاد است.

مهمان بودن دانشجو کاملا داستان پیچیده‌ای دارد.

اهمیت سناریو‌ها به شدت بالاست.

چه کاری
کی انجامش میده
چرایی کار 
توضیح چرایی یا what if چه می‌شود اگر. دیدن همان شرایط خاص Exceptionها

سناریو را در sequence دیاگرامی مشخص میکنم که در آن what وجود دارد، what if وجود
دارد ولی why ندارد.

متوازی الاضلاغ چرایی سیستم را مشخص می‌کند.

در خیلی سناریو‌ها 

انواع سناریو‌ها را موقعی که در sequence نشان می‌دهیم در حقیقت بعد why را ندارد.

منفی: کارایی که سیستم نباید بکنه، روز‌هایی که در تقویم هر کدام افراد سازمان
نمی‌تونن شرکت نند باید کار محرمانه باشه. کارایی که نباید انجام بشه.
مثبت: رفتاری که انتاظر داریم. لیست دروس قابل ارائه شده. چه کارایی انجام میده.
نرمال: مجموعه منفی و مثبت‌ها را بیان می‌کنیم بعد what if ندارد. اول نرمال سناریو
رو می‌نویسم بعد برای what ifها از غیرنرمال‌ها انجام میدم. در مورد استیت صحبت میکند.
غیرنرمال: اگه همه چی خوب و خوش نبود باید سناریو غیرنرمال رو بگیم. تمام چیزایی که
اونطوری که ما انتظار داریم پیش نمیره. یه آینده نگری برای سیتسم می‌باشد. در مورد
استیت صحبت میکند.

مزایا و معیاب سناریو:
بعد why در سناریو وجود نداره انگار حذف شده.
قصه گفتن سخته. پیچیدگیش بالاست.
حسنی که دارد اینه که سادست.


\subsubsection{\lr{Prototypes, Mock-ups for early feedback}}

یه بخش‌هایی باید پروتوتایپ باشه. توش مشخص می‌کنیم که چه قابلیت‌هایی داریم.
می‌تونه ui

مستقیم ترین فراینورده ایک ه برای استخراج نیازمندی‌های استفاده می‌شود. عملیاتی در
این قسمت بیشتر دیده میشن.

اطلاعات تکمیلی:
سرچ هممزان با چند چیز

یک کار فانکشناله ولی نان فاکنشنالیش توی یوزر اینترفیسه.

فوکسش برای بحث روی بدست آوردن نیازمندی‌هایی گنگ بوده.

کار تقریبا سریعه تا نشون بدیم چالش‌های سیستمی چی بوده.

سرچ شود که سخت افزار می‌تواند اسم prototype را بگیرد یا خیر. استاد میگوید
testbed.

\subsubsection{\lr{Knowledge reuse: Domain-independent, Domain specific}}

بازیابی دانش در چه حوزه‌ای؟

دانش در مورد سازمان، دامنه و اسکوپ‌ها رو می‌خوایم.

بانکداری‌ها معمولا یه سری رول‌های مشابه ندارند. (پوشا)

نمایش دانش، المان‌های دانشی که تشکیل میدن رو شناسیی کنیم. 

المان‌های دانشی از جنس زیر هستند:

کانسپت
اهداف
تسک‌ها
افراد
نیازمندی‌ها
دامنه‌ها

دانش باید نمایش داده بشه. 

روش نمایش گرافیکی است. به صورت تکست هم هست اما رسمی نخواهد بود.

در خیاطی نمایش دانش را الگو می‌گویند. نمایش گرافیکی برای تسکمون.

سه مرحله:

مثال خیاطی:

retrieve
دانش مناسب را از بقیه سیستم دریافت کنیم.

ممکن الگوی کاملی وجود نداشته باشه

پس ماژپلار براساس الگو پیش می‌رویم.

سیستم ثبت‌نام باشگاه:

آیا عضو گیری داریم. نتها برای باشگاه نیست بلکه یه کاری است که تکرار آن زیاد است.
پس باید المان‌های دانشی وجود داشته باشه که بگه بخش‌هایی داره دانشی وچه
دامین‌هایی داره چه کانسپت‌هایی داره.

راه حل مشترک برای خانواده مشترکی از مسائل

قدم اول پیده کردن الگو است.

قدم دوم transpose:

ماهیتی جنس‌ها فرق دارد.، الگو راهنما است، یسری چیزا به درد میخوره. یه سری چیزا
باید تبدیل بشه. مثال فروش راکت تنیس نسبت به فروش پنیر که ماهیتی فرق دارد.

قدم آخر Validate اعتبار سنجی و تطبیق الگو با نیاز‌هامون.

الگو‌ها بایستی با یکدیگر تطابق ببینند چون ممکن است یه تغییری داشته باشه تا بتونن
کنار همین دیگه قرار بگیرن.


transpose به سه طریق انجام میشه. چرا که سخت‌تره:

نمونه سازی: Instantiation: مستقل از دامنه
خاص سازی: Specialization: دقیق در مورد دامنه میگه. : Domain specific
عمل واجب برای بیان فرمول با کلمات سیستم: reformulation

نمونه سازی در سازمان تطبیق دادنش به سیستم سخته چرا که یک مفهوم کلی را بیان میکند
که باید BP شناخته بشه مدل‌هاش یادگرفته بشه. چون د رمورد domain اصلاً صحبت
نمیکنه.

اینکه خاص دامنه باشه جزئیات بیشتری خواهد داشت.

---
 
دامنه را باید با مثال بیان بشه:

سیستم هدف چیست از چه دامنه‌ای است. دامنه مدیریت منبع
مدیریت کتاب خانه

المان‌های دانشی الگوی انتخاب شده بالا:

کانسپت: کتاب میشه منبع، 
اسلاید مربوط به آن خوانده شود.

borrow -> resource user

دامنه اگه تغییر بکنه دیگه شرایط و خاص بودن نمیتونه همان نمونه قبلی باشد (الگو
تغییر میکند.)

روش مستقل از دامنه:

در نظر گرفته سلسله موراد نیازمندی‌ها
بازیابی متا مدل‌ها

نان‌فانکشنال‌ها مستقل از دامنه هستند، مانند امنیت و الگوریتم جست و جو

متامدل‌ها در مورد فانکشین‌ها هستند که در سطح سازمان عمل میکنه.

تاکسونومی یا نان فانکشنال:
درخت بررسی شود.

نیازمندی‌های فانکشنالی که در سطح سازمان انجام میشه متامدل است.

چون جزئیات رو نمیگه خیلی کمتر استفاده می‌شود.


\subsubsection{\lr{Interviews}}

تمام سوالاتی که برای آن‌ها جوابی نداریم را در بخش \lr{Interview} می‌پرسیم. مثلا
بعد از ثبت‌نام کاربر برای او اعلاناتی ارسال شود؟ یا مثلاً می‌گویم که در سناریو
تغییر کلاس کاربر اعلانات ارسال شود.

پرسشنامه هیچ وقت کامل نمیشه. برای کامل شدنش نیاز به مصاحبه داره. بدهیات در
مصاحبه پرسیده نمی‌شه.

حالات ساخت یافته و غیر ساخت یافته دارد:

ساخت یافته براساس سوال است که پرسیده میشود.

بدون ساختار گفت و گو آزاد و بدون پیش در آمدی است.

\subsubsection{\lr{Observation and ethnographic studies}}

تنها system-as-is مشاهده میش‌شود. هم فانکشن‌ها دیده میشه هم نان فانکشن‌ها
استخراج میشده.  حالا انجام کار افراد‌ها را بصری در صورت آنها می‌توان دید.
چهره خسته کارمندان.

می‌تواند به صورت غیر فعال باشه: ناظر باشد
می‌تواند به صورت اکتیو به صورت درگیر شدن با فرآیند‌ها که همراه با یادداشت برداری
است.


\subsubsection{\lr{Group sessions}}

ساخت یافته: هر کسی زمان دارد، همه چیز به صورت مشخص می‌باشد. کاملا نقش‌ها مشخص
است.

ساختار نیافته برای brainstorme: خودش انگار داره صورت جلسه می‌نویسه.

ساسنور و مسخره کردن وجود ندارد.
