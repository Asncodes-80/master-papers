یک عملی عامل انجام میشه که توی یه کلاس تغییرات اعمال شده رو تاثیر میذاره.

دیاگرام کلاس می‌خوایم

توی uml دیدیم

با دیاگرام کلاس با تو فاز طراحی فرق میکنه ولی جزئیات نداره.

incapsulaction ندارد

method ندارد

abstraction ندارد

اونا را طراح باید تصمیم بگیره

یک نمای کلی از کلاس‌های فضای مسئله

فضای راه‌حل مال طراحه

نقطه گذاری انجام میشه که بتونیم به صورت دنبال به هم وصل کنیم.

فقط کلاس
اتریبیوت‌های کلاس
تعریف یک کلاس Annotation
ارتباطات کلاس از نوع

\section{Assosiation یا انجمنی}

رابطه بین دو کلاس که از طریق فعل نمایش داده می‌شود.


\section{Inheritance وراثت}

یک کتاب هست یک محتوای آموزشی

یک فیلم هست یک محتوای آموزشی

یه پدری دارند که ویژگی‌های کلی توی کلاس پدر نوشته میشود و بچه‌ها به ارث می‌برند.

اساساً در قدم‌های ابتدایی مسئله دیده می‌شود.

\section{Composition ترکیب}

گاهی وقتا یک شی مستقیل نداریم، شی به نام ماشین نداریم. هویت کلاس باید ایجاد شود.
پیام تشکیل شده از بدنه، هدر و فوتر. اگه این سه تا رو بذاری کنار هم میشه پیام. یک
چیز مستقل نیست. سفارش هم مسقل نیست. رابطه کل به جز هستش. از کل به جز بخونیم میشه
کلاس کل شامل این کلاس اجزاست. 

سشن میشه کل و جز میشه واریزها وغیره توی ATM

سشن تموم شد باید دیستوری

باید ببینیم نیاز سیستم چیه که چه اتفاقی بیوفته.

اینا قانون نداره و تصمیمه. یک راهنمایی از نظر تصمیم‌گیری بهت میشه.

حیات جز به حیات کل وابسته است.

\section{Aggrigation تجمیع}

شبیه ترکیب هستش و هر دو یه جور خونده میشه. 

تجمیع میگه کل از بین بره جز از بین نمیره، ترکیب میگه که کل ازبین بره اجزا هم از
بین میره.

کل مرد به جز کاری نداره.

\section{Multiplicity یا تعدد }

تعداد نمونه‌هایی از یک کلاسه که با تعداد نمونه‌های یک کلاس دیگه که باهاش توی
رابطه است شرکت.

تعریف میکنیم حداقل یک

مثل n...m است

+ 0...*

+ *

تعدد از کجا اومد؟ تعدد چیست؟

از دو طریق ایجاد میشه.

Assumption

System requirement

Domain properties

در کل جملات Prescriptive و Descriptive

وصل به گول باید بشه هر المانی.

همه المان‌ها باید توجیه داشته باشن.

بتواند ۵ تا کتاب قرض بگیرد. میشه Prescriptive و اینکه شرط میذاریم رو توی System
requirement می‌باشد.

تعدد‌ها همه تو سیستم توجیه پذیر هستن.

\section{کلاس انجمنی}

یک کلاسه و رابطه نیست. که به کلاس انجمنی متصل است.

با نقطه چین به کلاس انجمنی متصل میشه.

بیشتر توی کلاس‌های n به n هستش.

یه دانشجویی کتابی رو قرض می‌گیره و تایم دیوریشنش ۲ هفتس.

با یه مثال:

یک فایل داریم

اسم کلاس دسترسیه
میتونه دانجشو‌های مختلفی داشته باشه که به فایل‌های مختلفی دسترسی داشته باشه.

سطوح دسترسی به هر فایلی متفاوت است. علی نسبت به فایل اکس خواندن و نوشتن می‌تواند
انجام دهد، رضا نسبت به فایل اکس می‌تواند تنها عمل نوشتن را انجام دهد.

این اتریبیوت اکسس تنها بین رابطه تعریف می‌شود:

\lr{Ali(id, Ali, other properties);}

\lr{Book1(bid, Book1, other properties);}

\lr{Assosiation(id, bid, R/W);}

نکته: هر نمادی باید وصل بشه به گول.

از تعریف گول تمام این المان‌ها را بدست می‌آوریم.

مثال قطار و بلاک رو بنویس از ارائه فصل ۱۰

گاهی فعله شامل شدنه و آن و فالو کردن نیست بلکه باید به صورت ترکیبی یا تجمیعی
باشد.

همه دامین پراپرتری‌ها تعدد رو نمیگن البته.

همه المان‌ها به کلاس ارتباط دارد.

می‌تواند چیز اضافی در سبد وجود داشته باشد چرا که ممکن است یه سیستم جامع از پیش
طراحی شده را در سیستم جاری بخواهیم الگو برداری کنیم که به یسری چیزاش نیاز داریم
به یسری چیزاش نیاز نداریم و میذاریمش کنار.

\section(\lr{Agent Diagram})

از این بخش سوال امتحانی خواهیم داشت.

عامل‌ها براساس وظایفشون مشخص میشه چی رو ببین و چی رو نبینن.

با تحمیل زنجیره آسیب مشخص میشه که تو ایجنت فعلی که داریم استفاده میکنم یا غلط
دادیم وظیفه رو یا تو انجام وظیفش تخوب نیست باید به عامل دیگه‌ای بدیمش.

کنترل یک ایجنت مانیتور یه ایجنت دیگست.

سه دسته دیاگرام عامل داریم:

Agent diagram کامل ترینه که گول داریم اجینت داریم و کلاس داریم.

Context diagram: همون مجموعه ارتباطی آسامپشن‌ها و غیره هستند.

depedency diagram: برای زنجیره آسیبه

اسلاید هشت فایل ۱۱

Ag1

مسئول یه گوله

زنجیره آسیب ارتباطات اینتپونت هر عامل دیگر را مشخص میکند.

همیشه از انتها به ابتدا می‌خوانیم.

تمرین:

در یک مثال یک گول در نظر بگیرید و یک زنجیره آسیب براش بسازید.

نوشته می‌شود. (نسبت به سکشن‌هایی بالایی ببین ببخشید واقعاً)

نهاد کلاس‌های تغییر میکند.

بقیش دیگه تو طراحی می‌مونه

قط کلاس‌ها و اتریبیوت‌های که از قضای مسئله می‌آیند.