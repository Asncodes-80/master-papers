% Continue of Chapter 3
اسلاید ۳۶ در کنترل لیست این چهارتا آیتم به شدت مهم است.

ریسک بشه و اتفاق بیوفته بایستی عواقب بعد از آن کنترل بشه. سعی بکنیم آثارش رو
کمتر کنیم یا حداقل مدیریت انجام بدیم.

نکته: قدم‌های مدیریت ریسک متفاوته

وای برای خود ریسکه

هوا برای پیامد‌ها و بعد از درده

کامپوننت‌ها زیاد بزرگ نباشد، باید مشتق بگیرم. چون تمام ۴ تا سوال را خیلی سخت
می‌توانیم پاسخ دهیم.

\subsection{درخت ریسک}

فصل نهم تدریس میشه.

همه نود‌های درخت ریسک هستند. از روت شروع میکنه به ریسک‌های کوچیک‌تر می‌شکند.

مثلا خونه آتیش بگیره.

انفجار گاز

اتصالی سیستم برق داخلی

یک اتفاق بزرگ و بد که چند عامل کوچک تاثیر گذار روی آن خواهد بود.

صرفاُ برای یک ریسک نوشته نمی‌شود. زمانی که سطح پیچیدگی اگر زیاد باشه نیازمند
کشیدن درخت خواهیم بود. تا وقتی که قابل فهم شود.

تنها برای یک ریسکه که خیلی بزرگه.

که ببینیم می‌تونیم کنترل کنیم یا با سطح تحمل آستانه ما حل میشه.

بعضی از اون معیار‌ها آماری هستند. طی اون ۵۰ سال مثلا دو بار آتیش گرفته.

همه صنایع ریسک وجود دارد.

نود یا با مستطیل نمایش داده میشه یا بیضی

نماد‌ها اند و اور هستند.

اگه نیاز به شکست باد مستطیل اگه دیگه آخرش باشیم و بلد باشیم که ریسک رو بدونیم و
نیازی به شکست نداره از بیضی استفاده می‌شود.

بچه‌ها همیشه بیضی هستند.

اسلاید ۳۸

در‌های درخت در هنگام حرکت باز نباشد

در‌های قطار در هنگام حرکت باز باشد.

عامل باز بودن رو ما نمی‌دونیم 
پس درخت ریسک می‌خوایم

برای اینکه پیدا کردن اتفاقاتی که اتفاق بالایی را می‌سازد ساده‌تر باشد درخت ریسک
را با یک الگوریتم تبدیل به درخت کات ست می‌کنیم.

اسلاید ۴۰ دیده شود:

اسلاید ۴۱ تفسیر شود.

میزان بزرگی توپ ریسک میزان بزرگی خطر را نشان نمی‌دهد.

احتمال وقوع ریسک اگر کم باشد نمیتوانیم بگوییم که میزانش کمتر بوده.

احتمال ممکنه که کم باشد ولی اگر رخ بده بدبخت بشیم.

ریسک اتفاق میوفته و شدتی که توی محیط هم رخ میده هم مهم است. چقدر از دست دادیم.
چقدر مشکل داریم. درد سوریتی یک بازه می‌باشد.

کل لیست میشه \lr{2n+1} به عکس گرفته شده توجه شود که چگونه می‌توان به این قاعده
رسید.

چگونه می‌توان اینا رو ترکیب کرد.

روش‌های ترکیب متفاوت می‌باشد.

n: میشه عاقبت

راه‌حل‌های ریسک ۵ تا می‌باشند. یا احتمال وقوع را کم میکنند یا صفر می‌کنند.

یادت باشه با عکس بازه بین ۱ و ۰ هستش که در ۱۰۰ می‌تواند ضرب شود.

روی عواقب باید تاکتیک نظیم کنیم.

Audit

درد صفر شدنی نیست پس کم شدنی است. چون اگر عواقب هم مورد بررسی قرار بگیرد
می‌تواند درد داشته باشه اگه درد نداشته باشه یعنی سیستم ما نیست.

جعبه کمک‌های اولیه تسریع اتفاقاتی است که بعد از اتفاق رخ داده است.

\subsection{استفاده از تکنیک‌های جمع‌آوری و استخراج}

اونایی که استیک هولدر هستند در قسمت استخراج استفاده میشه. که از اکسپرت‌هاشون
بپرسیم که چه خبره ازشون زیاد یاد بگیریم. مثل گروپ سشن و اینترویو.

ریسک افزایش سرعت در حوزه حمل و نقل قطار
فاصله بین دو قطار رعایت نشود
ناتش میشه نشود.

ارزیابی

کنترل

رزولوشن راه‌حل تضاد بود

و کانترویژن راه حل و کنترل ریسکه

اول راه‌حل رو ارزیابی کن
بعد بهترینشو انتخاب کن.
