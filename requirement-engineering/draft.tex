% Overall effect of countermeasure

% احتمال اهمیتی نداره ولی این ریسک به هر دلیلی چقدر دردسرساز بوده برای مون اهمیت
% داره.

% سطر آخر دردسر کلش رو مشخص میکنه یا \lr{Risk criticality}

% $overallEffect(cm) = \sigma r(Reduction(cm, r) x Criticality(r))$

% ستون \lr{Overall effect of countermeasure} مشخص میکند این راه‌حل چقدر خوب است؟

% میتونی بگی که این سبد راه‌حل روی ریسک ۱ تا ان چقدر تاثیر می‌گذارد.

% یک ریسکی که به خاطر راه‌حل ‌های مختلف ریدیوس میشه. 

% ۱ - ریداکش میشه باقی مانده

% چرا ضرب میشه؟ چون همپوشانی دارند.

% (فرمول ریداکشن هستش)

% ۱ - باقی مانده = ریداکشن

% برای کامبایند میشه بالایی

% ۱ - ریداکشن میشه باقی مانده. 

% ضربه میگه تکه‌های باقی مانده در کاهش یافته

% یک سبد پیشنهاد میده و به همراه لیبل‌هاش مطرح میکنه.

% ۲۰ تا راه‌حل میذاره برای هر کار

% جمله نتیجه به صورت \lr{NP Hard} است.

% تا اسلاید ۶۷