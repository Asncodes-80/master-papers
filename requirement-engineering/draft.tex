% Chapter 2

transpose به سه طریق انجام میشه. چرا که سخت‌تره:

نمونه سازی: Instantiation: مستقل از دامنه
خاص سازی: Specialization: دقیق در مورد دامنه میگه. : Domain specific
عمل واجب برای بیان فرمول با کلمات سیستم: reformulation

نمونه سازی در سازمان تطبیق دادنش به سیستم سخته چرا که یک مفهوم کلی را بیان میکند
که باید BP شناخته بشه مدل‌هاش یادگرفته بشه. چون د رمورد domain اصلاً صحبت
نمیکنه.

اینکه خاص دامنه باشه جزئیات بیشتری خواهد داشت.

---
 
دامنه را باید با مثال بیان بشه:

سیستم هدف چیست از چه دامنه‌ای است. دامنه مدیریت منبع
مدیریت کتاب خانه

المان‌های دانشی الگوی انتخاب شده بالا:

کانسپت: کتاب میشه منبع، 
اسلاید مربوط به آن خوانده شود.

borrow -> resource user

دامنه اگه تغییر بکنه دیگه شرایط و خاص بودن نمیتونه همان نمونه قبلی باشد (الگو
تغییر میکند.)

روش مستقل از دامنه:

در نظر گرفته سلسله موراد نیازمندی‌ها
بازیابی متا مدل‌ها

نان‌فانکشنال‌ها مستقل از دامنه هستند، مانند امنیت و الگوریتم جست و جو

متامدل‌ها در مورد فانکشین‌ها هستند که در سطح سازمان عمل میکنه.

تاکسونومی یا نان فانکشنال:
درخت بررسی شود.

نیازمندی‌های فانکشنالی که در سطح سازمان انجام میشه متامدل است.

چون جزئیات رو نمیگه خیلی کمتر استفاده می‌شود.
