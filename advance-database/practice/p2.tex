\documentclass[a4paper]{article}
\usepackage{listings}
\usepackage{hyperref}
\usepackage{plantuml}
\usetikzlibrary{arrows, shapes, automata, petri, positioning, calc}
\usepackage{graphicx}
\usepackage{ragged2e}
\usepackage{color}
\usepackage{xepersian}
\usepackage{subfiles}
\settextfont[Scale=1]{XB Roya}
\renewcommand{\baselinestretch}{1.5}

\begin{document}
\centerline{پایگاه داده پیشرفته تمرین دوم}
\centerline{دکتر شجاعی مهر}
\centerline{علیرضا سلطانی نشان}
\centerline{\today}

\section{سوال اول}

براساس گراف سابقه ترسیم شده، برای یک زمانبندی، آیا این زمانبندی پی در پی پذیر در
برخورد است یا خیر؟ پاسخ خود را توضیح دهید

برای پی در پی پذیر در برخورد می‌باشد:

\begin{LTR}
    \begin{table}[h]
        \centering
        \begin{RTL}
            \caption{نمونه‌ای از فرایند ACA}
        \end{RTL}
        \scalebox{0.7}{
            \begin{tabular}{c|c|c|c|c|c|c|c|c}
                $T_{1}$ & & & & & & & & \\
                $T_{2}$ & & & & & & & & \\
                $T_{3}$ & & & & & & & & \\
                $T_{4}$ & & & & & & & & \\
                $T_{5}$ & & & & & & & & \\
            \end{tabular}
        }
    \end{table}
\end{LTR}


\section{سوال دوم}

برای هر یک از سطوح انزوای زیر یک مثال از زمانبندی ارائه دهید، که در آن Isolation
باشد اما پی در پی پذیر در برخورد نباشد.

\begin{enumerate}
    \item \lr{Read uncommitted}
    \item \lr{Read commited}
    \item \lr{Repeatable read}
\end{enumerate}

\end{document}