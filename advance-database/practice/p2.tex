\documentclass[a4paper]{article}
\usepackage{geometry}
\usepackage{listings}
\usepackage{hyperref}
\usepackage{plantuml}
\usetikzlibrary{arrows, shapes, automata, petri, positioning, calc}
\usepackage{graphicx}
\usepackage{ragged2e}
\usepackage{color}
\usepackage{xepersian}
\usepackage{subfiles}
\newgeometry{left=2cm, right=2cm, bottom=2cm}
\settextfont[Scale=1]{XB Roya}
\renewcommand{\baselinestretch}{1.5}

\begin{document}
\centerline{پایگاه داده پیشرفته تمرین دوم}
\centerline{دکتر شجاعی مهر}
\centerline{علیرضا سلطانی نشان}
\centerline{\today}

\section{سوال اول}

براساس گراف سابقه ترسیم شده، برای یک زمانبندی، آیا این زمانبندی پی در پی پذیر در
برخورد است یا خیر؟ پاسخ خود را توضیح دهید

بله پی در پی پذیر در برخورد می‌باشد زیرا هیچ گونه حلقه‌ای بین تراکنش‌ها وجود
ندارد. همانطور که قبلا مفصل توضیح داده شد با وجود یک حالت ناسازگار و زمانی که
بین دو تراکنش حلقه‌ به وجود آید آن تراکنش پی در پی پذیر در برخورد نیست.


\section{سوال دوم}

برای هر یک از سطوح انزوای زیر یک مثال از زمانبندی ارائه دهید، که در آن Isolation
باشد اما پی در پی پذیر در برخورد نباشد.

\begin{enumerate}
    \item \lr{Read uncommitted}
    \item \lr{Read commited}
    \item \lr{Repeatable read}
\end{enumerate}


\subsection{\lr{Read uncommitted}}

در این روش تراکنش $T_{j}$ مقداری را میخواند که در ابتدا کامیت شده است. در صورتی
که در تراکنش $T_{i}$ مقدار جدیدی نوشته شود ولی آن تراکنش کامیت نشده باشد دائما
مقدار قبلی خوانده می‌شود تا زمانی که تراکنش $T_{i}$ مقدار جدید خود را کامیت کند.

\begin{LTR}
    \begin{table}[h]
        \centering
        \scalebox{0.9}{
            \begin{tabular}{c|c}
                $T_{i}$ & $T_{j}$ \\ \hline
                W(A) &  \\
                 &  \\
                 & R(A)  \\
                W(A) &  \\
                W(A) &  \\
                 & R(A)  \\
            \end{tabular}
        }
    \end{table}
\end{LTR}

برای مثال اگه تراکنشی مقدار ۵ را خوانده باشد، در اولین تراکنش که نوشتن روی منبع
(A) وجود دارد هیچ کامیتی صورت نگرفته پس در تراکنش بعدی مقدار قبلی یعنی ۵ خوانده
می‌شود. بعد از دو بار نوشتن مجدد اما چون هیچ کامیتی صورت نگرفته بازم مقدار ۵
خوانده خواهد شد. با وجود حلقه در این مثال می‌توان گفت که این مثال در حالت انزوا
تعریف شده اما پی در پی پذیر در برخورد نیست.

\newpage

\subsection{\lr{Read commited}}

\begin{LTR}
    \begin{table}[h]
        \centering
        \scalebox{0.9}{
            \begin{tabular}{c|c|c|c|c|c|c}
                $T_{i}$ & W(A) &  & W(Q) & W(Q) C & & R(F)\\ \hline
                $T_{j}$ & & R(A) & & R(Q) & W(F) C & \\
            \end{tabular}
        }
    \end{table}
\end{LTR}

در این روش انزوا وجود دارد چرا که به صورت همزمان میتواند هر دو تراکنش انجام شوند
و به مقادیر یک دیگر دسترسی داشته باشند. اما در ادامه اگر گراف آنها ترسیم شود در
هنگام خواند منبع (F) امکان ایجاد یک حلقه بین دو تراکنش وجود خواهد داشت.

\subsection{\lr{Repeatable read}}

در این روش دو حالت وجود دارد:

\begin{enumerate}
    \item  آیا در حال خواندن مقدار به صورت تکرار پذیر در یک خط تراکنشی هستیم که
    در تراکنش مقابل آن کامیتی صورت گرفته است؟
    \begin{LTR}
        \begin{table}[h]
            \centering
            \scalebox{0.9}{
                \begin{tabular}{c|c}
                    $T_{i}$ & $T_{j}$ \\ \hline
                    W(A) & R(A)  \\
                     C & R(A)  \\
                     & R(A)  \\
                    W(A) & R(A)  \\
                    C & R(A)  \\
                \end{tabular}
            }
        \end{table}
    \end{LTR}

    \item هیچ کامیتی صورت نگرفته و تراکنش مقدار قبلی خود را به صورت دائمی در حال
    خواندن می‌باشد.
    \begin{LTR}
        \begin{table}[h]
            \centering
            \scalebox{0.9}{
                \begin{tabular}{c|c}
                    $T_{i}$ & $T_{j}$ \\ \hline
                    W(A) & R(A)  \\
                    & R(A)  \\
                    & R(A)  \\
                    W(A) & R(A)  \\
                    & R(A)  \\
                \end{tabular}
            }
        \end{table}
    \end{LTR}
\end{enumerate}

\end{document}