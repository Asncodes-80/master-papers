یکی از ابزار‌های خیلی معروف تست گراف هستش.

از کاربرد‌های اصلی گراف آزمون نرم‌افزار می‌باشد.

فصل ۲۱ و ۲۲

نه تنها سورس کد

بلکه از اون روزی که به ما گفته میشه که نرم‌افزار باید آزمون کنیم.

قید و بند‌هایی که میدن رو ما نمیتونیم بفهممیم که با سناریو‌هایی که وجود داره
برای پیاده‌سازی نرم‌افزار کانفلیکت دارن یا نه پس از زبان فرمال استفاده می‌کنیم
تا تست و تاییدش انجام شود و سپس بعد از آن از usecase control flow sequence
diagrams و همگی باید مو به مو تست بشن. خود برنامه‌ای که برنامه‌نویسان می‌نویسن
هم کد میشن و اونا رو باید تست کنیم.

نباید تهی باشه یک گراف از شروط اصلی گراف هستش.

لبه‌ها و رئوس بایستی متناهی باشن.

گره‌های ابتدایی باید داشته باشین initial nodes

تمام برنامه‌های حاوی شروع و پایان هستن.

وقتی یه گرافی برای سیستمون می‌کشیم کاوریج اون گراف برامون مهم هستش.

اینکه چقدرشن باید کاور بکنیم رو می‌تونیم کاملاً روی گراف تعریف کنیم.

که چند بار باید یک مسیر رفته شود اشاره به حلقه‌ها

چه مسیر‌هایی باید بررسی شود.

چه ورودی‌ها و خروجی‌هاییی باید آماده شود.

تست‌کیس‌هایی که (در رابطه بالا) میدیم تنها یک نوع نیستند بلکه بیشتر ورودی‌ها رو
برای خروجی‌های مختلفی تعریف می‌کنیم.

اگه گراف رو فول کاوریح کنیم بهترین حالته اما به هزینش می‌ارزه؟

کاوریج‌هایی که ابداع شده رو می‌خوایم الان بررسی کنیم:

ریترن یه نقطه پایانی.

خروجی نداشته باشه و نود فاینال داشته باشه.

تعاریف گراف رو داره میگه.

reach: یک گراف اگه از یک نود قابلیت دسترسی داشته باشن رو میگن.

اسلایدی که در مورد ریچ گفته رو قشنگ بنویس.

ریچ میتونه هم از طریق نود اتفاق بیوفته هم لبه

هر جایی که میگه ریچ مثلا چند نقطع میگه که الا بلا باید گراف از اون دو نود یا لبه
عبور کنه.

عکس گرفتم این نکتات رو نسبت به عکس ببین.

\subsection{Test Path}

مسیری است که از نود initial استارت میشه و با یک نود انتهایی تمام میشه.

چیزی که پای تخته گفته در رابطه با شکل گراف یک if و جامپ کردنش که میتونه باعث شکل
متفاوتی از گراف رو تشکیل بده.

SESE -> Single Entry Single Exit

یه نود ابتدایی داره و یه نود انتهایی

\subsection{Visiting and Touring}

p یک مسیر است اگر نود  را در این مسیر مشاهده کند.

Tour کردن برعکس بالاییه

یه سابگرافی هستش که تور میکنه

از حروف کوچیک اگه استفاده کنیم منظور یک دونه تست کیسه t

اما اگه T باشه میشه مجموهعه تست کیس‌ه

N -> محوع نود‌ها
n -> یک نود به خصوص

اسلاید شماره ۹ دیده شود.

این مسیر این این تست با هم صادق هستن که میشه statisifaction

Stracuctural coverage critria: 

یک گراف برنامه رو بهت نشون میده.

Data flow coverage criteria

روی یال‌هاش و ود هاش لیب داره که مقادیر رو مینویسه.

همه چیز رو در رابطه با دیتاش نشون میده.

تعاریف تموم شد.

روش‌های تست یا test coverage

node coverage or nc

تستی بدی که حتما به هر نود یک بار رو سر بزنه.

تنها یک تست نیستا یه تست کیس میدیم که از تک تک نود یک بار حرکت کنه

Edge coverage

تک تک لبه‌ها رو بیا تست کن.

ضریب اطمینان در edge coverage بیشتر میشه
به عکس گراف ۱۲۳ مراجعه کن خیلی قشنگ توضیح داده.

نود کاوریج یه زیر مجوعه‌ای از اج کاوریجه

TR -> Test Requirement

روش داره فورس میکنه که روش نود کاوریج رو باید تک تک نود‌ها رو باید ویزیت کنه.

x = 2 

x = 8

گراف تک نود اج کاورج نداره.

Edge pair coverage

جفت جفت رد بکنه بشه test rq

مسیر‌هایی به طول ۲ رو انتخاب کنه.

Complete Path Coverage (CPC)

هدف نهایی تست است که تمام مسیر‌ها رو پوشش بده.

Specified Path Coverage (SPC)

با یه متغییری که تعریف میشه میتونیم تست بکنیم لزومی نداره که کلش تست بشه.

% اگه تو nc بودیم نود تکراری دیدم رو هیچ وقت تکراری نمی‌نویسم یعنی یه بار دیدیمش
% دیگه.

مثالی که توی اسلاید هستش رو بنویس.

باید توی edge pair مسیر رو بنویسیم

(0, 1), (0, 2)

Test path: از نقطه شروع حتما شروع و حتما بایستی به یه نقطه پایانی برسن.

مجموع نقاط باید staticify بکنن.

Complete ها شبیه عملکرد Star یا * در ماشین‌ها هستش.

test path ها در Complete نا متناهی هستش اگر لوپ داشته باشه.

وقتی لوپ داریم ته گراف بازه چون میتونه چندین بار بچرخه.

Thus cpc is not feasible when software's graph has loop.

spc کاملاً وابسته به شروطی هستش که بهش وارد میشه.

objective: 

توش هیچ حرفی نیست و یه فکت در حقیقت دامنه تستمون هستن.

subjective:

ولی کاملاً دلخواه و اصلاً معیاری نداره و معیارش کاملاً وابسته به دیدگاهه.


SPC is not statisfactory because the results are subjectives and vary with the
testre.

