\documentclass[a4paper]{article}
\usepackage{forest}
\usepackage{float}
\usepackage{geometry}
\usepackage{listings}
\usepackage{hyperref}
\usepackage{makecell}
\usepackage{algorithm}
\usepackage{algpseudocode}
\usepackage{plantuml}
\usepackage{graphicx}
\usepackage{ragged2e}
\usepackage{color}
\usepackage{xepersian}
\usepackage{subfiles}
\usetikzlibrary{shapes, arrows, positioning}
\newgeometry{left=1.4cm, right=1.4cm, bottom=2.0cm, top=2.0cm}
\settextfont[Scale=1]{XB Roya}
\renewcommand{\baselinestretch}{1.5}
\definecolor{dkgreen}{rgb}{0,0.6,0}
\definecolor{gray}{rgb}{0.5,0.5,0.5}
\definecolor{mauve}{rgb}{0.58,0,0.82}
\definecolor{commentColor}{rgb}{0.6,0.6,0.60}

\title{آزمون نرم‌افزار \\ خانم دکتر فرشته جدیدی میاندشتی}
\author{علیرضا سلطانی نشان}

\begin{document}
\maketitle
\tableofcontents

\section*{مجوز}

به فایل license همراه این برگه توجه کنید. این برگه تحت مجوز GPLv3 منتشر شده است
که اجازه نشر و استفاده (کد و خروجی/pdf) را رایگان می‌دهد.

\section{معرفی}

میخواد ببینه که تست چیه چه کاری رو باید تو درس تست انجام بدیم و آینده تستینگ
چیه.

ضریب اطمینان
هیچ چیز در امیت به طور ۱۰۰ درصد نیست.

برای سیستم‌های کریتیکال باید اکسپشن هندلینگ به طور مناسب انجام بشه.

بودجه‌ای که برای تست داریم استفاده می‌کنیم کم می‌باشد و عاقلانست که بودجه زیادی
را برای تستینگ اختصاص بدیم تا از اتفاقاً قبل از رخداد جلوگیری کنیم.

بیشتر اوقات نسخه آلفا را کاربر می‌آزماید.

پیش پا افتاده‌ترین مشکلات را کاربر متوجه می‌شود.

با روشهای معمول آشنا میشیم که بالاترین ضریب اطمینان رو به ما می‌دهند.

اگه برای تست کردن هزینه‌ای در ابتدا نشود مطمئناً هزینه بسیار زیاد در آینده
خواهند داد.

مدلی را نرم‌افزار مشخص می‌کند و تمام راه‌ها را در آن بررسی می‌کند.

سعی می‌کنیم تستی که کاور خوبی داشته باشه رو تعیین کنیم.

انواع روش‌های تست را مرور می‌کنیم و آن‌ها را در کار به صورت کاربردی بررسی خواهیم
کرد.

زمان زیاد منجر به افزایش هزینه‌ها می‌شود.

ملاک تست

هر تستی یک test requirement را در آن تعیین می‌کنیم که ببینم برای آزمایش چه 

تعریف می‌کنیم که چی می‌خواهیم تست کنیم کدام مفاد، چه چیزی برامون مهمه

برای هر روش تستی یک test requirement تعریف می‌شود.

برای مثال یک مگا اپلیکیشن که امکانش نیست که همه دستورات و شاخه تست شود. فقط
قسمت‌های تصمیم‌گیری. مثلا فقط حلقه‌ها تست شود فقط شروط بررسی شود که چه خروجی
دارند. چون بررسی کردیم که احتمال خطا در آن‌ها زیاد‌تر است.

test critriation

مجموعه‌ای از قوانین وفرایند‌هایی که test requirement ها را تعریف می‌کند.

test case

دقیقا ورودی‌هامون رو چی بدیم و چه خروجی‌هایی رو انتظار داریم.

جریان تست کیس جریان zero division exception هستش.

اسلایدی رو بخون که ساختار ۴ راه مدل‌های نرم‌افزاری رو می‌گه.

شکل دوم که عبارات منطقی رو میگه قبل از هر گونه تست نوشتن و برنامه نویسیه.

TODO: Coverage Overview tree

اینکه شکست در نرم‌افزار چیه

\lr{Error}

خطا‌های موجود در سیستم را گویند.

\lr{Fault}

\lr{Bug}

اوایل به مشکلات سخت‌افزاری میگفتن و بعد از آن در نرم‌افزار هم ورود کرد.



\end{document}