\documentclass[10pt, a4paper]{article}
\usepackage{geometry}
\usepackage{float}
\usepackage{listings}
\usepackage{hyperref}
\usepackage{makecell}
\usepackage{xepersian}
\newgeometry{left=1.4cm, right=1.4cm, bottom=2.0cm, top=2.0cm}
\settextfont[Scale=1]{XB Roya}

\title{تفاوت میان \lr{Failure} و \lr{Fault} \\ استاد: خانم دکتر فرشته جدیدی}
\author{علیرضا سلطانی نشان}

\begin{document}
\maketitle

\section{\lr{Defect} یا نقص}

زمانی رخ می‌دهد که بین خروجی واقعی و خروجی مورد انتظارمان تفاوت وجود داشته باشد
\cite{jpoint}. این نوع نقص‌ها را مهندسان آزمون نرم‌افزار مطلع می‌شوند و آن‌ها را
به توسعه‌دهنده کد آن برنامه ارجاع می‌دهند که بایستی اصلاحات را انجام دهد.

نقص‌ها انواع مهمی دارند:

\begin{itemize}
    \item نقص‌هایی که بایستی به ترتیب اولویت بر طرف شوند مانند (\lr{High,
    Medium, Low}).
    \item نقص‌هایی که باید از نظر اهمیت برطرف شوند (\lr{Critical, Major,
    Miner, Trivial}).
\end{itemize}

\section{\lr{Error}}

\lr{Error} زمانی رخ می‌دهد که یک اشتباهی در کد وجود داشته باشد \cite{jpoint} و
مانع انجام شدن آن در کامپایلر یا مفسر شود به گونه‌ای که برنامه نوشته شده با وجود
آن خطا اجرا نخواهد شد. تفاوت اصلی \lr{Error} با باگ‌ها این است که در نرم‌افزاری
که حاوی باگ است می‌توان فرایندی را انجام داد و تنها آن قسمتی که باگ دارد
نمی‌تواند وظیفه‌اش را کامل انجام دهد اما در نرم‌افزاری که با \lr{Error} رو به رو
شده است برنامه اصلاً در مرحله اجرا و خروجی وارد نمی‌شود \cite{geekflare}.
معمولاً می‌تواند توسط مهندسان نرم‌افزار و مهندسان واحد آزمون نرم‌افزار گزارش
شود. می‌تواند انواع مختلفی داشته باشد:

\begin{itemize}
    \item خطای نوشتاری (گرامری) در برنامه نوشته شده
    \item خطای منطقی
    \item خطا‌های محاسباتی
    \item خطای هنگام کامپایل مانند استفاده از ماکرو‌هایی که قبل از کامپایل انجام
    می‌شوند.
\end{itemize}

\section{\lr{Fault} یا عیب}

یک وضعیتی است که باعث می‌شود نرم‌افزار نتواند یک عملکرد ضروری را انجام دهند که
انسان با انجام اشتباه باعث رخ دادن آن می‌شود \cite{jpoint}. به بیانی بهتر،
\lr{Fault} یک رفتار ناخواسته یا نادرستی است که توسط برنامه‌های کاربردی آن را
مشاهده خواهیم کرد \cite{geekflare}. عیب‌ها باعث ایجاد اخطار در برنامه می‌شوند.
اگر برطرف نشوند، ممکن است منجر به شکست در کار کد مستقر شده شوند. اگر اجزای مختلف
کد نوشته شده به یکدیگر متصل باشند (به عنوان ماژولی از یکدیگر باشند) یک عیب ممکن
است رفتار کل سیستم را تحت تاثیر خود قرار دهد \cite{geekflare}. دلایل مختلفی
می‌تواند وجود داشته باشد که باعث ایجاد عیب در سیستم شوند. یک عیب در یک برنامه
می‌تواند زمانی رخ دهد که یک عمل نادرست در فرایند‌ها یا تعریف داده‌ها صورت گرفته
باشد یا می‌تواند به صورت ناسازگاری در برنامه‌ها شناخته شود.

برای مثال در یک برنامه پایتون از کتابخونه‌ای جهت خواندن فایل‌های مخصوصی استفاده
شده است. این برنامه شامل هیچ خطا و باگی نیست و به درستی عمل می‌کند. این برنامه
برای فرمت‌های مشخص شده می‌تواند به درستی عمل کند ولی به محض آن که فایلی با فرمت
مشخص نشده به آن وارد می‌شود ممکن است آن را اصلاً نپذیرد یا خروجی مورد نظر را
ایجاد نکند و باعث خروجی از برنامه شود.

یا اینکه برنامه‌ای نوشته شده است که به یک پروتکلی با نسخه‌ای مشخص متصل می‌شود
اما وقتی آن برنامه را به همان پروتکل ولی نسخه‌ای قدیمی‌تر متصل می‌کنیم تغییراتی
که بین نسخه‌ها وجود دارد باعث می‌شود برنامه به درستی وظیفه خود را انجام ندهد.

عیب‌های مختلفی می‌تواند در یک محصول نرم‌افزاری در بخش‌های مختلف آن ظاهر شوند
\cite{jpoint}:

\begin{itemize}
    \item عیب‌های مربوط به \lr{Business Logic}
    \item عیب‌های عملکردی و منطقی
    \item عیب‌های نمایش (\lr{GUI})
    \item عیب‌های اجرایی
    \item عیب‌های امنیتی
    \item عیب‌های سخت‌افزاری/نرم‌افزاری
\end{itemize}

\section{\lr{Failure} یا شکست}

گاهی وقت‌ها، زمانی که برنامه در حال اجرا است سیستم ممکن است نتیجه‌ غیرمنتظره‌ای
را به عنوان خروجی تولید کند که می‌تواند به برنامه اجازه دهد که به شکست منجر شود.
در موقعیت و محیط‌های مشخصی نقص‌ها می‌تواند دلیلی برای ایجاد شکست در برنامه باشد
\cite{geekflare}. یا به عبارتی دیگر، اگر برنامه‌ای تعداد زیادی از نقص‌ها را
داشته باشد منجر به شکست می‌شود یا باعث شکست می‌شود. این شکست‌ها در تست‌های دستی
معمولی توسط مهندسان در فرایند چرخه توسعه نرم‌افزار کشف می‌شوند \cite{jpoint}.

شکست در برنامه می‌تواند توسط خطا‌های انسانی توسط کاربرانی که با نرم‌افزار تعامل
دارند رخ دهد. برای مثال زمانی که کاربر ورودی اشتباهی به برنامه کاربردی وارد کند
باعث شود که نرم‌افزار با شکست رو به رو شود. با این وجود می‌توان گفت که یک شکست
می‌تواند عمداً یا سهواً توسط یک فرد در سیستم ایجاد شود \cite{geekflare}.


\bibliographystyle{plain-fa}
\bibliography{failure_fault_refs.bib}
\end{document}

