تعریف معماری

اسلاید معماری نرم‌افبزار مقدماتش رو میگه.

معماری اسلاید ۳ یه تعریفی داره

تعریف تقیقاً گارلا و شار رو بنویسید

یا مثلاً از نظر ساختار بنویسید

اسلاید شماره ۶ خیلی مهمه

اسلاید ۷ هم ادامه همون ۶ عه

تعریف معماری اسلاید ۱۶ گارلا و شا

اسلاید ۱۷ تا ۲۵ مهمه چیزی که در امتحان می‌پرسه
refrence model 
arch pattern
pattern چیه

ممکنه بگه با یه مثال تعریف کنید.

اسلاید ۳۶ سافت ور آرک و سیستم آرک چیه

تعریف استراکچر یا ویو اسلاید ۳۷

از اسلاید ۳۹ تا اسلاید ۵۴ به این دلیل زیاده که چون فارسی هم داره، ساختار‌های
معماری نرم‌افزار را نام ببرید. باید اون ماژول و سی اند سی و الوکیشن رو بگیم.

ساختار اساتیک و ساختار داینامیکش رو نامببرید که میشه اسلاید ۳۹

----------------------------------------------------

چیزی که خیلی مهمه بحث سند ۴ بعلاوه ۱ یا سند معماری نرم‌افزار قشنگ باید تعریف بشه.

اسلاید سافت ویر کوآلیتی اتریبیوت فور اویل ابیلیتی

اسلاید ۳ نیاز‌های وظیفه مندی و غیر وظیفه مندی به یکدیگر متعامد هستند.

Software performance measures:

صفحه ۳ تعاریفش مهمه wating time, queue length, task length utilization,
throughput and goodput.

اسلاید ۸ NTBF اسلاید ۹ MTTF یه قسمت قرمز رنگ داره که تفاوت‌هاشون بگیم.

اسلاید ۱۰ تعریف MTTR

اسلاید ۱۱ یه سوال داده که بخواد یه قطعه میتونه اویل باشه ولی ریلای نباشه.

مثال ماشین رو گفته

میتواند اویل باشه ولی ریلای نباشه تنها به خاطر تعریف محیط هستش.

میتونه باشه میتونه نباشه ریلای

اسلاید ۱۲ Mean Down Time

------------------------------------------------------

خصوصیات کیفی قابل مشاهده و غیر قابل مشاهده اسلاید ۸
کارایی امنیت قابلیت دسترسی
قابلیت استفاده مجدد اصن به زمان اجرا ربطی ندارد.

اسلاید ۱۴ از اسلاید ۱۲ جوابش شروع میشه سوال زیر
سناریو عمومی و عینی چیه.

سناریو از اسلاید ۱۲ بخش‌هاشو توضیح داده که ۶ بخش داره و هر ۶ بخش رو گفته توضیح
داده یا می‌تونیم شکل صفحه ۱۳ رو بکشیم.

اسلاید ۱۶ تاکتیک چیه

اسلاید ۱۸ و ۱۹ یه جنرال سناریو گفته و بعد کانکریت رو گفته و ممکنه استاد بگه که
سناریو جنرال اویل ابینلیتی رو بنویسی و یه سناریو عینی براش مثال بزنیم.

-----------------------------------------------------

دسترس پذیری

اسلاید ۵ فرمول اویل ابیلیتی

اسلاید ۱۱ نحوه محاسبه داون تایم که خیلی مهمه.

اسلاید ۱۳ جنرال سناریو اویل ابینلتیه

توی اسلاید ۱۴ ممکنه بگه که انواع فالت‌ها رو بنویسه که میشه اون ۴ تا

توی جزوه نبوده ولی 

Fault failure error خییییییییییییییلی مهمه

اسلاید ۱۵ کانکریت سناریو دیگه از اویل ابیلیتی مثال زده

اسلاید ۱۸ تاکتیک‌های اویل ابیلیتی رو تو شکل نشون داده

تاکتیک‌ها اسمشو اشکال نداره بنویسیم ولی توضیحش باید باشه که با استدالال باشه.

ممکنه بگه از دیتکتفالت یکی رو توضیح بده یا اسمشو میده که بخواد ازمون توضیح بدیم.

اسلاید ۲۲ که وتینگه و ۳ تاتینکی که زیر مجموعه اونه

تاکتیک بعدی که میخواد بدونیم صفحه ۳۰ اکتیو ریدان پسیو و اسپیر

این تاکتیک‌های رو باید نسبت به یه سناریو بدونیم.

-------------------------------------------------

interoperatibility

بحث سرویس و معماری سرویس گرا اون حتما باشه ۳ بخش و ۴ بخش که اگه تعریف خواست اون
۴ تعریفه رو اوکیه.

اسلاید ۳ جنرال سناریو اینترآپریتیبیلیتی مهمه

interface A , B  Serivce contract Implementation
بیزینس لاجیک و دیتا مال پیاده‌سازیه

اسلاید ۴ عینیش مهمه

اسلاید ۷ تاکتیک‌ها و تفاوت اورکستریشن و کریوگرافی هستش.

اسلاید ۱۳ و ۱۴ بحث اورکستریشن و کریو رو گفته گفته دو تا زبان هم گفته بشه.

TODO:
Definition Language رو باید بگیم.

-------------------------------------------------------

بحث کلاس بود سر امتحان پیدا کردن کلاس مطرح نیست. برای پروژست فقط

--------------------------------------------------

بیزینس اکتور
بیزینس
اکتور
بیزینس ورکر
یوزکیس
بیزینس یوزکیس

اینو باید بدونیم برای امتحان قطعاً مهمه.

------------------------------------------------------

کشیدن نمودار واسه امتحان نیست.