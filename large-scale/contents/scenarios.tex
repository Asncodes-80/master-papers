\newpage
\section{سناریو‌های ویژگی‌های کیفی \lr{Availability} و \lr{Interoperability}}

برای دو سناریو زیر چه ویژگی‌های کیفی می‌توان وضع کرد و چه تاکتیک‌هایی بایستی
انتخاب شوند تا سیستم بتواند برای کاربران در کارایی رضایت بخش باشد.

\subsection{سیستم مراقبت از سلامت}

یک سیستم مراقبت از سلامت داریم که شامل یک ساعت هوشمند و یک سیستم آنالیز داده
می‌باشد. ساعت هوشمند بر اساس سنسور‌های تعبیه شده در آن اطلاعات را دریافت می‌کند
و اطلاعات را پس از فیلتر شدن در اپلیکیشن سبکی که در سمت ساعت است آنالیز اولیه را
انجام می‌دهد. با توجه به اینکه آنالیز نهایی می‌بایست بر اساس سابقه و پرونده
پزشکی بیمار انجام شود، اطلاعات به سمت ابر فرستاده می‌شود. در محیط ابر سیستم
آنالیز اطلاعات بیمار با توجه به سوابق، آنالیز نهایی را انجام می‌دهد و در صورت
نیاز پیام مناسب را برای بیمار ارسال می‌کند و نتیجه را در پرونده پزشکی ثبت
می‌کند. همچنین اطلاعات بدست آمده را برای پزشک بیمار ارسال می‌کند. در سیستم مورد
نظر، هشدار‌هایی نیز به بیمار برای مصرف دارو یا اقدامات احتمالی داده می‌شود.

این سناریو نیازمندی دو ویژگی کیفی \lr{Availability} به همراه
\lr{Interoperability} می‌باشد. همچنین وجود ویژگی کیفی امنیت در نگهداری داده‌ها و
اطلاعات بیمار جهت حفظ حریم‌خصوصی وی الزامی می‌باشد.

\subsubsection*{حفظ \lr{Availability}}

\begin{enumerate}
    \item وجود عملیات \texttt{Ping/echo} برای بررسی صحبت ارتباط ساعت هوشمند به
    تلفن همراه. درست است که استفاده از این تکنیک باعث می‌شود ساعت صبر کند تا
    تلفن هوشمند بیمار پیام آزمایشی را دریافت کند اما این عملیات کاملاً می‌تواند
    به صورت \texttt{Async} باشد و همزمان ساعت پایش علائم حیاتی بیمار را ادامه
    دهد. استفاده از این تکنیک به دلیل آن است که اگر ارتباط با تلفن هوشمند از کار
    افتاد، در ابتدا ساعت در حالت آنالیز اولیه داخلی خود رفته و سپس به کاربر
    اعلام کند که ساعتش با تلفن همراه وی متصل نمی‌باشد.
    \item استفاده از تکینک \texttt{Timestamp}: چون داشتن علائم حیاتی بیمار
    کاملاً ضروری است برای مثال در سیستم مذکور اطلاعات هر ۵ دقیقه یکبار به سمت
    موبایل ارسال می‌شود و سپس به سمت سرور‌ها ارسال خواهند شد. به همین خاطر
    مانیتور کردن رکورد‌های ثبت شده توسط ساعت امری بسیار مهم است تا بررسی کنیم
    توالی \texttt{timestamp}ها کاملاً براساس برنامه بوده و اطلاعات به صورت منظم
    ارسال می‌شود. در غیر این صورت بایستی وجود \lr{Fault} را در این بخش بررسی
    کنیم.
    \item استفاده از تکنیک \lr{Condition monitoring}: به جهت آن است که کاربر
    ساعت را به صورت مناسب به مچ خود متصل کرده باشد تا علائم حیاتی با دقت بیشتر
    از بدن بیمار پایش شوند.
    \item استفاده از سیستم‌های \lr{Voting} برای حفظ پایداری ذخیره‌سازی اطلاعات
    در سرور‌های ابری: به دلیل آنکه پرونده پزشکی بیمار بایستی کامل باشد تا فرایند
    هوش‌مصنوعی به صورت مناسب صورت گیرند، براساس استاندارد \texttt{TMR} برای حفظ
    \lr{Redundancy} ۳ سرور تعریف می‌کنیم که کاملاً به صورت \lr{Hot Spare} یا
    \lr{Active Redundant} درحال سینک شدن اطلاعات با یکدیگر هستند. به دلیل حیاتی
    بودن حفظ اطلاعات نیاز به \lr{State resync} بین سرور‌ها وجود دارد. طبق
    قانون \texttt{TMR} هر ۳ سرور به صورت \texttt{Replication} دقیقاً کلونی از
    یکدیگر هستند.
    \item در تمام بخش‌هایی که برنامه نویسی انجام می‌شود برای شناسایی \lr{Fault}
    بایستی تمام برنامه‌های \texttt{Exception-detection} داشته باشند که کاملاً
    خطا‌های غیرمنتظره را هندل کنند تا سیستم نیاز به راه‌اندازی و تعمیر نداشته
    باشد و کاملاً صحیح به کار خود ادامه دهند.
    \item استفاده از \texttt{Degradation}: در قسمت تعمیر و آماده‌سازی، اگر بخشی
    از اپلیکیشن موبایل با سرور قادر به دریافت اطلاعات نبود، کل اپلیکیشن به خاطر
    آن قسمت از دسترس خارج نشود و تنها از قسمت مورد تعمیر و آماده‌سازی قرار گیرد.
    \item استفاده از \lr{non-stop forwarding}: به محض اینکه یکی از
    الگوریتم‌های رمزنگاری اطلاعات باعث اختلال در سیستم و بار اضافی شد سریعاً از
    الگوریتم‌های جایگزین که سبک‌‌تر هستند استفاده شود. همچنین در قسمت آنالیز
    اولیه به محض اینکه یکی از مدل‌های \lr{ML} با کندی فرایند پیشبینی را انجام
    دادند از مدل جایگزین استفاده شود تا بیمار کمترین تاخیر را داشته باشد.
    \item استفاده از تکنیک \lr{Remove fault from service} به محض اینکه
    \lr{Fault} در سیستم یافت شد سریعاً آن را رفع کنیم و به جوان‌سازی نرم‌افزار
    به بپردازیم.
\end{enumerate}

\subsubsection*{حفظ \lr{Interoperability}}

با توجه به اینکه می‌توان به سناریو به وسعت جهانی نگاه کرد، بیماران هر کشوری قرار
است از سرویس‌های ما استفاده کنند، به همین خاطر بایستی سعی شود سرویس‌ها در موقعیت
مکانی مناسب با هر کشور تعبیه شوند از نظر \lr{Locate} و \lr{Service discovery} تا
سریع‌ترین پاسخ را نسبت به بیمارستان منطقه شهر خود دریافت کنند (به دلیل ارتباط
سرویس نرم‌افزاری با بیمارستان‌ها و درمانگاه‌های داخل کشوری، استانی). از سوی دیگر
به دلیل آنکه این سرویس به چندین زیر سرویس براساس معماری \lr{Micro service} تقسیم
می‌شود وجود یک \lr{Orchestrator} به عنوان بخش مرکزی مدیریت ارسال اطلاعات بیماران
به سرویسی مناسب پیشنهاد می‌شود. چرا که کاملاً سیستم در سطح مقیاس بزرگ می‌باشد و
قطع شدن یکی از سرویس‌ها لازم است سرویس‌های \lr{Back-up} مورد استفاده قرار گیرند.

\subsection{سیستم مدیریت بحران شهری}

یک سیستم مدیریت بحران شهری را در نظر بگیرید که با استفاده از مدل جمع‌سپاری،
بحران‌های شهری را مدیریت می‌کند. این بحران‌ها شامل آتش‌سوزی و حوادث رانندگی
می‌باشند. کاربران امکان ارسال پیامک، تصویر و ویدئو را از طریق سیستم دارند. سیستم
مدیریت بحران از ابتدا، راستی آزمایی نسبت به وقوع حادثه را انجام می‌دهد، سپس
براساس اطلاعات دریافتی اقدام به حل موضوع می‌کند. همچنین پیامی از سمت سیستم در
صورت صحت حادثه کاربران برای اطلاع‌دهنده ارسال خواهد شد. سیستم تعامل مشخصی با
اورژانس و آتش‌نشانی خواهد داشت.

\subsubsection*{حفظ \lr{Availability}}

\begin{enumerate}
    \item استفاده از \lr{Heartbeat} برای بررسی صحت وجود سرویس‌ها در سیستم بین
    سرویس‌های دیگر.
    \item استفاده از تکنیک مانیتورینگ در جهت پایش فعالیت‌های \lr{HTTP} که اگر
    مورد حمله \lr{DDoS} قرار گرفت سریعاً منبع ارسال‌کننده حجم سنگین درخواست را
    شناسایی و بلاک کنیم.
    \item استفاده از \lr{Timestamp}: به خاطر حیاتی بودن اطلاعات باید
    \lr{timestamp} رکورد‌ها را داشته باشیم که اگر توالی نامنظمی دریافت شد سریعاً
    رفع \lr{Fault} را در سیستم انجام دهیم.
    \item استفاده از \lr{Condition monitoring}: به دلیل ارسال تصاویر توسط
    کاربران مختلف، امکان این وجود دارد تصاویر به صورت ناواضح و نامناسب گرفته
    شوند. به همین خاطر در همگام ارسال تصاویر و ویدئو‌ها باید به کاربران آموزش
    داده شود که گوشی را به صورت افقی برای ثبت وقایع در دست داشته باشند.
    \item استفاده از \lr{Retry}: به دلیل ثبت تصاویر و ویدئو‌های حجیم باید این
    قابلیت پیاده‌سازی شود.
    \item استفاده از \lr{Exception-detection-handling}: بایستی در تمام
    برنامه‌های اعم از فرانت و بک، مدیریت استثنات نرم‌افزاری صورت گیرد.
    \item در خصوص تکنیک \lr{Voting} استفاده از ۳ سرور به ازای هر منطقه به صورت
    \lr{Replicatino} مناسب می‌باشد تا تمام سیستم‌ها اطلاعات و توابع یکسانی داشته
    باشند و اطلاعات وقایع به صورت منظم نگهداری و بک‌آپ گرفته شوند.
    \item با توجه به \lr{Protection group} استفاده از تکنیک \lr{Warm spare} یا
    \lr{Passive Redundancy} مناسب می‌باشد که در زمان‌های مشخصی که سازمان مدیریت
    بحران شهری مشخص می‌کند بتوان تمام اطلاعات را بین سرور‌ها دیگر \lr{Sync} کرد.
    \item استفاده از تکنیک \lr{Resync} در \lr{Warm spare} سیستم‌ها.
    \item استفاده از تکینک \lr{Rollback} دقیقاً زمانی که کاربر نتوانسته است
    تصاویر وقایع را ارسال کند و سریعاً کاربر اعلام‌دهنده را بایستی به نقطه امن
    قبلی‌ای که معمار نرم‌افزار مشخص کرده است باز گردانیم.
    \item استفاده از تکینک \lr{Software update}: اگر هر گونه \lr{Bug} و
    \lr{Software fault} رخ داده باشد در نسخه بعدی رفع شده باشد.
    \item استفاده از تکنیک جوان‌سازی برای رفع \lr{Fault} در نرم‌افزار.
    \item استفاده از تکنیک \lr{NSF} برای مثال در پردازش تصویر. اگر هر کدام از
    سرور‌ها قادر به دریافت درخواست پردازش تصویر نبودند سریعاً به سرور‌های دیگر
    درخواست ارسال شود.
\end{enumerate}

\subsubsection*{حفظ \lr{Interoperability}}

به دلیل تعامل سرویس‌ها با یکدیگر و همچنین با سرویس‌های بیرونی مانند اورژانس و
آتش‌نشانی‌های هر منطقه بایستی دو تکنیک استفاده از \lr{Orchestrator} برای هدایت
درخواست‌ها به سرویسی مناسب و \lr{Locate} برای دسترسی به سرویس‌های بیرونی با
آدرسی مناسب که می‌تواند \lr{DNS} یا آدرس شبکه باشد استفاده شود. همچنین به دلیل
تبادل اطلاعات بین سرویس‌ها می‌تواند از \lr{Queue}ها یا \lr{ESB} استفاده کرد. اگر
از \lr{ESB} استفاده شود می‌توان علاوه‌بر ارسال داده‌ها درون آن، منطق کسب و کار
نیز وضع کرد یا داده‌ها را در همان بین \lr{Nomalize} کرد.