\section{کلاس‌ها، یوزکیس‌ها و سیکوئنس‌ها}

کلاس‌ها ثابت‌هایی هستند که به سه دسته زیر تقسیم می‌شوند:

\begin{enumerate}
    \item کلاس‌های مرزی یا \lr{Boundary classes}
    \item کلاس‌های موجودیتی یا \lr{Entity classes}
    \item کلاس‌های کنترلی \lr{Control classes}
\end{enumerate}

\subsection{کلاس‌های مرزی}

کلاس‌هایی هستند که مرز بین سیستم با دنیای بیرون را مشخص می‌کنند که شامل
سخت‌افزار و پروتکل‌های ارتباطی می‌توانند باشند.

برای مثال، کیبوردی که در \lr{ATM} استفاده می‌کنیم از نوع کلاسی مرزی می‌باشد.

\subsection{کلاس‌های موجودیتی}

کلاس‌هایی هستند که در مورد یک چیز یا موجودیتی در سیستم حاضر هستند و بخواهیم در
مورد آن‌ها اطلاعاتی را ذخیره‌سازی کنیم.

\subsection{کلاس‌های کنترلی}

کلاس‌هایی هستند که کنترل بقیه قسمت‌های سیستم را بر عهده دارند.

\subsubsection*{نکته‌}

\begin{itemize}
    \item حداقل یک کلاس کنترلی در سیستم وجود دارد، اگر وجود نداشته باشد در حقیقت
    آن سیستم خودش یک کلاس کنترلی می‌باشد.
    \item ساختار کلاس‌ها عبارت‌اند از: نام کلاس، ویژگی‌های موجودیت و متد‌های آن
    \item یک نیازمندی که به مهندس نرم‌افزار گفته می‌شود بایستی زیر تمام اسم‌ها
    خط بکشد. این اسم‌های خط کشیده شده در حقیقت کلاس‌های سیستم هستند و بایستی
    مشخص کنیم که از چه نوع کلاس‌هایی می‌باشند. زیر تمام افعال نیز بایستی خط بکشد
    که متد‌های آن کلاس‌ها را مشخص کند.
    \item کلاس‌ها زمانی که تکرار می‌شوند، یا ممکن است به آن افزوده شود که تبدیل
    به کلاس بزرگ‌تری شوند یا یک کلاس به زیر کلاس‌هایی برای سادگی پیاده‌سازی
    شکسته می‌شود.
    \item کلاس‌های کنترلی را خودمان بایستی ایجاد کنیم.
    \item در کشیدن نمودار‌ها بایستی اول از همه \lr{Usecase} دیاگرام کشیده شود و
    از روی موجودیتی که با افعال کار می‌کند به کلاس‌ها رسید و سپس دیاگرام
    \lr{Sequence} که بالای هر مورد در \lr{Sequence diagram} کلاس‌های ما قرار
    دارند. با اینکار در نمودار \lr{Sequence} مشخص می‌کنیم که هر موجودیت با چه
    موجودیت دیگری در ارتباط است.
\end{itemize}

\subsection{اجزای \lr{Usecase diagram}}

\begin{enumerate}
    \item \lr{Actor}
    \item \lr{Usecases}: در حقیقت افعالی که هر \lr{Actor} انجام می‌دهد را شامل
    می‌شود. در حقیقت به صورت کاربردی منو‌های یک سیستم یا آیتم‌های تنظیمات یک
    نرم‌افزار (دستگاه) \lr{usecase}های آن است.
\end{enumerate}

\subsection{تعریف \lr{Actor}}

شخص یا سیستمی بیرون از سیستم ما است که با سیستم تعامل دارد. این تعامل به صورت
مستقیم و \lr{Single hop} می‌باشد و به وسیله واسط یا \lr{Two step} نمی‌توان نام
آن را \lr{Actor} گذاشت. بین \lr{Actor}ها عملیاتی انجام نمی‌شود.

\subsection{تعریف \lr{Business actor}}

شخص یا سیستمی که بیرون از سازمان با سیستم ما در تعامل است را می‌گوییم. سطح تجرید
در \lr{Business} بالاتر می‌باشد. هر \lr{Business} هدف کلی یک سازمان را مشخص
‌میکند. \lr{Business} داشنگاه آموزش دادن و \lr{Business} بانک انجام تراکنش‌ها می‌باشد.

\begin{itemize}
    \item برای مثال شخصی وارد دانشگاه می‌شود یک عنوانی دارد. یا استاد، یا دانشجو
    یا کارمند آنجاست. این فرد قرار است یک سرویسی دریافت کند.
    \item برای مثال شخصی وارد بانک می‌شود و به باجه‌ای از بانک مراجعه می‌کند جهت
    افتتاح حساب. آن مشتری به عنوان \lr{Business actor} است و آن شخصی که پشت باجه
    در حال استفاده از سیستم بانکی جهت افتتاح حساب است به عنوان \lr{Actor} محسوب
    می‌شود.
    \item شخصی که داخل دانشگاه است چه استاد باشد، چه دانشجو چه کارمند، از نوع
    \lr{Business actor} می‌باشد ولی داخل سیستم آموزشیار به صورت \lr{Actor} خواهد
    بود.
    \item مشخص کردن نقش موجودیت‌ها داخل سیستم به موقعیت آن‌ها نسبت به سیستم
    وابستگی دارد.
\end{itemize}

\subsection{تعریف \lr{Business process}}

فرایند‌هایی که یک \lr{Business} پوشش می‌دهد را در نظر دارد. در \lr{Business}
آموزش یا دانشگاه، \lr{Business process} جذب داشنجو و استاد را داریم.

\subsection{تعریف \lr{Business worker}}

به شخصی گفته می‌شود که داخل سیستم فعالیت دارد و روی موجودیت‌ها کاری را انجام
می‌دهد.

\subsection{ابهامات}

برای اینکه مشخص کنیم وضعیت یک موجودیت در سیستم \lr{Actor} است یا \lr{Business
actor} یا \lr{Business worker} نیازی به الزام دانستن همه چیز نداریم بلکه همه چیز
به شرایط و محیط سیستم بستگی دارد:

\begin{itemize}
    \item هر \lr{Actor} الزاماً یک \lr{Business worker} است؟ هر \lr{Actor}
    می‌تواند \lr{Business worker} باشد?
    \begin{itemize}
        \item الزامی وجود ندارد و یک \lr{Actor} می‌تواند \lr{Business worker}
        باشد. برای مثال استاد نسبت به دانشگاه یک \lr{Business worker} است چرا که
        با موجودیت‌هایی به نام دانشجو سروکار دارد، اما نسبت به سیستم آموزشیار به
        صورت مستقیم \lr{Actor} می‌باشد.
        \item شخصی در بانک کار می‌کند پس به عنوان \lr{Business worker} معرفی
        می‌شود. همان شخص در تعطیلات وقتی می‌خواهد از خدمات بانک استفاده کند به
        عنوان \lr{Actor} خواهد بود. همان شخص اگر بخواهد پول خود را از سیستم
        برداشت کند باید به عنوان \lr{Business actor} ظاهر شود تا کارش را انجام
        دهد.
    \end{itemize}
    \item هر \lr{Actor} الزاماً \lr{Business actor} است؟ هر \lr{Actor} می‌تواند
    \lr{Business actor} باشد؟
    \begin{itemize}
        \item در حالت کلی دانشجو یک \lr{Business worker} است ولی نسبت به
        آموزشیار صرفاً \lr{Actor} خواهد بود.
        \item سیستم صندوق رفاه باید نسبت به تمام سیستم‌ها \lr{Actor} باشد.
    \end{itemize}
\end{itemize}

\subsubsection*{نکات}

\begin{itemize}
    \item از \lr{Include} زمانی استفاده می‌کنیم که از یک \lr{usecase} به
    \lr{usecase} دیگر اشاره داشته باشیم.
    \item \lr{Uses} تحت شرایطی خاص استفاده می‌شود. برای مثال یک \lr{usecase} به
    نام «تخفیف دادن» داریم که تنها برای مشتریانی است که بیشتر از ۶۰ میلیون تومان
    خرید داشته‌اند.
    \item در \lr{usecase}‌ها هیچ وقت در مورد زمان انجام کاری به صورت مرحله به
    مرحله صحبت نمی‌کنیم بلکه انجام فرایند داخل هر \lr{usecase} را با نمودار
    \lr{Sequence} مشخص می‌کنیم.
    \item هر آنچه که ماهیت لمس کردن داشته باشد در سیستم حساب نمی‌شود و به عنوان
    کلاس تعریف نخواهد شد.
\end{itemize}