\newpage
\section{قابلیت تعامل یا \lr{Interoperability}}

ویژگی کیفی دوم بعد از \lr{Availability} قابلیت تعامل می‌باشد. تعامل باید بیشتر
از دو سرویس باشد که قادر باشند به صورت معناداری تبادل اطلاعات
\footnote{\lr{Information exchange}}داشته باشند.

برای مثال در سیستم دانشگاهی، یک سیستم به نام آموزشیار وجود دارد، یک سیستم به نام
سیستم مالی دانشگاه و سیستم دیگر به نام پژوهشیار و غیره. این سیستم‌ها باید
بتوانند به صورت معناداری با هم تعامل و تبادل اطلاعات داشته باشند. همانند
\lr{Availability} جواب ما به صورت «بلی» یا «خیر» نمی‌باشد.

\subsubsection*{نکته}

هر کدام از ویژگی‌های کیفی یک سناریو عمومی دارند و حداقل یک سناریو عینی.

\subsection{سناریو عمومی}

\begin{enumerate}
    \item منبع تحریک: یک سیستم
    \item محرک: یک درخواست جهت تبادل اطلاعات بین سیستم‌ها
    \item مصنوعات یا \lr{Artifact}: سیستمی که قرار است منبع تحریک با آن تعامل
    داشته باشد.
    \item محیط: سیستم‌هایی که در زمان اجرا یا قبل از زمان اجرا شناخته می‌شوند و
    می‌خواهد برای تبادل اطلاعات شرکت داشته باشند.
    \item پاسخ: حداقل یکی از موارد زیر:
    \begin{itemize}
        \item درخواست به صورت مناسبی رد می‌شود و به روشی مناسب به کاربر یا
        سیستم‌ها اطلاع داده می‌شود.
        \item درخواست به طور موفق در سیستم پذیرفته می‌شود و اطلاعات به صورت موفق
        بین سیستم‌ها ارسال می‌شود.
        \item درخواست به وسیله یک یا چند سیستم گزارش‌گیری
        \footnote{\lr{logging}} می‌شود.
    \end{itemize}
    \item اندازه‌گیری پاسخ: حداقل یکی از موارد زیر:
    \begin{itemize}
        \item درصد تبادل صحیح اطلاعات به صورت موفقیت آمیز بین سیستم‌ها
        \item درصد عدم تبادل اطلاعات بین سیستم‌ها
    \end{itemize}
\end{enumerate}

\subsection{سناریو عینی}


سیستم اطلاعات خودرو (\lr{ECU}) موقعیت فعلی ما را به سیستم نظارت بر ترافیک ارسال
می‌کند. سیستم نظارت بر ترافیک موقعیت مکانی ما را با اطلاعات دیگر ترکیب می‌کند و
این اطلاعات را روی نقشه گوگل قرار می‌دهد و آن را پخش می‌کند. اطلاعات مکان ما به
درستی با احتمال ۹۹/۹٪ گنجانده شده است \footnote{\begin{LTR} Our vehicle
information system sends our current location to the traffic monitoring system.
The traffic monitoring system combines our location with other information,
overlays this information on a Google Map, and broadcasts it. Our location
information is correctly included with a probability of $99.9$\%. \end{LTR}}.

\begin{enumerate}
    \item منبع تحریک: سیستم اطلاعات خودرو
    \item محرک: موقعیت فعلی کاربر
    \item مصنوعات: سیستم دیگر (سیستم نظارت بر کنترل ترافیک)
    \item محیط: انتقال اطلاعات کاربر در \lr{Google Maps}
    \item پاسخ: نمایش اطلاعات مسیریابی و ترکیب اطلاعات موقعیت مکانی با اطلاعات
    دیگر
    \item معیار پاسخ: اطلاعات ارسال شده به کاربر با احتمال ۹۹/۹٪ درست خواهد بود.
\end{enumerate}

\subsection{هدف تاکتیک‌های تعامل‌پذیری}

ورودی در این ویژگی کیفی درخواست تبادل اطلاعات است که با استفاده از تاکتیک‌هایی
می‌خواهیم درخواست را به درستی کنترل کنیم که انجام شود.

\subsection{تاکتیک‌های تعامل‌پذیری}

تاکتیک‌های تعامل‌پذیری به دو دسته زیر تقسیم می‌شوند:

\subsubsection{\lr{Locate} و \lr{Dicover service}}

قبل از تعامل با هر سیستم باید اول از همه ببینیم که سیستم کجا قرار دارد. پس
بایستی موقعیت مکانی سیستم که می‌خواهیم با آن تبادل اطلاعات داشته باشیم را بدست
آوریم. چیزی مانند \lr{DNS} را استفاده کنیم که به صورت سلسله مراتبی جست و جو را
انجام دهد و فضای جست و جو را همیشه محدود کنیم.

سرویس هرگونه خدمتی است که کاربر آن را تقاضا داشته است. سرویس‌ها می‌توانند از روی
نام، آدرس محل قرارگیری، آدرس‌های شبکه‌ای و غیره کشف شوند و توسط زیر سیستم‌ها
مورد استفاده و تعامل قرار گیرند.

برای مثال سرویس خدمات شهری، وقتی کاربری از شهر بندر عباس درخواست خود را ارسال
می‌کند باید به سرویس‌هایی ارسال شوند که نزدیک‌ترین نقطه به کاربر هستند تا کاربر
بتواند به صورت مناسب به پاسخ خود برسد (در مقیاس دو شهر داخل کشور یک مقدار
نامناسب است ولی اگر سرویس \lr{Netflix} را در نظر بگیرید که یک کاربر ایرانی
بخواهد به سرور \lr{USA} درخواست ارسال کند و از آن خدمات دریافت کند بهتر است به
سرویسی درخواستش هدایت شود که نزدیک‌ترین نقطه به محل زندگی وی است.).

\subsubsection{\lr{Manage Interface}}

\begin{enumerate}
    \item \lr{Orchestration}: عموماً یک سرویس مرکزی وجود دارد که نقش هماهنگ
    کننده را در سیستم اجرا می‌کند و مشخص می‌کند که هر تسکی بایستی از چه سرویس به
    چه سرویسی منتقل شود.
    \item \lr{Choreography}: سرویس‌هایی هستند که کاملاً به یکدگیر وابستگی دارند
    و مرحله به مرحله یکی پس از دیگری درخواست را پردازش و خروجی را به سرویس بعدی
    منتقل می‌کنند.
    \item \lr{Tailor interface}: واژه‌ای به نام \lr{Decorate} است که با نظر
    خودمان سیستم مورد نظر را دکور می‌کنیم که این دکور کردن نیز براساس الگوی دکور
    اتفاق می‌افتد. قابلیت‌هایی که یک سیستم دارد را براساس شرایط ممکن تغییر
    می‌دهیم. برای مثال اتفاقی که برای اپلیکیشن \lr{Snapp} در \lr{AppStore} شرکت
    اپل رخ داد. وقتی شرکت اسنپ برنامه \lr{iOS} خود را در فروشگاه اپل منتشر
    می‌کرد به دلیل ایرانی بودن توسعه‌دهندگان برنامه حذف می‌شد. برنامه‌نویسان
    انسپ برنامه‌ای را طراحی کردند که تنها در شرایط \lr{IP} ایران برنامه اسنپ را
    برای مشتریان نشان دهد. در غیر این صورت در هر موقعیت مکانی به غیر از ایران
    اپلیکیشن \lr{Music player} را به کاربران نمایش می‌داد.
\end{enumerate}

\subsubsection*{نکات}

\begin{itemize}
    \item دو سرویس تنها با یکدیگر به صورت مستقیم صحبت نمی‌کنند بلکه نیاز به روشی
    داریم به نام \lr{Enterprise Service Bus (ESB)} داریم تا مدیریت پیام‌ها از
    طریق آن صورت گیرد.
    \item \lr{ESB} حاوی تعدادی \lr{Adapter} دارد که هر موقع سرویس (آ) به سرویس
    (ب) می‌خواهد ارتباط داشته باشد پیام‌های آن‌ها را بررسی می‌کند که چگونه
    بایستی با یکدیگر تعامل داشته باشند.
\end{itemize}

\subsection{سه بخش اصلی سرویس}

\begin{itemize}
    \item \lr{Interface}
    \item \lr{Service contract}: قراردادی است که مشخص می‌کند نحوه استفاده از
    سرویس به چه صورتی باید باشد.
    \item \lr{Implementation}
    \begin{itemize}
        \item \lr{Business logic}
        \item \lr{Data}
    \end{itemize}
\end{itemize}

\subsection{\lr{Service composition} یا ترکیب سرویس‌ها}

ترکیب سرویس‌ها عموماً براساس \lr{Workflow} یک سازمان انجام می‌شود. اینکه بتوانیم
برآورد کنیم در بقیه \lr{Workflow}ها چه ترکیب‌هایی وجود دارد که می‌توانند در
سیستم ما نیز مورد استفاده قرار گیرند. در مباحث مقیاس‌پذیری باید بررسی شود که
تماماً شامل هزینه خواهد بود. بعد از \lr{Workflow} بایستی \lr{Sub-workflow} آن
سیستم را پیدا کنیم که کمک می‌کند مجموعه‌ای از تسک‌ها را در یک دسته‌بندی قرار
دهیم که اولاً بتواند تعامل داشته باشد و ثانیاً بتواند روی منابع، بهتر اجرا شود.
شکست \lr{Workflow} به چند \lr{Sub-workflow} باید به منابع سیستم هم نگاه شود.
مثال وسایل خانه نسبت به متراژ و مقدار وسیله‌ای که یک خانه می‌تواند داشته باشد.
ترکیب سرویس‌ها باید هم کاربر را نسبت \lr{SLA} راضی نگه دارد هم منابعی را تعریف
کند که بتواند خدمات را به بهترین شکل ارائه دهند.