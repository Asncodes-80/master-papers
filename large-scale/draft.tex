Ignore faulty behavior: نسبت به رفتار‌هایی که به نظر فالتی میرسه ایگنور بشه.
سیستم نسبت به پیام‌هایی که از یه منبع مشکوک میاد هیچ واکنشی نشون نده. منبع مشکوک
یعنی منبعی که توی سیستم ناشناخته است و احتمالاً شما رو به سمت فیلیور شدن سیستم
می‌برد. خارج از حوزه تعریفش باشه.

عموماً یه تاکتیک استفاده نمیشود بلکه مجموعه‌ای از تاکتیک‌ّها مورد استفاده قرار
می‌یگرد. سیستم هر موقع به اون آستانه رسید باید یه استراتژی رو طراحی کنیم که
سیستم رو از اون شلوغی که رخ داده خالی کنیم. آستانه‌ها کاملاً به مشاهده مهندس از
سیستم بستگی داره. Single criteria عمل نمی‌کنیم هیچ وقت.

degradation تنزل:
سینک آشپزخونه خراب میشه خونه رو تعطیل نمیکنیم و میگیم که اونجا خرابه استفاده
نکنین. تنزل انجام میدیم توی سیستم ولی سیستم رو اوت او سرویس نمی‌کنیم. مثل جریان
جیمیل که یه حالتی داشت می‌برد ما رو به حالت ساده که بتونیم کارایی اولیمون رو
استفاده کنیم.

reconfiguration: اساین کردن وظایف به منابع باقی مانده. reassignment وظایف میاد
به نود‌ها اساین میشه که در حقیقت کارشون اون نبوده چرا میشه چون که اون نود‌های
اصلی دیگه نیستن و باید این نود‌هایی که باقی مونده کار رو انجام بدن. هر کدوم از
نود‌های ریاساین شده حداقل کار‌ها رو میکنن.

\subsection{reintroduction}

shadow: کامپوننتی که از دسترسی ما خارج شده یک بازه زمانی در حالت شدو بمونه.
ورود‌ها از به این نود هم داده بشه که ببینم اون نود اکتیو به اون خروجی رسیده اینم
رسیده یا نه. بازه زمانی رو طراح مشخص میکنه. برای سیستم‌هایی که critical هستش که
در لحظه سیستم باید پاسخ درست رو نشون بده. هزینه بره قطعاً چون دو تا سیستم باید
همزمان کار کنن.

state resync: اگر اکتیو و پسیو باشه باشه باید ریسینک بشه کولد اسپیر که اصلاً
سینک نمیشده که بخواد ریسینک بشه. وابستگی به مانیتورینگ داره که تمامی اطلات بگه.
از این تاکتیک باید استفاده بشه.

NSF or Non Stop Forwarding: امتحانیه و جز اون سوالست. اگر تابع یه سیستمی رو
بخوایم بررسی کنیم که الگوریتم رو ورودی‌ها فعال میکنن میشه فانکشن سیستم. فرض شود
که روتری داریم سوپروایزور یا الگوریتمش درست کار نمی‌کن. به هر دلیلی الگوریتم
مسیریابی فیلد بشه. این الگوریتم رو با همون جدول هماسگی بتونه بازم مسیریابی کنه.
معضلات اینه که باید کانکتیویتی از بین نره. اصلاً نباید اتصالات از بین بره نود
خاموش شد اینا باید بتونه مسیر جدید رو پیدا کنه. دو تا الگوریتم هستن 

escalaling restart: بالابرده میشه اضافه کردن. مثال داره، شما عموماً توسیستم با
سلسله مراتب مواجه هستیم که پدر فرزندی هستند. طراح تصمیم میگیره که وقتی یه سیستمی
آبرومندانه تنزول کرده باید سلسله مراتبی بالا بیاد که کامپوننت به کامپوننت بوده و
برای بالا اومدنش هم کامپوننت هستش. اول بچه‌ها بعد پدر اول بچه‌ها بعد پدر.
graceful degradation.

\subsection{Prevent faults}

Removal from service: با software rejuvenation جوان سازی نرم‌افزار ارتباط دارد.
یک مفهنومی به اسم ممولیک وجود داشت، کامپوننت‌هایی رو می‌بینی که سیستم رو به عدم
دسترسی می‌رسونن باید قبل از اینکه فالته اتفاق بیوفته باید ازش جلوگیری بشه
وبرگرده توی مدار.

transactions

بحثی وجود داره که مسابقه بر سر منابع. که این تاکتیه که تمرینه این مشکل رو حل
میکنه.
TODO: جلسه بعد

predictive model: مانیتورینگ استیت سیستم باید باشه. مدل‌های پیشبینی. اگر ما این
تاکتیک رو انتهاب کنیم می‌توینم تصمیم‌های بهتری نسبت به سیستم بگیریم. نرخ ورود
دیتا درخواستی که زیاد میشه رو داری از مانیتور + پریدیکشن مشاهده میکنی. طول صف رو
میدونی و نسبت به اون عمل میکنی. تصمیم‌گیری‌های مهم رو می‌بریم تو سیستمی که از
پریدیکت استفاده میکنیم.

exception prevention:

increate competence set: افزایش مجموعه شایستگی‌ها. کامپوننت‌ها طوری دیزاین بشه
که مجموعه شاسیتگیشون از حوزه رفتاری بیشتر شده باشه. فرض کنیم که کامپوننت داریم
که میخوایم بهش دسترسی پیدا کنیم. در خواست رو به کامپوننت میدیم که به هر دلیلی
درخواست اکسپت نمیشه. مثلا درخواست قابل انجام نبود و دسترسی کنسل شد. یه خورده صبر
کن و یه ترای بکنه که این درخواست رو داشته باشه. کامپوننته نرخ عدم دسترس پذیری
کمتر بود دیگه تو یه فرصتی اومده توی جریان.

یعنی کامپوننت داشته باشیم که مثلا ۲ ثانیه دیگه دوباره ترای کنه دسترسی براش
برقرار بشه.