% transactions

% بحثی وجود داره که مسابقه بر سر منابع. که این تاکتیه که تمرینه این مشکل رو حل
% میکنه.
% TODO: جلسه بعد

مبحث کلاس دیاگرام‌ها

کلاس‌ها ثابت هستن

مرزی، موجودیتی و کنترلی
Boundary entity control

کلاس مرزی:

مرز بین سیستم با دنیای بیرون را مشخص میکند. که شامل سخت افزار میتواند باشد یا
پروتکل ارتباطی

کیوردی که داریم استافده میکنیم کلاس مرزی میشه توی ATM

موجودیتی:
نسبت به یک چیزی در سیستم بخوایم اطلاعاتی رو ذخیره کنیم.

کلاس کنترل:

کلاس کنترل بقیه قسمت‌ها رو به دست داره.

حتماً یه دونه کلاس کنترلی الزمای و بیشتر هم می‌تونه داشته باشه.

ساختار کلاس‌ها
نام ویژگی متد

یک نیازمندی که به ما گفته میشه. زیر تمام اسم‌ها خط می‌کشیم که این اسم‌ها یا نام
کلاس هستن یا ویژگی کلاس هستن. زیر تمام افعال هم خط می‌کشیم که میتونن احتمالاً
متد‌های کلاس باشن.

کلاسمون در تکرار‌ها یا داره جوین میشه که یه کلاس بزرگ‌تر میشه یا یه کلاس که
شکسته میشه به چند تا زیر کلاس.

کلاس‌های کنترلی رو خودمون ایجاد می‌کنیم.

requirement ro gofeteh
OOP رو گفته. یه کتابی هستش برای ATM

اول باید همیشه نمودار یوزکیس رسم بشه
قبل از سیکوئس باید کلاس دیاگرام باشه
چون بالای سیکوئس کلاس هستش که میگه کی با کی ارتباط داره.

مفهوم اینایی که میگیم تو سطح بالا متفاوته: 

Actor:

شخص یا سیستمی بیرون از سیستم ما باز سیستم تعامل دارد.
تعامل باید Single Hop باشه. two step نمیشه.

بین actor ها عملیاتی انجام نمیشه.

usecase یک کاری است که از سیستم انتظار داریم انجام بشه از جنس verb انتظار داره
یه کارایی رو بکنه مثل اخذ واحد. اگه یه سیستم مشابهی داشتیم منو‌های اون سیستم
usecase هاشه. ستینگ یه تلوزیون یوزکیسشه.

Business actor: شخص یا سیستمی که بیرون از سازمان با ما در تعامل است. سطح تجرید
درمورد بیزینس بالاتره.
business یعنی هدف اصلی سازمان. آموزش دادن هدف اصلی دانشگاهه. بیزینس بانک انجام
تراکنش‌ها. هر بیزینسی هدف کلی یک سازمان را مشخص می‌کند.

B:

BProcess ها داره کاور میشه.

مثلاً بیزینس آموزش داریم که بیزینس پراسسش جذب دانشجو هستش. 

بیزینس پراسس جذب استاد
جذت دانشجو
استپ‌هایی داره که بتونه بیزینس رو کاور کنه.

Bussiness actor: 

یارو میاد تو دانشگاه یه تایتل داره یا دانشجوعه یا استاده یا کارمنده.

دانشجو میاد دانشگاه نقش من business actor عه که قراره یه سرویسی گرفته بهش.

business worker شخصی است که داخل سیستم داره کار میکنه و روی entity داره کاری رو
انجام میده.

شخصی که داخل دانشگاه اگر استاد باشه اگر دانشجو باشه هر کاری باشه business actorعه 
ولی داخل آموزشیار actorعه.

مثلا داخل بانک یه مشتری داره میاد
اون مشتریه که هنوز کاری نکرده business actor
مشتری میره پیش یارو که براش کارو انجام بده بازم مشتری business actor عه

% هر actor الزماً بیزینس ورکر است؟ : هر اکتور می‌تواند بیزینس ورکر باشد.
% هر actor الزماً بیزینس actor است؟: هر اکتور می‌ةواند بیزینس اکتور باشد.

% هر اکتور می‌تواند بیزینس ورکر باشد.

% هر اکتور الزاماً یک بیزینس ورکر است؟

میتواند:

مثال: استاد نسبت به دانشگاه business worker عه نسبت به آموزشیار میشه اکتور
پس هر اکتور می‌تواند بیزینس ورکر باشد.

الزاما:

صدونق رفاهه باید نسبت به تمام سیستم‌ها اکتور باشه یا نه.

هر اکتور الزما بیزینس اکتوره

دانشجو بیزینس اکتوره
دانشجو الزماً اکتور آموزشیاره

پس الزامی نداره.

در عان واحد میتونین یه جا کار کنین و خدمات بگیرین

یا تو سیستم business worker ی یا business actor

تا قبل تو بانک کار میکردی
تعطیل شده دیگه رفتی یه بانک دیگه میشی busniess actor

بیزینس سطح تجریدش بالاست

business هم خیلی سطح از اون بالاتره.

یوزکیس atm رو بنویس از pdf

include یعنی یه یوزکیس از یه یوزکیس دیگه استفاده میکنه

uses: تحت شراطی استافده می‌کند.

یه یوز کیس تخفیف داریم. داریم که فقط برای مشتری‌های ۶۰ میلیونه که میشه uses

در یوزکیس اصلاً زمان نداریم. اون مال قسمت سیکوئنسه.

هر چیزی که ماهیت لمس کردن باشه اینا وتس سیستم هیچی حساب نمیشه. کلاس حساب نمیشه.

هر وقت کلاس کنترلی پیدا نکردیم خود سیستم کلاس کنترلی هستش.

--- جلسه پنجشنبه

ویژگی کیفی دوم Interoperability یه بخشی رو خودش میگه یه بخشی رو بچه‌های سرویس
ارائه میده.

تعامل باید بیشتر از ۲ تا باشه دیگه که می‌تونن به صورت معناداری تبادل اطلاعات
بکنن. برای مثال سیستم آموزشیار هستش سیستم مالی داریم سیستم پژوهشیار داریم که
باید بتونن به صورت معناداری با هم تعامل داشته باشند. یس اور نو نیست جوابمون.

هر کدوم ۱ جنرال دارن
ان تا عینی دارن.

اون جدول رو بنویس که در مورد سناریو جنرال تعامل‌پذیریه

مثال مربوط به لوکشین که سناریو عینی هستش رو مطرح کن اینجا.

پس اینم یسری تاکتیک داره

ورودی اینجا درخواست تبادل اطلاعات هستش که با یسری تاکتیک‌ها میخوایم ریکوست با
موفقیت انجام بشهک.

دو تا داره

locate

discover serivce: قبل از تعامل با اکس باید ببینیم کجاست. پس لوکیت این قسمت چی
میگه؟ چیزی مثل dns که بیایم لوکیشن سرویسی که میخوایم تعامل داشته باشیم رو بدست
بیاریم که به صورت سلسله مراتبی جست و جو را انجام می‌دهیم. فضای سرچ همیشه محدود
باید باشه.

سرویس هر گونه خدماتی هستش. سرویس چطوری ثبت میشه؟ بعضی وقتا با تایپش مثلاً سرویس
آب و هوایی. بعضی وقتا اسم سروی سرو میاریم مثل آکوآ ودر. بعضی وقتا سرویس را با
لوکیشن می‌شناسیم. نرخ Availability بالایی داشته. سرویسی که  یک یا ترکیبی با
چندین اتریوبیوت رو مشه توش بگیم. 

Mange interface:

orchestrate:

شکل‌ها رو بکش که یه شروع کننده ای داره که داره بقیه رو هندل و مدیریت میکنه.
عموماً یه سرویسی است که نقش هماهنگ کننده رو داره کار خاصی انجام نمیده.

Choreography:

قبلی و بعدی ما هم در ارتباط هستن شروع نداره که هدی پاسخگو باشه.

% تمرین: زبان Orchestrate و زبان Chroegraphy را تحقیق کنید.

tailor interface:

یه واژه ای هست به اسم دکوریت که خودمون با نظر خودمون میایم سیتم رو دکور می‌کنیم.
Decorate pattern هم داریم. که قابلیت‌هایی که یک سیستم داره براساس شرایط ممکنه
تغییرش بدیم. مثلا دو تا فرم داریم اگر کاربر ip ایران بود این ببینه در غیر این
صورت اون رو ببینه.

یه نمودار سلسه مراتبی هستش.

enterprise serice bus یا ESB که تو سرویس هستش.

نمیشه دو تا سرویس داشته باشیم که تنها با هم صحبت کرد. پس نیاز به esb داریم که
کارش اینه مدیریت پیام‌ها از طریق این انجام بشه.

esp یکسری ادامپتر داره که هر موقع سرویس آ به سوریس ب می‌خواد ادرتبط بر قرار کنه
پیام‌هاشون رو میان بررسی می‌کنن که چطوری باید تعامل داشته باشن.

% ویژگی‌های ESP رو توی ارائه بگو
% که شناسایی پترن‌های ناهنجار رو باید بگی SEP
% چهارتا ویژگی SOA رو بگیم.
% بخش service composition مال ماست.

سرویس سه تا قسمت داره که سوال امتحانه
این همون دکوریت کردنست.
interface A , B  Serivce contract Implementation
بیزینس لاجیک و دیتا مال پیاده‌سازیه

قراردادی که مشخص میکند نحوه استفاده از سرویس به چه صورتیه. Serivce contract

Service Composition:

یکی از چالش‌های مهمی که توی پایان‌نامه میشه تعریف کنیم ترکیب سرویس‌هاست. که
عموماً براساس ورکلوی یک سازمان انجام می‌دهیم. بتونیم حدس بزنیم که تو بقیه فلوها
چه ترکیب‌هایی هستن که می‌تونن تو سیستم ما مورد استفاده قرار بگیرن. در مباحث
اسکیل و مقایس پذیری باید مورد بررسی قرار بگیره که تماماً شامل هزینه هستش.

بعد از ورک‌فلو باید ساب ورکفلو رو پیدا کنیم که کمک میکنه مجموعه‌ای از تسک‌ها در
یه دسته‌بندی قرار بگیره که اولاً بتونه تعامل داشته باشه و بتونه روی منابع بهتر
اجرا بهش. شکست ورکفلو به چند ساب باید به طرف منبع هم نگاه کنیم. مثال وسایل خونه
نسبت به فضای خونه هستش.

ترکیب باید هم یوزر رو نسبت به sla راضی نگه داره هم نسبت به منبعی که وجود داره
بتونه خدمات رو بده.

