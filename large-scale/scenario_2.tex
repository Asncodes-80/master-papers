سیستم مدیریت بحران شهری

Quality measures:

+ Availability:

کاربران امکان ارسال پیامک، تصاویر و فیلم‌ها رو دارند.

Heartbeat:
استفاده از تکنیک هارت بیت برای اینکه ببینیم سرویس‌ها حضور دارن یا نه

Monitoring:
استفاده از مانیتورینگ تا اگر درخواست‌های نامربوط ارسال شد شناسایی بشه.

Timestamp:
داده‌هایی که ممکنه به صورت توالی این از ایونت‌ها ذخیره کنیم چون میتونه به صورت
توزیع شده باشه از تایم‌استمپ استفاده کنیم.

Voting -> Replication:
استفاده از وتینگ مناسب که میشه ریپلیکیشن چون باید چندین سرور بتونن اطلاعاتمون رو
نگهدارن نا از دست نره.

Passive Redundancy:
این سرویس‌ها باید پسیو ریداندنت باشن که سرویس در صورت خرابی در دسترس باشه.

Condition monitoring:
چون ممکنه از فیلم و عکس پشتیانی کنیم احتمالاً بایستی روی شرایط ارسال اطلاعات هم
دقتی داشته باشیم، مثلاً ممکنه توی اپ موبایل امیج پراسسینگ داشته باشیم.

Retry
چون حجم فیلم ممکنه بسته موبایلی که کپچر میکنه متغیر باشه باید امکان ری‌ترای رو
براش پیاده‌سازی کنیم.

Degradation:
چون ممکنه کار خیلی اساسی باشه و یه قسمتی از سرویس اوت او د سرویس بشه می‌تونیم
فیچر‌های مهم رو توی یه سیستم دیگه براش بذاریم که بتونه کار اس او اس رو انجام
بده.

قطعا از عمل NSF استفاده می‌کنیم.

Software update

+ Interoperatibility:

من از orchestration استفاده می‌کنم.

می‌تونیم لوکیشن رو در نظر بگیریم که سرویس‌ها رو اگه orchestrator میذاریم بتونه
بسته به منطقه هندل کنه که لود‌بالانسر بهتری داشته باشه.