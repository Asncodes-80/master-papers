\documentclass[a4paper]{article}
\usepackage{forest}
\usepackage{float}
\usepackage{makecell}
\usepackage{geometry}
\usepackage{listings}
\usepackage{hyperref}
\usepackage{graphicx}
\usepackage{ragged2e}
\usepackage{color}
\usepackage{xepersian}
\usepackage{subfiles}
\settextfont[Scale=1]{XB Roya}

% Code command color
\definecolor{dkgreen}{rgb}{0,0.6,0}
\definecolor{gray}{rgb}{0.5,0.5,0.5}
\definecolor{mauve}{rgb}{0.58,0,0.82}
\definecolor{commentColor}{rgb}{0.6,0.6,0.60}

% Code style configuration
\lstset{frame=tb,
  language=python,
  aboveskip=1mm,
  lineskip=0.9mm,
  belowskip=1mm,
  showstringspaces=false,
  showspaces=false,
  columns=flexible,
  basicstyle={\small\ttfamily},
  numbers=none,
  keywordstyle=\color{mauve},
  commentstyle=\color{commentColor},
  stringstyle=\color{dkgreen},
  numberstyle=\small\color{black},
  numbers=left,
  stepnumber=1,
  breaklines=true,
  breakatwhitespace=true,
  tabsize=3
}


\title{جلوگیری از \lr{Exceptions} در برنامه‌های نرم‌افزاری}
\author{علیرضا سلطانی نشان}

\begin{document}
\maketitle

\section{چکیده}

در این روش که یک نوع تاکتیک برای افزایش ویژگی کیفی \lr{Availability} مورد
استفاده قرار می‌گیرد هدف آن است که نرم‌افزار نوشته شده عاری از هر گونه
\lr{Fault} باشد که توسط توسعه دهنده‌ در هنگام توسعه مورد بررسی و آزمون
نرم‌افزاری قرار نگرفته است. این کار باعث می‌شود که نرم‌افزار با کمترین مشکل به
دست استفاده کننده برسد و در صورتی که \lr{Fault} در نرم‌افزار رخ داد،‌این
\lr{Fault} باعث از بین رفتن چرخه عمر نرم‌افزار نشود و کاربر کمافی‌سابق بتواند از
نرم‌افزار مورد نظرش استفاده کند و نیاز‌هایش را بر طرف سازد.

ما به عنوان توسعه‌دهنده می‌توانیم با استفاده از استراتژی‌هایی از هر گونه
\lr{Exception} در نرم‌افزار جلوگیری کنیم. این استراتژی‌ها عبارت‌اند از:

\section{\lr{Smart Pointers}}

در زبان \lr{C++} برای مدیریت حافظه به شکل هوشمند و خودکار طراحی شده‌اند نقش مهمی
برای جلوگیری از نشتی حافظه \footnote{\lr{Memory Leak}} دارند. سه نوع رایج از
\lr{Smart pointer}ها وجود دارد که هر کدام ویژگی منحصر به فرد خود را دارا هستند:

\subsection{\texttt{\large std::unique\_ptr}}

این نوع از پوینتر‌ها به شکل اختصاصی مالکیت یک منبع را بر عهده می‌گیرند و هیچ وقت
اجازه نمی‌دهد که به صورت همزمان دو پوینتر متفاوت به یک منبع دسترسی داشته باشند
که به \lr{Deadlock} برخورد کند.

\begin{LTR}
    \texttt{std::unique\_ptr<int> aPointer = std::make\_unique<int>(10);}
\end{LTR}

در این حالت \texttt{\large aPointer} تنها حافظه را در اختیار دارد و پس از
اتمام استفاده از حافظه، حافظه را به صورت خودکار آزاد می‌کند.

\subsection{\texttt{\large std::shared\_ptr}}

از این پوینتر زمانی استفاده می‌کنیم که چند پوینتر می‌خواهند به یک منبع دسترسی
داشته باشند. در این پوینتر شمارنده‌ای وجود دارد که تعداد پوینتر‌هایی که در حال
استفاده از منبع هستند را نشان می‌دهد و به محض اینکه حافظه آزادسازی شود، منابع
آزاد می‌شوند و شمارنده نیز به روز می‌شود.

\begin{LTR}
    \texttt{std::shared\_ptr ptrA = std::make\_shared<int>(20);}
    \texttt{std::shared\_ptr ptrB = ptrA;}
\end{LTR}

در مثال بالا پوینتر \texttt{ptrB} به منبع‌ای اشاره دارد که \texttt{ptrA} از آن
استفاده می‌کند.

\subsection{\texttt{\large std::weak\_ptr}}

نسخه ضعیف شده‌ای از پوینتر‌های \texttt{shared\_ptr} هستند که مانع بروز
\lr{Circular refs} می‌شوند. اگرچه به منبع مورد استفاده اشاره می‌کند اما هیچ وقت
مالک منبع نیست و در سمت شمارش منابع در پوینتر افزایش رخ نمی‌دهد.

\begin{LTR}
    \texttt{std::shared\_ptr<int> ptrA = std::make\_shared<int>(30);}    
    \texttt{std::weak\_ptr<int> ptrC = ptrA}
\end{LTR}

به طور کلی استفاده از پوینتر‌ها باعث مدیریت حافظه می‌شوند که در نهایت از هر نشت
حافظه جلوگیری به عمل می‌آورد. از طرفی دیگر کد را ایمن‌تر و مقاوم‌تر در برابر
\lr{Fault}ها می‌کند. بیشتر از \lr{Smart pointer}ها برای مدیریت پویا و امن حافظه
استفاده می‌شود.

\section{\lr{Abstract data type ADT} یا نوع داده انتزاعی}

یک نوع ساختار داده‌است که در آن تعریف می‌شود چه عملیاتی می‌توان روی داده‌ها
انجام داد. برای توضیحی ساده‌تر رفتار کلاس‌ها را می‌توانیم به صورت چکیده و به دور
از جزئیات در یک ساختار داده تعریف کنیم.

مهم‌ترین مزیت استفاده از آن در جلوگیری از \lr{Exception}ها این است که به دلیل
انتزاع بالا برنامه نویس نیازی به دانستن جزئیات پیاده‌سازی در برخی قسمت‌ها ندارد
و آن را با تعریف در \lr{ADT} می‌تواند در قسمت‌های بعدی مدیریت کند و سپس وارد
جزئیات شود. مزیت دیگر آن فهمیدن بهتر کد است که پیاده‌سازی رفتار کلاس‌ها را
آسان‌تر می‌کند.

\begin{LTR}
    \begin{lstlisting}
        class Value {
            public int value;
            public increase() int;
            public decrease() int;
            private initialized(v: int) void;
        }
      \end{lstlisting}
\end{LTR}

\section{\lr{Wrapper}ها}

رپر‌ها نوعی ساختار یا کلاس هستند که به دور یک \lr{Api} یا ساختمان داده پیچیده‌تر
قرار می‌گیرند و رابطی ساده‌تر و ایمن‌تر را تشکیل می‌دهند. رپر‌ها معمولاً برای
ساده‌سازی و محافظت از منطق پیچیده‌ای استفاده می‌شوند که در داخل یک \lr{Api} یا
کتابخانه‌ای وجود دارند.

برای مثال در \lr{Api}هایی که سطح پایین هستند و منابع سیستمی به صورت دستی مدیریت
می‌شوند می‌توان با استفاده از یک رپر این منابع را به صورت خودکار در اختیار گرفته
و در پایان آزاد کند و از بروز استثنا و خطا‌های حافظه جلوگیری کند.

% \bibliographystyle{unsrt-fa}
% \bibliography{normalization_refs.bib}
\end{document}
