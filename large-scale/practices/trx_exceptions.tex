\documentclass[a4paper]{article}
\usepackage{forest}
\usepackage{float}
\usepackage{makecell}
\usepackage{geometry}
\usepackage{listings}
\usepackage{hyperref}
\usepackage{graphicx}
\usepackage{ragged2e}
\usepackage{color}
\usepackage{xepersian}
\usepackage{subfiles}
\settextfont[Scale=1]{XB Roya}

\title{تحقیق در مورد \lr{2 Phase Commit} و \lr{3 Phase Commit}}
\author{علیرضا سلطانی نشان}

\begin{document}
\maketitle

\section{معرفی پروتکل‌ها}

پروتکل‌های \lr{2PC} و \lr{3PC} در سیستم‌های توزیع شده جهت یکپارچگی داده‌ها مورد
استفاده قرار می‌گیرد. هدف اصلی این پروتکل‌ها این است که \lr{Participants} یا
مشارکت‌کنندگان در یک تراکنش توزیع شده، همگی یا با تراکنش موافقت کنند یا همگی آن
را لغو یا \lr{Abort} کنند که از ناسازگاری داده‌ها جلوگیری شود
\cite{npc,consensus}.

\subsection{پروتکل \lr{2PC}}

این پروتکل در دو مرحله به تراکنش‌ها و درخواست‌ها رسیدگی می‌کند:

\subsubsection{مرحله آمادگی یا \lr{Prepare Phase}}

در این مرحله دستگاهی به نام \lr{Coordinator} وجود دارد که درخواست اجرای تراکنش
را به تمام مشارکت‌کنندگان حاضر در شبکه توزیع شده ارسال می‌کند. هر گره بررسی
می‌کند که شرایط اجرای تراکنش را دارد یا نه:

\begin{enumerate}
    \item اگر هر گره بتواند تراکنش را اجرا کند وضعیت \lr{Ready} یا آماده بودن را
    در \lr{State} خود ذخیره می‌کند.
    \item اگر هر گره نتواند تراکنش را اجرا کند وضعیت خود را در حالت \lr{Abort}
    ذخیره می‌کند.
\end{enumerate}

اگر حتی یک گره نتواند تراکنش را اجرا کند، کل تراکنش در سیستم توزیع شده متوقف
می‌شود.

\subsubsection{مرحله کامیت یا لغو \lr{Commit/Abort Phase}}

اگر همه گره‌ها پاسخ آماده را در \lr{State} خود ذخیره کرده باشند،
\lr{Coordinator} فرمان \lr{Commit} را ارسال می‌کند تا تراکنش در همه گره‌ها اجرا
شود. در این حالت است که اگر حتی یک گره وضعیت لغو را ذخیره کرده باشد فرمان
\lr{Abort} ارسال می‌شود و تمام گره‌ها تمام فعالیت‌ها و فرایند‌هایی که طی کرده
بوده‌اند را \lr{Rollback} می‌کنند.

نکته بسیار مهم آن است که اگر در پروتکل \lr{2 Phase Commit} سیستم
\lr{Coordinator} خراب شود و یا بر هر نحوی در دسترس نباشد، تمام گره‌های داخل شبکه
توزیع شده در حالت بلاتکلیفی یا \lr{Blocked} قرار می‌گیرند. همچنین اگر گره‌ای در
شبکه قطع شود بقیه گره‌ها نیز در وضعیت \lr{Blocked} قرار می‌گیرند زیرا در انتظار
جواب آن گره هستند.

\subsection{پروتکل \lr{3 Phase Commit}}

این پروتکل یک پروتکل بهبود یافته نسبت به \lr{2PC} می‌باشد که مشکل بلاتکلیفی
گره‌ها را حل می‌کند.

این پروتکل برای اجرای تراکنش در سیستم‌های توزیع شده ۳ مرحله را طی می‌کند:

\subsubsection{مرحله \lr{Prepare Phase}}

دقیقاً مشابه با پروتکل \lr{2PC} گره \lr{Coordinator} همه گره‌ها را مطلع می‌کند
که تراکنشی می‌خواهد انجام شود.

\subsubsection{مرحله \lr{Pre-commit Phase}}

اگر همه گره‌ها پاسخ \lr{Ready} را داشته باشند سیستم \lr{Coordinator} یک پیام
\lr{Pre-commit} ارسال می‌کند. این پیام به سایر گره‌ها اطلاع می‌دهد که به احتمال
زیاد تراکنش انجام و کامیت خواهد شد و گره‌ها باید وضعیت خود را ذخیره کنند. ذخیره
این پیام در سطح \lr{Pre-commit} می‌تواند کاملاً بهینه‌تر نسبت به پروتکل پیشین
باشد.

\subsubsection{مرحله کامیت یا لغو \lr{Commit/Abort}}

سیستم \lr{Coordinator} بعد از دریافت پیام تاییدیه \lr{Pre-commit} از تمام
گره‌های شبکه پیام کامیت را ارسال می‌کند. اگر مشکل پیش آید پیام \lr{Abort} را
ارسال می‌کند.

\subsection{نکته}

\begin{itemize}
    \item پروتکل \lr{3PC} همانند پروتکل \lr{2PC} نیست و احتمال بلاتکلیفی سایر
    گره‌ها را به شدت کمتر می‌کند چرا که اگر سیستم \lr{Coordinator} از بین رود،
    هر گره‌ای در شبکه اطلاعات \lr{Pre-commit} خود را ذخیره دارد و می‌تواند به
    گره‌های دیگر اطلاع دهد که آیا می‌تواند تراکنش را به طور کلی کامیت کنند یا
    نه. در حقیقت جایگزینی برای نود \lr{Coordinator} هستند.
    \item پروتکل \lr{3PC} نسبت به \lr{2PC} در طراحی پیچیده‌تر می‌باشد.
    \item در پروتکل \lr{2PC} بلاک شدن سریع می‌تواند اتفاق بیوفتد اما در \lr{3PC}
    بلاتکلیف شدن سایر گره‌ها به شدت کاهش پیدا می‌کند.
    \item پروتکل \lr{2PC} مناسب برای سیستم‌های توزیع شده کوچک می‌باشد که ایمنی
    زیادی ندارد اما در مقابل \lr{3PC} می‌تواند به دلیل مقاوت بیشتر از نظر
    \lr{Reliability} در سیستم‌های توزیع شده بزرگ مورد استفاده قرار گیرد.
\end{itemize}

\bibliographystyle{unsrt-fa}
\bibliography{trx_except_refs.bib}
\end{document}