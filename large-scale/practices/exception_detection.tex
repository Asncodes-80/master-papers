\documentclass[a4paper]{article}
\usepackage{forest}
\usepackage{float}
\usepackage{makecell}
\usepackage{geometry}
\usepackage{listings}
\usepackage{hyperref}
\usepackage{graphicx}
\usepackage{ragged2e}
\usepackage{color}
\usepackage{xepersian}
\usepackage{subfiles}
\settextfont[Scale=1]{XB Roya}

\title{شناسایی \lr{Fault}های نرم‌افزاری}
\author{علیرضا سلطانی نشان}

\begin{document}
\maketitle

\section{\lr{Parameter fence}}

در حقیقت همانند \lr{Checksum} می‌ماند. که شامل گنجاندن یک الگوی مشخصی از داده از
پیش تعریف در حافظه مانند \texttt{0xDEADBEEF} می‌باشد که بلافاصله پس از
\lr{Variable-length parameters} یک شیء قرار می‌گیرد. این الگوی داده به عنوان یک
\lr{sentinel} عمل می‌کند تا به هنگام دسترسی یا پردازش آن پارامتر‌های متغیر از
بروز \lr{Buffer overflow} جلوگیری کند.

\section{\lr{Parameter typing}}

روشی است که در آن نوع پارامتر‌های ورودی توابع به صورت دقیق تعریف و کنترل می‌شود.
این روش باعث می‌شود که تنها مقادیر سازگار با نوع مورد نظر بتواند به عنوان ورودی
تابع یا \lr{Method} ارسال شود و هر بار که داده‌ای با نوع اشتباه وارد شود به خطای
کامپایل یا \lr{Run-time} دچار می‌شود.

استفاده از حافظه‌های پویا همانند زبان‌های \lr{Python} و \lr{Javascript} بسیار
پیچیده و ناایمن هستند. چرا که شما در یک ورودی می‌توانید داده‌ای از نوع
\texttt{str} را وارد کنید و در ورودی دیگر از همان تابع می‌توانید متغیری از نوع
\texttt{int} وارد کنید. این گونه زبان‌ها اصطلاحاً \lr{Type-safe} نیستند.
زبان‌هایی مانند \lr{C++}، \lr{Rust}، \lr{Dart} و مشابه آن‌ها در تعریف متغیر‌ها و
نوع آن‌ها کاملاً سختگیرانه عمل می‌کنند.

به طور کلی استفاده از نوع داده‌ای مشخص می‌تواند از \lr{TypeError} به عنوان یک
\lr{Exception} در نرم‌افزار در حال اجرا به شکل ویژه‌ای جلوگیری کند.

\section{\lr{Timeout}}

همانطور که از نامش پیداست عموماً در سیستم‌ها یک مقدار مشخصی از زمان را تعیین
می‌کنند که در آن محدود سیستم بایستی نهایتاً قادر به پاسخگویی باشد. در صورتی که
سیستم نتواند در آن زمان به مولفه‌ای پاسخ دهد تا زمان مشخص شده صبر می‌کند و اگر
تا انتهای آن مولفه پاسخی را دریافت نکرد سیستم \lr{Exception} مربوط به
\lr{Timeout} را اعلام می‌کند. برای مثال سیستم‌های \lr{Modbus} مقدار زمان
\lr{Timeout} آن‌ها ۱۰ ثانیه می‌باشد که پس از آن به ارور \lr{Timeout} برخورد
می‌کند.
\end{document}