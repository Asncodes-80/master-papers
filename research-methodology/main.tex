% !TEX program = xelatex
\documentclass[20pt, a4paper]{article}
\usepackage{hyperref}
% Justify paragraphs
\usepackage{graphicx}
\usepackage{ragged2e}
\usepackage{color}
\usepackage{xepersian}
\usepackage{subfiles}
\settextfont[Scale=1]{XB Roya}
\renewcommand{\baselinestretch}{1.5}

\begin{document}
\centerline{روش تحقیق}
\centerline{علیرضا سلطانی نشان}
\centerline{\today}
\tableofcontents

\newpage

\section{مقدمه}

\subsection{پیدا کردن سر فصل کاری مناسب با سمینار}

\begin{enumerate}
    \item پیدا کردن موضوعی که برای محقق اهمیت دارد
    \item پیدا کردن زمینه مطالعاتی مفید و مناسب
    \item صحبت با اساتید مربوطه
    \item همفکری
    \item سخنرانی‌ها و مطالب اخیر نشریات مختلفی را در زمینه مورد نظر خود مطالعه
    کنید
\end{enumerate}

\subsection{مطالعه و ارزیابی مقالات انتخاب شده به طور کامل}

\begin{enumerate}
    \item چه فرضیه‌هایی است که به نظر می‌رسد بیشتر محققان انجام می‌دهند
    \item محققان از چه روش‌هایی استفاده می‌کنند؟
    \item یافته ‌های تحقیقاتی و نتیجه‌گیری‌های انجام شده را ارزیابی و تلفیق کنید
    \item نظریه‌ها، نتایج و روش‌های متضاد را مد نظر داشته باشید
    \item محبوبیت نظریه‌ها و چگونگی تغییر یا عدم تغییر آنها را در طول زمان بررسی
    کنید
\end{enumerate}

\subsection{سازماندهی مقالات انتخاب شده با دنبال کردن الگو‌ها و توسعه زیر
مجموعه‌ها}

\begin{enumerate}
    \item یافته‌های رایج و یا بحث برانگیز
    \item دو یا سه روش مهم در تحقیق
    \item تاثیر گذار ترین نظریه‌ها
\end{enumerate}

\subsection{ادامه سازماندهی}

\begin{itemize}
    \item فصل سرفصل‌های سمینار را بنویسید
    \item اگر موضوع پبیشینه پژوهی شما گسترده‌است، یک میز بزرگ تهیه کنید و از
    استیکر‌ها یا کارت‌های دسته‌بندی برای سازماندهی یافته‌های خود در دسته‌های
    کوچک استفاده کنید.
\end{itemize}

\subsection{سند سمینار}

\subsection{نگاهی به آنچه که نوشته‌اید. تمرکز بر روی تجزیه و تحلیل، نه تشریح}

\section{پروپوزال}

برگه‌ای است که در آن پیشنهاد مکتوب در مورد موضوع و ماله پژوهش و بیان هدف و
انگیزه و زمان‌بندی انجام کار را حاوی است.  ارزیابی این برگه توسط شورای تخصصی
گروه آموزشی صورت می‌گیرد.

\section{پایان‌نامه}

گزارش علمی از کار انجام شده در راستای دستیابی به اهداف مشخص و یافته‌های موضوع
پژوهش در پروپوزال است. ارزیابی آ» توسط کمیته داوران گروه با لحاظ نمره انجام
می‌شود.

\section*{مقاله}

ارائه نتایج کار اصیل که دارای نوآوری است به جامعه علمی متخصص در زمینه موضوع
پژوهشی. ارزیابی آن توسط متخصصین بین المللی قبول یا رد می‌شود.


\section{اهمیت تحقیق}

\section{فرایند تحقیق}

\section{سمینار و پروپوزال و پایان‌نامه}

\section{سامانه پژوهشیار}

\section{سرقت علمی}

\section{پایگاه‌های اطلاعاتی}

\section{همایش و مجله}

\section{ارائه علمی}

\end{document}