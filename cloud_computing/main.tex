\documentclass[a4paper]{article}
\usepackage{float}
\usepackage{geometry}
\usepackage{listings}
\usepackage{hyperref}
\usepackage{plantuml}
\usepackage{graphicx}
\usepackage{ragged2e}
\usepackage{color}
\usepackage{xepersian}
\usepackage{subfiles}
\newgeometry{left=1.4cm, right=1.4cm, bottom=2.0cm, top=2.0cm}
\settextfont[Scale=1]{XB Roya}
\renewcommand{\baselinestretch}{1.5}
\definecolor{dkgreen}{rgb}{0,0.6,0}
\definecolor{gray}{rgb}{0.5,0.5,0.5}
\definecolor{mauve}{rgb}{0.58,0,0.82}
\definecolor{commentColor}{rgb}{0.6,0.6,0.60}

\title{رایانش ابری، دکتر سیدجوادی}
\author{علیرضا سلطانی نشان}

\begin{document}
\maketitle
\tableofcontents

\section{مجوز}

به فایل license همراه این برگه توجه کنید. این برگه تحت مجوز GPLv3 منتشر شده است
که اجازه نشر و استفاده (کد و خروجی/pdf) را رایگان می‌دهد.

\section{مراجع}

براساس مقاله جلو میریم.

سه سرویسی که محیط ابر اختصاص میده به کاربران:

سیستم‌های توزیع شده: یک سیستم توزیع شده متشکل از سیستم‌های کامپیوتری است که به
شکل مستقل یک کامپیوتر می‌باشد. که در نقاط مختلف می‌توانند با هم تعامل کنند
ساختار منابع آن‌ها می‌تواند به صورت اشتراکی باشند. یک سیستم توزیع شده باید
مقایس‌پذیر باشه.

شفافیت از ویژگی‌های مهم توزیع شده‌هاست.


پیاده‌سازی و شبیه‌سازی مربوط به محیط کلاد و محاساباتی که در آن انجام می‌شود.

transparent: پنهان بودن پشت زیر ساخت سیستم‌های توزیع شده. دقیقاً‌مثال گوگل که که
معلوم نیست پشت پردسش. مثلا یه سرور قطع بشه سرویس‌دهی باید ادامه پیدا کنه.

ابعاد مختلف transparency:

accessibility

Relocation: جا به جایی داده‌ها بین سیستم‌های دیگر، به طوری که جا به جایی اگر
اتفاق افتاد، کاربر بازم بتواند با سیستم بدون مشکل و بدون اینکه متوجه شود کار
کند.

Scaleability

Openness: سیستم‌هایی که کاملا آزادانه می‌توان آنها را سفارشی‌سازی کرد. انقدر
سیستم ماژولار باشه تا بتوانیم سیستم را محدود یا گسترش بدیم.




\end{document}