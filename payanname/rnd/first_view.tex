\documentclass[10pt, a4paper]{article}
\usepackage{geometry}
\usepackage{listings}
\usepackage{hyperref}
\usepackage{graphicx}
\usepackage{ragged2e}
\usepackage{color}
\usepackage{xepersian}
\usepackage{subfiles}
\newgeometry{left=3cm, right=3cm, bottom=3cm}
\settextfont[Scale=1]{XB Roya}
\renewcommand{\baselinestretch}{1.5}

\begin{document}
\centerline{علیرضا سلطانی نشان}
\centerline{بررسی مقالات جست و جو شده}
\centerline{\today}
\tableofcontents

\section{مقدمه}

این برگه خلاصه‌ای از بررسی اولیه مقالات در مورد IoT در حوزه سلامت می‌باشد.
ساختار کلی این برگه به شکل زیر است:

\begin{enumerate}
    \item نام مقاله به صورت واضح نوشته شود
    \item کلیدواژه‌های مقاله مورد نظر را یادداشت کنید
    \item تاریخ نگارش مقاله مورد نظر را ذکر کنید
    \item مشخص کنید که این مقاله دستاورد دانشجویان کدام دانشگاه بوده است
    \item توضیح اولیه در مورد مقاله بنویسید
\end{enumerate}

\newpage

\section{\lr{The metaverse in medicine}}

نام مقاله: \lr{The metaverse in medicine}

کلیدواژه‌های مقاله: 
\lr{Metaverse},
\lr{Digital teching},
\lr{Telemedicine},
\lr{Artificial},
\lr{intelligence},
\lr{Virtual surgery}

ژورنال سال ۲۰۲۳ از جامعه اروپایی قلب و عروق

گروه علوم قلب و عروقی، پلی کلینیک بنیاد آگوستینو، رم، ایتالیا، دانشگاه کاتولیک
قلب مقدس

دیدگاه کلی:

متاورس یک جایگزین برای زندگی دیجیتالی است که به صورت اختصاصی هر فردی می‌تواند از
آن استفاده کند. در دنیای امروز اشکال مختلفی از حضور هوش مصنوعی را شاهد هستیم،
مانند استفاده از آوتار‌ها که یک فرم کپی شده از انسان‌های حقیقی در دنیای دیجیتال
هستند. اینگونه فناوری‌ها امکانات زیادی را فراهم کرده‌اند، برای مثال جراحی از راه
دور \footnote{\lr{Remote surgery}} که جراح و بیمار از هزاران کیلومتر دورتر
می‌توانند در ارتباط با یکدیگر باشند که می‌تواند امکان تجسم داده‌های بالینی بیمار
را در زمان واقعی داشته باشد مانند تصویر برداری‌های معنادار
\footnote{\lr{Diagnostic images}}.  همچنین می‌تواند امکان این را فراهم کند که
درمان‌های پزشکی \footnote{\lr{Medical treatments}} و پروتکل‌های دارویی
\footnote{\lr{Pharmacological}} روی انسان‌های مجازی مشابه با بیماران حقیقی به
صورت درمان‌های بالینی انجام شود. به همین ترتیب مشاهده تاثیر درمان‌ را به صورت
پیشرفته میسر می‌کند و طول انجام آزمایشات بالینی را به طور چشم‌گیری کاهش می‌دهد.
متاورس ابزار آموزشی استثنایی را ارائه می‌دهد که می‌توان از طریق آن آموزش‌های
تعاملی دیجیتال را با شبه بیماران یاد گرفت. کاربرد آن نه تنها در آموزش و یادگیری
بلکه در جهت جراحی پیشرفته یا اقدامات پزشکی بر روی شبه بیماران برای مشاهده تکامل
فرایند آسیب شناسی نیز مورد استفاده قرار می‌گیرد. از آنجایی که امروزه هوش مصنوعی
در حوزه تشخیص درمان بالینی بیماران شناخته شده و ثابت شده است، استفاده از متاورس
در این حوزه می‌تواند بسیار مفید باشد در حالی که پیشرفت زیادی را از این تکنولوژی
انتظار می‌رود.

در پایان، ارتباط بین هوش مصنوعی و متاورس به شدت پیچیده می‌باشد. اغلب مدل‌های
مختلفی از هوش مصنوعی می‌توانند به عنوان پایه و اساس اجزای یک مدل خلاقانه‌ای از
برنامه متاورس باشد. (منظور آن است که هر بخش متاورس به عنوان یک ماژول پیشرفته به
آن می‌بایست پرداخت که هر کدام موتور هوش مصنوعی مخصوص به خودشان را دارند و حتی
بسیاری از اجزای دیگر). با توجه به پتانسیلی که متاورس به صورت بی‌نهایت دارد، به
صورت فوری و سریع نمی‌توان از هوش مصنوعی به متاورس آپگرید کرد. اما در حوزه سلامت
و درمان می‌تواند به عنوان یکی از تاثیرگذارترین ابزار‌هایی باشد که در آینده توسعه
زیادی روی آنها انجام خواهد شد.

\newpage

\section{\lr{IoT based medical image encryption}}

نام کامل مقاله: \lr{IoT based medical image encryption using linear feedback shift
register Towards ensuring security for teleradiology applications}

کلیدواژه‌های مقاله: 
\lr{IoT},
\lr{Encryption},
\lr{Teleradiology},
\lr{Image processing},
\lr{Computed tomography}

کالج دانشگاه لینکلن، کوتا بهارو، مالزی - ژانویه ۲۰۲۳

مقاله انتشار یافته در ELSEVIER

دیدگاه کلی:

پزشکی از راه دور امروزه سناریوی یک دستاورد بزرگ است و از آن مهم‌تر محافظت از
دارایی‌هایی که از آن ایجاد می‌شود مانند تصاویر نقش حیاتی را برای انتقال داده‌های
احراز هویت شده ایفا می‌کند. زمانی که ارسال اطلاعات پزشکی از طریق یک شبکه ابری
انجام می‌شود، بایستی از امنیت اطلاعات اطمینان حاصل کرد. این تحقیق بر روی
رمزنگاری تصاویر پزشکی با استفاده از ساختار \lr{Linear Feedback Shift Register}
کار می‌کند. این الگوریتم اعداد شبه تصادفی ایجاد می‌کند و جایگاه پیکسل‌ها را از
حالت صحیح بهم می‌زند. بعد از انجام فرایند رمزنگاری تصویر
\footnote{Stagnography}، این تصویر از طریق شبکه ابری به سمت دریافت کننده اطلاعات
ارسال می‌شود، دریافت کننده اطلاعات تصویر را رمزگشایی و آن را به تصویر اصلی
بازیابی می‌کند. تصویر برداری دیجیتال و ارتباطات در درمان یا DICOM
\footnote{\lr{Digital Imaging and Communications Medicine}} تصاویر توموگرافی
پردازش شده در این تحقیق مورد استفاده قرار گرفتند و برای پیاده‌سازی آن در IoT از
یک پردازنده دستگاه رزپبری پای \lr{B+} استفاده شده است. پیاده‌سازی اینترنت اشیا
انتقال داده‌ها را از طریق شبکه ابری تسهیل می‌کند. محققان این مقاله توانستند از
طریق انجام PSNR \footnote{\lr{Peak to Signal Noise Ration}} و میانگین مربعات خطا
یا MSE \footnote{\lr{Means Square Error}}، کیفیت تصاویر رمزگشایی شده را ارزیابی
کنند که نتیجه کار تحقیقاتی راه را برای محققان در انتقال امن داده‌های پزشکی از
طریق پلتفرم‌های ابری را آسان می‌کند.

\newpage


\section{\lr{The Metaverse in Current Digital Medicine}}

کلیدواژه‌های مقاله: 
\lr{Metaverse},
\lr{Digital medicine},
\lr{Virtual Reality},
\lr{Internet of Things},
\lr{COVID-19},
\lr{COPD}

مقاله سلامت الکترونیک بالینی - ژولای ۲۰۲۲

گروه پزشکی ریوی و مراقبت‌های ویژه، بیمارستان شانگهای، دانشگاه فودان، مرکز
تحقیقات مهندسی و فناوری چین - اینترنت اشیا برای پزشکی تنفسی با همکاری دانشگاه
چین و نیویورک

متاورس در کنار فناوری‌های دیگر مانند واقعیت مجازی، اینترنت اشیا و بلاک‌چین، وارد
زندگی مردم شده است. در سیستم مراقبت‌های بهداشتی کنونی، مدیریت بیماری‌های مزمن از
قبیل انسداد مزمن ریه \footnote{\lr{Chronic obstructive pulmonary diseases}} یا
به اختصار (COPD) و سندروم آپنه انسدادی خواب-هیپونه یا به اختصار (OSAHS) هنوز با
چالش‌هایی رو به رو است. این چالش‌ها مانند توزیع نابرابر منابع پزشکی و دشواری در
پیگیری و غیره، می‌باشند. فناوری پلتفرم‌های درمانی متاورس، با فناوری‌های پیشرفته
هوش مصنوعی ترکیب شد‌ه‌اند، از قبیل دو قلو‌های دیجیتال در مقیاس صنعتی، ممکن است
این چالش‌ها را برطرف کنند. در این مقاله به چشم انداز کاربرد این فناوری‌ها در
حوزه درمان دیجیتال و آینده پزشکی در متاورس پرداخته می‌شود.


\end{document}