تحلیل

طراحی

پیاده‌سازی

تست

فرایند‌هایی که در گذشته بکار می‌بردیم.

کارش با طراح و پیاده‌سازی و مدیر پروژه کاملاً متفاوته

داکیومنت چطوری پیاده‌سازی بشه، وظایف معمار چیه و چه چیزی باید تحویل داده بشه.


انتظارات در طول ترم:

۱۴ نمره نهایی
سمینار و پروژه

گروه‌های دو تا سه نفر.

ارائه از هفته سوم می‌تونه شروع بشه.

تاپیک‌ها رو استاد میده.

بعضیا زمان ارائه داره یعنی بعد از درس دادن استاد باید ارائه داده بشه.

حدودا تایم ارائه ۱۵ دقیقه تا ۲۰ دقیقه هستش.

پرزنت شفاهی کافیه

پروژه براساس ددلاین باید تحویل داده بشه.

نمودار یوزکیس در مهندسی نرم‌افزار

یسری یوزکیس داره

یوز کیس‌ها verb که مانند انتخاب کردن واحد

یوزکیس اصل کاره

تنها نموداریه که برای مشتری و طراح قابل فهمه

دانشجو انتظاراتش از سیستم چیه.

یوزکیس، نیاز‌های وظیفه‌مندی و غیر‌وظیفه‌مندی رو مشخص میکنه.

فانکشنال، انتظارات ما از سیستم هستش. از سیستم ثبت‌نام قاعدتاً انتظار نداریم رزرو
غذا انجام بدیم.

انتظار نداریم که بیمه کنیم.

انتظارات از سیستم میشه.

برای چه کاربرد و هدفی سیستم ایجاد میشه.

نان فانکشن

با چیزی سروکار داره که ویژگی‌های کیفی را در آن مشخص می‌کند.

Quality attribute

بتونیم اخذ واحد کنیم زیر دو ثانیه.

در حقیقت کیفیت اون کار رو مشخص میکنه.

سیستمی در حوزه سلامت نوشتیم می‌خواهیم این سیستم همیشه در دسترس باشه بر خلاف تمام
ویژگی‌هایی که داره.

اینا رو کیفیت رو انتظار داریم.

سند معماری نرم‌افزار

سناریو‌هایی که بالاتر گفتیم

یه سناریو در ازای یه سولوشن.

قدیم دستی می‌نوشتیم و میگفتیم این عدد 

تست‌ها رو از اول پروژه میزنیم و تا زمانی که مستقر بشه.

RUP متدولوژوی هستش که کاملاً قابل کاستومایز کردن هستش آرتیفکت‌های بسیار زیادی
وجد داره از جمله اکتیویتی‌ها و نقش‌ها.

محبوبیتش به خاطر اینه که به صورت توتاله

برای مقایس بزرگ تعریف میشه.

استاندارد ایارن که دکتر شمس تعریف کرده با 
زمان و انسانی

زمان بیشتر از ۶ ماه

و نفرات ۱۲ نفر

قطعاً هزینش میره بالا برای پیاده‌سازی

یک سال به بالا ۲۰ تا ۲۲ نفر استاندارد بین‌المللی.

فاز اول طراحی معماری نرم‌افزار هستش.

معماری چیست؟ متفاوته

میگه چه کامپوننت‌هایی داریم چه ارتباطاتی دارند و چه قید‌هایی درونش وجود داره

نمود خارجی المان‌ها رو میگن.

در معماری میگیم که یه کامپوننتی میخوایم برای مانیتورینگ.

جزیئیات استفاده از الگوریتم توش مطرح نمیشه.

معماری علاوه بر نیاز‌های جاری نیاز‌های آتی رو هم میدونه.

RUP متدولوژی رو توضیح باید بدن.

۴ فاز دارد.

فار آغاز

تشریح 

ساخت یا کانساتراکشن

دیپلوی یا استقرار

هر کدوم از این فاز‌ها انتهاش چیزی داره به اسم milestone یک بازه زمانی که
می‌بینیم از اون زمان قبلی چه کار‌هایی باید انجام میشده و آیا انجام دادم اگر
انجام ندادم تکرار میکنیم تا اون کار تموم بشه.

major milestone

داخل خودش minor milestone داره.

تکرار‌ها n تا هستند مدیر پروژه یا طراح سیستم باید به ما تعداد تکرار‌ها رو به
صورت تقریبی بگه.

هر فردی مسئول چک کردن بحث QC خودشو بر عهده داره.

یه مطالعه اولیه در مورد RUP داشته باشیم.

بعد داینامیک معماری و بعد استاتیک

چگونه در sequence diagram لوپ ایجاد کنیم.

ساده‌ترین داکیومنت دیپلوی منته.

کاموننت فاز ساخت هستش.

سیکوئنس بین فاز elaboration و build هستش

کلاس مهم‌ترین خروجی فاز elaboration


۹ تا نظم داره rup و اکتیویتی‌های فاز آخر خیلی زیاد خواهند بود.

منظور از نظم چیست؟

ون‌هایی که تو فرایند توسعه استفاده میشه.

نظم مدل سازی حرفه هستش

business processing modeling

نظم بعدیی در مورد requirement هستش.

هفت نظم دیگر در rup خوانده شود.

این نظم‌ها به ما کمک میکنن که چه زمانی چه اکتویتی ها رو به چه میزان در چه
بازه‌هایی باید اجرا کنم. اکتیویتی انجام میدیم خروجی چیه.

مثلا نظم نیازمندی فرایند زیاده که خرجوی این نظم تو این فاز میشه یوزکیس و سند
معماری نرم‌اغزار. 

خروجی تنظم در فاز elaboration دو تا داک یوزکیس و سندمعماریه.

یکی فانکشنال رو کاور میکنه یکی نان‌فانکشن رو کاور میکنه.

inception, Elaboration, Constructure and deploy are phase of RUP methodology.

قسمت دومی:

service orientend

یه تاپیک به عنوان ارائه

جزء ارائه آخر خواهد بود.

بحث استایل رو خواهیم داشت

رفنرس آرچیتکچل

و معماری

و بعدش ADL

Architectural pattern

adabi.sa@gmail.com

هفته آینده نمودار‌هایی که گفته شده رو بخونیم یوز کیس و اکتیویتی خیلی مهمه.

نحوه استفاده از یوزکیس‌ها هم خوانده شود مثل include و uses

