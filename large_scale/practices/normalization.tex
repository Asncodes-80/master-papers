\documentclass[a4paper]{article}
\usepackage{forest}
\usepackage{float}
\usepackage{makecell}
\usepackage{geometry}
\usepackage{listings}
\usepackage{hyperref}
\usepackage{graphicx}
\usepackage{ragged2e}
\usepackage{color}
\usepackage{xepersian}
\usepackage{subfiles}
\settextfont[Scale=1]{XB Roya}

\title{نرمال‌سازی و استانداردسازی داده‌ها در آمار}
\author{علیرضا سلطانی نشان}

\begin{document}
\maketitle

هیچ وقت ما نمی‌توانیم به صورت پیش فرض دو مقداری که دارای معیار‌های مختلفی هستند
را با یکدیگر محاسبه کنیم. برای مثال هیچ وقت نمی‌توانیم حاصل ضرب زیر را بدست
آوریم:

\begin{equation}
    40_{Km/h} \times 100_{kg}
\end{equation}

زیرا وقتی دو ویژگی مانند سرعت (کیلومتر در ساعت) و وزن (کیلوگرم) را باهم مقایسه
یا ترکیب می‌کنیم، تنها نرمالیزه کردن کافی نمی‌باشد و ممکن است به یک مفهوم فیزیکی
نرسیم. برای چنین شرایطی که مقیاس‌ها و واحد‌ها متفاوت هستند، چند راهکار وجود دارد
که می‌تواند به استاندارد‌سازی کمک کند:

\section*{بی واحد کردن داده‌ها یا \lr{Dimensionless Scaling}}

برای ایجاد مقیاس‌های یکسان، می‌توانیم داده‌ها را با استفاده از میانگین یا مقادیر
ماکسیمم و مینیمم خودشان بی واحد کنیم.

\begin{equation}
    \frac{(\frac{speed}{speed_{max}}) + (\frac{weight}{weight_{max}})}{2}
\end{equation}

\begin{equation}
    \frac{(\frac{40}{80_{max}}) + (\frac{100}{120_{max}})}{2} = 0.6665
\end{equation}

براساس معادله بالا داده‌های ما مقیاسی بدون واحد دارند.

در نظر داشته باشیم که با استفاده از محدوده داده‌های فراوان می‌توانیم به روش‌های
مختلفی محاسبات خود را انجام دهیم \cite{enwiki:1203519063}.

\begin{itemize}
    \item سرعت: $40_{km/h}$
    \item وزن: $100_{kg}$
\end{itemize}

فرض کنیم که از همان مثال بالا در دیتاست خود اطلاعات زیر را هم داریم:

\begin{itemize}
    \item میانگین سرعت در داده‌ها: $50_{km/h}$
    \item انحراف معیار سرعت: $20_{km/h}$
    \item میانگین وزن: $80_{kg}$
    \item انحراف معیار وزن: $25_{kg}$
\end{itemize}

\section{استانداردسازی با روش \lr{Z-score}}

هر مقدار را به صورت تفاضل از میانگین و تقسیم بر انحراف معیار، نرمال‌سازی می‌کنیم
تا نتیجه \lr{Z-score} بدست آید:

\begin{equation}
    \frac{x - \mu}{\sigma}
\end{equation}

برای سرعت $40_{km/h}$ خواهیم داشت:

\begin{equation}
    \frac{40 - 50}{20} = -0.5
\end{equation}

برای وزن $100_{kg}$ خواهیم داشت:

\begin{equation}
    \frac{100 - 80}{25} = 0.8
\end{equation}

حال دو مقداری که از روش نرمال‌سازی \lr{Z-score} حاصل شده است را می‌توان به عنوان
مقادیر بدون واحد در نظر گرفت و تمام مقایسه‌ها و تحلیل‌های مورد نظر خود را روی
آن‌ها انجام داد.

\section{واحد‌های نسبی یا \lr{Relative units}}

در واحد‌های نسبی ما باید مقدار ورودی مسئله را با یک مقدار استاندارد یا مبنا در
یک جامعه خاص بسنجیم:

فرض کنیم:

\begin{itemize}
    \item مقدار استاندارد سرعت به عنوان ورودی $40_{km/h}$ بوده است و در جامعه
    محدوده این داده $80_{km/h}$ می‌باشد.
    \item همچنین در محاسبه وزن داده ورود $100_{kg}$ بوده است و در جامعه محدوده
    این داده $120_{kg}$ می‌باشد.
\end{itemize}

فرمول:

\begin{equation}
    \frac{speed}{base speed} = \frac{40}{80} = 0.5
\end{equation}

\begin{equation}
    \frac{weight}{base weight} = \frac{100}{120} = 0.833
\end{equation}

حال نتایج بدست آمده کاملاً بدون واحد هستند و با روش واحد‌های نسبی محاسبه
شده‌اند:

\begin{itemize}
    \item سرعت نسبی: $0.5_{km/h}$
    \item وزن نسبی: $0.833_{kg}$
\end{itemize}

\bibliographystyle{unsrt-fa}
\bibliography{normalization_refs.bib}
\end{document}
