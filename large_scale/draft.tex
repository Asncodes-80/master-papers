تاکتیک چیست؟

برای هر ویژگی کیفی با مجموعه‌ای از تاکتیک‌ها رو داریم. هر تاکتیک روی یه ویژگی
کیفی مشخص اعمال میشه. تمرکز تاکتیک روی یه ویژگی کیفی هستش.

اگر چند ویژگی کیفی داریم باید پترن‌ها رو در نظر بگیریم.

اگر بیشتر از یه ویژگی کیفی رو بخوای باید بری سراغ arch patternsها

ویزگی کیفی کارایی رو در نظر بگیر:

برای اینکه به عدد برسیم به تاکتیک معمول زمان‌بندی می‌رسیم. باید این بشکنیم به یه
سری استراتژی‌ها.

به عنوان معمار اگر کارایی رو میخوایم باید تاکتیک اسکژولینگ استفاده کنی که
استراتژی این تاکتیک انتخاب الگوریتم هستش.

برای ویژگی کیفی یه سری تاکتیک داریم.

منبع تحریک هر چیزی به صورت خارجی چه نرم‌افزار باشه چه آدم باشه.

فالتی که شناسایی نشده باشه توی محرک هستش.

همه سیستم با هم دیگه کرش نمیکنه که میشه تنزل آبرو مندانه برگشتش هم همینه سیستم
تکه تکه بر میگرده.

۴ نوع فالت سوال امتحانه

فالت محرک برای ویژگی کیفی اویل ابیلیتی میشه. چون سیستم از دسترس خارج میشه.

omission زمانی که یه کمپوننتی نسبت به یک ورودی پاسخ نمیده.

تمرین: در یک سیستم خروجی مد نظر باشد که خروجی تولید نشود.

crash: اگه همه کامپوننت‌ها از امیژن رنج ببرن میشه کرش

timing: یه کامپوننت که پاسخ میده ولی انقدر پاسخش دیره یا زوده که دیگه جواب نمیده.

response: سیستم پاسخ میده ولی پاسخ صحیح نمیده. خطرناک‌ترین حالت ممکنه. باید از
تاکتیک‌هایی استفاده کنیم بتونیم متوجه بشیم.

تاکتیک‌ها

یا باید فالت‌ها رو اوکی کنیم و ریپیر مودمون رو بیاریم پایین.

تاکتیک‌ها

سوال امتحان، یه مسئله میگه که براساس اون میگه که ویژگی کیفی مشخص بکنه، و
تاکتیک‌هایی که براساس صورت مسئله مناسب میاد رو با ذکر دلیل بنویسیم.

تاکتیک‌های available سه دسته هستن.

دیتکت و ریکاور و جلوگیری کردنه. 

ممکنه بگه دو تا رو نام ببر و تعریف کنین

اسم تاکتیک بیاد اوکیه: 

ping/echo: 

یک پیام ریکوئست ریسپانس آسنکرون هستش که بین نود‌ها رد و بدل میشه برای اینکه
ببینم نودی در دسترس هست یا نه. یه پیغام هلو می‌فرستی و اگه جواب بده یعنی در
دسترس است. بیشتر برای فهمیدن تاخیر در شبکه استفاده میشه.

monitor: پایش کنی وضعیت سلامت یه اکسی رو که میتونه تو سطح شبکه باشه و میتونه سطح
کامپوننت باشه مثلا حملات ddos میتونه استفاده بشه. اکثر سیستم‌ها این سیسیتم رو
نیاز دارن.

heartpeat: پیغام رو نمیفرسته که متنظر جواب باشه به صورت دوره‌ای پیغام می‌فرسته
نگاه نمیکنه که کسی دریافت کرده یا نه. برای چک کردن ارتباطات استفاده می‌شود.

تاکتیک‌ها معمولاً ترکیبی مورد اسفتاده قرار میگیرد اینکه ما چطوری از مانیتور
استفاده میکنیم بستگی به سناریو دارد.

timestamp: برای متوجه شدن توالی اشتباه ایونت‌ها می‌باشد. در بحث distributed
message passing استفاده می‌شود.

Sanity checking: عملیات یه کامپوننت یا یه اوت پوت رو میگه که ببینه منطقی و قابل
تایید است یا نه، ذات اطلاعاتی که داریم روش دقت می‌کنیم. knowledge of the
internal design, the state of the system, or nature of the information under
security.

Condition Monitoring: چک کردن شرایطی که اون دستگاهه قرار بوده توش قرار بگیره. که
ببینیم مفروضات حفظ شده در طی دیزاین یا نه. یخچاله داغ شده باید از دیوار فاصله
می‌داشته. رعایت شدن مفروضات زمانی که سیستم‌ها ناهمگن است.

از صفحه voiting تا صفحه ۲۵ حتماً سوال امتحانه

voting: TMR Triple modular redundancy ۳ باشه چرا ۵ نگرفته چرا ۷ نگرفته به خاطر
هزینه‌ها هستش.

ممکنه جوابی تولید شود ولی درست و غلطش رو ندونیم میتونیم از این سیستم استفاده
کنیم. 

چیزی که تو امتحان می‌پرسه:
Voiting tactics:

replication: هر کامپوننتی که دقیقاً کلون از یه کامپوننت دیگر هستند. پس ورودی و
فانکشن یکسانی دارن و حتی یه خروجی طراحی می‌کنند. اگه سخت افزار فیلد بشه باید
جایگزین وجود داشته باشه. فرمی از تنوع وجود ندارد، چرا که همه چیز مثل هم هستن پس
هیچ واگرایی نسبت به هم ندارن. ماهیتش هستش.

functional redundancy:

سه تا کمپوننت داریم ورودی‌ها یکسانه ولی با فانکشن‌های مختلف انجام میده، مثلا
میخواد تقسیم کنه ۳ تا روش مختلف برای رسیدن بهش وجود دارد. ۴ داده و ۲ گرفته. بدرد
زمانی میخوره که اگه خطایی یه جا باشه تو جای دیگه نیست. یعنی روش‌ها چون فرق میکنه
ولی به جواب یکسان میرسه بهتره نسبت به قبلی که همه چیشون یکی بود.

% سومی جلسه بعدی:

% مسئله:

% یک سیستم مراقبت از سلامت داریم، که شامل یک ساعت هوشمند و یک سیستم آنالیز دیتا
% می‌باشد. ساعت هوشمند بر اساس سنسور‌های تعبیه شده در آن اطلاعات را میگیرد و
% اطلاعات پس از فیلتر شدن در اپلیکیشن سبکی که در سمت ساعت استیک آنالیز اولیه میشود
% با توجه به اینکه آنالیز نهایی می‌بایست براساس سابقه و پرونده پزشکی بیمار انجام
% شود اطلاعات به سمت ابر فرستاده می‌شود. در محیط ابر سیستم آنالیز اطلاعات بیمار با
% توجه به سوابق آنالیز نهایی را انجام می‌دهد و در صورت نیاز پیام مناسب را برای
% بیمار ارسال می‌کند. و نتیجه را در پرونده پزشکی ثبت می‌کند. همچنین اطلاعات بدست
% آمده را برای پزشک بیمار نیز ارسال می‌کند. در سیستم مورد نظر هشدار‌هایی نیز به
% بیمار برای مصرف دارو یا اقدامات احتمالی داده می‌شود.

% چه ویژگی کیفی داره و چه تاکتیک‌هایی باید روی آن اعمال شود.

% هفته بعد کلاس‌ها رو مشخص کنه بعد تاکتیک‌ها رو بگه.