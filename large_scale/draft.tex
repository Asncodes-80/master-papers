سر امتحان چطوری میشه، اول میگه توضیح بدین یا مثلاً مقایسه کنیم فرمولشون هم باید
بلد باشیم.

از سلاید پرفورمنس اسلاید ۱۱ رو باید توضیح دهد.

یک سری قطعاتی که می‌تواند vailable باشد ولی reliable نباشد؟

نیازمند تعریف availble هستش.

در زمانی که میخوایم از یه چیزی استافده کنیم ارائه سرویس انجام بده.

uptime / total time

reliable قابلیت اطمینان

یک ماشین رو در نظر بگیرید که ۶ دقیقه در ساعت داونه.

54 / 60 = 0.9 

available هستش

total time تو اویل ابیلتی خیلی مهمه

باید سناریو‌های معنا دار رو بررسی کنیم.

سیستم آموزشیار رو باید در ابتدای ترم یعنی اون بازه یک ماهه انتخاب واحد باید
availability استفاده بشه.

مثلا در مورد بانک اصلاً صادق نیست. باید ۲۴/۷ روز هفته در دسترس باشه.

تعریف reliability

یک موتور شارژی داریم که هر بار شارژش می‌کنی ۲ ساعت رانندگی میکنی.

میشه بری پنیر بخری آره

ولی نمیتونی بری باهاش کیش

یعنی براساس موقعیت میتونه قابل اطمینان باشه.

یک گوشی که ۱۰ ساعت باتری نگه می‌داره میشه گفت

x < 10 ساعت قابلیت اطمینان رو داره.

اسلاید: اماشین در یک ساعت ۶ دقیقه داون بوده.

میشه گفت کمتر از یک ساعت reliable

دقیقه ۵۴ دیگه reliable نیست.

یک سیستم می‌تواند availble باشد ولی برای یک صورت مسئله می‌تواند reliable باشد
برای یه صورت مسئله نمی‌تواند reliable باشد.

\lr{MDT, Mean Down Time}

متوسط زمانی که یک سیستم عملیاتی نیست.

معمولاً \lr{scheduled downtime} هستش.

System failure:
+ Fault detection and isolation فاقد هر کدونه از فالت‌های نرم‌افزاری باشه.
+ Parts Procurement منتظر خرید یه قطعه‌ای هستش.
+ System Repair

Scheduled downtime:
+ Preventive maintenance
+ Systemupgrade
+ Calibration
+ Other administration actions

این هم باید عدد کوچکی باشد که هر چقدر کمتر باشد availability سیستم بیشتر خواهد
بود.

دسترس پذیری اولین ویژگی کیفی که بسیار مهم هستش که در همه سیستم‌ها وجود دارد:

سناریو عمومی

سناریو عینی یا concreat

Availability

سیستم زمانی که آماده خدمات دهی هستش زمانی که می‌خوای ازش خدمات بگیری. دقیقاً
موقعی که می‌خوای ازش استفاده کنی آماده خدمت رسانی باشد. به نظر میرسه که با
reliability رابطه دارد که reliability پایه availability هستش. reliability +
repair = Availability. مینیمایز کردن زمان‌هایی که سیستم در حال سرویس دهی نیست.

اعداد availability:

سطر اول

زیر availability ۹۹٪

زیر downtime x

سیستمی که ۹۹ درصد اویل هست در بازه زمانی ۹۰ روز چقدر در حالت عدم سرویس‌دهی (داون
تایم) است.

99 / 100 = x / 99 * 24 تبدیل رو به ساعت

فرمولش نوشته بشه.

سوال امتحان جنرال سناریو و عینی سناریو

جنرال سناریو:

سناریو عمومی برای هر ویژگی کیفی یکی هستش. فاقد از ویژگی دامنه هر جایی یه سناریو داره.

کانکرید براساس دامین مسئله سناریو بزنی.

پارت‌ها ممکنه یکی باشه ولی جزئیات این سناریو‌ها در ویژگی‌های کیفی مختلف متفاوت هستش.

منبع تحریک
محرک 
محیط  درونش فراویده
پاسخ و معیار پاسخ
% فیگورش رو بنویس

جدول Availability genral scenario

۶ تا بخش برای تمام ویژگی‌های کیفی وجود دارد.

منبع تجریک، هر منبعی که ممکنه انسان باشه یا یه سیستم کامپویتری باشه. هر منبع که
تحریکی را برای سیستم ایجاد می‌کند. درخواست کارنامه. منبع میشه دانشجو. پرینت
شهریه محرکی در سیستم آموزشیار داره درخواست ریز نمره رو میده.

محرک: شرایطی است که اگر در سیستم ایجاد شود باید بهش پاسخ داده بشه. شرایطی که در
سیستم تعریف نکردیم و جز طراحی نبوده و اگه اتفاق بیوفته شرایط خاصی رو به بار
میاره. اکسپشن میشه.

درخواست‌هایی که یه دانشجو داره میشه محرک که بایستی بررسی بشه.

مدیر گروه لیست دروس اخذ شده آقای اکس رو ببینه، آیا باید بررسی بشه نه! شرایطی رو
در نظر میگیریم که براساس منبع تحریک اگه سیستم دریافتش کنه ممکنه سیستم دچار
مشکلاتی شود.

نتایج کنکور

تمام منابع تحریک و محرک‌ها قابل بررسی نیستن.

محیط: وضعیتی است که منبع تحریک داره تحریکی رو ایجاد میکنه و باید بررسی شود.
سیستم آموزشیار باید در زمان زیر بار کاری مورد بررسی قرار گیرد. محیط همیشه بار
کاری نیست. 

آرتیفکت یا فرآورده، کار نامه در حقیقت میشه موجودیتی که مبع تحریک میخواد چیزی رو
دریافت بکنه یا تغییرش بده. 

پاسخ زمانی که سیستم تحریک میشه باید بهش جواب بده. مثلاً پیام مناسبی به کاربر
نشون بده. سیستم زیر باره پیام میده که چند دقیقه دیگه برای ورود به سیستم تلاش
کنید.

معیار پاسخ، آن چیزی که انتظار داشتیم پاسخ بده یا انجام بده سیستم آیا انجام داده
یا خیر. مثلا گفتیم زیر ۲۰ ثانیه پیام رو نشون بده، آیا اتفاق افتاده یا نه؟

موجودیتی که میخواد روش عملیاتی انجام بشه که هدف تحریکه میشه آرتیفکت.

تاکتیک چیست؟

برای هر ویژگی کیفی با مجموعه‌ای از تاکتیک‌ها رو داریم. هر تاکتیک روی یه ویژگی
کیفی مشخص اعمال میشه. تمرکز تاکتیک روی یه ویژگی کیفی هستش.

اگر چند ویژگی کیفی داریم باید پترن‌ها رو در نظر بگیریم.

اگر بیشتر از یه ویژگی کیفی رو بخوای باید بری سراغ arch patternsها

ویزگی کیفی کارایی رو در نظر بگیر:

برای اینکه به عدد برسیم به تاکتیک معمول زمان‌بندی می‌رسیم. باید این بشکنیم به یه
سری استراتژی‌ها.

به عنوان معمار اگر کارایی رو میخوایم باید تاکتیک اسکژولینگ استفاده کنی که
استراتژی این تاکتیک انتخاب الگوریتم هستش.

برای ویژگی کیفی یه سری تاکتیک داریم.

جنرال سناریو برای availability دیاگرامشو بکش

منبع تحریک هر چیزی به صورت خارجی چه نرم‌افزار باشه چه آدم باشه.

فالتی که شناسایی نشده باشه توی محرک هستش.

همه سیستم با هم دیگه کرش نمیکنه که میشه تنزل آبرو مندانه برگشتش هم همینه سیستم
تکه تکه بر میگرده.

کانکریت توی سیستم و مختص سیستم هستش.

یه قسمتی از کوالیتی رو نگفته.

۴ نوع فالت سوال امتحانه

فالت محرک برای ویژگی کیفی اویل ابیلیتی میشه. چون سیستم از دسترس خارج میشه.

omission زمانی که یه کمپوننتی نسبت به یک ورودی پاسخ نمیده.

تمرین: در یک سیستم خروجی مد نظر باشد که خروجی تولید نشود.

crash: اگه همه کامپوننت‌ها از امیژن رنج ببرن میشه کرش

timing: یه کامپوننت که پاسخ میده ولی انقدر پاسخش دیره یا زوده که دیگه جواب نمیده.

response: سیستم پاسخ میده ولی پاسخ صحیح نمیده. خطرناک‌ترین حالت ممکنه. باید از
تاکتیک‌هایی استفاده کنیم بتونیم متوجه بشیم.

مثال هارت‌بیت مانیتور رو بنویس:

منبع تحریک سیستم مانیتورینگ ضربان قلب

محرک پیام برقراری ارتباط بوده

محیط نرماله

سرور میشه فرآورده

پاسخ میشه پاسخ به اوپراتور

معیار پاسخ میشه بدون هیچ داون تایمی

تاکتیک‌ها

یا باید فالت‌ها رو اوکی کنیم و ریپیر مودمون رو بیاریم پایین.

تاکتیک‌ها

سوال امتحان، یه مسئله میگه که براساس اون میگه که ویژگی کیفی مشخص بکنه، و
تاکتیک‌هایی که براساس صورت مسئله مناسب میاد رو با ذکر دلیل بنویسیم.

تاکتیک‌های available سه دسته هستن.

دیتکت و ریکاور و جلوگیری کردنه. 

ممکنه بگه دو تا رو نام ببر و تعریف کنین

اسم تاکتیک بیاد اوکیه: 

ping/echo: 

یک پیام ریکوئست ریسپانس آسنکرون هستش که بین نود‌ها رد و بدل میشه برای اینکه
ببینم نودی در دسترس هست یا نه. یه پیغام هلو می‌فرستی و اگه جواب بده یعنی در
دسترس است. بیشتر برای فهمیدن تاخیر در شبکه استفاده میشه.

monitor: پایش کنی وضعیت سلامت یه اکسی رو که میتونه تو سطح شبکه باشه و میتونه سطح
کامپوننت باشه مثلا حملات ddos میتونه استفاده بشه. اکثر سیستم‌ها این سیسیتم رو
نیاز دارن.

heartpeat: پیغام رو نمیفرسته که متنظر جواب باشه به صورت دوره‌ای پیغام می‌فرسته
نگاه نمیکنه که کسی دریافت کرده یا نه. برای چک کردن ارتباطات استفاده می‌شود.

تاکتیک‌ها معمولاً ترکیبی مورد اسفتاده قرار میگیرد اینکه ما چطوری از مانیتور
استفاده میکنیم بستگی به سناریو دارد.

timestamp: برای متوجه شدن توالی اشتباه ایونت‌ها می‌باشد. در بحث distributed
message passing استفاده می‌شود.

Sanity checking: عملیات یه کامپوننت یا یه اوت پوت رو میگه که ببینه منطقی و قابل
تایید است یا نه، ذات اطلاعاتی که داریم روش دقت می‌کنیم. knowledge of the
internal design, the state of the system, or nature of the information under
security.

Condition Monitoring: چک کردن شرایطی که اون دستگاهه قرار بوده توش قرار بگیره. که
ببینیم مفروضات حفظ شده در طی دیزاین یا نه. یخچاله داغ شده باید از دیوار فاصله
می‌داشته. رعایت شدن مفروضات زمانی که سیستم‌ها ناهمگن است.

از صفحه voiting تا صفحه ۲۵ حتماً سوال امتحانه

voting: TMR Triple modular redundancy ۳ باشه چرا ۵ نگرفته چرا ۷ نگرفته به خاطر
هزینه‌ها هستش.

ممکنه جوابی تولید شود ولی درست و غلطش رو ندونیم میتونیم از این سیستم استفاده
کنیم. 

چیزی که تو امتحان می‌پرسه:
Voiting tactics:

replication: هر کامپوننتی که دقیقاً کلون از یه کامپوننت دیگر هستند. پس ورودی و
فانکشن یکسانی دارن و حتی یه خروجی طراحی می‌کنند. اگه سخت افزار فیلد بشه باید
جایگزین وجود داشته باشه. فرمی از تنوع وجود ندارد، چرا که همه چیز مثل هم هستن پس
هیچ واگرایی نسبت به هم ندارن. ماهیتش هستش.

functional redundancy:

سه تا کمپوننت داریم ورودی‌ها یکسانه ولی با فانکشن‌های مختلف انجام میده، مثلا
میخواد تقسیم کنه ۳ تا روش مختلف برای رسیدن بهش وجود دارد. ۴ داده و ۲ گرفته. بدرد
زمانی میخوره که اگه خطایی یه جا باشه تو جای دیگه نیست. یعنی روش‌ها چون فرق میکنه
ولی به جواب یکسان میرسه بهتره نسبت به قبلی که همه چیشون یکی بود.

سومی جلسه بعدی:

مسئله:

یک سیستم مراقبت از سلامت داریم، که شامل یک ساعت هوشمند و یک سیستم آنالیز دیتا
می‌باشد. ساعت هوشمند بر اساس سنسور‌های تعبیه شده در آن اطلاعات را میگیرد و
اطلاعات پس از فیلتر شدن در اپلیکیشن سبکی که در سمت ساعت استیک آنالیز اولیه میشود
با توجه به اینکه آنالیز نهایی می‌بایست براساس سابقه و پرونده پزشکی بیمار انجام
شود اطلاعات به سمت ابر فرستاده می‌شود. در محیط ابر سیستم آنالیز اطلاعات بیمار با
توجه به سوابق آنالیز نهایی را انجام می‌دهد و در صورت نیاز پیام مناسب را برای
بیمار ارسال می‌کند. و نتیجه را در پرونده پزشکی ثبت می‌کند. همچنین اطلاعات بدست
آمده را برای پزشک بیمار نیز ارسال می‌کند. در سیستم مورد نظر هشدار‌هایی نیز به
بیمار برای مصرف دارو یا اقدامات احتمالی داده می‌شود.

چه ویژگی کیفی داره و چه تاکتیک‌هایی باید روی آن اعمال شود.

هفته بعد کلاس‌ها رو مشخص کنه بعد تاکتیک‌ها رو بگه.