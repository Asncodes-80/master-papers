
% تا اونجایی که می‌تونیم نرخ موفقیت یک تسک رو میخوایم بالا ببریم.

% چون ماهیت تسک‌ها با هم متفاوت و ما معمولاً ملقمه‌آی از همه تسک‌ها داریم. به همین
% دلیل الگوریتم‌های زمانبندی زیاد هستش.

% یه وقت اولویت مطرح هستش.

% تز اول میتونه، الگوریتم اولویت‌دهی تسک‌ها باشه

% مجموعه‌ای از تسک‌ها اومده کدوم تسک رو اساین بکنم به منبع task priority

% تسکی که اندازه در آن مهم است. موقعیت در گراف مهم است.

% ترکیبی از تسک‌های مهم و اولویت پایین روی هر منبعی داشته باشیم. اگه اون منبع با
% شکست خود ری‌اسکژولینگ نکنه.

% هفته بعد:

% اولویت‌دهی منابع به عنوان تز گفته شود. فصل زمانبندی تننباوم رو بخونیم توی
% ویژگی‌های کیفی و پرفورمنس اینا رو داریم.

% ددلاین هم میتونه پایان در نظر گرفته بشه هم در ابتدا.

% monitoring Broaker Resource monitoring Receiving task Task schedule