Useful concepts

پترن یا الگو راه‌حلی برای خانواده‌ای از مشکلات است.  در حقیقت به پترن استایل هم
میگن. وقتی می‌گیم که پترن کلاینت سرور یعنی دامنه حوزه کاریش مهم نیست. از نانوایی
بگیر تا سیستم‌های دیگر

الگوی بلک‌برد.

هر آن چیزی که قراره خوانده بشه و نوشته بشه در این الگو مشخصه برای حافظه مشترک
توی کامپیوتر استفاده زیادی داره.

layering

هر لایه نسبت به لایه‌های دیگر متمایزه وقتی نیازمند ایجاد امنیت هستیم نیازمند این
الگو هستیم.

مثلا لایه بندی برای مسئولین یه دانشگاه رو در نظر بگیرم که مسئولین سنگین بالا
طبقه هستن.

Arch patterns

الگو به چه دردی می‌خوره؟

الگو روی ویژگی کیفی کار میکنه. Quality Attribute 
الگو ارتباط مستقیم به ویژگی کیفی داره.

براساس ویژگی‌های کیفی الگوها رو انتخاب میکنیم بعضیا پرفورمنس رو کاور میکنن بعضیا
امنیت بعضیا ha رو و غیره.

اگه بخوایم ویژگی کیفی جدید ایجاد کنیم با الگو رو تغییر بدیم.

پترن‌ها عموماً به صورت ترکیبی استفاده می‌شوند.

تحلیل قضای مسئله رو مشخص میکنه
طراحی حل مورد نیاز برای فضای مسئله رو بررسی می‌کنه.

توی تحلیل سیستم رو می‌شناسی، ارتباط رو می‌شناسی، مشکل رو متوجه شدی.
برای پاسخ بهش توی طراحی وارد عمل می‌شیم.

نمودار‌ها در فضای حل مسئله هستش.
بحث نیازمندی‌ها در قسمت فضای مسئله هستش.

ما نمی‌توانیم در فاز طراحی نمی‌تونیم بپرسیم که این کامپوننت باید به اون کامپوننت
وصل بشه؟

هر معماری یک طراحی است. درست است

یا هر طراحی یک معماری است. غلط است چون جزئیات داره.

سوال کنکور بوده.

در معماری وارد جزئیات نمیشیم جریان ساخت خونه 

طراحی در مورد جزئیات میگه
معماری در مورد high abstract layer صحبت میکنه.

گام اول طراحی معماری هستش.

طراحی کله معماری میاد پایین پایین

در RUP فرایند تشریح اولین اسناد

AD: سند معماری هم باید نهایی شده باشه.

RD: ۸۰ درصد باید برآورد و نهایی شده باشه.

مایل استون

ref model 

مدل مرجع تقسیم وظیفه‌مندی یک سیستم. با جریان دیتا اون قسمت.

مثلا تو امور مالی هیچ وقت نمیریم انتخاب واحد کنیم.

واحد زمانبندی یک بروکر داره، یه واحد اسکور و اولویت‌دهی داره. یه پکیجه که داخلش
اینا هستش.

ref arch

اگر (لایه‌هاشو پیدا کرده فهمیده ۳ تا لایه میخواد واسه امنیت دایره‌ها بخش‌هاشه)
نگاشت المان‌های نرم‌افزاری و جریان بینشون اون دیاگرام رو بکش که فیگر ۲.۲. میشه.

معماری مرجع

سوال،

اگر بخوایم یه یونی بزنیم می‌تونیم از معماری مرجع دانشگاه‌های دیگه استفاده کنیم.

از ۱۷ تا ۲۵ سوال امتحانی

برای چی معماری ممهمه:

ارتباطات بین ذینفعان رو مشخص میکنه. هر ذینفعی مجموعه‌از نیازمندی‌ها رو داشت که
تضاد هم بینشون زیاد بود.

گام اول طراحی معماری نیاز‌های سطح بالای سیستم رو باید همون لحظه متوجه بشیم.

روش تخمین هزینه و زمانمون از اینجا مشخص میشه.

روش‌های کوکومو درمورد تخمین هزینه رو بخوانیم.

تو line of code اصلاً هیچ خلاقیتی دیده نمیشه.

معماری تصمیم اولیه طراحی می‌باشد.

ساختار سازمانی را می‌تواند به ما دیکته کند.

چه بخشی‌هایی رو پوشش میده؟ ارتباطات رو پوشش میده.

آموزش داریم پژوهش داریم.

قید بند‌ها رو روی سطح کد هم معماری حرفی برای گفتن داشته باشه و قید‌ها رو لحاظ
میکنن.

نیاز‌های آتی رو چون معمار میدونه میتونه تاکیید کنه که تکنولوژی رو مشخص کنه.

گاهی معمار و طراح یه نفر هستند تو بعضی پروژه‌ها

دائما یادت باشه که معماری با ویژگی‌های کیفی در ارتباطه

هم میتونیم زمان و هم میتونم هزینه‌ها رو دقیق‌تر تخمین بزنیم.

از یک معماری میتونیم استفاده مجدد هم بکنیم. به شرطی که ویژگی‌های کیفی یکسانی
داشته باشه.

تو آزمون جامع پرسیده می‌شود.

چرا نمیگیم سیستم آرچ به جای سافتور آرچ

یه بخشی‌هایی نادیده گرفته میشه موقع معماری مثلا نرم‌افزار یه طرفه، سخت افزار یه
طرف دیگه، منبع بهتر یکی از فاکتور‌هاست که cpuش سریع تره
یا مثلاً ارتباطات شبکه‌ای اصلاً در نظر گرفته نمی‌شود.

سرعت‌ها هزینه‌های تراکنش اینا هیچ وقت در معماری نرم‌افزار دیده نمیشه ولی در
معماری سیستمی دیده میشه.

توی سخت افزار اصلاً نمیتونیم منعطف باشیم بیشتر سمت نرم‌افزار هستیم ولی به طوری
که ویژگی کیفی ما کاملاً با سخت افزار در ارتباط هست.

ویو و ساختار کاملاً روی یک سکه هستند

از اسلاید ۳۹ نا اونجایی که ۵۴ سوال امتحانه:

معماری رو دیدگاه استاتیک در نظر میگیرن به جای داینامیک بچه‌های سافتور

General SA Structure داریم.

معماری دید ایستا نیست.

ماژول

به چه بخش‌هایی تقیسیم میشود

ارتباطات چه طوریه یوزز

هر کدام در چه لایه ای قرار می‌گیرند لیر

معمولاً کلاس رو می‌بینن میگن که استاتیکه یادت باشه در مورد پراپتری و رفتار بگی اخذ درس

cnc component and connector

پترن چیه

بحث همزمانی رو داریم

همروندی قسمتی موازی اجرا میشه

بحث داینامیک در بحث پراسس

یعنی پراسس در اکتیویتی هستش که داره یه فلو رو از صفر تا صد میگه.

سیکوئسن داری فراخوانی متد رو می‌بینی. که دید دانامیک رو داره سی اند سی نشون
میده.

یه نمودار کلاس می‌خوایم یه نمودار اکتیویتی می‌خوایم که با Business processing
modelling notation.

Allocation

در مورد اختصاص کار‌ها و اسناد مربوط به استقرار

\subsection{ساختار‌های تخصیص یا \lr{Allocation structure}}

ماژول‌ها دید استاتیک

سی اند سی دید داینمایمک برای ران‌تایمه

الکویشن هم اختصاص منابع هستش که همون استقراره

اسناد
استقرار
اکتیویتی
bpml

رفتار زمان اجرا میشه کامپوننت

اساین المان سخت افزاری به نرم‌افزاری میشه الوکیشن.

تمرین:

سند معماری یا سند ۴+۱ یا سند kruchten چه بخش‌هایی دارد و بخش چه سندی را تولید
خواهد کرد.

نیاز‌های نان فانکشن نمی‌تواند خارج از در نظر گرفتن فانکشنال سیستم باشد.

نیاز‌های علمیاتی سیستم را یوزکیس‌ها نمایش میدیم که سناریو‌ها میشه یوزکیس‌ها

جمع‌بندی:

اسلاید دوم، به صورت ویژه به ویژگی‌های کیفی می‌پردازد.

Responsiveness: چقدر یک تسک سریع می‌تواند توسط یک سیستم تمام شود. waiting time
and queue length.

Usage level: چقدر ما به صورت خوب و مطلوب می‌توانیم از المان‌های مختلف در سیستمان
استفاده کنیم. Throughput و Utilization معیار‌های اندازه گیری هستن.

تعاریف اسلاید دوم wating time

تمرین، فرمول نرمال سازی چیست اینکه تمام معیار‌ها یک بعد بشه.

Good put دیتایی است که باز ارسال نشهد یا سربارهیی که به ابتدا و انتهاش اضافه
می‌کنیم گود پوت نیست بلکه یه پکتی که سالم بدون هیچ باز ارسالی از سمت مبدا به
مقصد ارسال می‌شود.

ماموریت پذیری نشون میده اون سیستم تو اون بازه زمانی داشته عملیاتی رو انجام
می‌داده که خیلی با واژه Availability در ارتباط می‌باشد.

Mean time to fauilre

طول زمانی که انتظار داریم یک دستگاهی یا هر چیزی توی مدار باقی بماند.

سوال امتحان، فرق mttf mtbf اینه که mtbf رای محصولاتی استفاده میشکه قابل ریپیر
شدن هستن و اون یک یبرای محصولاتی است که قابل ریپیر نیستن.

HW is MTTF
SW is MTBF

Mean time to repair or recovery

اونجاهایی که نوشته one failure داخل اسلاید هست ازش استفاده کن.

این عدد باید کوچک شود تا سریع دوباره برنامه به مدار برگردد و عدد آپ بودن سیستم
باید بزرگ باشد.

تعداد فیل شدن هم باید عددش کوچیک بشه.

باز سیستم باید یک بازه صحیح باشد.

Mean time to between failures

تعداد شکتس‌ها کم بشه و تایم شکست خوردنمون هم کمتر بشه.

همه از نوع زمان‌ هستند.

سر امتحان چطوری میشه، اول میگه توضیح بدین یا مثلاً مقایسه کنیم فرمولشون هم باید
بلد باشیم.

باز‌های زمانی بررسی سسیستمان به صورت مشخص هفتگی ماهانه و غیره هستش.

% از سلاید پرفورمنس اسلاید ۱۱ رو باید توضیح دهد.