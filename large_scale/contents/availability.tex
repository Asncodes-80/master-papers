\section{دسترس‌پذیری}

دسترس‌پذیری یا \lr{Availability}  یعنی زمانی که می‌خواهیم از چیز استفاده کنیم و
آن چیز بایستی ارائه سرویس را انجام دهد.

\begin{equation}
    Availability = \frac{uptime}{Total Service Time}
\end{equation}

\subsection*{مثال}

یک ماشین هر یک ساعت، ۶ دقیقه داون است. مطلوب است محاسبه \lr{Availability} و
\lr{Reliability}:

\begin{equation}
    Uptime = 60-6 = 54
\end{equation}

\begin{equation}
    Availability = \frac{Uptime}{Total Service Time} = \frac{54}{60} = 0.9 or 90\%
\end{equation}

برای محاسبه قابلیت اطمینان می‌توان گفت که وقتی در یک ساعت ۶ دقیقی با قطع کارکرد
خودرو همراه هستیم، پس قابلیت اطمینان زیر یک ساعت یا کمتر از ۵۴ دقیقه است.

\subsection{بازه زمانی یا \lr{Total time}}

در دسترس‌پیذیرپذیری بررسی \lr{Total time} بسیار مهم است، چرا که سرویس در آن زمان
بایستی بدون مشکل در دسترس باشد و به صورت صحیح تا انتهای بازه مشخص \lr{Total
time} به کار خودش ادامه دهد. برای مثال سیستم آموزشیار بایستی در ابتدای ترم جهت
اخذ واحد درسی دانشجویان، در یک بازه یک ماهه به طور مثال کاملاً در دسترس و قابل
اطمینان باشد. اما با تغییر دامنه از سیستم انتخاب واحد دانشگاه به دامنه بانکی این
گفته صادق نیست، زیرا محصولات و سرویس‌‌های بانکی بایستی ۲۴ ساعته ۷ روز هفته در
دسترس باشند و کاملاً قابلیت اطمینان را به همراه داشته باشند.

\subsection{ارتباط میان \lr{Availability} با \lr{Reliability}}

عموماً وقتی سیستمی \lr{Reliable} است یعنی دارای \lr{Availability} بالایی است اما
وقتی سیستمی \lr{Available} است ممکن است آن سیستم قابل اطمینان باشد و ممکن است
قابل اطمینان نباشد. زمانی کاملاً قابل اطمینان است که تمام آن سیستم با آزمون‌ها و
ارزیابی‌ها پوشش داده شده باشد و فاقد هر گونه \lr{Fault} باشد و از سمتی در هنگام
استقرار نیز تمام نکات \lr{Availability} به عنوان ویژگی کیفی رعایت و پیاده‌سازی
شده باشند. به این صورت هم دسترس‌پذیری بالایی خواهد داشت هم از قابلیت اطمینان
بالایی برخوردار خواهد بود.

\subsection*{نکته}

در قابلیت اطمینان وابستگی به موقعیت می‌تواند عامل مشخص‌کننده‌ای باشد. برای مثال
با استفاده از یک موتور شارژی می‌توان درون شهر فعالیت کرد، اما با همان موتور
شارژی نمی‌توان به جنوب کشور سفر کرد.

\begin{equation}
    Availability = Reliability + Repair
\end{equation}

\subsection{\lr{Mean Down Time (MDT)}}

میانگین زمانی که یک سیستم قابل استفاده نباشد. \lr{MDT} با فاکتور‌های زیر همراه
است:

\begin{itemize}
    \item \lr{System failure}:
    \begin{itemize}
        \item سیستم به طور کلی فاقد هر گونه \lr{Fault} باشد.
        \item منتظر تامین قطعات نباشد
        \item سیستم نیاز به تعمیر داشته باشد.
    \end{itemize}
    \item \lr{Scheduled downtime}:
    \begin{itemize}
        \item نگهداری پیشگیرانه
        \item به روزرسانی سیستم
        \item کالیبراسیون
        \item سایر اقدامات اداری (\lr{Administrative})
    \end{itemize}
\end{itemize}

مقدار \lr{MDT} هر چقدر کمتر باشد دسترس‌پذیری نیز بیشتر خواهد بود.

\subsection{بدست آوردن مدت زمان \lr{Down time}}

برای محاسبه مدت زمان قطع سرویس یک سیستم از فرمول زیر استفاده کنیم:

\begin{equation}
    (Availability - 1) * Total Time = DownTime
\end{equation}

اگر یک سیستم در یک سال $99.99\%$ دسترس‌پذیری داشته باشد چند دقیقه \lr{Down time}
خواهد داشت:

\begin{equation}
    (0.9999 - 1) * 365D = Down Time
\end{equation}

\begin{equation}
    (0.0001) * 8760Hr = 0.876Hr
\end{equation}

\begin{equation}
    0.876Hr \rightarrow 52.56Min
\end{equation}

\begin{equation}
    52.56Min \rightarrow 52 Min \rightarrow 0.56Min
\end{equation}

در نهایت پاسخ ۵۲ دقیقه و ۳۴ ثانیه قطعی سرویس با $99.99\%$ دسترس‌پذیری می‌باشد.'

\subsection{تعریف کیفیت}

در استاندارد \lr{IEEE 1990} کیفیت به دو صورت تعریف می‌شود:

\begin{enumerate}
    \item چقدر یک سیستم، یک مولفه یا یک فرایند در برابر رویارویی با نیازمندی‌های
    پروژه موفق بوده است.
    \item چقدر یک سیستم، یک مولفه یا یک فرایند در برابر با رفع نیاز‌های کاربران
    موفق بوده است.
\end{enumerate}

\subsection{خصوصیات کیفی قابل مشاهده در زمان اجرای نرم‌افزار}

\begin{enumerate}
    \item کارایی
    \item امنیت
    \item قابلیت استفاده
    \item قابلیت دسترسی
\end{enumerate}

\subsection{خصوصیات کیفی غیرقابل مشاهده در زمان اجرای نرم‌افزار}

\begin{enumerate}
    \item قابلیت اصلاح
    \item قابلیت آزمانیش
    \item قابلیت استفاده مجدد
    \item قابلیت یکپارچگی
    \item قابلیت حمل
\end{enumerate}

\subsection{سناریو‌های خصوصیات کیفی}

\subsubsection{\lr{Source of Stimulus} یا منبع تحریک}

منبع تحریک شامل بعضی از موجودیت‌ها از قبیل، انسان، سیستم کامپیوتری، نرم‌افزار‌ها
و هر محرک دیگری است که یک تحریک را در سیستم ایجاد می‌کند یا به بیانی دیگر موجود
تولید یک تحریک می‌شود.

\begin{itemize}
    \item در درخواست کارنامه توسط دانشجو، دانشجو منبع تحریک می‌باشد.
    \item در چاپ شهریه منبع تحریک کسی است که درخواست آن را ارسال می‌کند.
\end{itemize}

\subsubsection{\lr{Stimulus} یا محرک}

محرک وضعیتی است که درسیستم ایجاد شده است و لازمه مورد بررسی قرار گرفتن (باید به
آن پاسخ داده شود) می‌باشد.

\subsubsection{\lr{Environment} یا محیط}

محیطی که در آن منبع تحریک یک وضعیتی یا محرکی ایجاد کرده است. برای مثال زمانی که
یک تحریک رخ می‌دهد ممکن است سیستم در حال اجرا باشد، یا هر وضعیت دیگری مانند
\lr{Shotdown} یا \lr{Standby}.

\subsubsection{\lr{Artifacts} یا فرآورده‌ها}

موجودیتی است که روی آن تحریکی انجام شده است. فرآورده ممکن است کل سیستم یا بخشی
از آن باشد.

\subsubsection{\lr{Response} یا پاسخ}

پاسخ فعالیتی است که سیستم بعد از تحریک شدن انجام می‌دهد.

\begin{itemize}
    \item پیام مناسبی را به کاربر نشان بدهد.
    \item سیستم زیر بار محاسباتی شدید است، در زمانی مشخص پیام دهد که «چند دقیقه
    بعد برای ورود تلاش کنید».
\end{itemize}

\subsubsection{\lr{Response Measure} یا معیار پاسخ}

وقتی که پاسخی بعد از تحریک شدن داده می‌شود باید بتوان آن را به روشی مناسب و مشخص
اندازه‌گیری کرد تا نیازمندی‌های مورد نظر بتواند مورد آزمایش قرار بگیرند.

\subsubsection*{نکته}

\begin{itemize}
    \item تمامی منابع تحریک و محرک‌ها قابل بررسی نیستند.
    \item محیط همیشه بار کاری نیست.
\end{itemize}

\begin{figure}[H]
    \centering
    \includegraphics[width=0.8\textwidth]{images/main_section_of_general_scenario.png}
    \caption{بخش‌های اصلی سناریو خصوصیات کیفی}
    \label{fig:generalScenarioMainSections}
\end{figure}

\subsection{\lr{General scenario} یا سناریو عمومی}

سناریو عمومی برای هر ویژگی کیفی \footnote{\lr{Quality attributes}} یک معیار
می‌باشد و فاقد از ویژگی‌های دامنه هر جایی یک سناریو دارد. یا به عبارتی دیگر،
سناریو‌های عمومی مستقل از سیستم هستند و در ارتباط با هر سیستمی می‌توانند باشند.

\subsection{\lr{Concrete scenario} یا سناریو عینی}

سناریو‌های عینی براساس ویژگی‌های دامنه هر پروژه‌ای متفاوت می‌باشند یا به عبارتی
دیگر برای سیستم‌های خاص مشخص می‌شوند.

\subsection*{نکته}

خصوصیات کیفی در سناریو‌های عمومی و عینی دقیقاً مانند هم هستند فقط موارد‌ آن‌ها
نسبت به دامته متفاوت مطرح می‌شوند.

\subsection{مثال سناریو عینی}

پایش ضربان قلب تعین می‌کند که سرور در شرایط نورمال پاسخگو نیست. سیستم به
اوپراتور اطلاع می‌دهد و فرایند‌های خود را بدون داون‌تایم انجام می‌دهد
\footnote{\begin{LTR}
    The heatbeat monitor determines that the server is nonresponsive during
    normal operations. The system informs the operator and continues to operate
    with no downtime.
\end{LTR}}.

\begin{enumerate}
    \item عامل تحریک: عامل خارجی پایش ضربان قلب
    \item محرک: ارسال پیام از کلاینت به سرویس برای بررسی دریافت پیام به صورت صحیح
    \item محیط: محیط انجام این عملیات کاملاً نورمال است.
    \item فرآورده: سرور در حقیقت مورد بررسی قرار گرفته است.
    \item پاسخ: اطلاع به اوپراتور که سرور کار نمی‌کند.
    \item معیار پاسخدهی: بدون داون تایم
\end{enumerate}

\begin{figure}[H]
    \centering
    \includegraphics[width=0.6\textwidth]{images/heartbeat-system-concrete-scenario.png}
    \caption{سناریو عینی برای قابلیت دسترسی در مثال بالا}
    \label{fig:heartbeatExampleConcreteScenrario}
\end{figure}

\subsection{سناریو عمومی برای ویژگی کیفی دسترس‌پذیری}

\begin{LTR}
    \begin{table}[H]
        \centering
        \begin{tabular}{|>{\raggedright\arraybackslash}p{0.25\textwidth}|>{\raggedright\arraybackslash}p{0.7\textwidth}|}
            \hline
            \textbf{Portion of Scenario} & \textbf{Possible Values} \\
            \hline
            Source & Internal/external: people, hardware, software, physical infrastructure \\
            \hline
            Stimulus & Fault: omission, crash, incorrect timing, incorrect response \\
            \hline
            Artifact & System’s processors, communication channels, persistent storage, processes \\
            \hline
            Environment & Normal operation, startup, shutdown, repair mode, degraded operation, overloaded operation \\
            \hline
            Response & Prevent the fault from becoming a failure \\
            & \textbf{Detect the fault:} \\
            & \quad $\bullet$ Log the fault \\
            & \quad $\bullet$ Notify appropriate entities (people or systems) \\
            & \textbf{Recover from the fault:} \\
            & \quad $\bullet$ Disable source of events causing the fault \\
            & \quad $\bullet$ Be temporarily unavailable while repair is being effected \\
            & \quad $\bullet$ Fix or mask the fault/failure or contain the damage it causes \\
            & \quad $\bullet$ Operate in a degraded mode while repair is being effected \\
            \hline
            Response Measure & Time or time interval when the system must be available \\
            & Availability percentage e.g., $99.999\%$ \\
            & Time to detect the fault \\
            & Time to repair the fault \\
            & Time or time interval in which system can be in degraded mode \\
            & Proportion e.g., $99\%$ or rate e.g., up to $100$ per second of a certain class of faults that the system prevents, or handles without failing \\
            \hline
        \end{tabular}
        \caption{Scenario Portions and Their Possible Values}
        \label{tab:scenario-possible-values}
    \end{table}
\end{LTR}