\section{دسترس‌پذیری}

دسترس‌پذیری یا \lr{Availability}  یعنی زمانی که می‌خواهیم از چیز استفاده کنیم و
آن چیز بایستی ارائه سرویس را انجام دهد.

\begin{equation}
    Availability = \frac{uptime}{Total Service Time}
\end{equation}

\subsection*{مثال}

یک ماشین هر یک ساعت، ۶ دقیقه داون است. مطلوب است محاسبه \lr{Availability} و
\lr{Reliability}:

\begin{equation}
    Uptime = 60-6 = 54
\end{equation}

\begin{equation}
    Availability = \frac{Uptime}{Total Service Time} = \frac{54}{60} = 0.9 or 90\%
\end{equation}

برای محاسبه قابلیت اطمینان می‌توان گفت که وقتی در یک ساعت ۶ دقیقی با قطع کارکرد
خودرو همراه هستیم، پس قابلیت اطمینان زیر یک ساعت یا کمتر از ۵۴ دقیقه است.

\subsection{بازه زمانی یا \lr{Total time}}

در دسترس‌پیذیرپذیری بررسی \lr{Total time} بسیار مهم است، چرا که سرویس در آن زمان
بایستی بدون مشکل در دسترس باشد و به صورت صحیح تا انتهای بازه مشخص \lr{Total
time} به کار خودش ادامه دهد. برای مثال سیستم آموزشیار بایستی در ابتدای ترم جهت
اخذ واحد درسی دانشجویان، در یک بازه یک ماهه به طور مثال کاملاً در دسترس و قابل
اطمینان باشد. اما با تغییر دامنه از سیستم انتخاب واحد دانشگاه به دامنه بانکی این
گفته صادق نیست، زیرا محصولات و سرویس‌‌های بانکی بایستی ۲۴ ساعته ۷ روز هفته در
دسترس باشند و کاملاً قابلیت اطمینان را به همراه داشته باشند.

\subsection{ارتباط میان \lr{Availability} با \lr{Reliability}}

عموماً وقتی سیستمی \lr{Reliable} است یعنی دارای \lr{Availability} بالایی است اما
وقتی سیستمی \lr{Available} است ممکن است آن سیستم قابل اطمینان باشد و ممکن است
قابل اطمینان نباشد. زمانی کاملاً قابل اطمینان است که تمام آن سیستم با آزمون‌ها و
ارزیابی‌ها پوشش داده شده باشد و فاقد هر گونه \lr{Fault} باشد و از سمتی در هنگام
استقرار نیز تمام نکات \lr{Availability} به عنوان ویژگی کیفی رعایت و پیاده‌سازی
شده باشند. به این صورت هم دسترس‌پذیری بالایی خواهد داشت هم از قابلیت اطمینان
بالایی برخوردار خواهد بود.

\subsection*{نکته}

در قابلیت اطمینان وابستگی به موقعیت می‌تواند عامل مشخص‌کننده‌ای باشد. برای مثال
با استفاده از یک موتور شارژی می‌توان درون شهر فعالیت کرد، اما با همان موتور
شارژی نمی‌توان به جنوب کشور سفر کرد.

\subsection{\lr{Mean Down Time (MDT)}}

میانگین زمانی که یک سیستم قابل استفاده نباشد. \lr{MDT} با فاکتور‌های زیر همراه
است:

\begin{itemize}
    \item \lr{System failure}:
    \begin{itemize}
        \item سیستم به طور کلی فاقد هر گونه \lr{Fault} باشد.
        \item منتظر تامین قطعات نباشد
        \item سیستم نیاز به تعمیر داشته باشد.
    \end{itemize}
    \item \lr{Scheduled downtime}:
    \begin{itemize}
        \item نگهداری پیشگیرانه
        \item به روزرسانی سیستم
        \item کالیبراسیون
        \item سایر اقدامات اداری (\lr{Administrative})
    \end{itemize}
\end{itemize}

مقدار \lr{MDT} هر چقدر کمتر باشد دسترس‌پذیری نیز بیشتر خواهد بود.

\subsection{بدست آوردن مدت زمان \lr{Down time}}

برای محاسبه مدت زمان قطع سرویس یک سیستم از فرمول زیر استفاده کنیم:

\begin{equation}
    (Availability - 1) * Total Time = DownTime
\end{equation}

اگر یک سیستم در یک سال $99.99\%$ دسترس‌پذیری داشته باشد چند دقیقه \lr{Down time}
خواهد داشت:

\begin{equation}
    (0.9999 - 1) * 365D = Down Time
\end{equation}

\begin{equation}
    (0.0001) * 8760Hr = 0.876Hr
\end{equation}

\begin{equation}
    0.876Hr \rightarrow 52.56Min
\end{equation}

\begin{equation}
    52.56Min \rightarrow 52 Min \rightarrow 0.56Min
\end{equation}

در نهایت پاسخ ۵۲ دقیقه و ۳۴ ثانیه قطعی سرویس با $99.99\%$ دسترس‌پذیری می‌باشد.